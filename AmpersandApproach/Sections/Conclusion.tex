% !TEX root = ../Parsing.tex

\section{Conclusion}
\label{sec:conclusion}
In \autoref{sec:libraries}, the choice was made to use a combinator library for the new parser of Ampersand.
The main reason to avoid the parser generators is that it is hard to generate useful feedback.
Then, in \autoref{sec:errors}, it was made even more clear that besides generating good messages, those messages should also be customizable.
Therefore, the advise is to use the combinator library that offers the highest level of customization in error messages, Parsec.

Although the uu-parsinglib seems to also be a very good choice, the experiences from the Helium compiler \cite{helium-parser} should be also considered.
Besides, the Parsec library offers better support.
Therefore, unless the error correction possibilities are deemed necessary or the customer is not convinced by these arguments and prefers to keep on using the uu-parsinglib, the Ampersand parser will be rebuilt with Parsec.

Besides the choice of the library, a list of important consideration points has been collected in the literature and can be found in \autoref{sec:errors}, more specifically \ref{subsec:errors-ampersand}.