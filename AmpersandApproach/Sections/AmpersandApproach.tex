% !TEX root = ../Parsing.tex

\section{The Ampersand Methodology}
\label{sec:AmpersandApproach}

\subsection{Software requirements, the problem statement}
\dict{Lexical analysis}{Separating text into tokens}%
\dict{Lexer}{Software that does the lexical analysis}%
\dict{Alex}{Lexer included in the Haskell Platform}%

In 1976,  T. E. Bell todo:cite4 investigated the domain of requirement engineering, todo:cite, sponsored by the Ballistic Missile Advanced Technology Center with the goal to determine the magnitude and characteristics of requirement related problems and to indicate what type of techniques could correct the issues. One of the main conclusions was that requirement errors were the most numerous and that these kind of errors are very time-consuming and hence costly to correct.
In his conclusion,  T. E. Bel advises to use methods and techniques during the requirement engineering process to ensure consistency within and  between the requirements, such as the unique naming of objects and correct relations between the requirements themselves, in addition, he stressed that the applied methods must aid the verification and validation of the requirements.
 
Although the research is ancient history from an IT point of view, the conclusion is actually very relevant and therefore methods and techniques to further improve the quality and consistency of the captured requirements are still a hot topic within the domain of Requirement Engineering.

An important distinction classification of requirements is made by  Prof.dr. Stef Joosten todo:cite1 and Alex Borgida todo:cite, into early phase requirements, the business requirements, and the late phase requirements, the functional specifications. 

\subsubsection{Formal functional specifications}
The initial focus, in which the research still continuous after 30 years, was on the functional specifications resulting in formal methods for functional specifications, (todo:cite 3) which are based on a mathematical representation of the specifications resulting in the ability to analyze, validate and transform them into useful artifacts during the subsequent design and implementation phases. 

As described by  Luqi todo:cite 10, these formal methods had a positve impact on the reliability of the software development process, specifically for the purpose of specification analysis, transformation and verification todo:cite 7, this  reflects  in the elaboration of several mature formal methods.
The formal methods are categorized into two domains:

\begin{description}
	\item[Algebraic languages] describing the system in terms of operations and their relationships, such as Z todo: cite 8, VDM todo: cite 11  and Alloy (url:alloy.mit.edu) todo: cite 12 todo: cite 17
	\item[Model based languages] using mathematical sets and sequences to to express the system specification as a system state model such as OBJ todo: cite 13, Larch todo: cite 14 and LOTOS todo: cite 15
	\item[Hybrid languages] combining features of both algebraic as model-based languages like RAISE todo: cite 16 and CafeOBJ, and enhancement of the OBJ language, todo: cite 17
\end{description}

\subsubsection{Business requirements}

Although the positive impact of the formal specification methods, it was recognized that these techniques were not sufficient. Concisely summarized by todo: cite 18, the formal specification methods focus on the `whats´ and the  `hows´ of the desired system without an understanding of the `whys´ behind them. 

Within the social environment in which a software systems will be introduced, several stakeholders have different goals, the business requirements, and based on these, they will express, often imprecise and inconsistent, expectations. Only by the correct capturing and understanding of the business requirements, to a feasible extend,the requirement engineers will truly understand the `whys´ needed to crystallize the correct  `whats´ and  `hows´ to support these different goals.

Several early phase requirement modeling languages, RMLs, are introduced and typically included, todo: cite 2, 
\begin{description}
	\item[An ontology of requirements] describing the needed information to capture and the desired properties and behavior of of the to-be system the view on the world from the perspective of the to-be system. This includes instances such as `Entity', `Activity' and `Assertion'. todo: cite 4
	\item[Modeling primitives] to model the concepts and the relations within the ontology.
	\item[Methods], sometimes automated, to verify consistency and to perform additional analysis to verify if the stated requirements will satisfy the business expectations.
\end{description}

Early RMLs such as RML, todo: cite 4, provided initial methods but suffered from drawbacks such as extensibility as the provided ontologies were rather fixed meaning that no new notions on par with the existing instances could be addressed. CML todo: cite 19, which evolved to Telos todo: cite 20, contained already additional flexibility. 

KAOS, todo: cite 21, introduced the notion of `stakeholder goal' where i*, todo: cite é&, even differentiated between independent and interrelated, joint goals.
Further research to further introduce the concept of goal priorities is ongoing for which a new abstract requirements modeling language `Techne' todo: cite 2 is designed, based on the CORE ontology for requirements todo: cite 22. This will provide the framework for new RMLs which will be able to compare candidate solutions and their compliance towards the business requirements, a feature that will come in handy as many software engineering projects nowadays includes COTS package based solutions.

\subsubsection{Limitations and frequent issues}
Although the significant improvements and their positive impact in software engineering projects, several important drawbacks related to the use of formal methods are identified todo: cite 10, todo: cite 1:
\begin{description}
	\item[Communication] Communication between the business users and the requirements engineers based on the mathematical model is difficult as the business users are not used to use mathematical notations. Although the functional specifications are analysed to make sure they are consistent, it still offers no guarantee that the business requirements are consistent and tranfsormed correctly into functional specifications.
	\item[Typical experience and knowledge of developers] Software developers are not used to develop based on the mathematical models, or even lack the skills of higher mathematics. It requires extensive training for these developers to understand the mathematical models and to develop efficiently in these kind of software projects.
	\item[Theoretical approaches] Many formal methods are theoretical and their appliance is demonstrated by means of very simplified example but when they are effectively used in practice, the gap between specification and effective coding appears to be problematic.
	\item[Supporting tools] Most formal methods don't have supporting tools making it very cumbersome to use them in larger scaled projects, even whem there are supporting tools available, they often lack a suitable user interfaces.

to add: agilie, COTS, transformation from early stage to later stage, structure...
	
\end{description}


\subsection{The goals of the Ampersand Methodology}
   
The Ampersand Methodology was founded in 2007 by the inventor Prof.dr. Stef Joosten with the vision to provide an answer to several of the main drawbacks of using formal techniques  in software engineering. The Ampersand Methodology is  developed to provide a method to unify the informal process of capturing the needs of users with the formal process of specifying an information system and to provide a formal translation method between both processes, including the necessary tools to ensure that the methodology is useful in real life projects. \cite{Joosten_derivingfunctional}. Special attentions is given to the verification that the functional specifications satisfies the specified business requirements´ todo: cite 1. 

Bottom line, the methodology presents the means to structure business needs in such a way that they can both validated by the users as well as be unambiguously interpreted by the system engineers after an automated transformation into functional specifications.

The Ampersand Methodology addresses several goals to achieve tis vision:
\begin{description}
	\item[Communication]  To assure that the stakeholders can correctly understand the requirements, the requirements must be documented in natural language to enable users, without any requirement engineering knowledge, to validate the system requirements. On the other hand, the requirements must be structured in a formal as functional specifications to make them useful during the actual software development phase. 
	\item[Completeness]A software system in which not all requirements are supported is useless. The Ampersand language bust be fully declarative, meaning that all the requirements must be supported to assure that all business requirements, relevant to the subsequent software system, are accommodated correctly in the system specification.
	\item[Consistency]One of the main goals of the Ampersand Methodology is to guarantee consistency, each specified requirement is therefore applicable to all processes in the context of this requirement. The methodology needs to provide the means to assure this consistency
	\item[Traceability] The requirements must be traceable 
	\item[User support] In order to facilitate the adoption of the Ampersand Methodology outside an academic environment, Prof.dr. Stef Joosten defined the additional goal that the Ampersand Methodology must be accompanied with supporting tools to support the requirement engineers and this by creating design artifacts to be used by the software developers.
\end{description}


\subsection{The Ampersand approach and rules}
\dict{DSL}{Domain specific language}%
The Ampersand Methodology introduces a specific approach how to achieve the specified goals. 

\subsubsection{Communication}

One of the key differentiators is that the business requirements are presented towards the stakeholder in natural language while the system engineers can use functional specifications which are automatically transformed from the the business requirements assuring the the functional specification is consistent with the business requirements.

The innovative aspect of the AM is that the business requirements are  represented as `business rules' using relational algebra, a business rule must be seen as a business requirement in the form of an invariant to be satisfied by the business. The business rules are transformed in an automated way by an accompanying tool, ADL, assuring the correctness of the functional specification based on the relational algebra of the business rules. Once the functional specifications are generated, the ADL tool will transform them back into the business rules for business user validation. When the re-translated requirements are still correct, the system engineers have the assurance that the functional specification are fully consistent with the business requirements. 

\subsubsection{Completeness}

\subsubsection{Consistency}

\subsubsection{Traceability}


\subsubsection{user support}




\noindent

\noindent

\subsection{Conclusion}
The following differences have been found between the two considered libraries:

\end{description}
%
Considering these differences, a deeper analysis of error messages is given in the next section.
The actual advice on the library choice is delayed until \autoref{sec:conclusion}.
