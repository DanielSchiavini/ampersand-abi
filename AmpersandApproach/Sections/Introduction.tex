% !TEX root = ../Parsing.tex
\section{Introduction}
\subsection{Identification}
This document contains the domain \& techniques analysis of the project `Useful feedback in the Ampersand parser'.
The document is the milestone product of the project phase 3a for Maarten Baertsoen, as specified in the project planning  \citenac{plan}.

This document is part of the graduation project of the computer science bachelor at the Open Universiteit Nederland.
The project `Useful feedback in the Ampersand parser' is executed in collaboration with Daniel S.C. Schiavini, with support of the supervisor Dr. Bastiaan Heeren and examiner Prof.dr. Marko C.J.D. van Eekelen.
The assignment is given by Prof.dr. Stef Joosten, who researches how to further automate the design of business processes and information systems by the development of the Ampersand project.

Ampersand is an approach for the use of business rules to define the business processes.
Users describe the business rules in a formal language (ADL), and Ampersand compiles those rules into functional specification, documentation and working software prototypes.
The main objective of this project is to improve the feedback and maintainability of the Ampersand parser.
See \citenac{plan} for more details on the project.

\subsection{Goals}
The main objective of this phase is to gather information that will support the execution of the project.
This document contains the results of the research on domain and techniques that will support the project team.
The Ampersand approach is twofold: Ampersand provides a methodology on one side and on the other hand, it provides the tools and accelerators to support actual usage of the methodology in software engineering projects, this document touches both topics:
\begin{description}
	\item[The Ampersand Methodology]
	In order to build the new Ampersand Parser (or re-factor the current one), it is crucial that the project members correctly and completely understand the drivers of the Ampersand methodology and the vision behind it.
	This part will outline the history and current status of formal requirements and formal specifications.
	 Against the backdrop of this status, and the known drawbacks, the vision behind the Ampersand methodology will be illustrated as well as how this vision is realized.
	The goal is to provide a sufficient insight in the `whys' and the `hows' behind the methodology to make sure that these can be respected in the next phases of the project.
	
	\item[Ampersand in practice]
	Besides the theoretical approach, it is important to know how Ampersand is actually used in practice. This section will outline how to get started, which tools that are used, the way the business requirements are gathered and how these can be further processed according to the methodology. Some remarks are made towards usability which can be taken into account taking design decisions in subsequent project phases.
	
\end{description}
%
The results of both subjects culminate in a single section with a SWOT analyze, the future road map and some conclusive remarks of this research.

\subsection{Document overview}
An introduction is given is this section.
In section \autoref{sec:AmpersandTheory} the history of functional methodologies is outlined followed by the Ampersand approach to solve some striking drawbacks.
Section \autoref{sec:InPractice} will briefly sketch the way Ampersand is used in followed by a summary of the benefits and issues of the methodology and how this methodology can further contribute to the domain of requirement engineering in section \autoref{sec:conclusion}.

Finally, in the appendix, a glossary of terms, definitions and abbreviations is given, just as a list of references.
