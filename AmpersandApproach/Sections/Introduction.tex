% !TEX root = ../Parsing.tex
\section{Introduction}
\subsection{Identification}
This document contains the domain \& techniques analysis of the project `Useful feedback in the Ampersand parser'.
The document is the milestone product of the project phase 3a for Maarten Baertsoen, as specified in the project planning \cite{plan}.

This document is part of the graduation project of the computer science bachelor at the Open Universiteit Nederland.
The project `Useful feedback in the Ampersand parser' is executed in collaboration with Daniel S.C. Schiavini, with support of the supervisor Dr. Bastiaan Heeren and examiner Prof.dr. Marko C.J.D. van Eekelen.
The assignment is given by Prof.dr. Stef Joosten, who researches how to further automate the design of business processes and information systems by the development of the Ampersand project.

Ampersand is an approach for the use of business rules to define the business processes.
Users describe the business rules in a formal language (ADL), and Ampersand compiles those rules into functional specification, documentation and working software
prototypes.
The main objective of this project is to improve the feedback and maintainability of the Ampersand parser.
See \cite{plan} for more details on the project.

\subsection{Goals}
The main objective of this phase is to gather information that will support the execution of the project.
This document contains the results of the research on domain and techniques that will support the project group.
The Ampersand approach is twofold: Ampersand provides a methodology on one side and on the other hand, it provides the tools to support actual usage of the methodology in software engineering projects together with some accelerators, this document both of these topics:
\begin{description}
	\item[The Ampersand Methodology]
	In order to build the new Ampersand Parser (or refactor the current one), it is crucial that the project members correctly and completely understand the Ampersand methodology and the vision behind it.
	This part will outline the Ampersand methodology and the practical use together with some additional investigation topics such as usability, future roadmap and the supporting tools.
	The goal is to provide a sufficient insight in the methodology to start the next phase of the project will providing the necessary references in case more detailed information is needed.
	
	\item[The practical use of the Ampersand tools]
	Embedding useful feedback in the tools that support the Ampersand tools requires a thorough understanding of the way the Ampersand tools are used.
	This part will further elaborate on the way the Ampersand methodology is used on a project, the way the  and how the tools need to be used within a project, the way th
	This part of the research is done by consulting literature.
\end{description}
%
The results of both subjects culminate in a single section with research conclusions.

\subsection{Document overview}
An introduction is given is this section.
Then, in \autoref{sec:libraries} the choices of user-friendly error messages are elaborated.
In \autoref{sec:errors} the qualities of user-friendly error messages are briefly described, and in \autoref{sec:conclusion} the final conclusion is given.

Finally, in the appendix, a glossary of terms, definitions and abbreviations is given, just as a list of references.
