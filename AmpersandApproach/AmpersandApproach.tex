\documentclass[a4paper,12pt,abstracton,titlepage]{scrartcl}
\usepackage{scrpage2}
\usepackage[utf8]{inputenc}
\usepackage[T1]{fontenc}
\usepackage[top=2.5cm, bottom=2.5cm, left=2cm, right=2cm]{geometry}
\usepackage[affil-it]{authblk}
\usepackage{lipsum}
\usepackage{url}
\usepackage[hidelinks]{hyperref}
\usepackage{graphicx}
\usepackage[table,xcdraw]{xcolor}
\usepackage{longtable}
\usepackage{multicol}

%citations
\usepackage{multibib}
\newcites{ac}{Academic references}
\newcites{nac}{Informal references}

% code for generating glossary, from http://tex.stackexchange.com/a/5837/59718
\usepackage[acronym,toc]{glossaries}
\usepackage{glossary-mcols}
\newcommand{\dict}[2]{%
  \newglossaryentry{#1}{name=#1,description={#2}}%
  \glslink{#1}{}%
}
\makeglossaries

% Here we set up the header, meta-information and front matter
%\date{January 5, 2015}      %// Today's date will appear when this is commented out.
\newcommand{\version}{0.1}

% title page
\author{Maarten Baertsoen}
\affil{Open Universiteit Nederland, faculteit Informatica \\
	T61327 - Afstudeerproject bachelor informatica}
\title{The Ampersand approach \&\\ Ampersand in practice}
\subtitle{Useful feedback in the Ampersand parser\\
	~\\
	Phase 3a: Domain \& Techniques}
\publishers{Version \version}

% header
\pagestyle{scrheadings}
\setheadsepline{0.2pt}
\clearscrheadings
\automark[section]{chapter}
\ihead{Maarten Baertsoen}
\ohead{The Ampersand Approach\&Ampersand in practice}
\cfoot{\pagemark}

% URL's
\renewcommand*{\UrlFont}{\footnotesize\ttfamily}

% hyphenation
\hyphenation{
	gua-ran-tee
	pro-duct
	cor-res-pon-ding
	me-cha-nism
	know-ledge
	de-ve-lo-pers
	do-cu-men-ta-tion
	sa-tis-fac-tion
	Schi-a-vi-ni
	Ba-ert-so-en}

% Now the document starts
\begin{document}
\maketitle
\newpage

\tableofcontents
%\listoffigures
%\listoftables
\clearpage

% !TEX root = ../Documentation.tex
\section{Introduction}
\subsection{Identification}
This document contains the domain \& techniques analysis of the project `Useful feedback in the Ampersand parser'.
The document is the milestone product of the project phase 3a for Daniel S.C. Schiavini, as specified in the project planning \citenac{plan}.

This document is part of the graduation project of the computer science bachelor at the Open Universiteit Nederland.
The project `Useful feedback in the Ampersand parser' is executed in collaboration with Maarten Baertsoen, with support of the supervisor Dr. Bastiaan Heeren and examiner Prof.dr. Marko C.J.D. van Eekelen.
The assignment is given by Prof.dr. Stef Joosten, who researches how to further automate the design of business processes and information systems by the development of the Ampersand project.

Ampersand is an approach for the use of business rules to define the business processes.
Users describe the business rules in a formal language (ADL), and Ampersand compiles those rules into functional specification, documentation and working software
prototypes.
The main objective of this project is to improve the feedback and maintainability of the Ampersand parser.
See \citenac{plan} for more details on the project.

\subsection{Goals}
\lipsum[3]

\subsection{Document overview}
\lipsum[4]
% !TEX root = ../Parsing.tex

\section{The Ampersand Methodology}
\label{sec:AmpersandApproach}

\subsection{Software requirements, the problem statement}

In 1976,  T. E. Bell todo:cite4 investigated the domain of requirement engineering, todo:cite, sponsored by the Ballistic Missile Advanced Technology Center with the goal to determine the magnitude and characteristics of requirement related problems and to indicate what type of techniques could correct the issues. One of the main conclusions was that requirement errors were the most numerous and that these kind of errors are very time-consuming and hence costly to correct.
In his conclusion,  T. E. Bel advises to use methods and techniques during the requirement engineering process to ensure consistency within and  between the requirements, such as the unique naming of objects and correct relations between the requirements themselves, in addition, he stressed that the applied methods must aid the verification and validation of the requirements.
 
Although the research is ancient history from an IT point of view, the conclusion is actually very relevant and therefore methods and techniques to further improve the quality and consistency of the captured requirements are still a hot topic within the domain of Requirement Engineering.

An important distinction classification of requirements is made by  Prof.dr. Stef Joosten todo:cite1 and Alex Borgida todo:cite, into early phase requirements, the business requirements, and the late phase requirements, the functional specifications. 

\subsubsection{Formal functional specifications}
The initial focus, in which the research still continuous after 30 years, was on the functional specifications resulting in formal methods for functional specifications, (todo:cite 3) which are based on a mathematical representation of the specifications resulting in the ability to analyze, validate and transform them into useful artifacts during the subsequent design and implementation phases. 

As described by  Luqi todo:cite 10, these formal methods had a positive impact on the reliability of the software development process, specifically for the purpose of specification analysis, transformation and verification todo:cite 7, this  reflects  in the elaboration of several mature formal methods.
The formal methods are categorized into two domains:

\begin{description}
	\item[Algebraic languages] describing the system in terms of operations and their relationships, such as Z todo: cite 8, VDM todo: cite 11  and Alloy (url:alloy.mit.edu) todo: cite 12 todo: cite 17
	\item[Model based languages] using mathematical sets and sequences to to express the system specification as a system state model such as OBJ todo: cite 13, Larch todo: cite 14 and LOTOS todo: cite 15
	\item[Hybrid languages] combining features of both algebraic as model-based languages like RAISE todo: cite 16 and CafeOBJ, and enhancement of the OBJ language, todo: cite 17
\end{description}

\subsubsection{Business requirements}

Although the positive impact of the formal specification methods, it was recognized that these techniques were not sufficient. Concisely summarized by todo: cite 18, the formal specification methods focus on the `whats' and the  `hows' of the desired system without an understanding of the `whys' behind them. 

Within the social environment in which a software systems will be introduced, several stakeholders have different goals, the business requirements, and based on these, they will express, often imprecise and inconsistent, expectations. Only by the correct capturing and understanding of the business requirements, to a feasible extend,the requirement engineers will truly understand the `whys' needed to crystallize the correct  `whats' and  `hows' to support these different goals.

Several early phase requirement modeling languages, RMLs, are introduced and typically included, todo: cite 2, 
\begin{description}
	\item[An ontology of requirements] describing the needed information to capture and the desired properties and behavior of of the to-be system the view on the world from the perspective of the to-be system. This includes instances such as `Entity', `Activity' and `Assertion'. todo: cite 4
	\item[Modeling primitives] to model the concepts and the relations within the ontology.
	\item[Methods], sometimes automated, to verify consistency and to perform additional analysis to verify if the stated requirements will satisfy the business expectations.
\end{description}

Early RMLs such as RML, todo: cite 4, provided initial methods but suffered from drawbacks such as extensibility as the provided ontologies were rather fixed meaning that no new notions on par with the existing instances could be addressed. CML todo: cite 19, which evolved to Telos todo: cite 20, contained already additional flexibility. 

KAOS, todo: cite 21, introduced the notion of `stakeholder goal' where i*, todo: cite, even differentiated between independent and interrelated, joint goals.
Further research to further introduce the concept of goal priorities is ongoing for which a new abstract requirements modeling language `Techne' todo: cite 2 is designed, based on the CORE ontology for requirements todo: cite 22. This will provide the framework for new RMLs which will be able to compare candidate solutions and their compliance towards the business requirements, a feature that will come in handy as many software engineering projects nowadays includes COTS package based solutions.

\subsubsection{Limitations and frequent issues}
Although the significant improvements and their positive impact in software engineering projects, several important drawbacks related to the use of formal methods are identified todo: cite 10, todo: cite 1:
\begin{description}
	\item[Communication] Communication between the business users and the requirements engineers based on the mathematical model is difficult as the business users are not used to use mathematical notations. Although the functional specifications are analysed to make sure they are consistent, it still offers no guarantee that the business requirements are consistent and tranfsormed correctly into functional specifications.
	\item[Typical experience and knowledge of developers] Software developers are not used to develop based on the mathematical models, or even lack the skills of higher mathematics. It requires extensive training for these developers to understand the mathematical models and to develop efficiently in these kind of software projects.
	\item[Theoretical approaches] Many formal methods are theoretical and their appliance is demonstrated by means of very simplified example but when they are effectively used in practice, the gap between specification and effective coding appears to be problematic.
	\item[Supporting tools] Most formal methods don't have supporting tools making it very cumbersome to use them in larger scaled projects, even when there are supporting tools available, they often lack a suitable user interfaces.

to add: agilie, COTS, transformation from early stage to later stage, structure, traceability...
	
\end{description}


\subsection{The goals of the Ampersand Methodology}
   
The Ampersand Methodology was founded in 2007 by the inventor Prof.dr. Stef Joosten with the vision to provide an answer to several of the main drawbacks of using formal techniques  in software engineering. 
The Ampersand Methodology is  developed to provide a method to unify the informal process of capturing the needs of users with the formal process of specifying an information system and to provide a formal translation method between both processes, including the necessary tools to ensure that the methodology is useful in real life projects.
Joosten derivingfunctional. Special attentions is given to the verification that the functional specifications satisfies the specified business requirements' todo: cite 1. 

Bottom line, the methodology presents the means to structure business needs in such a way that they can both validated by the users as well as be unambiguously interpreted by the system engineers after an automated transformation into functional specifications.

The Ampersand Methodology addresses several goals to achieve tis vision:
\begin{description}
	\item[Communication]  To assure that the stakeholders can correctly understand the requirements, the requirements must be documented in natural language to enable users, without any requirement engineering knowledge, to validate the system requirements. On the other hand, the requirements must be structured in a formal as functional specifications to make them useful during the actual software development phase. 
	\item[Completeness]A software system in which not all requirements are supported is useless. The Ampersand language must be fully declarative, meaning that all the requirements must be supported to assure that all business requirements, relevant to the subsequent software system, are accommodated correctly in the system specification.
	\item[Consistency]One of the main goals of the Ampersand Methodology is to guarantee consistency, each specified requirement is therefore applicable to all processes in the context of this requirement. The methodology needs to provide the means to assure this consistency
	\item[Traceability] The requirements must be traceable 
	\item[Supporting tools] In order to facilitate the adoption of the Ampersand Methodology outside an academic environment, Prof.dr. Stef Joosten defined the additional goal that the Ampersand Methodology must be accompanied with supporting tools to support the requirement engineers and this by creating design artifacts to be used by the software developers.
\end{description}

todo: review based on the underlying topics mentioned

\subsection{The Ampersand approach}
The Ampersand Methodology introduces a specific approach how to achieve the specified goals. The Ampersand Methodology can support different languages, a specific instance called `A Description Language', ADL, is implemented to support the Ampersand Methodology, the reasoning used to explain the achievement of the different goals is based on ADL.

\subsubsection{Communication}

One of the key differentiators is that the business requirements are presented towards the stakeholder in natural language while the system engineers can use functional specifications which are automatically transformed from the the business requirements assuring the the functional specification is consistent with the business requirements.

The innovative aspect of the AM is that the business requirements are  represented as `business rules' using relational algebra, a business rule must be seen as a business requirement in the form of an invariant to be satisfied by the business. The business rules are transformed in an automated way by an accompanying ADL tool, assuring the correctness of the functional specification based on the relational algebra of the business rules. Once the functional specifications are generated, the ADL tool will transform them back into the business rules for business user validation. When the re-translated requirements are still correct, the system engineers have the assurance that the functional specification are fully consistent with the business requirements. 

The use of natural language is further facilitated by the fact that is is not a specific language that can be used, it is any language that satisfies a predefined set of axioms that can be used, making it possible to address each business user in his own language.

\subsubsection{Completeness}
Initial methodologies to define business requirements lacked the power to add new notions besides the existing instances. ADL is designed as a purely declarative language to avoid this drawback. ADL features user specified rules, in relational algebra, design patterns, defined as sets of rules, contexts in which rules are applied and a signaling construct. In ADL, rules are specified without specific actions to avoid that they become too narrow compromising the declarative aspect of ADL.

\subsubsection{Consistency}
Expressing each business requirement as a business rule using relational algebra provides a mathematical foundation to check all rules against each other and to identify inconsistency between two or more business rules.  Checking each rule on his consistency towards the full set of defined business rules is quite demanding and would compromise the efficiency of the method in real life projects, the consistency check is therefore automated by the supporting tools. The supporting tools are described in more detail in section: todo: reference.

\subsubsection{Traceability}
Business requirements have a on-to-one relationship with business rules offering traceability back from a specific rule to the business requirement making it possible for the end-user to correctly identify the reasons why the business rule was identified.

\subsubsection{Supporting tools}
The practical use of the Ampersand Approach is supported by a tool built in the functional programming language Haskell, this tool produces of a wide range of design artifacts to support the validation, consistency checks and the subsequent software development steps:
\begin{description}
	\item[Business rules and consistency issues] The inputted business rules in ADL are compiled and typed checked, this is realized by using the Swierstra's combinator package for the compiler, todo: cite??,  and the AG-preprocessor by Dijkstra and Swiestra for the type-checker, todo: cite ??
	\item[Data model]  Class diagrams and entity-relationships are created using the the GraphViz package, todo: cite 24. 
	\item[Service catalogue] Al possible services on the defined classes and relationships, such as create, get, update and delete,  are specified formally. This formal representation together with the pre- and postconditions of the service make it possible that several developers can program the services independently of each other. 
	\item[Function point analysis] A function point analysis providing an insight of the complexity of the to be built system is generated according to the IFPUG guidelines, todo: cite 25. This degree of complexity can be used to estimate the remaining system development effort taking into account the impact of the Ampersand tool as an accelerator
	\item[Software prototype] A remarkable aspect of the Ampersand tool is that it makes it possible to generate a working prototype. The prototype is generated as a web application and is using a MySQL database.
\end{description}
 
\subsubsection{Training}
Although the Ampersand Methodology, including ADL, is not yet widely adopted in the domain of requirement engineering, the methodology, including the supporting tools, is educated by the Open University of the Netherlands, course `Ontwerpen met bedrijfsregels'. a specific site is dedicated to the methodology is as well for self-tuition or co-development purposes.

\noindent




% !TEX root = ../Parsing.tex

\section{Ampersand in practice}
\label{sec:InPractice}

\subsection{Getting started}
To use the Ampersand tools, several components needs to be installed:
\begin{description}
	\item[Ampersand installer] is an installation package containing the full Ampersand tool. This package can be downloaded from the Ampersand home page, wiki.tarski.nl. The necessary installation instructions are provided on that same website.
	\item[MiKTex,], a LaTeX compiler used to generate documents which are presented in pdf format.
	\item[Graphviz] is used to generate the pictures, such as the class diagrams and the entity-relationship diagrams. 
	\item[XAMPP] can be used in case your environment doesn't support an HTTP/PHP server and/or a MySQL database.
\end{description}

After some post-install actions, the Ampersand tool can be driven through the MS-DOS command prompt in MS Windows. 
No graphical user interface is yet available. All commonly used command line parameters are listed on the Ampersand web page.

\subsection{Requirement gathering}
The business rules are recored in a plain ASCII text file, created using a simple text editor such as Notepad++.
No specific user interface supporting on the fly checks on structure and types, e.g. brackets and names, is yet available 

\subsection{Practical usage}
Although it is no nuclear physics, the correct installation and usage of the Ampersand tools and the additional tools requires some IT awareness, especially in case some unexpected errors occur when the Ampersand tool uses the additional tools. 

To get used to the specific syntax of how business rules are defined in ADL scripts requires some training. 
Besides the base syntax, a specific patters hub is available, to support new and seasoned ADL users,  which can be used for training, inspiration and re-usage which comes in very handy for new Ampersand users.

As already mentioned in the introduction of this document, the feedback generated from the Ampersand tool, in case  issues are detected in the input file, such as syntax and types, appears to be insufficient and need to be revised.
% !TEX root = ../Parsing.tex

\section{Conclusion}
\label{sec:conclusion}
In \autoref{sec:libraries}, the advice was given to use a combinator library for the new parser of Ampersand.
The main reason to avoid the parser generators is that it is hard to generate useful feedback.
Then, in \autoref{sec:errors}, it was made even more clear that besides generating good messages, those messages should also be customizable.

Therefore, the advice of this research is to use the combinator library that offers the highest level of customization in error messages, Parsec.
Although the uu-parsinglib seems to also be a very good choice, the experiences from the Helium compiler \citeac{helium-parser} should be also considered.
Besides, the Parsec library offers better support.

A list of important consideration points has also been collected through the literature and can be found in \autoref{sec:errors}, more specifically \ref{subsec:errors-ampersand}.
\part*{Appendices}
\addcontentsline{toc}{part}{Appendices}
\appendix
% !TEX root = ../Parsing.tex

\small
\printglossary[style=mcolindex,title=Glossary]
\label{sec:glossary}

\newpage
% !TEX root = ../Parsing.tex
\addcontentsline{toc}{section}{References}
\label{sec:bibliography}

\begin{thebibliography}{99}

\bibitem{plan}
	Planning for the project `Useful feedback in the Ampersand parser'\\
	Maarten Baertsoen and Daniel S. C. Schiavini\\
	Version 2.0 -- November 29, 2014\\
	\url{http://git.io/NeHuLg}

\bibitem{heeren-error}
	Top Quality Type Error Messages\\
	Bastiaan Heeren\\
	ISBN 90-393-4005-6, September 20, 2005\\
	\url{http://www.open.ou.nl/bhr/phdthesis}

\bibitem{monadic-parsing}
	Functional pearls -- Monadic Parsing in Haskell\\
	Graham Hutton (University of Nottingham) and Erik Meijer (University of Utrecht)\\
	\url{http://www.cs.nott.ac.uk/~gmh/monparsing.pdf}

\bibitem{convert-ebnf}
	 From EBNF to BNF \\
	 Christoph Zenger\\
	 June 4, 2000\\
	 \url{http://lampwww.epfl.ch/teaching/archive/compilation-ssc/2000/part4/parsing/node3.html}

\bibitem{bnf-ebnf}
	BNF and EBNF: What are they and how do they work?\\
	Lars Marius Garshol\\
	August 22, 2008\\
	\url{http://www.garshol.priv.no/download/text/bnf.html}

\bibitem{parser-examples}
	Haskell Parser Examples\\
	Geoff Hulette\\
	August 22, 2014\\
	\url{https://github.com/ghulette/haskell-parser-examples}

\bibitem{hugs-parser}
	Source code of the Hugs parser\\
	March 25, 2007\\
	\url{https://github.com/fuzxxl/Hugs/blob/master/src/parser.y}

\bibitem{ghc-parser}
	GHC: The Parser\\
	December 1, 2014\\
	\url{https://ghc.haskell.org/trac/ghc/wiki/Commentary/Compiler/Parser}
	%\url{https://ghc.haskell.org/trac/ghc/browser/ghc/compiler/parser/Parser.y}
	%https://www.haskell.org/pipermail/haskell-cafe/2013-August/109557.html

\bibitem{helium-parser}
	Helium, for Learning Haskell\\
	Bastiaan Heeren, Daan Leijen, Arjan van IJzendoorn\\
	Utrecht University\\
	\url{http://www.open.ou.nl/bhr/heeren-helium.pdf}
	
\bibitem{gcc-c-parser}
	GCC 4.1 Release Series Changes, New Features, and Fixes\\
	\url{https://gcc.gnu.org/gcc-3.4/changes.html}

\bibitem{gcc-cpp-parser}
	GCC 3.4 Release Series Changes, New Features, and Fixes\\
	\url{https://gcc.gnu.org/gcc-4.1/changes.html}
	
\end{thebibliography}

\end{document}