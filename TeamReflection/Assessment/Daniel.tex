% !TEX root = ../Thesis.tex

% individueel verslag van de persoonlijke ervaringen en leermomenten
\subsection{Report: Daniel S. C. Schiavini}
\label{subsec:assessment-daniel}
Many of my fellow colleagues that work in the IT area are somewhat afraid of functional programming languages.
When I tell some of them that I decided to do my bachelor thesis in an open-source Haskell project, they think I am crazy.
My answer is usually that I like to learn and that I love challenges.

However, this answer does not really tell how I feel about this project.
I have learned that Haskell is actually a very suited language for larger systems.
Being so strict, it helped me to spot errors and `fail fast, fail often'.
Failing fast is a key feature in the most popular software development methodologies and that is for a good reason.
Besides, programming without side effects may really be the way to scale software systems to more and more cores.

So I am happy that I chose this project.
It has been a great couple months learning about a multitude of subjects.
Not only we studied more about business rules, we also had the opportunity to build up some experience with the functional development environment, work with Git and to learn about many different projects.

On the other hand, I am also proud of the software we delivered.
I believe the new parser is a large step forward and that the users will really appreciate it.
Software is never perfect, so there are always things that I still want to improve.
But who knows I may want to continue working on Ampersand, since it is open source?

It was fun, but it was heavy as well.
By now I actually expected to be finished with three other OU courses that I still have open.
Also because of having a new house, my free time shrank and it ended up being hard to even reach the amount hours per week for this project.

Around the same time, Maarten also could not work on the project for several weeks.
We were falling late on most deadlines while I was working alone.
When Maarten got back he had to put off a lot of effort to learn the technicalities.
Until now he could not catch me up in the amount of work done, but with a highly demanding job and two kids at home I am impressed he pulled it off at all.
While the lack of time and knowledge delayed the project, I must admit that Maarten is one of the most hard-working persons I ever worked with.

Gladly for me and my team mate, neither the supervisor nor the customer were in a hurry, so we could delay the project.
Because of our irregular times, the process we planned ended up being too cumbersome.
However, I am very satisfied with the way we worked around it: by communicating.

I must honestly say that my next OU courses may be a bit lonely.
I got used to being online in the evenings working with Maarten.
Hopefully we can keep in touch when we are done.
