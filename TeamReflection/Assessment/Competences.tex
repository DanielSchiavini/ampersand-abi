% !TEX root = ../Thesis.tex

\subsection{Academic competences}
\label{assessment:competences}
As part of the graduation project, each project member chose five academic competences that were considered important.
In this section, we discuss these competences and explain in which way they were part of the project.

\subsubsection{Daniel S.C. Schiavini}
In the beginning of the project, I decided to focus on the following competences:
\begin{enumerate}
    \item \textbf{Customer focus:}
        Whether you are working in the academia or in a company, you are always working for a customer.
        For example, a researcher is supposed to fulfil the wishes of their fund-raiser, while the fund-raiser usually wants the research results in order to make their own customers happy.
        Since customers are so important, I decided to focus on the customer satisfaction as my first competence.
        During the project, we tried to keep in constant contact with the customer.
        While this could not happen as often as we wanted, we kept the wishes of the customer as the most important goal of the project.
        Now that we have delivered the system, I believe the customer is satisfied with the results.
        
    \item \textbf{Collaboration:}
        While I usually do well in interpersonal relationships, I always find it hard to entrust someone else with the work and have the tendency to do everything myself.
        Even though Maarten probably felt this tendency of mine, I believe our task division went very well and that we communicated often and clear.
    
    \item \textbf{Evaluation:}
        Currently there are so many products in the IT world that building something from scratch is most of the time unnecessary.
        However, since there are so many options, it is very hard to choose the one best suited for the needs of the project.
        In this project, I executed the research for parsing libraries in Haskell.
        This allowed me to learn a lot about parsers, Haskell and multiple frameworks.
        With this knowledge, I could exercise my evaluation skills and I believe both the team and the customer were satisfied with the results.
    
    \item \textbf{Discussing, arguing:}
        After understanding a problem and the solution for it, the next step is convincing others that your choice was the best possible with the available knowledge.
        In an academic context, this us usually done by writing papers with well-constructed arguments and clear evidence.
        After my research on parsing libraries, I believe I have successfully convinced my fellow project team that we should use Parsec.
        Besides, during the project several discussions were started on design and architecture decisions.
        I believe that these discussions mostly reached a good agreement within the team.

    \item \textbf{Problem definition:}
        If you do not know what your study question / problem statement is, how can you possibly search the solution for it?
        In this project I have formulated problems and using the definition created, I could research it further.
        For example, I have researched what constitutes a good error messages and formulated a definition for it.
\end{enumerate}

\subsubsection{Maarten Baertsoen}
Below you can find a retrospective on the competences I identified in the beginning of the project:
\begin{enumerate}
    \item \textbf{Problem definition:}
        Defining the quality of error messages is subjective to the target audience using the application.
		To fulfil the goal of the project we needed, however, a clear problem definition to acquire a good understanding of the problem we needed to solve.
		On the second hand, we needed a solid baseline to measure our project achievements againts.
		By defining an objective definition of good error messages, against the backdrop of the Ampersand project, combined with a detailed analysis on the quality of error messages, I was able to  provide a clear view on the exact needs and results of the project.
        
    \item \textbf{Presentation:}
        While writing this reflection, the main achievement in this topic still to be executed, our final presentation.
		I however believe that I was able to present a correct view on the situation, approach and next steps towards the customer during alignment sessions.

    \item \textbf{Design and implementation:}
        In the beginning of this project, I indicated that I had the feeling that my programming skills were a bit rusty, although I used to be a developer at the beginning of my professional career.
		This feeling proved to be totally correct.
		It was however very difficult to get back in the game as my project counterpart was technically very skilled and progressed at a fast pace.
		In this respect, we needed to find the balance between project progress and my personal education.
		Although I personally would have realized a better progress in this domain, I believe this project provided me a good developing reboot.
		This topic is, however, something I still need to take up after this project to continue my personal evolution on this topic.
		
    \item \textbf{Collaboration:}
        In contrary to what one would suspect, the tools used to collaborate efficiently in a professional context are often absent or used inefficiently.
        It was my goal to get back up to date with the new tools used to support collaboration.
		Where some tools such as Skype and Dropbox were very easy to use, other topics like GitHub and Latex required some additional effort to be understood correctly.
		After this project, I now have a better and updated view on interesting collaboration tools to be used in a professional environment. 
		Especially the tools which needed some extra effort to get used to, are the most interesting ones to be used in my further professional live.
		
    \item \textbf{Develop a project plan:}
        We took more time than indicated to deliver the project plan, initially we thought that this was an excessive overrun.
		Later on in the project, we realized that the time we needed to clearly understand the actual project and to determine the approach how to realize it was something that paid off well.
		The main impact on the project plan was related to personal interferences, not foreseen at the beginning of the project.
        I must admit that planning a project for team members having a demanding professional career, and for myself a lovely family, is hard.
		The reason was partly due to the unforeseen personal interferences, on the other hand, we noticed that sometimes, after a hard day work, it could be very difficult to focus on a demanding project like this one.
		We revised our project planning once after we realized we could not deliver the estimated hours per week.
		This revision was well aligned with our supervisor and our customer.
		Besides these interferences, we were able to deliver the solution within the planned time frame.
\end{enumerate}
