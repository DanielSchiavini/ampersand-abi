% !TEX root = ../Thesis.tex

\subsection{Academic competences}
\label{assessment:competences}
As part of the graduation project, each project member chose five academic competences that were considered important.
In this section, we discuss these competences and explain in which way they were part of the project.

\subsubsection{Daniel S.C. Schiavini (R-M)}
In the beginning of the project, I decided to focus on the following competences:
\begin{enumerate}
    \item \textbf{Customer focus:}
        Whether you are working in the academia or in a company, you are always working for a customer.
        For example, a researcher is supposed to fulfil the wishes of their fund-raiser, while the fund-raiser usually wants the research results in order to make their own customers happy.
        Since customers are so important, I decided to focus on the customer satisfaction as my first competence.
        During the project, we tried to keep in constant contact with the customer.
        While this could not happen as often as we wanted, we kept the wishes of the customer as the most important goal of the project.
        Now that we have delivered the system, I believe the customer is satisfied with the results.
        
    \item \textbf{Collaboration:}
        While I usually do well in interpersonal relationships, I always find it hard to entrust someone else with the work and have the tendency to do everything myself.
        Even though Maarten probably felt this tendency of mine, I believe our task division went very well and that we communicated often and clear.
    
    \item \textbf{Evaluation:}
        Currently there are so many products in the IT world that building something from scratch is most of the time unnecessary.
        However, since there are so many options, it is very hard to choose the one best suited for the needs of the project.
        In this project, I executed the research for parsing libraries in Haskell.
        This allowed me to learn a lot about parsers, Haskell and multiple frameworks.
        With this knowledge, I could exercise my evaluation skills and I believe both the team and the customer were satisfied with the results.
    
    \item \textbf{Discussing, arguing:}
        After understanding a problem and the solution for it, the next step is convincing others that your choice was the best possible with the available knowledge.
        In an academic context, this us usually done by writing papers with well-constructed arguments and clear evidence.
        After my research on parsing libraries, I believe I have successfully convinced my fellow project team that we should use Parsec.
        Besides, during the project several discussions were started on design and architecture decisions.
        I believe that these discussions mostly reached a good agreement within the team.

    \item \textbf{Problem definition:}
        If you do not know what your study question / problem statement is, how can you possibly search the solution for it?
        In this project I have formulated problems and using the definition created, I could research it further.
        For example, I have researched what constitutes a good error messages and formulated a definition for it.
\end{enumerate}

\subsubsection{Maarten Baertsoen (M)}
%TODO: Reflect on the chosen competences in the beginning of the project.
