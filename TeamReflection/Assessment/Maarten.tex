% !TEX root = ../Thesis.tex
 (M)
%individueel verslag van de persoonlijke ervaringen en leermomenten
\subsection{Report: Maarten Baertsoen (M)}
\label{subsec:assessment-maarten}
To start with, I must honestly confess that I underestimate this project.
Being active on IT projects since 2001, I thought this project would be easy to combine with my professional and personal activities.
My experience has proven me wrong, in this section, I'll provide my personal insights in how I experienced this project and what I learned form it.

My first misconception was my estimation of the time I could spend to the project.
My estimation was that an average of 15 hours would fit easily in my time schedule.
Combining this project with a full time job in combination with a household with two lovely little children is just not easy.
In addition, I got seriously ill during this project due to which I lost two weeks directly, due to the fact of being hositalised, and indireclty due to the fact that it took several months to get back into shape.
I found it difficult to get back into the game on this project.
The lesson I learned was that one needs personal, non working time to relax and that there is something like: just one project too much.

A second point was that, although I'm an IT professional for many years, my programming experience was already buried deep down.
It took me some time to get back  up to speed, and as well as up to date.
While getting back up to speed I focused in the beginning too much on the technical capabilities of my counterpart Daniel, who is much more advanced in the domain of programming. 
After a while, I released that it was not needed to both have the same expertise, and that we better could be complementary instead of identical.
The speed in which we realized project results improved due to this as of then.

This project learned me a lot of new topics such as GitHub, the research domain of parsers, Latex, Haddock, Haskell.
It is a typical statement that the world of IT evolves quickly but it took me by surprise to realize that I was already outdated.
After this project I do have again the feeling to be more up to date then a couple of months ago.

And last but not least, I once again experienced the importance of team work.
I want to take this opportunity to express my gratefulness towards Daniel.
Without his patience and efforts to help me back on track, this on a technical level as on a human level during my recovery period after my illness, this project would have become very difficult. 
This is immediately the most important lesson learned from the project: how hard and difficult something is, if you keep on supporting each other, a strong team will find its way to the finish.




