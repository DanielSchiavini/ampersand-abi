% !TEX root = ../Planning.tex
\section{Risk management}
\label{sec:risk-management}
Risk management is one of the measures taken within the project to guarantee the process quality and hence project success.
This approach is loosely based on the PMI Risk Management approach \cite{pmi-risks}.
This risk management approach consists out of 5 steps: Risk identification, risk qualification, determination of risk mitigation measures, monitoring \& control and evaluation \& fine-tuning.

\subsection{Risk identification}
Every identified risk within the full life-cycle of the project will be followed up by the project team.
The definition of a risk in this project is  `an event that can influence the project in a tangible negative way'.
Of course there can be risks with a positive effect, but these will be merely considered as a `lucky coincidence'.

The risks will be kept together in a central repository, the risk register, in which a summary with the main characteristics is provided such as: description, risk qualification, status and identified mitigation actions.

\subsection{Risk qualification}
Not every identified risk will have the same importance, and the focus will be on the risks that are worth paying attention to.
This means that the focus will be on the medium or heavy risks which, if they occur, will have a bad or sometimes a catastrophic effect on the project.

The exact qualification of the risk type is based on the risk likelihood (what is the chance of the risk becoming a fact), and the risk impact (how hard would the risk hit the project if it happens).
\autoref{tab:risk-qualification} identifies the risks qualification based on these 2 parameters:

\begin{longtable}{|p{1,5cm}|p{1,5cm}|p{1,5cm}|p{1,5cm}|}\hline
       & \multicolumn{3}{l|}{Likelihood} \\\hline
Impact & Low           & Medium          & High \\\hline
High   & \risklow & \riskhigh   & \riskhigh \\\hline
Medium & \risklow & \riskmedium   & \riskhigh \\\hline
Low    & \risklow & \risklow   & \riskmedium \\\hline
\caption{Risk qualification}
\label{tab:risk-qualification}
\end{longtable}

\noindent
Based on the identified risk type, the risk will be managed accordingly:
\begin{description}
    \item [L - Low Risk:]
    Low risks will be monitored on a regular basis, but no specific actions will be initiated with regards to risk mitigation and/or risk avoidance.
    \item [M - Medium Risk:]
    Medium risk will be managed in a more proactive way.
    Risk mitigation options will be documented, and these will be taken into account during the risk's life-cycle.
    It is up to the project team to decide whether the risk is managed actively or merely monitored.
    \item[H-High risks:]
    The project team will act on high risks with a high sense of urgency.
    High risks will be investigated in detail, a risk mitigation approach will be installed and the risk will be monitored intensively throughout the risk's life-cycle.
\end {description}

\subsubsection{Determination of risk mitigation measures}
The possible risk management strategies to cope with each distinct risk are the following:
\begin{description}
    \item [Risk avoidance]
    Actions are undertaken to avoid the risk, meaning that the risk will disappear or become less likely due to the taken actions.
    \item[Risk reduction]
    The risk itself cannot be avoided, but measures are taken to limit the negative consequences in case the risk materializes.
    \item[Risk sharing]
    This kind of risk cannot be coped with by the project team on its own, therefore the risk mitigation responsibility is shared with other resources or project teams to identify a joint risk approach.
    \item[Risk acceptance]
    The project team decides to just accept the consequences of the risk.
    The team identifies the risk impact and foresee time budget to handle with the negative consequences in case the risk materializes.
\end {description}

\noindent
The risk management strategy will be documented for each distinct risk in the risk register.

\subsubsection{Monitoring and control}
The risks are monitored on a weekly basis and the necessary re-qualifications and new/ongoing mitigation actions are discussed when appropriate.

The risk register, together with a status overview, will be discussed with the project supervisor on a regular basis.
The most important highlights and ongoing mitigation actions or results will be explained in more detail.

\subsubsection{Evaluation and fine-tuning}
The evaluation of the risk management approach is not planned in a formal way, but each time an improvement point is identified, the team will evaluate fine-tuning activities and implement these whenever considered useful.

\subsection{Known risks at the beginning of the project}
Risk management is a continuous process during the full project life cycle and the team will react quickly on new risks and will monitor the open risks with care.
At the beginning of the project, some risks are already identified and documented in the risk register.
Given the holistic goal of this project plan document, the identified high risks are summarized below:

\begin{longtable}{R}
\hline
\risk{Some requirements are unclear and/or ambiguous} \\\hline
Poss. impact:    & \riskline{The features might be built incorrectly} \\\hline
ID:  & 1        & Domain:        & Scope     & Impact:           & High \\\hline
Likelihood:      & Medium   & Qualification: & \riskhigh     & Mitigation:  & Avoidance \\\hline
Approach:        & \riskline{For each unclear requirement, a detailed description of the need including the solution approach will be documented. This will presented towards the customer to ensure that the delivered functionality meets expectations.} \\\hline
\end{longtable}

~\\
\begin{longtable}{R}
\hline
\risk{Requirements are not ordered by priority} \\\hline
Poss. impact:   & \riskline{Lower priority features might be built before the high priority features} \\\hline
ID: & 2         & Domain:       & Scope    & Impact:             & High \\\hline
Likelihood:     & Medium    & Qualification:            & \riskhigh     & Mitigation:    & Avoidance \\\hline
Approach:       & \riskline{Each time an additional topic can be initiated, the most urgent one will be indicated by the customer.} \\\hline
\end{longtable}

\begin{longtable}{R}
\hline
\risk{The problems of the current parser are still unknown} \\\hline
Poss. impact:    & \riskline{A parser with the same problems may be delivered} \\\hline
ID:              & 3        & Domain:          & Scope    & Impact:             & High \\\hline
Likelihood:      & High    & Qualification:    & \riskhigh     & Mitigation:    & Avoidance \\\hline
Approach:        & \riskline{A meeting with the customer will be organized to gather knowledge of the existing system, focussing on the current issues, pain points and pitfalls.} \\\hline
\end{longtable}

\begin{longtable}{R}
\hline
\risk{The code, grammar description, code comments and examples aren't consistent.} \\\hline
Poss. impact:     & \riskline{The delivered code may not be according to the expected coding conventions} \\\hline
ID:               & 4        & Domain:          & Scope    & Impact:             & High \\\hline
Likelihood:       & High    & Qualification:    & \riskhigh     & Mitigation:    & Avoidance \\\hline
Approach:         & \riskline{The team will document the differences in the current code and plan a meeting with the customer to decide on the correct coding conventions.} \\\hline
\end{longtable}

\begin{longtable}{R}
\hline
\risk{`Useful feedback' is a subjective term}\\\hline
Poss. impact:    & \riskline{If the project team and the project customer have a different opinion regarding this term, the team could end up delivering a project according to specifications but of which the quality is judged as insufficient.} \\\hline
ID:               & 5       & Domain:           & Quality  & Impact:             & High \\\hline
Likelihood:       & High    & Qualification:    & \riskhigh     & Mitigation:    & Avoidance \\\hline
Approach:         & \riskline{The team will document the differences in the current code and plan a meeting with the customer to decide on the correct coding conventions.} \\\hline
\end{longtable}

\begin{longtable}{R}
\hline
\risk{The full list of wishes cannot be implemented within the time budget} \\\hline
Poss. impact:    & \riskline{Unsatisfied customer, non acceptance of the project.} \\\hline
ID:               & 6       & Domain:           & Budget   & Impact:             & Medium \\\hline
Likelihood:       & High    & Qualification:    & \riskhigh     & Mitigation:    & Reduction \\\hline
Approach:         & \riskline{The customer was informed that the full list cannot be implemented.
	Within the project plan, it is clearly stated that the feedback mechanism in the parser receives full priority.
	The additional requirements will only be considered when the project team is able to integrate them within budget.}
\\\hline
\end{longtable}
