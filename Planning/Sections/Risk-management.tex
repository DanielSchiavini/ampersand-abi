% !TEX root = ../Planning.tex
\section{Risk management}
\label{sec:risk-management}
Risk management is one of the measure taken within the project to guarantee the process quality and hence project success.

The risk management approach consist out of 5 steps

\begin{enumerate}
	\item Risk identification 
	\item Risk qualification
	\item Determination of risk mitigation measures
	\item Monitoring and control
	\item Evaluation and fine-tuning
\end {enumerate}
\subsection{Risk identification}
Every identified risk within the full life-cycle of the project will be followed up by the project team.
The definition of a risk in this project is  `every event that can influence the project in a tangible negative way'.
Of course there can be risks with a positive effect, but these will be merely considered as a `lucky coincidence'.

All risks will be kept together in a central repository, the risk register, in which a summary with the main characteristics is provided.
The following information is registered for each risk:

\begin{itemize}
	\item Risk ID
	\item Status
	\item Domain
	\item Identification date
	\item Short description
	\item Additional information
	\item Mitigation actions
	\item responsible
	\item date next review
	\item Impact
	\item Likelihood
	\item Risk evolution (status quo, increased or decreased probability/evolution)
\end{itemize}

\subsection{Risk qualification}
Not every identified risk will have the same importance, and the focus will be on the risks that are worth paying attention to.
This means that the focus will be on the medium or heavy risks which, if they occur, will have a bad or sometimes a catastrophic effect on he project.

The exact qualification of the risk type is based on the risk likelihood (what is the chance of the risk to become a fact), and the risk impact (how hard will this risk hit the project if it happens).

\autoref{tab:risk-qualification} identifies the risks qualification based on these 2 parameters:

\begin{longtable}{|l|c|c|c|}
\hline
       & \multicolumn{3}{l|}{Likelihood}                                                     \\ \hline
Impact & Low                       & Medium                      & High                      \\ \hline
High   & \cellcolor[HTML]{FFCB2F}M & \cellcolor[HTML]{FE0000}H   & \cellcolor[HTML]{FE0000}H \\ \hline
Medium & \cellcolor[HTML]{34FF34}L & \cellcolor[HTML]{FFCB2F}M   & \cellcolor[HTML]{FE0000}H \\ \hline
Low    & \cellcolor[HTML]{34FF34}L & \cellcolor[HTML]{34FF34}L   & \cellcolor[HTML]{FFC702}M \\ \hline
\caption{Risk qualification}
\label{tab:risk-qualification}
\end{longtable}

\noindent
Based on the identified risk type, the risk will be managed accordingly:
\begin{description}
	\item [L - Low Risk:]
	Low risks will be monitored on a regular basis, but no specific actions will be initiated with regards to risk mitigation and/or risk avoidance.
	\item [M - Medium Risk:]
	Medium risk will be managed in a more proactive way.
	Risk mitigation options will be documented, and these will be taken into account during the risk's life-cycle.
	It is up to the project team to decide whether the risk is managed actively or merely monitored.
	\item[H-High risks:]
	The project team will act on high risks with a high sense of urgency.
	High risks will be investigated in detail, a risk mitigation approach will be installed and the risk will be monitored intensively throughout the risk's life-cycle.
\end {description}

\subsubsection{Determination of risk mitigation measures}
The possible risk management strategies options to cope with each distinct risk are the following:
\begin{description}
	\item [Risk avoidance]
	Actions are undertaken to avoid the risk, meaning that the risk will disappear or become less likely due to the taken actions.
	\item[Risk reduction]
	The risk itself cannot be avoided, but measures are taken to limit the negative consequences in case the risk materializes.
	\item[Risk sharing]
	This kind of risk cannot be coped with by the project team on its own, therefore the risk mitigation responsibility is shared with other resources or project teams to identify a joint risk approach.
	\item[Risk acceptance]
	The project team decides to just accept the consequences of the risk.
	The team identifies the risk impact and foresee time budget to handle with the negative consequences in case the risk materializes.
\end {description}

\noindent
The risk management strategy option will be documented for each distinct risk in the risk register.

\subsubsection{Monitoring and control}
The risks are monitored on a weekly basis and the necessary re-qualifications and new/ongoing mitigation actions are discussed when appropriate.

The risk register, together with a status overview, will be discussed with the project supervisor on a regular basis.
The most important highlights and ongoing mitigation actions or results will be explained in more detail.

\subsubsection{Evaluation and fine-tuning}
The evaluation of the risk management approach is not planned in a formal way, but each time an improvement point is identified, the team will evaluate fine-tuning activities and implement these whenever considered useful.

\subsection{Known risks at the beginning of the project}
Risk management is a continuous process during the full project life cycle and the team will react quickly on new risks and will monitor the open risks with care.
At the beginning of the project, some risks are already identified and documented in the risk register.
Given the holistic goal of this project plan document, the identified high risks are summarized below:

\begin{description}
	\item [\underline{ID}]
	\underline{1}
	\item[Likelihood]
	Medium
	\item[Impact]
	High
	\item[Qualification]
	High
	\item[Domain]
	Scope
	\item[Risk]
	Some requirements are unclear and/or ambiguous
	\item[Possible impact]
	We might build the features incorrectly
	\item[Mitigation type]
	Avoidance
	\item[Mitigation approach]
	For each requirement, a detailed description of the need including the solution approach will be documented and presented towards the customer to ensure that the delivered functionality meets expectations.
\end {description}

\begin{description}
	\item [\underline{ID}]
	\underline{2}
	\item[Likelihood]
	Medium
	\item[Impact]
	High
	\item[Qualification]
	High
	\item[Domain]
	Scope
	\item[Risk]
	Requirements are not ordered by priority.
	\item[Possible impact]
	We might build the lower priority features before the high priority features.
	\item[Mitigation type]
	Avoidance
	\item[Mitigation approach]
	The project team will draw an overview of the perceived priorities and will align with the customer to reach an agreement regarding the priorities.
\end {description}

\begin{description}
	\item [\underline{ID}]
	\underline{3}
	\item[Likelihood]
	High
	\item[Impact]
	High
	\item[Qualification]
	High
	\item[Domain]
	Scope
	\item[Risk]
	We don't know what the problems of the current parser are.
	\item[Possible impact]
	We might deliver a parser with the same problems.
	\item[Mitigation type]
	Avoidance
	\item[Mitigation approach]
	A meeting with the customer will be organized to gather knowledge of the existing system, especially regarding the current issues, pain points and pitfalls.
\end {description}

\begin{description}
	\item [\underline{ID}]
	\underline{4}
	\item[Likelihood]
	High
	\item[Impact]
	High
	\item[Qualification]
	High
	\item[Domain]
	Scope
	\item[Risk]
	The Haskell code, grammar description, code comments and examples are not consistent.
	\item[Possible impact]
	We don't know which one is leading.
	\item[Mitigation type]
	Avoidance
	\item[Mitigation approach]
	The team will document the differences in the current code and plan a meeting with the customer to decide on the correct coding conventions.
\end {description}

\begin{description}
	\item [\underline{ID}]
	\underline{5}
	\item[Likelihood]
	High
	\item[Impact]
	High
	\item[Qualification]
	High
	\item[Domain]
	Quality
	\item[Risk]
	Useful feedback' is a subjective term, if the project team and the project customer have a different opinion regarding this term, the team could end up delivering a project according to spec but of which the quality is judged as 'insufficient'.
	\item[Possible impact]
	unsatisfied customer, non acceptance of the project.
	\item[Mitigation type]
	Avoidance
	\item[Mitigation approach]
	A formal error message overview ill be created to establish a clear and traceable overview regarding which input error types will result in which error
\end {description}

\begin{description}
	\item [\underline{ID}]
	\underline{6}
	\item[Likelihood]
	High
	\item[Impact]
	Medium
	\item[Qualification]
	High
	\item[Domain]
	Budget
	\item[Risk]
	It already clear that the full list of current enhancement points cannot be treated within the current budget (hrs).
	\item[Possible impact]
	Unsatisfied customer, non acceptance of the project.
	\item[Mitigation type]
	reduction
	\item[Mitigation approach]
	Within the project plan, it is clearly stated that the feedback mechanism in the parser receives full priority. 
	The additional requirements will only be considered when the project team is able to integrate them without budget overrun. 
	This will be discussed with the customer.
\end {description}

