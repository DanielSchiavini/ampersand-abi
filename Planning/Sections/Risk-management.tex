% !TEX root = ../Planning.tex
\section{Risk management (M)}
\label{sec:risk-management}
Risk management is one of the measure taken within the project to guarantee the process quality and hance project success.

The risk management approach consist out of 5 steps

\begin{enumerate}
	\item Risk identification 
	\item Risk qualification
	\item Determine risk mitigation and risk avoidance measures
	\item Monitor and control
	\item Evaluate and fine-tune
\end {enumerate}
\subsection{Risk identification }
Every identified risk withtin the full life cycle of the project will be listed. A risk is considered every event that can influence the project in a negative way. 
All risks will be kept together in a central repsository, the risk register, in which a summary with the main charateristics is provided.
The following information is registered for each risk:

\begin{enumerate}
	\item Risk ID
	\item Status
	\item Identification date
	\item Short description
	\item Additional information
	\item Mitigation actions
	\item responsible
	\item date next review
	\item Impact
	\item Probability
	\item Risk evolution (status quo, increased or decreased probability)
\end {enumerate}

\subsection{Risk qualification }
Not every identified risk will have the same importance and hence, we will only focus on the risk which are worth our attention.
Thi smeans that we will focus on the heavy risks which, if they occur, will have a very bad or catastrophic effect on he project and those that havce a fair chance of occuring but will still have quite a negative impact.

The exact qualification of the risks we will actively manage is based on the risks probability, what is the chance of the risk to becoma a fact, and the risk impact, how hard will this risk impact our project if it happens.

Following matris identifies the risks category based on these 2 parameters:

\begin{center}
    \begin{tabular}{ | p{2cm} | p{2cm} | p{2cm} | p{2cm} |}
    \hline
    Probability & Low & Medium & High \\ \hline
    Monday & 11C & 22C & A clear day with lots of sunshine.
    However, the strong breeze will bring down the temperatures. \\ \hline
    Tuesday & 9C & 19C & Cloudy with rain, across many northern regions. Clear spells 
    across most of Scotland and Northern Ireland, 
    but rain reaching the far northwest. \\ \hline
    Wednesday & 10C & 21C & Rain will still linger for the morning. 
    Conditions will improve by early afternoon and continue 
    throughout the evening. \\
    \hline
    \end{tabular}
\end{center}

 

\subsubsection{Functional risks}
\lipsum[1]
\subsubsection{Technical risks}
\lipsum[1]
\subsection{Risk mitigation plan}
\lipsum[1]

\subsection{Interferences with other projects}
\lipsum[1]