% !TEX root = ../Planning.tex
\section{Tools, methodologies and accelerators (R-M)}
\label{sec:tools-methodologies}

\subsection{Collaboration}
\dict{Dropbox}{File hosting service that offers cloud storage, file synchronization, personal cloud, and client software}%
\dict{Git}{A distributed revision control and source code management}%
\dict{GitHub}{Git repository web-based hosting service which offers all of the distributed revision control and source code management (SCM) functionality of Git}%
\dict{Microsoft Office}{Office suite of desktop applications devleoped by Microsoft}%
For the collaboration between the project members, the following tools will be used:
\begin{description}
	\item[Dropbox] For sharing time tracking and other reference documents;
	\item[GitHub] For sharing git repositories of code, besides managing issues and documentation;
	\item[Microsoft Office] For writing internal documents, e.g. time tracking;
\end{description}

\subsection{Documentation}
\dict{Haddock}{A software documentation generator for the Haskell programming language}%
\dict{TeXworks}{Graphical user interface for editing and compiling \LaTeX{} documents}%
\dict{LaTeX}{Document preparation system and document markup language for the TeX typeset}%
For writing the documentation, the following tools will be used:
\begin{description}
	\item[Haddock] For annotating documentation on the code;
	\item[TeXworks] For writing and compiling \LaTeX{} documents;
\end{description}

\subsection{Design}
\dict{yEd}{Software for editing graphs}%
For designing the software and its architecture, the following tools will be used:
\begin{description}
	\item[yEd] For creating diagrams and graphs;
\end{description}

\subsection{Development}
\dict{IDE}{Integrated Development Environment}%
\dict{GHC}{Glasgow Haskell Compilation system}%
\dict{Cabal}{Library for managing Haskell builds and packages}%
For software development, the following tools will be used:
\begin{description}
	\item[IDE] No standard integrated development environment will be chosen: the project members are free to use any IDE, e.g. Eclipse, Leksah, Notepad++;
	\item[GHC] The compiler GHC (Glasgow Haskell Compilation System), version 7.8.3, will be used;
	\item[Cabal] For managing Haskell packages and compilation, cabal-install version 1.18.0.5 and Cabal library version 1.18.1.3.
\end{description}

\subsection{Testing}
\dict{Hpc}{Library for checking, recording and displaying code coverage}%
\dict{Sentinel}{Test server for the Ampersand project}%
\dict{QuickCheck}{Library for testing Haskell code}%
Finally, the following tools will be used for testing the software:
\begin{description}
	\item[Hpc] might be used for checking, recording and displaying the code coverage of tests;
	\item[Sentinel server] can be used for the integration tests;
	\item[QuickCheck] for automating tests on the code and property based testing (e.g. pretty-print and reparsing, random code generation);
\end{description}
