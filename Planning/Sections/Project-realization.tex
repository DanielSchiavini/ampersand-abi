% !TEX root = ../Planning.tex
\section{Project realization}
\label{sec:project-realization}
\dict{Phase 3c}{Phase of the project when the software is actually designed, developed and tested. See \autoref{sec:project-realization}}%
During the project phase 3d, the software solution for the customer is designed, developed and tested.

\subsection{Coding conventions}
\label{subsec:coding-conventions}
\dict{Wiki}{Website that allows users to create web pages collaboratively}%
The coding conventions are described on the Ampersand wiki\cite{ampersand-wiki} and the project team will strictly adhere to these coding conventions.
In order to guarantee consistency, the conventions will not be duplicated in this document.

\subsection{Iterative approach}
\dict{Agile}{Group of software development methods in which requirements evolve through collaboration during the project's execution}%
\dict{Iteration}{An iteration is the basic unit of development. In the end of each iteration, working software should be delivered}%
For this project, an agile iterative approach has been chosen for the following reasons:
\begin{itemize}
	\item Offering fast feedback to the customer;
	\item Allowing adaptation during the project;
	\item Delivering software as often as possible;
	\item Often merging code with other ongoing projects (see \autoref{sec:risk-management});
	\item Maintaining a sustainable pace of development;
	\item Allowing team self-organization;
\end{itemize}
%
Considering the member's experience with agile methods, some inspiration has been taken from Scrum.
This project will be executed in a part-time way, thus the amount of work in each iteration will be very limited.
That makes the use of Scrum very challenging, so the following remarks are to be considered:
\begin{itemize}
	\item Team meetings with the supervisor are initially once per 3 weeks, so the iterations shall take three weeks.
	\item This meeting will work also as a review and planning meeting.
	\item The customer cannot be present in every team meeting; this will be done partially by distance, via e-mail and other means (see \autoref{sec:tools-methodologies}).
	\item The project members shall get in contact as often as possible, for backlog refinement and for answering the three basic questions:
		\begin{itemize}
			\item What have you done since yesterday?
			\item What are you planning to do today?
			\item Any impediments/stumbling blocks?
		\end{itemize}
\end{itemize}

\subsection{Validation}
In order to validate that the right product is being built, the first step before the planning meeting is to define what is going to be built.
This will be done by means of writing requirements and sharing them with the customer.

An important step on defining the requirements for the Ampersand parser will be defining which situations require improvement.
This can be done by collecting user data, although this data might be hard to analyze afterward.
The best and most efficient approach will be defined in cooperation with the customer.

Whenever applicable, additional documents, prototypes and examples will be shared with the customer or stakeholders.
This will require a big amount of communication (see \autoref{subsec:communication}) between the project members and other stakeholders.
Even though the customer has limited time, the end product can only be as good as the information given to the project team.

The validation of the work products will be facilitated by the project team using a pre-defined validation approach:
\begin{itemize}
	\item Upon the completion of a work product, the project team will do the verification and validation of the work product against the agreed on requirements;
	\item The verification and validation will be documented and handed over to the customer, the supervisor will be added in cc;
	\item After review by the customer's team, all remarks and issues of topics not corresponding to the agreed on scope will be delivered by the customer within a time frame of 1 week;
	\item The project team will treat and handle the remarks and issues within 1 week;
	\item The final work product is presented to the customer for acceptance.
\end{itemize}

\subsection{Test approach}
In order to verify the requirements, the approach is to automate as much as possible of the testing platform.
Although this will take some extra effort at the beginning of the project, the automation guarantees that a lot of time is saved in regression testing.

See also \autoref{sec:tools-methodologies} for a list of tools that will be used for testing.

\subsection{Test methodology}
Automating tests for Haskell programs should always be possible.
Indeed, the programs should be always deterministic.
The Ampersand parser, specifically, can be used in a feedback way: the program code is parsed, and then converted back to source code.
By parsing this code yet again, it can be verified that the result is the same.

\subsection{Test documentation}
\label{subsec:test-documentation}
All test scripts shall be added to source control (see \autoref{sec:tools-methodologies} for the used tools).
If necessary, a simple test report will be delivered at the end of the sprint.
Note that this report can be shared earlier with all stakeholders, as soon as a deliverable (backlog item) is ready.

\subsection{Milestones}
Considering the iterative approach, defining all backlog items beforehand is not possible.
Defining which backlog items will be delivered in each sprint is thus also impossible, this will be done in a lean and agile way during phase 3c.
However, some products are known beforehand and can be defined here.
See \autoref{tab:realization-milestones} for the initial planned delivery dates.
%
\begin{table}[h]
  \begin{tabular}{|l|l|}\hline
    \textbf{Date} & \textbf{Deliverable (s)} \\\hline
    2014-10-22 & Project planning v0.1 \\\hline
    2014-11-13 & Project planning v1.0 \\\hline
    2014-12-03 & End of research: domain \& techniques (phase 3a) \\\hline
    2014-12-24 & No delivery (Christmas holidays) \\\hline
    2015-01-14 & Delivery sprint 1\\\hline
    2015-02-04 & Delivery sprint 2\\\hline
    2015-02-25 & Delivery sprint 3\\\hline
    2015-03-18 & Delivery sprint 4\\\hline
    2015-04-08 & Delivery sprint 5\\\hline
    2015-04-29 & Delivery research context (phase 3b)\newline{}Final software delivery (phases 3c/3d)\\\hline
    2015-05-20 & Final thesis delivery\newline{}Presentation\\\hline
  \end{tabular}
  \caption{Realization milestones}
  \label{tab:realization-milestones}
\end{table}