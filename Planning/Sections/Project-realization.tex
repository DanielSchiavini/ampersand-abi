% !TEX root = ../Planning.tex
\section{Project realization}
\label{sec:project-realization}
\dict{Phase 3c}{Phase of the project when the software is actually designed, developed and tested. See \autoref{sec:project-realization}}%
During the project phase 3c, the software solution for the customer is designed, developed and tested.
This section describes the approach for the development and testing.

\subsection{Coding conventions}
\label{subsec:coding-conventions}
\dict{Wiki}{Website that allows users to create web pages collaboratively}%
The coding conventions are described on the Ampersand wiki \cite{ampersand-wiki} and the project team will strictly adhere to these coding conventions.
In order to guarantee consistency, the conventions will not be duplicated in this document.

\subsection{Iterative approach}
\dict{Agile}{Group of software development methods in which requirements evolve through collaboration during the project's execution}%
\dict{Iteration}{An iteration is the basic unit of development. In the end of each iteration, working software should be delivered}%
For this project, an agile iterative approach has been chosen for the following reasons:
\begin{itemize}
	\item Offering fast feedback to the customer;
	\item Allowing adaptation during the project;
	\item Delivering software as often as possible;
	\item Often merging code with other ongoing projects (see \autoref{sec:risk-management});
	\item Maintaining a sustainable pace of development;
	\item Allowing team self-organization;
\end{itemize}
%
Considering the member's experience with agile methods, some inspiration has been taken from Scrum.
This project will be executed in a part-time way, thus the amount of work in each iteration will be very limited.
That makes the use of Scrum very challenging, so the following remarks are to be considered:
\begin{itemize}
	\item Team meetings with the supervisor are initially once per 3 weeks, so the iterations shall take three weeks.
	\item This meeting will work also as a review and planning meeting.
	\item The customer cannot be present in every team meeting; this will be done partially by distance, via e-mail and other means (see \autoref{sec:tools-methodologies}).
	\item The project members shall get in contact as often as possible, for backlog refinement and for answering the three basic questions:
		\begin{itemize}
			\item What have you done since yesterday?
			\item What are you planning to do today?
			\item Are there any impediments or stumbling blocks?
		\end{itemize}
\end{itemize}

\subsection{Validation}
In order to validate that the right product is being built, the first step before the planning meeting is to define what is going to be built.
This will be defined in communication with the customer.

An important step on defining the requirements for the Ampersand parser will be defining which situations require improvement.
This can be done by collecting user data, although this data might be hard to analyze afterwards.
The best and most efficient approach will be defined in cooperation with the customer.

Whenever applicable, additional documents, prototypes and examples will be shared with the customer or stakeholders.
This will require a big amount of communication (see \autoref{subsec:communication}) between the project members and other stakeholders.
Even though the customer has limited time, the end product can only be as good as the information given to the project team.

The validation of the work products will be facilitated by the project team using a pre-defined validation approach:
\begin{itemize}
	\item Upon the completion of a work product, the project team will do the verification and validation of the work product;
	\item The verification and validation will be reported and handed over to the customer for acceptance (with a copy to the supervisor);
	\item After review by the customer, all remaining remarks and non-conforming topics will be solved;
	\item The final work product then is presented to the customer for acceptance.
\end{itemize}

\subsection{Test approach}
In order to verify the requirements, the approach is to automate as much as possible of the testing platform.
Although this will take some extra effort at the beginning of the project, the automation guarantees that a lot of time is saved in regression testing.

Automating tests for Haskell programs should be possible, since the programs should be deterministic.
The Ampersand parser, specifically, can be used in a feedback way: the program code is parsed, and then converted back to source code.
By parsing this code yet again, it can be verified that the result is the same (also known as pretty-printing).

All test scripts shall be added to source control (see \autoref{sec:tools-methodologies} for the used tools).
If necessary, a simple test report will be delivered at the end of the sprint.
Note that this report can be shared earlier with all stakeholders, as soon as a deliverable is ready.

See also \autoref{sec:tools-methodologies} for a list of tools that will be used for testing.
