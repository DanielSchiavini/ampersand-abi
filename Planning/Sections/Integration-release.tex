% !TEX root = ../Planning.tex

\section{Integration \& release}
\label{sec:integration-release}
\dict{Branching}{Branching is the duplication of an object under revision control so that modifications can happen in parallel along both branches}%
\dict{Forking}{Forking is the duplication of a whole repository, including all branches, mostly in order to allow changes without needing authorization of the owner of the main fork}%
\dict{Merging}{The reintegration of a branch into the original code}%
\dict{Pull request}{Method of making contributions available to a project for which the requester have no direct access to}%
The current developments in the Ampersand project are done in the master branch of the Ampersand GitHub repository (\texttt{AmpersandTarski/ampersand}).
The repository will be forked by this project, in order to allow changes independent of access to the main fork.
The code changes of this project will be done in a separate branch of this fork, in order to simplify the pull requests and merging.

By the end of each sprint, the changes to the project branch will be delivered via a pull request.
This allows the customer to merge the project branch into the master branch whenever wished.
The merge is however only expected to happen by the end of the project.

The client will limit all changes to the parser within the master branch as much as possible, mainly syntax changes.
However, may urgent changes occur, those must be integrated in the project branch by the project team.
Also other changes unrelated to the parser must be often integrated in the project fork.
