% !TEX root = ../Planning.tex

\section{Issue management}
\label{sec:issue-management}
During the project execution phase, mainly in phase 4, software and documentation issues will be found.
Whenever the project members or other users encounter such an issue, it should be logged and tracked.
In \autoref{sec:tools-methodologies} a list of tools is given that will be used for issue tracking.

If the issue is related to the scope of the project and if it is simple to solve, the project members shall take immediate action.
This simplifies the process and removes some noise, so that the focus can be on the important issues.

Bigger or more complicated issues will be logged, prioritized and assigned, taking in consideration the experience and availability of the project members.
For prioritizing issues, the following properties shall be considered:
\begin{itemize}
	\item The expected effort to solve the issue;
	\item How often the issue occurs;
	\item What is the impact when the issue occurs;
	\item How many people are influenced by it;
\end{itemize}

If there is a risk the issue cannot be solved within the time constraints, a more extended conversation must be started.
In case the issue is caused by changes in this project, canceling the regarding feature might be an option.
Otherwise, escalation to other Ampersand developers can be considered.
In all these cases, the project supervisor and the customer will be consulted as a sounding board.

Some issues can occur that are not related to the project's initial scope or that introduce a deviation from the initial scope.
In this case, all these issue, and corresponding changes, will be logged in a Change Request list in which they will be evaluated based on project impact, cost (in hrs), added value and corresponding risk. 
Change request will be handled in full transparency with the customer and supervisor to avoid scope creep and the risk that the focus drifts away from the initial project goal.

See also \autoref{sec:risk-management} for risk management procedures.
