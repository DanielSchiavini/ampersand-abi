% !TEX root = ../Planning.tex
\section{Introduction}
\subsection{Identification}
\dict{Phase 2}{Planning phase of the project}%
This document contains the planning for the execution of the graduation project `Useful feedback in the Ampersand parser'.
This planning gives the high-level requirements, the risks and a timeline for the project.
It is the milestone product of the project phase 2.
As such, the planning provides the steps for reaching the project objectives, and provides criteria that are used to validate and accept the results of the graduation.

This document is part of the graduation project of the computer science bachelor at the Open Universiteit Nederland.
The project `Useful feedback in the Ampersand parser' is assigned to the students Daniel Schiavini and Maarten Baertsoen, with support of the supervisor Dr. Bastiaan Heeren and examiner Prof.dr. Marko C.J.D. van Eekelen.
The assignment is given by Dr. Stef Joosten, who researches how to further automate the design of business processes and information systems by the development of the Ampersand project.

\subsection{Goal of this document}
The main goal of this document is to capture the taken decisions and agreements around the execution, the management and the control of the project.
In order to make the targets clear, allowing the project team to put the right focus, the project context is also depicted in the document.

The document describes the current situation and the issues it presents, making clear why the project has been started.
The purpose is thus to describe the management approach and to propose a high-level solution, as well as changes, risks and problems that might occur.

\subsection{Document overview}
An introduction is given is this chapter.
Afterwards, a general description of the project is given in \autoref{sec:project-description}.
Then, in \autoref{sec:knowledge-acquisition}, strategies are proposed for the acquisition of knowledge.
In \autoref{sec:project-approach} the project approach is explained and the management strategy is given in \autoref{sec:project-management}.

Assumptions and constraints are given in \autoref{sec:assumptions-constraints}.
Strategies for risk management are then proposed in \autoref{sec:risk-management}, possible interferences with other projects and an overview of known risks, including the mitigation approach, at the beginning of the project.

Before the actual realization approach is described, a clear view on the project requirements is set in  \autoref{sec:requirements}.

Details of the development techniques, including design, implementation and testing, are given in the project realization strategies of \autoref{sec:project-realization}.
Issue management is explained in \autoref{sec:issue-management}.
The strategy for integration of the released software is given in \autoref{sec:integration-release}, while the documentation strategy is given \autoref{sec:documentation} and the used tools and methodologies are given in \autoref{sec:tools-methodologies}.

Finally, in the appendix, a glossary of terms, definitions and abbreviations is given, just as a list of references.
