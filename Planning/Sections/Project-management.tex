% !TEX root = ../Planning.tex
\section{Project management}
\label{sec:project-management}
This sections describes the project management approach and used techniques to guarantee project success.
As the project team consists only out of 2 team members, special attention is given to make the project management approach holistic but light.
Holistic to assure that all important aspects with regards to the management of this project are properly covered and lightweight to avoid that the project team loses time in handling unnecessary tasks which only have added value in larger project teams.

\subsection{Project governance / roles \& responsibilities}
We distinguish 5 different project parties in the project environment:
\begin{description}
	\item[The project team]
	The project team consists out of Daniel Schiavini and Maarten Baertsoen.
	This team takes the full responsibility for the day to day project management as well as the actual delivery of all project work products consisting out of
	\begin{enumerate}
		\item The actual Haskell program code in full conformity with the coding conventions
		\item Exhaustive code documentation
		\item All necessary, final and intermediate, deliverables to ensure project's success such as: project planning, action/issues/risks lists and intermediate status reports based on the project methodology.
	\end {enumerate}
	The full list of all work products with their corresponding description is included in \autoref{sec:project-approach}.
	
	\item[The project supervisor]
	The project supervisor is Bastiaan Heerlen.
	He is the main contact person for the project team.
	It is the role of the project supervisor to monitor the project team on a regular basis as well as to provide support and guidance to the project team members both on content and approach.
	It is the responsibility of the project team to send project status updates towards the project supervisor; these status updates will form the basis for the regular alignment meetings.

	During project execution, it might occur that the effort needed to meet the customer expectations is way over the foreseen budget, measured in hours, of the project.
	Should this situation occur, the supervisor will facilitate customer alignments to adjust the customer's expectations.

	All efforts spent are reported on a regular basis to the supervisor who will track the spent effort to make sure that every aspect of the project receives the proper attention it deserves.

	\item[The customer]
	The project will deliver an actual product with all corresponding documentation in a clear and controlled way.
	The customer's satisfaction is one of the key success factors of this project.

	Although this project is conducted in an educational environment, the customer shall be treated just like he would be in a professional environment.

	\item[End-Users]
	The Ampersand Parser is actively used by several users groups.
	Although these user groups are not addressed directly, they will be affected when the project team releases new topics in the Ampersand Parser.
	All necessary communication and change management techniques applied to support the introduction of new functionality are described within this document.

	All topics in which the end user groups are involved need to be carefully aligned with the customer as it is the customers responsibility to keep the Ampersand Parser up and running and in good shape.
	Introduction of bugs, unclear functionality or even the total unavailability will harm the customer reputation.

	\item[The examiner]
	Last but not least, the examiner will evaluate all aspects of the project to which he will grant a score.
	These sub-score will be used to determine the final and individual course result.
	It's obvious that the customer's satisfaction is one of the main benchmarks, however the examiner will also evaluate the used project approach, academic skills and personal investigation areas.
\end {description}

\subsection{Communication}
\label{subsec:communication}
Regular alignment and communication sessions are foreseen within the project.
The communication plan described below identifies the way the project team will align and communicate with the involved project parties.

\subsubsection{Internal}
The project team members, Daniel Schiavini and Maarten Baertsoen will align on a weekly basis.
During this meeting, following topics will be discussed:
\begin{enumerate}
	\item Tasks done
	\item Tasks due
	\item Alignment on and assignment of the upcoming project tasks
	\item Actions
	\item Issues \& Risks
\end {enumerate}
%
Given the project size of the team, Daniel and Maarten will communicate informally on a regular basis besides this recurrent weekly meeting.

\subsubsection{OU}
\dict{OU}{Open Universiteit Nederland}%
A recurrent meeting between the project supervisor and the project team is planned every 3 weeks.
This meeting is an official feedback meeting, from team to supervisor, providing the full status of the project encompassing all important project aspects such as progress, issues and risks.

During these meetings the supervisor will monitor and steer the project progress and will check the proper project management by consulting the project's statistics.

All documents will be delivered to the project supervisor at least 24 hours before the actual meeting to allow the project supervisor to perform an in depth review.

\subsubsection{Customer}
The sessions with he customer will be planned carefully, both on timing and content, to make sure that the customer is not disturbed too much with project questions not relevant for them as a customer, but without losing the insight in the customer's needs.

The main principle to distinguish between the questions that can be posed to the customer or not, is by reflecting this project to a professional business project.
Every question or issue that should be handled by a professional project team, based on knowledge and experience, will be handled internally within this project as well.

All questions, issues and status updates towards the customer will be listed by the project team and will be revised with the project supervisor to assure to customer meetings are efficient, clear and most importantly, that these meetings provide the customer a trustful feeling regarding the project.

\subsection{Time keeping}
The project team members will record every effort they have spent on the project an overview over:
\begin{enumerate}
	\item Date
	\item Phase
	\item Domain
	\item Member
\end {enumerate}
This overview will be shared with the project supervisor on a regular basis and whenever requested.

\subsection{Project reporting}
The project team will share periodic status updates during the project, this reporting will cover following project aspects:
\begin{enumerate}
	\item General project status
	\item Timing
	\item Results achieved
	\item Budget (spent hours)
	\item Risks and issues
	\item Quality
	\item Focus for the upcoming weeks
\end {enumerate}

\subsection{Quality assurance}
The customer's perception of the project quality will have a tremendous impact on the overall rating of the project deliverables and therefore, it is for the project team of utmost importance to demonstrate the measures taken to guarantee project success.

Having correct quality assurance processes is not sufficient when these are not used on a day-to-day basis.
The project team will present quality reports towards the project supervisor and the customer to gain their confidence that the project team will deliver the promised quality.

Following topics will be monitored and reported: 
\begin{enumerate}
	\item Issue handling
	\item Follow up of actions
	\item Release procedures to put project coding into the production environment
	\item Communication \& change management
	\item Test management
	\item Tool effectiveness
	\item Documentation management
\end {enumerate}

\subsubsection{Process quality \& monitoring}
The used project management processes must be both light and effective.
The project team will monitor the effectiveness of these processes on a regular basis and carry through useful changes after alignment with the project supervisor.

\subsubsection{Deliverables quality \& monitoring}
Each project deliverable will be reviewed by the project team member's counterpart within the project team.

Where needed, several revision iterations will be foreseen until the project team comes to an agreement regarding the sufficient quality level of the deliverable.
Deliverables approved within the project team will be shared with the project supervisor who can perform an additional quality check before the deliverables are shared with the customer and the project examiner.

Each delivery towards the customer can be followed by a subsequent revision phase to perform a final fine-tuning to assure customer satisfaction.

Especially for the work items that consist out of programmed code, a thorough review will be done to assure that the programmed code is in line with the coding conventions as delivered by the customer (see \autoref{subsec:coding-conventions}).
