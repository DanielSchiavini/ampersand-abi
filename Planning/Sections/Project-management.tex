% !TEX root = ../Planning.tex
\section{Project management (M)}
\label{sec:project-management}
This sections descirbes the project management approach and used techniques to guarantee project succes.
As the project team consists only out of 2 team members, special attention is given to make the project management approach holistic but light.
Holistic to assure that all important aspects with regards to the management of this project are properly covered and lightweight to avoid that the project team loses time in handling unnecessary tasks which only have added value in larger project teams.
\subsection{Project governance / roles \& responsibilities}
We distinguish 5 different project parties in the project environment:
\begin{enumerate}
	\item The project team
	The project team consists out of Daniel Schiavini and Maarten Baertsoen.
	This team takes the full responsibility for the actual delivery for all project work products consisting out of
	\begin{enumerate}
		\item The actual Haskell program code in full conformity with the coding conventions
		\item Exhaustive code documentation
		\item All necessary final and intermediate deliverables to ensure project success such as: project planning and, project methodology as described within this document. 
	\end {enumerate}
	The full list of all work products with their corresponding description is included in this document.
	\item The project supervisor
	The project supervisor is the main contact person for the project team. It is the role of the project supervisor to monitor the project team on a regular basis as well as to provide support and guidance to the project team members both on content and approach.
	It is the responsibility of the project team to send project status updates towards the project supervisor; these status updates will form the basis for the regular alignment meetings. 
	During project execution, it might occur that the effort needed to meet the customer expectations is way over the foreseen budget of the project, measured in hours. 
	Should this situation occur, the supervisor will facilitate customer alignments to adjust the customer’s requirements.
	All efforts spent are therefore reported on a weekly basis to the supervisor who will track the spent effort to make sure that every aspect of the project receives the needed attention.

	\item The customer
	The project will deliver an actual product with all corresponding documentations in a controlled way. 
	The customer's satisfaction is one of the key success factors of this project.

	Although this project is conducted in a educatinal environment, the customer will be treated just like he would be served in a professional environment. 
	\item End-Users
	The Ampersand Parser is currently used by several users groups 
	\item The examiner

\end {enumerate}


\subsection{Communication}
Regular alignment sessions and communications s are foreseen withtinthe project. The communication plan described below will identify the way the project team will align and communicte with the dfferent project parties

\subsubsection{Internal}
The project team memebers, Dniel and Maarten will align on a weekly basis.
During this meeting, following topics will be discussed:
\begin{enumerate}
	\item Tasks done
	\item Tasks due
	\item Assignment of next project tasks wihtin the projetc team
	\item Actions
	\item Issues & Risks
\end {enumerate}
Given the project size of the team, Daniel and Maarten will communicate on a regular basis between each other on a informal basis besides this recurrent weekly meeting.
\subsubsection{OU}
A recurrent meeting bewteen the project supervisor and the project team is planned each 3 weeks.
This meeting is an offical feedback meeting regarding the full status of the projects encompassing all important project aspects such as progress, issues and risks. 

During these meetings the supervisor will monitor and steer the project progress and will check the proper project management by consulting the project's statistics. All formal project delivery products which needs to be delivered towards the examinator. 

All documents will be delivered to the project supervisor at least 24hrs before the actual meeting to allow the project supervisor to perform an in depth review .
It is the goal of the projec team members to deliver the to be reviewed documents a couple of days earlier to respect the supervisor's agenda.

\subsubsection{Customer}
The meetings with he customer will be planned carefully both on timing and content to make sure that the customer is not disturbed too much with project questions not relevant for the customer. 

The main principle to distinguish between the questions that can be posed to the customer or not, is by reflecting our project to a real life project. 
Every question or issue that should be handled by a professional project team, based on knowledge and experience, will be handled internally within this project as well.
All questions, issues and status updates towards the customer will be listed by the project team and will be revised with the project supervisor to assure to customer meetings are efficient, clear and most importantly, that these meetings provide the customer a trustful feeling regarding the project.
\subsection{Time keeping}
The project team members will record every effort they have spent on the project an overview over:
\begin{enumerate}
	\item Date
	\item Project phase
	\item Project domein
	\item Executor
\end {enumerate}
This overview will be shared with the project supervisor on a regular basis.
\subsection{Project reporting}
The project team will share periodic status updates during the project, this reporting will cover following project aspects:
\begin{enumerate}
	\item General project status
	\item Timing
	\item Results achieved
	\item Budget (spent hours)
	\item Risks and issues
	\item Quality
	\item Focus for the upcoming weeks
\end {enumerate}

\subsection{Quality assurance}
\lipsum[1]

\subsubsection{Process quality \& monitoring}
\lipsum[1]

\subsubsection{Deliverables quality \& monitoring}
Each project deliverable will be reviewed by the project team member’s counterpart within the project team.
Where needed, several revision iterations will be foreseen until the project team comes to an agreement regarding the quality level of the deliverable.
Internally approved deliverables will be shared with the project supervisor who can perform an additional quality check before the deliverables are shared with the customer and the project examiner.
Each delivery towards the customer can be followed by a subsequent revision phase to perform a final fine-tuning to assure customer satisfaction.
The necessary information to assure that the initial deliverable towards the customer is as spot on as possible will be gathered during the customer alignment meetings. 




