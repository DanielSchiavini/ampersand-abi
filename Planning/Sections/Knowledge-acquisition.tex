% !TEX root = ../Planning.tex

\section{Knowledge acquisition (D/M)}
\label{sec:knowledge-acquisition}

\subsection{Domain \& techniques}
In the project phase 3a, the project members shall investigate the domain and the techniques available.
Each project member will do a separate analysis, and will present a separate document with their findings.
The planning of the investigation is presented a section per project member.

Some ideas discussed with Bastiaan:
\begin{itemize}
  \item more info on ampersand users
  \item user-friendly error messages (specific to compilers or not)
  \item parser-libraries
  \item Stef's research
  \item Haskell parsers (e.g. Helium), monadit or combinators
\end{itemize}

\subsubsection{Daniel Schiavini (D)}
\lipsum[1]

\subsubsection{Maarten Baertsoen (M)}
\lipsum[1]

\subsection{Research context (R-M)}
After understanding more about the domain and the techniques involved with the project, the next step is to understand the context in which the project has been requested.
Besides, the relation of this project with the main research should be made clear.
In the project phase 3b, the focus is on the the use of business rules for IT systems.

Since the relevance of this project in the research can be only well defined later on, only a high-level planning can be given.
This project part will be done in three steps:
\begin{itemize}
	\item Understanding the research, by reading existing papers and other literature.
	\item Consulting, by formulating questions that can then be asked to the customer (Stef Joosten).
	\item Reporting the findings, in a separate section of the thesis.
\end{itemize}

\subsection{Knowledge documentation (R-M)}
The project phases 3a and 3b will result in three reports.
Each report should have around 5 pages and will be added to the bachelor thesis.

