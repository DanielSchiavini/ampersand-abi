% !TEX root = ../Planning.tex

\section{Knowledge acquisition}
\label{sec:knowledge-acquisition}

\subsection{Domain \& techniques}
\dict{Phase 3a}{Phase of the project when domains and techniques are investigated}%
In the project phase 3a, the project members shall investigate the domain and the techniques available.
Each project member will do a separate analysis, and will present a separate document with his findings.
The planning of the investigation is presented in a section per project member.

\subsubsection{Haskell parsing libraries \& User-friendly error messages}
This part of the research is going to be executed and documented by Daniel Schiavini.
It is divided in two interrelated parts:
\begin{description}
	\item [Haskell parsing libraries:]
	The current Ampersand parser is created with the Utrecht University parser combinator library (UU.Parsing).
	This library supports both monads and combinators and has quite some good features.
	One big advantage for this library is that the current code base can be very helpful as a source of documentation.
	However, being an academic project, it has more experimental features, less users and may be maintained less well than other libraries.
	Other libraries may also be able to provide other feature sets more suitable for the project.
	
	The purpose of this research is to choose the library best suited to the development of the new Ampersand parser.
	Besides the UU.Parsing library, the Parsec library will be also analyzed.
	Another option, in case the grammar is documented, is to use a parser generator, e.g. the Happy Parser Generator.
	The result of this research could also be that the UU.Parsing library is the best choice.
	
	There are also advantages and disadvantages on using monads and combinators, which must be well understood by the team.

	The baseline approach for researching the libraries will be to check whether they have all required features and their corresponding effectiveness.
	Also, existing tools built with the parsing library will be checked, specially the maintainability of the code and quality of the error messages.
	If considered feasible within the time budget, a small proof-of-concept will be created to test the libraries.

	\item [User-friendly error messages]
	The most important feature of the parser that will be built, is that it should generate user-friendly error messages.
	To understand what kinds of messages can be (and should be) generated, a research will be done on what good errors are and how to generate them.
	
	This is also related to the choice of the parsing library, since the chosen library should support the generation of good errors without extensive effort.
	
	Coincidently, a good source of knowledge are the papers of the supervisor, Bastiaan Heeren, who has done his PhD thesis on the generation of top quality type error messages \cite{heeren-error}.
\end{description}

\noindent
The results of this research will be reported in a separate document, and be added as an appendix to the thesis.

\subsubsection{Business Rules \& Ampersand Users}
This part of the research is going to be executed and documented by Maarten Baertsoen.

The goal of the Ampersand project is to provide a methodology of defining business requirements using natural language.
This research will analyze the following topics:

\begin{description}
	\item [The Ampersand methodology ]
	The Ampersand methodology provides a way to transform rules from business stakeholders into consistent functional specifications and a working software prototype.
	Within this section, the theoretical approach of the Ampersand Methodology will be investigated:
	\begin{itemize}
		\item Goals of the methodology
		\item The approach and rules
		\item How to use the methodology
		\item Usability, when to use the methodology
		\item Status and future road map
		\item Tools supporting the use of the ampersand methodology and their corresponding artifacts
	\end{itemize}
	
	\item [The practical use of the Ampersand tools]
	After a theoretical research of the domain, the practical use, including the tools, will be highlighted.
	
	This topic will analyze how effective and useful the Ampersand Methodology is when correctly applied.
	An overview will be given regarding the way it is used:
	\begin{itemize}
		\item Getting started with the Ampersand methodology
		\item Requirement gathering, documentation and validation: the Ampersand way
		\item How to use the automatically generated artifacts
		\item Correctness and efficiency
		\item Ease of use
		\item The benefits and disadvantages
		\item Potential enhancement areas
	\end{itemize}
	
\end{description}

\subsection{Research context}
\dict{Phase 3b}{Phase of the project when the research context is investigated}%
After understanding more about the domain and the techniques involved with the project, the next step (phase 3b) is to understand the context in which the project has been requested.
Besides, the relation of this project with the main research should be made clear.
In this project phase, the focus is on the use of business rules for defining IT systems.

Since the relevance of this project in the research can be only well defined later on, only a high-level planning can be given.
This project part will be done in three steps:
\begin{itemize}
	\item Understanding the research, by reading existing papers and other literature.
	\item Consulting, by formulating questions that can then be asked to the customer (Stef Joosten).
	\item Reporting the findings, in a separate section of the thesis.
\end{itemize}

\subsection{Knowledge documentation}
The project phases 3a and 3b will result in three reports.
Each report should have around 5 pages and will be added to the bachelor thesis.
The specific deliverables are described in \autoref{sec:documentation}.

