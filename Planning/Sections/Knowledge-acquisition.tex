% !TEX root = ../Planning.tex

\section{Knowledge acquisition (D/M)}
\label{sec:knowledge-acquisition}

\subsection{Domain \& techniques}
\dict{Phase 3a}{Phase of the project when domains and techniques are investigated}%
In the project phase 3a, the project members shall investigate the domain and the techniques available.
Each project member will do a separate analysis, and will present a separate document with their findings.
The planning of the investigation is presented a section per project member.

\subsubsection{Haskell parsing libraries \& User-friendly error messages (D)}
Haskell parsers (e.g. Helium), monadit or combinators.
user-friendly error messages (specific to compilers or not).

\subsubsection{Business Rules \& Ampersand Users (M)}

\subsection{Research context (R-M)}
\dict{Phase 3b}{Phase of the project when the research context is investigated}%
After understanding more about the domain and the techniques involved with the project, the next step is to understand the context in which the project has been requested.
Besides, the relation of this project with the main research should be made clear.
In the project phase 3b, the focus is on the the use of business rules for IT systems.

Since the relevance of this project in the research can be only well defined later on, only a high-level planning can be given.
This project part will be done in three steps:
\begin{itemize}
	\item Understanding the research, by reading existing papers and other literature.
	\item Consulting, by formulating questions that can then be asked to the customer (Stef Joosten).
	\item Reporting the findings, in a separate section of the thesis.
\end{itemize}

\subsection{Knowledge documentation (R-M)}
The project phases 3a and 3b will result in three reports.
Each report should have around 5 pages and will be added to the bachelor thesis.

