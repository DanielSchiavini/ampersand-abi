% !TEX root = ../Planning.tex
\section{Project approach (M)}
\label{sec:project-approach}
Just like any other project, this project needs a fit-for-use approach. 
This section 'project approach' summarizes the project methodology we will apply together with an exhaustive overview of the project planning, milestones and corresponding deliverables.
\subsection{Project methodology}
The main drivers to determine our project methodology is based on the kind of deliverables that needs to be produced due to this project, more specific, the actual technical realization of the project.

We can group the technical realization into 2 categories:
\begin{enumerate}
	\item The introduction of a new error message approach to provide clear and useful error messages and warnings towards the end-users
	\item Stand-alone modification points toward the Ampersand Parser based on
	\begin{enumerate}
		\item Reported issues
		\item Enhancements
		\item Extensions
	\end {enumerate}
\end {enumerate}

Given the stand-alone aspect of the modification points, the use of an iterative approach is quite obvious as we already have a backlog of topics which can be taken up independently of each other. 

The error messages need somewhat more attention to guarantee that we build up a solid error message architecture. 
The project team will therefore realize a Proof-of-Concept to demonstrate the correct approach and base architecture. 
After validation of this POC, the project team will realize the error message architecture and actual implementation in an iterative approach.
More information on the approach is described in \autoref{sec:project-realization}.

Within this project methodology, following aspects are identified as important to guarantee a qualitative project within time, budget en corresponding to the agreed on project scope : 
\begin{enumerate}
	\item Project management
	\item Knowledge acquisition 
	\item Project realization approach
	\item Integration \& release
	\item Testing \& validation
	\item Communication
	\item Documentation
\end {enumerate}

All these approach topics are fully documented within this project plan.


\subsection{Project phases}

 \begin{enumerate}
	\item Phase 2 - Task allocation and Project plan
	todo: description of phase
	\item Phase 3a - Domains \& Techniques
	todo: description of phase
 	\item Phase 3b - Investigation context (todo: find good translation for 'Onderzoekscontext)
	todo: description of phase
 	\item Phase 3c - Design \& implementation
	todo: description of phase
 	\item Phase 3d - Project documentation
	todo: description of phase
	\item Phase 4 - Project closure and final project presentation	todo: description of phase
\end {enumerate}

\subsection{Project planning, milestones and corresponding deliverables }
The project is planned based on the project milestones and the detailed project planning. The milestone will be formalized by the delivery of their corresponding deliverables

A commitment is given by the project team to adhere to the following milestone planning, officiated by the delivery their corresponding deliverables, grouped by their corresponding project phase:

 \begin{enumerate}
	\item Phase 2 - Task allocation and Project plan
 	\begin{enumerate}
		\item Delivery of the detailed project plan 			-	13/11/2014
		\item Detailed allocation of tasks within the project team 	- 	13/11/2014
	\end {enumerate}
	\item Phase 3a - Domains \& Techniques
 	\begin{enumerate}
		\item Delivery of scientific article by Daniel			- 	03/12/2014
		\item Delivery of scientific article by Maarten  			- 	03/12/2014
	\end {enumerate}
 	\item Phase 3b - Investigation context (todo: find good translation for 'Onderzoekscontext)
 	\begin{enumerate}
		\item Detailed report of the 'investigation context'  		- 	29/04/2014
	\end {enumerate}
 	\item Phase 3c - Design \& implementation
 	\begin{enumerate}
		\item Analysis \& design documents  				- 	30/03/2014
		\item Test report  							- 	20/03/2014
		\item Source code  							- 	27/03/2014
		\item The actual release of the realized product  		- 	20/04/2014
		\item IT documentation  						- 	20/03/2014
		\item User documentation \& user manual  			- 	20/03/2014
	\end {enumerate}
 	\item Phase 3d - Project documentation
 	\begin{enumerate}
		\item Project documentation document				- 	04/05/2014
	\end {enumerate}
	\item Phase 4 - Project closure and final project presentation
 	\begin{enumerate}
		\item Project essay							- 	20/05/2014
		\item Project presentation						- 	date to be jointly determined, final due date 29/05/2014
	\end {enumerate}
\end {enumerate}

\begin{center}
    \begin{tabular}{ | l | l |  p{10cm} |}
    \hline
    	Date & Phase & Deliverable \\ 
    \hline
    	13/11/2014 & 2 & Delivery of the detailed project plan \\ 
    \hline
    	13/11/2014 & 2 & Detailed allocation of tasks within the project team \\ 
    \hline
    	03/12/2014 & 3a & Delivery of scientific article by Daniel\\ 
    \hline
    	03/12/2014 & 3a & Delivery of scientific article by Maarten \\ 
    \hline
    	29/04/2014 & 3b & Detailed report of the 'investigation context'  \\ 
    \hline
  	 30/03/2014 & 3c & Analysis \& design documents \\ 
    \hline
   	20/04/2014 & 3c & Test report \\ 
    \hline
  	27/04/2014 & 3c & Source code \\ 
    \hline
  	27/04/2014 & 3c & The actual release of the realized product \\ 
    \hline
  	20/04/2014 & 3c &  IT documentation   \\ 
    \hline
  	20/04/2014 & 3c & User documentation \& user manual  \\ 
    \hline
  	04/05/2014 & 3d & Project documentation document \\ 
    \hline
  	20/05/2014 & 4 & Project essay \\ 
    \hline
 	tbc, final due date 29/05/2014 & 4 & Project presentation	 \\ 
    \hline
    \end{tabular}
\end{center}

The project activities will be managed in a detailed project planning by the project team, and the status will be shared during project alignment meetings.
Deviations, changes and issues within the project plan will be managed by the project team itself, in full openness towards the project stakeholders. 
As long as the changes to the detailed project plan aren't  compromising the milestone plan, the project is considered to be 'in control'. 

Any deviation of the project plan (due to risks, delays, issues,...) that can have an impact on the project milestone planning will be reported even before they really hit the project plan. 
This way, all project stakeholders will be informed and can cooperate in the corrective actions.




