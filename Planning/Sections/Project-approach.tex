% !TEX root = ../Planning.tex
\section{Project approach (M)}
\label{sec:project-approach}
Just like any other project, this project needs a fit-for-use approach. 
This section 'project approach' summarizes the project methodology we will apply together with an exhaustive overview of the project planning, milestones and corresponding deliverables.
\subsection{Project methodology}
The main drivers to determine our project methodology is based on the kind of deliverables that needs to be produced due to this project, more specific, the actual technical realization of the project.

We can group the technical realization into 2 categories:
\begin{enumerate}
	\item The introduction of a new error message approach to provide clear and useful error messages and warnings towards the end-users
	\item Stand alone modification points toward the Ampersand Parser based on
	\begin{enumerate}
		\item Reported issues
		\item Enhancements
		\item Extensions
	\end {enumerate}
\end {enumerate}

Given the stand alone aspect of the modification points, the use of a scrum based approach is quite obvious as we already have a product backlog of topics which can be taken up independently of each other. 


The error messages need somewhat more attention to guarantee that we build up a solid error message architecture. 
The project team will therefore realize a Proof-of-Concept to demonstrate the correct approach and base architecture. 
After validation of this POC, the project team will realize the error message architecture and actual implementation in a scrum like approach.
More information on the scrum like approach is described in section ‘Project realization’.



Within this project methodology, following aspects are identified as important to guarantee a qualitative project within time, budget en corresponding to the agreed on project scope : 
\begin{enumerate}
	\item Project management
	\item Knowledge acquisition 
	\item Project realization approach
	\item Integration \& release
	\item Testing \& validation
	\item Communication
	\item Documentation
\end {enumerate}

All these approach topics are fully documented within this project plan.


\subsection{Project phases}
todo Maarten: overview of project phases as available on the OU site

\subsection{Project planning, milestones and corresponding deliverables }
The project is planned based on the project milestones and the detailed project planning. The milestone will be formalized by the delivery of their corresponding deliverables

A commitment is given by the project team to adhere to the following milestone planning, officiated by the delivery their corresponding deliverables, grouped by their corresponding project phase:

 \begin{enumerate}
	\item Phase 2: 1,5 maand (medio september - eind oktober)          20 uur
 	\begin{enumerate}
		\item Delivery of the detailed project plan
	\end {enumerate}
	\item Phase 3a: 2 maanden (eind september – eind november)       50 uur
 	\begin{enumerate}
		\item
	\end {enumerate}
 	\itemPhase 3c: 5 maanden (eind november – eind april)                230 uur
 	\begin{enumerate}
		\item
	\end {enumerate}
 	\itemPhase 3d: 5 maanden (eind november – eind april)                50 uur
 	\begin{enumerate}
		\item
	\end {enumerate}
 	\itemPhase 3b: 2 maanden (eind februari – eind april)                   30 uur
 	\begin{enumerate}
		\item
	\end {enumerate}
	\item Phase 4: 1,5 maand (eind maart – medio mei)                      20 uur
 	\begin{enumerate}
		\item
	\end {enumerate}
\end {enumerate}

The project activities will be managed in a detailed project planning by the project team, and the status will be shared during project alignment meetings.
Deviations, changes and issues within the project plan will be managed by the project team itself, in full openness towards the project stakeholders. 
As long as the changes to the detailed project plan aren't  compromising the milestone plan, the project is considered to be 'in control'. 

Any deviation of the project plan (due to risks, delays, issues,...) that can have an impact on the project milestone planning will be reported even before they really hit the project plan. 
This way, all project stakeholders will be informed and can cooperate in the corrective actions.




