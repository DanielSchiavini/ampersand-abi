% !TEX root = ../Planning.tex
\section{Project approach}
\label{sec:project-approach}
Just like any other project, this project needs a fit-for-use approach. 
This section `project approach' summarizes the project methodology that will be applied together with an exhaustive overview of the project planning, milestones and corresponding deliverables.

\subsection{Project methodology}
The main drivers to determine our project methodology is based on the kind of deliverables that needs to be produced as a result of this project, more specific, the actual technical realization of the project.

The technical realization can be grouped into 2 categories:
\begin{enumerate}
	\item The introduction of a new error message approach to provide clear and useful error messages and warnings towards the end-users
	\item Stand-alone modification points toward the Ampersand Parser based on:
	\begin{enumerate}
		\item Reported issues
		\item Enhancements
		\item Extensions
	\end {enumerate}
\end {enumerate}

\noindent
Given the stand-alone aspect of the modification points, the use of an iterative approach is quite obvious as a backlog of topics is already available and the topics can be taken up independently of each other. 

\dict{PoC}{Proof-of-concept}%
The error messages need somewhat more attention to guarantee that a solid error message architecture is built up. 
The project team will therefore realize a Proof-of-Concept to demonstrate the correct approach and base architecture. 
After validation of this PoC, the project team will realize the error message architecture and actual implementation in an iterative approach.
More information on the approach is described in \autoref{sec:project-realization}.

Within this project methodology, the following aspects are identified as important to guarantee a qualitative project within time, budget en corresponding to the agreed on project scope: 
\begin{enumerate}
	\item Project management
	\item Knowledge acquisition 
	\item Project realization approach
	\item Integration \& release
	\item Testing \& validation
	\item Communication
	\item Documentation
\end {enumerate}

All these approach topics are fully documented within this project plan.


\subsection{Project phases}
The project activities are grouped in several project phases. 
Each project phase has it own milestone, but each phase can start whenever there is sufficient information available from the previous preceding phase.
This means that the team doesn't need to wait until a previous phase is finished.

 \begin{description}
	\item [Phase 2 - Task allocation and Project plan]~\\
		The key foundation to guarantee a successful project is a well thought and appropriate project plan and corresponding project management techniques.
		The project plan will document the full approach that will be applied to manage the project to a successful delivery and satisfied customer.
		
		Based on the individual learning goals of each project team member, specific tasks and roles will be assigned to them, allowing the development of these learning goals during the project.
		A detailed view on the allocated roles and tasks, will clarify the expectations towards the team members and will draw a clear view towards the project environment regarding who is doing what.
	\item [Phase 3a -- Domains \& Techniques]~\\
		Project success has a direct interdependency with the actual knowledge of the project team regarding the system itself, the context and most importantly, the business drivers behind the system.
		This phase will allow to the project team members to acquire the necessary knowledge to fully understand  the system, more precisely by investigation of the following topics:

 		\begin{enumerate}
			\item The business goals of the system: the actual added value the system promises towards the end-users
			\item Academic, business and system knowledge in the specialism to understand the way the system is used and to be able to use  the system as it would be used in a real life situation
			\item Specific terminology and professional language of the specialism
			\item Relations towards other specialisms, organizations and systems
			\item Important and relevant theoretical and practical evolutions in the specialism
		\end {enumerate}

		This phase will be processed by the project team members individually.
 	\item [Phase 3b -- Research context]~\\
		After gaining a thorough knowledge of the specialism and the current system, the project team will further analyze the specific drivers behind the project requirements.
		The higher the understanding of the goal of the project in relation to the specialism, including the current evolutions, and the way this project can contribute towards this evolution, the better the project team can focus on the topics that matter.
	
		This phase will form the basis of the actual design and implementation phase and therefore, this phase be initiated before the start of the design \& implementation phase but will continue simultaneously as additional  topics will need to be investigated based on moving insights. 

 	\item [Phase 3c -- Design \& implementation]~\\
	In phase 3c, the project team will start the in-depth analysis of the requirements and create a design for the new or changed functionality based on the research context fundamentals.
	The actual implementation and careful testing of the coding are part of this phase. 

	Special attention will be given to support both functional as non-functional requirements such as correct appliance of coding conventions and the creation of maintainable, durable and future-proof code.

 	\item [Phase 3d -- Project documentation]~\\
	Although often neglected, or under-appreciated, clear and target group oriented documentation is crucial towards short and long term satisfaction of the customer.
	On a short term basis, the customer can have a wrong perception of the actual realization of the requirements within the project. On a long term, if each future system modification or design is hampered due to inconsistent and incomplete documentation, chances are low that the customer will embark again on a new project with the previous project team members, or in real life, with the company that delivered the project.

	Several documentation deliverables, targeting a specific audience, are therefore listed in \autoref{sec:documentation} to assure the short and long-term customer happiness.

	\item [Phase 4 -- Project closure and final project presentation]~\\
	The final closure of the project will assure that all closing activities are correctly addressed and that all deliverables are handed over and accepted.

	This phase will be closed by a thesis, containing the project overview, a retrospection of the project realization and execution together with the lessons learned and conclusions.

	The formal closure is embodied by a final presentation.
\end {description}

\subsection{Project planning, milestones and corresponding deliverables}
\label{subsec:planning-milestones-deliverables}
The detail project planning is based on the project milestones as depicted in this section.

A commitment is given by the project team to adhere to the following milestone planning, officiated by the delivery of all their corresponding deliverables, grouped by their corresponding project phase:

 \begin{enumerate}
	\item Phase 2 - Task allocation and Project plan
 	\begin{enumerate}
		\item Delivery of the detailed project plan 			-	13/11/2014
		\item Detailed allocation of tasks within the project team 	- 	13/11/2014
	\end {enumerate}
	\item Phase 3a - Domains \& Techniques
 	\begin{enumerate}
		\item Delivery of scientific article by Daniel Schiavini		- 	03/12/2014
		\item Delivery of scientific article by Maarten Baertsoen		- 	03/12/2014
	\end {enumerate}
 	\item Phase 3b - Research context
 	\begin{enumerate}
		\item Detailed report of the `research context'  		- 	29/04/2014
	\end {enumerate}
 	\item Phase 3c - Design \& implementation
 	\begin{enumerate}
		\item Analysis \& design documents  				- 	30/03/2014
		\item Test report  							- 	20/03/2014
		\item Source code  							- 	27/03/2014
		\item The actual release of the realized product  		- 	20/04/2014
		\item IT documentation  						- 	20/03/2014
		\item User documentation \& user manual  			- 	20/03/2014
	\end {enumerate}
 	\item Phase 3d - Project documentation
 	\begin{enumerate}
		\item Project documentation document				- 	04/05/2014
	\end {enumerate}
	\item Phase 4 - Project closure and final project presentation
 	\begin{enumerate}
		\item Project essay							- 	20/05/2014
		\item Project presentation						- 	date to be jointly determined, final due date 29/05/2014
	\end {enumerate}
\end {enumerate}

\noindent
The project activities will be managed in a detailed project planning by the project team, and the status will be shared during project alignment meetings.
Deviations, changes and issues within the project plan will be managed by the project team itself, in full openness towards the project stakeholders.

As long as the changes to the detailed project plan aren't  compromising the milestone plan, the project is considered to be `in control', the status will be green.
If there are issues that can impact the qualitative and timely delivery of the milestones, but which are in control of the project team, the project status will be orange.
If there is an impact (on scope, timing and quality) not in control anymore by the project team members, the project is considered to be `at risk' and the status will be red.
In this case, additional support from outside the project is needed.

Any deviation of the project plan (due to risks, delays, issues,...) that can have an impact on the project milestone planning will be reported even before they really hit the project plan.
This way, all project stakeholders will be informed and can cooperate in the corrective actions.




