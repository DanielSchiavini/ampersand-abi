% !TEX root = ../Planning.tex
\section{Project approach}
\label{sec:project-approach}
Just like any other project, this project needs a fit-for-use approach. 
This section `project approach' summarizes the project methodology that will be applied together with an exhaustive overview of the project planning, milestones and corresponding deliverables.

\subsection{Project methodology}
The main drivers to determine our project methodology are based on the kind of deliverables that need to be produced as a result of this project, more specifically, the actual technical realization of the project.

The technical realization can be grouped into 2 categories:
\begin{enumerate}
	\item The introduction of a new error message approach to provide clear and useful error messages and warnings towards the end-users
	\item Stand-alone modification points toward the Ampersand Parser based on:
	\begin{enumerate}
		\item Reported issues
		\item Enhancements
		\item Extensions
	\end {enumerate}
\end {enumerate}

\noindent
Given the stand-alone aspect of the modification points, the use of an iterative approach is ideal as a backlog of topics is already available and the topics can be taken up independently of each other. 

\dict{PoC}{Proof-of-concept}%
The error messages need somewhat more attention to guarantee that a solid error message architecture is built up. 
The project team will therefore realize a proof-of-concept to demonstrate the correct approach and base architecture. 
After validation of this `PoC', the project team will realize the error message architecture and actual implementation in an iterative approach.
More information on the approach is described in \autoref{sec:project-realization}.

\subsection{Project milestone planning}
\label{subsec:planning-milestones-deliverables}
The project activities are grouped in several standard project phases as specified by the OU.
Each project phase has it own milestone date, but they are not iterative, meaning that a specific phase doesn't need to wait until a previous phase is finished, but it can start once sufficient information is available from the previous phase.

More information regarding the actual content of the project phases can be found on the course site of the OU. 

A commitment is given by the project team to adhere to the following milestone planning, officiated by the delivery of all their corresponding deliverables. To avoid the delivery of several separate documents, all subtopics within a specific phase are grouped together into a single document.

 \begin{enumerate}
	\item Phase 2 - Task allocation and Project plan			- 	13/11/2014
 	\begin{enumerate}
		\item Delivery of the detailed project plan 			
		\item Detailed allocation of tasks within the project team 
	\end {enumerate}
	\item Phase 3a - Domains \& Techniques					- 	03/12/2014
 	\begin{enumerate}
		\item Delivery of scientific article by Daniel Schiavini
		\item Delivery of scientific article by Maarten Baertsoen
	\end {enumerate}
 	\item Phase 3b - Research context
 	\begin{enumerate}
		\item Detailed report of the `research context'  		- 	29/04/2015
	\end {enumerate}
 	\item Phase 3c - Design \& implementation				- 	30/03/2015
 	\begin{enumerate}
		\item Analysis \& design documents  				
		\item Test report  							
		\item Source code  					
		\item The actual release of the realized product  	
		\item IT documentation  			
		\item User documentation \& user manual  
	\end {enumerate}
 	\item Phase 3d - Project documentation					- 	04/05/2015
 	\begin{enumerate}
		\item Project documentation document	
	\end {enumerate}
	\item Phase 4 - Project closure and final project presentation
 	\begin{enumerate}
		\item Project essay							- 	20/05/2015
		\item Project presentation						- 	date to be jointly determined, final due date 29/05/2014
	\end {enumerate}
\end {enumerate}

Considering the iterative approach, defining all backlog items, which will be realized in phase 3c, beforehand is not possible.
Defining which backlog items will be delivered in each sprint is thus also impossible, this will be done in a lean and agile way.

However, the sprints and their corresponding delivery dates  are listed below:
%
\begin{longtable}{|l|l|}\hline
    \textbf{Date} & \textbf{Deliverable (s)} \\\hline
	\endhead
    2015-01-14 & Delivery sprint 1\\\hline
    2015-02-04 & Delivery sprint 2\\\hline
    2015-02-25 & Delivery sprint 3\\\hline
    2015-03-18 & Delivery sprint 4\\\hline
    2015-04-08 & Delivery sprint 5\\\hline
  \caption{Realization milestones}
  \label{tab:realization-milestones}
\end{longtable}

\noindent
The project activities will be managed by the project team, and the status will be shared during project alignment meetings.
Deviations, changes and issues within the project plan will be communicated within the team itself, in full openness towards the project stakeholders.

As long as the changes to the detailed project plan aren't  compromising the milestone plan, the project is considered to be `in control' and the status will be green.
If there are issues that can impact the qualitative and timely delivery of the milestones, but which are in control of the project team, the project status will be orange.
If there is an impact (on scope, timing and quality) not in control anymore by the project team members, the project is considered to be `at risk' and the status will be red.
In this case, additional support from outside the project is needed.

Any deviation of the project plan (due to risks, delays, issues,...) that can have an impact on this project milestone planning will be reported towards the project supervisor even before they really hit the project plan.
This way, all project stakeholders will be informed and can cooperate in the corrective actions.
