% !TEX root = ../Planning.tex
\section{High-level requirements (D)}
\label{sec:requirements}
In this section, a list of known requirements is given.
Please note that because the project plan is made in the beginning of the project, only the highest-level requirements can be described.
Also note that the project will be executed in an iterative/agile way, so the requirements will change, and when they do this document might be not updated.

~\newline\noindent
In this section, the abbreviation NAP is used, with the meaning ``new Ampersand Parser''.
\dict{NAP}{Ampersand Parser}%

\subsection{Functional requirements}
\req{PL-FUN-010}{Ampersand code base}{
	The NAP shall be implemented in the Ampersand code base.\\
	Note: This also means that NAP shall be implemented in the Haskell programming language, and will use the same compiler as Ampersand does.}

\req{PL-FUN-020}{User-friendly errors}{
	The NAP shall give user-friendly error messages for the most common parsing errors.\\
	Note: Discovering which errors are the most common is part of the assignment.}

\req{PL-FUN-030}{Command-line interface}{
	The NAP shall only have a command-line interface.}

\req{PL-FUN-040}{Windows environment}{
	The NAP shall run in the Microsoft Windows 7 operational system.}

\req{PL-FUN-050}{Git repository}{
	The NAP code shall be managed in the ``AmpersandTarski/ampersand'' GitHub repository, in the ``ABI\_Parser'' branch.}

\req{PL-FUN-050}{P-structure}{
	The NAP code shall generate the P-structure in the same way the old Ampersand parser currently does.}

\subsection{Non-functional requirements}

In case the new parser is successfully implemented and accepted, while the project members still have time budget available, the list of open user wishes issues can be addressed.
Some of these wishes are substantial, so that most of them cannot be fulfilled during the graduation project.
The current list of open issues has been provided\cite{open-issues}, although it must be clear that the issues are strictly seen as lower priority.
