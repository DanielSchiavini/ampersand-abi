% !TEX root = ../Planning.tex
\section{High-level requirements}
\label{sec:requirements}
In this section, a list of known functional and non-functional requirements is given.
Please note that because the project plan is made in the beginning of the project, only the highest-level requirements can be described.
The requirements will be further clarified in phase 3c based on the stated scope `useful feedback in the Ampersand parser'.
Also note that the project will be executed in an iterative/agile way, so the requirements will change, and when they do this document might be not updated, however, a full backlog off all functionalities realized within the project will be maintained by the project team.

\subsection{Main requirement: useful and user friendly error messages}
The main requirement is that the NAP shall give user-friendly error messages for the most common parsing errors.\\
Note: Discovering which errors are the most common and what user-friendly messages consist of, is the most important part of the assignment as there is at the start of the project no list available of the erroneous error messages.
It is up to the project team, as a part of the actual project, to identify all the wrong error messages and to identify the actual requirements to implement the useful feedback mechanism in the ampersand parser.

The project team will investigate the current parser to be able to draw up these requirements.
This input will be gathered by the actual usage of the ampersand parser allowing the project team to discover the wrong error messages, the investigation of the Haskell code and by interviewing the students of the OU who are using the parser.

The strategies for achieving the required results are described in \autoref{sec:knowledge-acquisition}.

~\newline\noindent
In this section, the abbreviation NAP is used, with the meaning `new Ampersand Parser'.
\dict{NAP}{New Ampersand Parser (the software deliverable of this project)}%

\subsection{Additional functional and non-functional requirements}
To assure that the listed requirements are as complete as possible, the project team used the standards ISO/IEC 9126 \cite{iso-9126} and J-STD-016 \cite{jstd-016} as requirement checklists towards the customer, the result of this checklist is listed below.
Since so little requirements have been identified, the requirements are listed here:

	\req{PL-FUN-010}{Command-line interface}
		{Capability requirement}
		{Project members}
		{The NAP shall only have a command-line interface and GUI interface}
		
	\req{PL-RSM-010}{Forward parsing}
		{Required state/mode}
		{Current parser}
		{The NAP shall provide forward parsing, from ADL to P-structure}
	
	\req{PL-RSM-020}{Backwards parsing}
		{Required state/mode}
		{Current parser}
		{The NAP shall provide backwards parsing, from P-structure to ADL (also known as pretty printing)}

	\req{PL-EIF-010}{File interface}
		{External interface requirement}
		{Current parser}
		{The NAP shall read input text files in the existing ADL format}

	\req{PL-EIF-020}{File encoding}
		{External interface requirement}
		{Stef Joosten}
		{The NAP shall handle input files with UTF-8 encoding}

	\req{PL-IIF-010}
		{Interface with Type Checker}
		{Internal interface requirement}
		{Current parser}
		{The NAP code shall interface with the existing Type Checker via the P-structure}

	\req{PL-IDR-010}{Generate P-structure}
		{Internal data requirement}
		{Current parser}
		{The NAP code shall generate the P-structure in the same manner as the old Ampersand parser currently does,
		this implies that the P-structure created by the old and new parser will be a perfect match}

	\req{PL-ENV-010}{Windows environment}
		{Environment requirement}
		{Current parser}
		{The NAP shall run in the Microsoft Windows 7 operational system}

	\req{PL-CSW-010}{Haskell language}
		{Software resource requirement}
		{Stef Joosten}
		{The NAP shall be written in the Haskell programming language}
		
	\req{PL-CSW-020}{Haskell compiler}
		{Software resource requirement}
		{Stef Joosten}
		{The NAP shall be compiled with GHC version 7.8.3}

	\req{PL-SQF-010}
		{User-friendly errors}
		{Software quality factor}
		{Project description}
		{The NAP shall give user-friendly error messages for the most common parsing errors in a consistent and clear way}

	\req{PL-SQF-020}
		{Code annotation}
		{Software quality factor}
		{Stef Joosten}
		{The NAP source code shall be documented in Haddock format}

	\req{PL-DIC-010}{Ampersand code base}
		{Design and implementation constraint}
		{Stef Joosten}
		{The NAP shall be implemented in the Ampersand code base.\newline
		Note: This also means that NAP shall be implemented in the Haskell programming language, and will use the same compiler as Ampersand does.
		The existing coding conventions shall be respected}

	\req{PL-PAK-010}{Git repository}
		{Packaging requirement}
		{Stef Joosten}
		{The NAP code shall be managed in the `AmpersandTarski/ampersand' GitHub repository, in the `ABI\_Parser' branch}
