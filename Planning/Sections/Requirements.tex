% !TEX root = ../Planning.tex
\section{High-level requirements (D)}
\label{sec:requirements}
In this section, a list of known requirements is given.
Please note that because the project plan is made in the beginning of the project, only the highest-level requirements can be described.
Also note that the project will be executed in an iterative/agile way, so the requirements will change, and when they do this document might be not updated.

~\newline\noindent
In this section, the abbreviation NAP is used, with the meaning ``new Ampersand Parser''.
\dict{NAP}{Ampersand Parser}%

In case the new parser is successfully implemented and accepted, while the project members still have time budget available, the list of open user wishes issues can be addressed.
Some of these wishes are substantial, so that most of them cannot be fulfilled during the graduation project.
The current list of open issues has been provided\cite{open-issues}, although it must be clear that the issues are strictly seen as lower priority.

\subsection{Capability requirements}
\req{PL-FUN-010}{Command-line interface}{
	The NAP shall only have a command-line interface.}

\subsection{Required states and modes}
No such requirements have been identified.

\subsection{External interface requirements}
No such requirements have been identified.

\subsection{Internal interface requirements}
\req{PL-IIF-010}{P-structure}{
	The NAP code shall generate the P-structure in the same manner as the old Ampersand parser currently does.}

\subsection{Internal data requirements}
No such requirements have been identified.

\subsection{Adaptation requirements}
No such requirements have been identified.

\subsection{Safety requirements}
Ampersand is not a safety critical application.

\subsection{Security and privacy requirements}
Ampersand is not a security or privacy critical application.

\subsection{Environment requirements}
\req{PL-ENV-010}{Windows environment}{
	The NAP shall run in the Microsoft Windows 7 operational system.}

\subsection{Computer resource requirements}
	\subsubsection{Hardware requirements}
	No such requirements have been identified.

	\subsubsection{Hardware resource utilization requirements}
	No such requirements have been identified.

	\subsubsection{Software requirements}
	No such requirements have been identified.

	\subsubsection{Communications requirements}
	No such requirements have been identified.

\subsection{Software quality factors}
\req{PL-SQF-010}{User-friendly errors}{
	The NAP shall give user-friendly error messages for the most common parsing errors.\\
	Note: Discovering which errors are the most common is part of the assignment.}

\subsection{Design and implementation constraints}
\req{PL-DIC-010}{Ampersand code base}{
	The NAP shall be implemented in the Ampersand code base.\\
	Note: This also means that NAP shall be implemented in the Haskell programming language, and will use the same compiler as Ampersand does.}

\subsection{Personnel-related requirements}
\req{PL-PER-010}{Project members availability}{
	The project members shall be available for at least an average of 10 hours a week.
}

\subsection{Training-related requirements}
No such requirements have been identified.

\subsection{Logistics-related requirements}
No such requirements have been identified.

\subsection{Other requirements}
No such requirements have been identified.

\subsection{Packaging requirements}
\req{PL-PAK-010}{Git repository}{
	The NAP code shall be managed in the ``AmpersandTarski/ampersand'' GitHub repository, in the ``ABI\_Parser'' branch.}

\subsection{Precedence and criticality of requirements}
No such requirements have been identified.
