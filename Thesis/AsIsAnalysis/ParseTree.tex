% !TEX root = ../Thesis.tex

\subsection{Parse tree (R-M)}
\label{analysis:parse-tree}
The parse tree (also known as P-structure) is a data structure that very much resembles the EBNF description.
The root of the tree is the \code{P\_Context} structure, and every leaf of the tree has a field for the location where it was found in the ADL code (the \code{Origin} structure).
The tree is consistently defined with the record syntax and is well documented.

However, the constructions are not completely pure, since some transformations are necessary from the ADL to the P-structure.
This forces the parser to do more than only parsing.
Also, the order of the fields can be confusing; sometimes \code{Origin} is the first field and sometimes it is not.

During this project, small changes to the parse tree have been done.
These changes are described in \autoref{design:parse-tree}.
