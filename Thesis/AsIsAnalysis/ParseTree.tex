% !TEX root = ../Documentation.tex

\subsection{Parse tree (R-M)}
\label{subsec:analysis-parse-tree}
The parse tree (also known as P-structure) is a data structure that very much resembles the EBNF description.
The root of the tree is the \texttt{P\_Context} structure, and every leaf of the tree has a field for the location where it was found in the ADL code (the \texttt{Origin} structure).
The tree is consistently defined with the record syntax and is well documented.

However, the constructions are not completely pure, since some transformations are necessary from the ADL to the P-structure.
This forces the parser to do more than only parsing.
Also, the order of the fields can be confusing; sometimes \texttt{Origin} is the first field and sometimes it is not.

During this project, small changes to the parse tree have been done.
These changes are described in \autoref{subsec:design-parse-tree}.

\label{subsec:design-parse-tree}
Improvements in the Ampersand parse tree are out of the scope of this project, because of the potential consequences to the rest of the Ampersand system.
However, during the development of the new parser a few constructions have been changed in order to make the parser more readable and maintainable.
The changes have been mostly in the order of the constructor parameters, and this was done consequently though all Ampersand modules.
The updated parse tree is depicted in the appendices (\autoref{fig:parse-tree}).
