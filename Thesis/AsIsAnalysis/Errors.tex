% !TEX root = ../Thesis.tex

\subsection{Errors}
\label{analysis:errors}
In this section, we analyze the error messages given by the old parser.
The results of this analysis are compared to the new parser in \autoref{design:errors}.

\subsubsection{Error message qualification}
The user friendliness and correctness of an error message is a subjective topic and therefore we need to start with a definition to objectively judge the quality of an error message.
After analyzing the current Ampersand parser error messages, we identified the following objective aspects of an Ampersand parser error message:
%
\begin{description}
	\item [Position]
	Each error messages is accompanied with the correct position (file, line number and column) of where the error was found.
	\item [Accuracy]
	The accuracy of an error message is measured based upon the following characteristics:
	\begin{enumerate}
		\item	\textbf{\small How does the provided error description exactly outlines the discovered error:}
				Providing the user with a good description of the encountered syntax issue will support a fast error resolution.
				When the issue is vaguely described without pinpointing the exact issue, the error resolution will be time consuming.
		\item	\textbf{\small Pinpointing the correct error}:
				An error can invoke multiple subsequent issues. 
				These issues are however irrelevant for the user and the Ampersand parser should provide the exact origin of the issues.
		\item	\textbf{\small Quality of the hint:}
			The parser provides a hint for a solution together with the error message to support the user with the error resolution.
	\end {enumerate}
    \item[Conciseness]
	Providing a good error description is one thing, but this one message can be hidden between several other error messages that result from the initial error.
	It is unlikely that users will easily find the exact originating issue in their source file when they are overwhelmed with a multitude of error messages. % using plural to avoid repeating his/her
\end {description}
%
Based on the objective Ampersand parser error characteristics, we defined the following definition to distinguish between good, bad and average error messages:
%
\begin{description}
	\item [Bad error message] A message is considered to be bad if one of the criteria blow is fulfilled:
		\begin{description}
			\item [Position]
			The position has a deviation of more then 1 line or 10 column positions from where the actual error is made.
			\item [Accuracy]~
				\begin{itemize}
					\item 	The provided error description is useless for the user to determine the actual error.
					\item 	The provided error description is not appointing the main, originating error without any correlation towards this main error.
				\end {itemize}
			\item[Conciseness]
			More then distinct 3 errors are mentioned by the Ampersand parser.
		\end {description}
	\item [Average error message] A message is considered to be of average quality if one of the criteria blow is fulfilled:
		\begin{description}
			\item [Position]
			The position has a deviation between 5 and 10 columns positions from where the actual error is made.
			\item [Accuracy]~
				\begin{itemize}
					\item 	The provided error description is not an exact description of the error but provides however useful information to discover the actual issue.
					\item 	The provided error description is not appointing the main, originating error but the link to the actual error can be discovered based on the provided information.
					\item 	The provided hint is incorrect.
				\end {itemize}
			\item[Conciseness]
			2 or 3 errors are mentioned by the Ampersand parser.
		\end {description}
		
	\item [Good error message] We can state that any error message being not bad nor average is good.
\end {description}

\subsubsection{Gathering process}

To gather the necessary input for the as-is analysis, an exhaustive list of all possible error messages is created.
This as-is analysis will be used as a reference base to verify the implementation of the new error mechanism with Parsec.
The errors are invoked by simulating all possible syntax errors that will invoke an error within the Ampersand parser.
Each syntax statement is therefore manipulated, introducing one specific error per time, and the resulting error message is then recorded together with the actual erroneous statement.
The exact same statements are afterwards pushed through the new parser, making it possible to make a quantitative `before and after' analysis.
Special attention is given to avoid redundant errors that could influence the quantitative analysis. 
An example of such an redundant error is the use of a capital letter in defining a specific reference. 
Although these references are used in several syntax statements, there is only one procedure in the parser to check all references starting with a capital letter.
An improvement in the error message of this check may only be taken into account one time.

\subsubsection{Results}
Based on our error message qualification definition, \autoref{tab:error-messages-analysis} visualizes the results of the as-is analysis.
This analysis clearly confirms the statement that the quality of the error messages is of low quality and that there is a lot of room for improvement.

% Please add the following required packages to your document preamble:
% \usepackage[table,xcdraw]{xcolor}
% If you use beamer only pass "xcolor=table" option, i.e. \documentclass[xcolor=table]{beamer}
\begin{table}[h]
  \centering
	\begin{tabular}{llrlr}
    Error quality  & \multicolumn{2}{c}{Old parser}     \\
		Good           & 19          & 22,35\%         \\
		Average        & 18          & 21,18\%       \\
		Bad            & 48          & 56,47\%           \\
		\rowcolor[HTML]{BBBBBB}
		\textbf{Total} & \textbf{85} & \textbf{100,00\%} 
	\end{tabular}
  \caption{Error message as-is analysis results}
  \label{tab:error-messages-analysis}
\end{table}
