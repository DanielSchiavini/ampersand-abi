% !TEX root = ../Thesis.tex

\subsection{Parser (R-M)}
\label{analysis:parser}
The previous Ampersand parser was generally well organized, so each ENBF rule could be mapped to a different parser.
However, several flaws were observed as improvement points.
We addressed these flaws in our project where possible.
The flaws we were not able to solve, due to the constraints of our project, are summarized them in section \autoref{sec:recommendations}.
During this project, we focused on the following issues:
\begin{description}
  \item[Lacking documentation]
    There was no documentation on the recognized grammar.
    The last EBNF available was not updated in a long while.
  
  \item[Ad-hoc transformations]
    The parser was built with the applicative interface of the uulib.
    The applicative operators were thus used in sequence to recognize each of the accepted grammar productions.
    However, the parser was often forced to change the order and format of the parsed structures, because the parse tree did not match the grammar productions (i.e. many rebuild functions).
    
  \item[Long file]
    Since all the grammar constructions -- plus help functions -- were in a single file, the parser was hardly readable.
    It summed a total of 823 lines of code only in the \texttt{Parser} module.
  
  \item[Pretty printing]
    It used to be impossible to print the parse tree back to ADL-code.
    That made it harder to develop and test the parser properly.
  
  \item[Test suite]
    There were no automated tests for the parser.
    Because of this, any code change would be hard to test and could potentially influence other Ampersand modules.
  
  \item[Duplicated code]
    A large part of the code was duplicated and not used (mainly in the \texttt{Parsing} module).
  
  \item[Error messages]
    As mentioned, the main reason for this project was the bad quality of the error messages generated.
    \autoref{analysis:errors} expands on this point.
\end{description}
%
Although it may be hard to resolve all the mentioned issues, we believe our efforts have played off well.
In \autoref{design:new-parser} we describe how the new parser was designed.
