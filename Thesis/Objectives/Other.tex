% !TEX root = ../Thesis.tex

\subsection{Other objectives}
While designing and implementing the new Ampersand parser, besides the generation of better user feedback, the following objectives were also important:
\begin{itemize}
  \item \textbf{Integration}: The new parser must interface with the remaining Ampersand modules.
    It must thus be implemented in Haskell.
  \item \textbf{Libraries}: Since different implementation options are available, it was important to choose the most suitable Haskell parsing framework.
  \item \textbf{Maintainability}: Well-written and maintainable code is a must for the Ampersand project, since it is an open-source project.
    The maintainability must be either preserved or improved; otherwise the parser will not be taken into production.
  \item \textbf{Tests}: Testing the parser well was a task for this project.
    The suggestion is to use testing tools to improve the tests, e.g. QuickCheck \citepr{project-announcement}.
  \item \textbf{Pretty-printing}: This is important in order to test the parser.
\dict{HPC}{Haskell Program Coverage}%
\dict{Haddock}{Software documentation generator for the Haskell programming language}%
\dict{HLint}{Statical analysis software that suggests maintainability improvements}%
  \item \textbf{Tools}: Within the Haskell community, several tools are popular to verify code quality and generate documentation.
    At least HPC, Haddock and HLint shall be used in the new parser.
  \item \textbf{Backwards compatible}: The new parser must process the same inputs and produce the same outputs as the old parser.
    Any changes to the parser output may have consequences to the rest of the system, and we have to adapt the rest of the system if we change the parser output.
  \item \textbf{Git/GitHub}: The changed software had to be integrated into the GitHub Ampersand project.
    The development itself happened in a separate branch of a separate fork, so that deliveries could be merged in a smooth way.
    This was an especially hard requirement for us, since we had no experience with Git.
  \item \textbf{Cabal}: The building system for Haskell had to be maintained as the building platform.
  \item \textbf{EBNF}: The syntax of the Ampersand grammar was specified in EBNF notation but was not up-to-date.
    Any changes to the syntax had to be documented in this notation.
    Updating the grammar in a separate file was one option.
    Another option was to add the EBNF as comment in the source code as well, in order to make clear that the complete grammar is implemented correctly.
	The last option is chosen to keep the EBNF close to the actual code to stimulate the simultaneous maintenance of both the code and the EBNF notations.
\end{itemize}

On top of the project goals, we also wanted to help the university and other students with our results.
