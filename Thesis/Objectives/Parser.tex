% !TEX root = ../Thesis.tex

\subsection{User-friendly parser}
The end-to-end process of the ampersand project, from compiling towards the generated artifacts, is correct, however there is a major improvement topic identified in the first step, the parsing of the input scripts.

One of the main complaints from users is the quality of the errors generated by the Ampersand parser, making it hard for the end users to correct faulty ADL statements.
Since the beginning of the project, the parser subcomponent never received special attention, and it has not been analyzed for improvements.

In order to generate better, useful and to the point error messages, it is assumed that a complete refactoring of the parser will be necessary.
The main challenge is to choose the correct kind of architecture and libraries.
Note that discovering which errors are the most common and what user-friendly messages consist of, is an important part of the assignment.
At the beginning of the project, no list of undesirable error messages is available.
Therefore it was up to us, as a part of the project, to judge which error messages were good enough and which were undesirable.
