% !TEX root = ../Thesis.tex

\section{Objectives}
Vraagstelling in detail.

In het kader van het Ampersand onderzoeksprogramma is een compiler gemaakt voor bedrijfsregels. Deze compiler is in Haskell geschreven, en is beschikbaar als open source project. Eén van de vraagstukken is de kwaliteit van de parser-foutmeldingen. Daar is nog nooit goed naar gekeken, met als gevolg dat de foutmeldingen niet altijd informatief zijn voor de gebruiker.

De opdracht is om de bestaande parser in z’n geheel te refactoren, met aandacht voor foutmeldingen en pretty-printing. De uitdaging is om zo informatief mogelijke foutmeldingen te genereren. Als randvoorwaarde geldt dat de Haskell code onderhoudbaar moet zijn en goed moet zijn getest. Wanneer het aan de onderhoudbaarheid mankeert, wordt de nieuwe parser niet in productie genomen.

Een ontwikkelstraat op basis van SVN is aanwezig. Dit project zal in een branch worden uitgevoerd. Bij succesvolle afronding zal het ABI-team deze branch in de productiestroom integreren.

\begin{itemize}
\item Analyse naar gebruiksvriendelijke foutmeldingen in compilers 
\item Vergelijken van diverse Haskell bibliotheken om te parsen en te pretty-printen. 
\item Inventarisatie van technieken en tools in Haskell ter bevordering van de software\-kwaliteit 
\item Analyse van de huidige ontwikkelstraat in relatie tot software engineering principes zoals continuous delivery/integration en het doen van aanbevelingen om de kwaliteit bewaking te vergroten 
\end{itemize}

De software waaraan wordt gewerkt draagt in belangrijke mate bij aan het onderzoek naar bedrijfsregels dat wordt uitgevoerd door Joosten en zijn onderzoeksgroep. De ABI studenten zullen zich voldoende moeten inwerken in dit onderzoek om een bijdrage te kunnen leveren. De opdracht biedt ook de ruimte om onderzoeksresultaten naar gebruiksvriendelijke foutmeldingen en testmethoden in functionele talen (bijv. QuickCheck) in te passen in het project.

\paragraph{Accenten}
\begin{description}
\item[analyse van een gebruikerscontext] er zal enigszins een beeld moeten worden gevormd van het soort gebruiker van Ampersand en hoe deze gebruiker omgaat met de huidige foutmeldingen
\item[ontwerpen van een toepassing] bestaande software moet worden aangepast: daarvoor is het noodzakelijk dat eerst het bestaande ontwerp wordt begrepen en dat dit zorgvuldig wordt uitgebreid met het oog op kwaliteit en onderhoudbaarheid
\item[bouwen en implementeren van een toepassing] delen van de software zullen herschreven moeten worden
\item[literatuuronderzoek of bureauonderzoek] mogelijke oplossingsrichtingen en documentatie van software bibliotheken zullen onderzocht moeten worden
\end{description}

