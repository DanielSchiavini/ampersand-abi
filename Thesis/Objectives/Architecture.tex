% !TEX root = ../Thesis.tex

\subsection{High-level architecture}
\label{subsec:architecture}
The compiler developed for the Ampersand research project runs in several steps, hence the Ampersand compiler is also divided in several subcomponents:
\dict{P-structure}{The parse-tree generated by the Ampersand parser, used as input for the type checker}%
\dict{A-structure}{The ADL code generated by the Ampersand type checker, used as input for the calculator component}%
\dict{ADL-structure}{See A-structure}%
\dict{F-structure}{The functional structure generated by the Ampersand calculator, used as input for the different output modules}%
\begin{itemize}
	\item \textbf{Parser}: This component receives the ADL code as input, and parses that code into a parse-tree (also known as P-structure).
	\item \textbf{Type checker}: The Ampersand type checker receives a P-structure as input and converts it into a relational algebra format, suitable for manipulation (also known as A-structure or ADL-structure).
		 The semantics of ampersand are expressed in terms of the A-structure.
	\item \textbf{Calc}: The Calc component receives an A-structure as input, and manipulates it according to the research rules, generating the functional structure (also known as F-structure).
		The F-structure contains all design artifacts needed to write a specification and generate the output.
	\item \textbf{Output components}: All design artifacts present in the F-structure are ready to be rendered.
		Several components use this data structure to generate the wished output.
		The output components currently implemented (and their output formats) are the following: 
		\begin{itemize}
			\item Atlas (HTML interface);
			\item Revert (Haskell source);
			\item Query (prototype generation);
			\item Documentation Generator (Pandoc structure).
		\end{itemize}
\end{itemize}
