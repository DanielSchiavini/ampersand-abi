% !TEX root = ../Parsing.tex

\subsection{Ampersand in practice}
\label{sec:InPractice}

\subsubsection{Getting started}
To use the Ampersand tools, several components need to be installed:
\begin{description}
	\item[Ampersand installer] is an installation package containing the full Ampersand tool. This package can be downloaded from the Ampersand home page (\url{http://wiki.tarski.nl/}). The necessary installation instructions are provided on that same website.
	\item[MiKTex,] a LaTeX compiler used to generate documents which are presented in pdf format.
	\item[Graphviz] is used to generate the pictures, such as the class diagrams and the entity-relationship diagrams. 
	\item[XAMPP] can be used in case your environment does not support an HTTP/PHP server and/or a MySQL database.
\end{description}

\noindent
After some post-install actions, the Ampersand tool can be started in the MS-DOS command prompt in MS Windows. 
No graphical user interface is yet available. All commonly used command line parameters are listed on the Ampersand web page.

\subsubsection{Requirement gathering}
The business rules are recorded in a plain ASCII text file, created using a simple text editor such as Notepad++.
No specific user interface supporting on the fly checks on structure and types, e.g. brackets and names, is yet available.

\subsubsection{Practical usage}
The correct installation and usage of the Ampersand tools and the additional tools requires some IT awareness, especially in case some unexpected errors occur when the Ampersand tool uses the additional tools. 

To get used to the specific syntax of how business rules are defined in the ADL language requires some training. 
Besides the base syntax, a specific patterns hub is available, to support new and seasoned ADL users, which can be used for training, inspiration and re-usage.

As already mentioned in the introduction of this document, the feedback generated from the Ampersand tool, in case  issues are detected in the input file, with respect to the syntax, appears to be insufficient and needs to be revised.