% !TEX root = ../Thesis.tex

% een korte samenvatting (plusminus 250 woorden)
\begin{abstract} 
Ampersand is een methode om businessregels een grotere rol te geven tijdens softwareontwikkeling.
Businessregels worden geschreven in scripttaal (ADL) en door Ampersand gecompileerd naar ontwerpartefacten, documentatie en prototypen.
Naarmate het project groter wordt, krijgen gebruikers steeds meer last van de slechte kwaliteit van de gegenereerde foutberichten.
Vooral nieuwe gebruikers worden hierdoor gehinderd en gefrustreerd.

Onze opdracht, gegeven door Prof.dr. Stef Joosten, is om betere gebruikersfeedback te ontwerpen en te implementeren.
Eerst hebben wij onderzocht wat goede berichten inhouden en hoe deze in Haskell geïmplementeerd worden, met de conclusie om de parser te herschrijven met Parsec.
Daarnaast hebben wij een onderzoek gevoerd naar het gebruik van businessregels en de Ampersand-aanpak.

De documentatie van de parser en de ADL-grammatica was niet up-to-date, waardoor wij 
reversengineering hebben toegepast om documentatie te vastgeleggen.
Wij hebben ons ingezet om door refactoring de leesbaarheid, uitbreidbaarheid, onderhoudbaarheid, documentatie en performance van de parser en de grammatica te verbeteren, zonder de ADL-taal te beinvloeden. 
Een automatisch testsysteem is geïmplementeerd met de mogelijkheid om de parsetree te `prettyprinten’ als ADL-code.

De nieuwe Ampersand parser is geïntegreerd en in productie genomen.
Onze analyse toont aan dat de goede foutberichten van 22\% naar 82\% zijn gegaan, terwijl slechte foutberichten zijn gedaald van 56\% naar 1\%.

In deze presentatie laten wij zien hoe de nieuwe parser de doelstellingen behaalt en hoe deze verbeteringen het mogelijk maken dat Ampersand kan blijven groeien.
Deze verbeteringen zullen tijd en inspanning besparen voor studenten, onderzoekers en commerciële gebruikers.

~\\
\centering{
De presentatie zal gegeven worden op:\\
Dinsdag 30 juni om 13u15\\
~\\
Open Universiteit Nederland\\
Studiecentrum Utrecht\\
Vondellaan 202, 3521 GZ Utrecht
}

\end{abstract}
