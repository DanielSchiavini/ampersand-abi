% !TEX root = ../Thesis.tex

% een korte samenvatting (plusminus 250 woorden)
\begin{abstract} 
Ampersand is een methode om businessregels een grotere rol te geven tijdens softwareontwikkeling.
De Ampersand compiler converteert businessregels vanaf een scripttaal (ADL) naar ontwerpartefacten, documentatie en protypen.
Naarmate het project groter wordt, krijgen gebruikers steeds meer last van de slechte kwaliteit van de Ampersand-foutberichten.
Vooral nieuwe gebruikers worden hier door gehinderd en gefrustreerd.

Onze opdracht, gegeven door Prof.dr. Stef Joosten, is om betere gebruikersfeedback te ontwerpen en implementeren in Ampersand.
Ons eerst stap was om te onderzoeken wat goede berichten inhouden en hoe dat in Haskell geïmplementeerd wordt.
De conclusie was om de parser te herschrijven met een andere bibliotheek (Parsec).
Daarnaast hebben wij een onderzoek gedaan naar businessregels en de Ampersand-aanpak.

Om de kwaliteit van de software te garanderen hebben wij een nieuwe testbibliotheek geïmplementeerd en de mogelijkheid om de parsetree te `prettyprinten’ als ADL-code.
Daarnaast hebben we ons ingezet om de leesbaarheid, uitbreidbaarheid, onderhoudbaarheid, documentatie en performance van de parser te verbeteren.

Naast de parser hebben wij ook aan de ADL-grammatica gewerkt.
De documentatie ervan was niet up-to-date, dus wij hebben middels reverse-engineering de grammatica vastgelegd als codeannotaties en documentatie.
Vervolgens hebben wij refactoring toegepast om de grammatica duidelijker en ondubbelzinnig te maken, zonder de ADL-taal te beïnvloeden.

De nieuwe Ampersand parser is nu geïntegreerd en in productie genomen.
Ten slotte, onze analyse toont aan dat de oude parser in 22\% van de gevallen goede foutberichten geeft, terwijl de nieuwe parser dit verbetert naar 82\%.
Foutberichten met slechte kwaliteit dalen van 56\% naar 1\%.

In deze presentatie laten wij zien hoe de nieuwe parser de eisen heeft vervuld en hoe deze verbeteringen het mogelijk maken dat Ampersand kan blijven groeien.
Onze resultaten zullen tijd en inspanning sparen voor studenten, onderzoekers en commerciële gebruikers.

~\\
\centering{
De presentatie zal gegeven worden op:\\
Dinsdag 30 juni om 13u15\\
~\\
Open Universiteit Nederland\\
Studiecentrum Utrecht\\
Vondellaan 202, 3521 GZ Utrecht
}

\end{abstract}
