% !TEX root = ../Thesis.tex

% een korte samenvatting (plusminus 250 woorden)
\begin{abstract} 
Ampersand is an approach for giving business rules a larger role in the software development process.
The Ampersand compiler allows users to write business rules in a domain specific language (ADL) and process them into design artifacts, documentation and software prototypes.
The project became ever larger and users started having significant issues with error messages of bad quality generated by the tool.
Mainly for new users the bad quality of error messages were a large obstacle and caused frustrations.

Our assignment, given by Prof.dr. Stef Joosten, is thus to redesign and implement a new parser for Ampersand providing user friendly error feedback.
In order to improve the error messages, Daniel Schiavini has researched the qualities of error messages and the options for parsing within Haskell.
The main conclusion was to rewrite the parser using another parsing library, namely Parsec.
On the other hand, Maarten Baertsoen researched business rules and the Ampersand approach more thoroughly.

To guarantee the quality of the software delivered, we implemented a new library for testing the parser automatically.
We added the possibility to `pretty print' the parsing tree back into ADL code.
Several efforts have been made to improve the code quality in the aspects of readability, extensibility, maintainability, documentation and performance.

During development, we often merged the development code with the other developments.
While this cost some effort, it allowed us to take the new parser into production as soon as it was ready.

Besides improvements in the parser, we have studied the ADL grammar itself.
The grammar documentation was not up-to-date, so it was our task to reverse engineer the code to determine the recognized language.
The updated grammar is now available as code annotations and documentation.
In order to make the grammar more clear, performant and unambiguous, refactorings were applied without changing the language accepted by the parser.

Finally, a thorough analysis on the error message quality showed that the old parser gave good error messages in 22\% of the time, while the new parser improved this percentage up to 82\%.
Also, bad quality messages went down from 56\% to 1\%.

In this presentation, we will show how the new parser achieved its goals and how this will allow for Ampersand to continue growing.
This will save time and effort for students, researchers and commercial users.

~\\
\centering{
The presentation will be held on:\\
Tuesday, June 30th 2015 at 13:15 ECT\\
~\\
Open University Netherlands\\
Studiecentrum Utrecht\\
Vondellaan 202, 3521 GZ Utrecht
}
\end{abstract}