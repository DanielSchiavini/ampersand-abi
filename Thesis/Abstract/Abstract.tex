% !TEX root = ../Thesis.tex

% een korte samenvatting (plusminus 250 woorden)
\begin{abstract}
Business rules promise to improve the quality and efficiency of the requirements elicitation process in information system projects.
Once the business rules are consistently and coherently defined, they form a knowledge repository used for the provisioning of information within organizations.
However, there used to be a large difference between the rules as seen by the Business Rules Manifesto and way these were supported by software systems.
With this issue in mind, the Ampersand approach was developed, giving focus to rules formulated in agreements instead of actions.

Ampersand is an approach for the use of business rules to define the business processes.
Users describe the business rules in a formal language (ADL) and Ampersand compiles those rules into functional specification, documentation and working software prototypes.

Over the years, the compiler has been extensively used, extended and improved.
However, the parser component has never been carefully analyzed for improvements and in particular, the quality, hence the user friendliness, of the error messages is an attention point.
Better user feedback in the Ampersand parser was thus the main objective in this project.

By understanding what kind of messages the Ampersand users expect, we have developed a new parser for Ampersand based on the Parsec library.
This new parser improves both the error messages and the software properties related to readability, extensibility and maintainability.
Besides the parser itself, a suite of automated tests has been developed to assure the quality of the parsing results.

In this thesis we show how the parser was implemented and how this improved parser will allow Ampersand to continue growing.
\end{abstract}
