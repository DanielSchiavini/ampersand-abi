% !TEX root = ../Thesis.tex

% een korte samenvatting (plusminus 250 woorden)
\begin{abstract} 
Ampersand is an approach for giving business rules a larger role in the software development process.
The Ampersand compiler allows users to write business rules in a domain specific language (ADL) and process them into design artifacts, documentation and software prototypes.
As the project becomes larger, users have significant issues with error messages of bad quality generated by the tool.
Mainly for new users, errors are a large obstacle and cause frustrations.

Our assignment, given by Prof.dr. Stef Joosten, is to design and implement user friendly feedback for Ampersand.
In order to improve the errors, Daniel Schiavini researched the qualities of good errors and the options for parsing within Haskell.
The conclusion was to rewrite the parser using another parsing library (Parsec).
In parallel, Maarten Baertsoen researched business rules and the Ampersand approach more thoroughly.

To guarantee the quality of the software delivered, we implemented a new library for testing the parser automatically.
We added the possibility to `pretty print' the parsing tree back into ADL code.
Several efforts have been made to improve the code quality in the aspects of readability, extensibility, maintainability, documentation and performance.

Besides improvements in the parser, we have studied the ADL grammar itself.
The grammar documentation was not up-to-date, so we reverse engineered the code and determined the recognized language.
The updated grammar is now available as code annotations and documentation.
In order to make the grammar more clear, performing and unambiguous, refactorings were applied without changing the language accepted by the parser.

The new parser is now merged with the main repository and is officially in production.
Finally, an analysis on the quality of the errors showed that the previous parser gave good errors in 22\% of the time, while the new parser improved this percentage up to 82\%.
Bad quality messages went down from 21% to 1\%.

In this thesis, we will show how the new parser achieved its goals and how this will allow for Ampersand to continue growing.
This will save time and effort for students, researchers and commercial users.

%~\\
%\centering{
%The presentation will be held on:\\
%Tuesday, June 30th 2015 at 13:15\\
%~\\
%Open University of the Netherlands\\
%Studiecentrum Utrecht\\
%Vondellaan 202, 3521 GZ Utrecht
%}
\end{abstract}
