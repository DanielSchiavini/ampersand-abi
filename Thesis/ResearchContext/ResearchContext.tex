% !TEX root = ../Thesis.tex

\section{Research context}
\label{sec:research-context}

In this project phase we identified the research contexts related to the Ampersand approach as a whole.
After identifying the different contexts and understanding them, for each identified context we described the influence of this project on the bigger picture.

The starting point of the investigation is done through literature research, allowing us to understand the actual research context.
During the literature study, key questions are formulated which are presented afterwards, together with the actual literature study, to a consult researcher.
For this investigation, we consulted the expert researcher Dr. Lloyd Rutledge, who was willing to help us widen our horizon.

Our conclusion is that the new Ampersand parser has a very small influence in its research context for business rules.
The parser does not offer any new features or improvements to the language.
However, we expect that the new parser will improve user experience, allowing commercial, research and educational users to do their work faster and more efficiently.

The complete research report is available as a separate document \citepr{research-context}.
