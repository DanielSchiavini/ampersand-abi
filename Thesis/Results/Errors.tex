% !TEX root = ../Thesis.tex

\subsection{Error messages}

\subsubsection{Error message qualification}
The user friendliness and correctness of an error message is a subjective topic and therefore we need to start with a definition to objectively judge the quality of an error message.
After analysis of the current Ampersand parser error messages, we identified following objective aspects of an Ampersand parser error message:

\begin{description}
	\item [Position]
	Each error messages is accompanied with the position, line number and column, where the error was found.
	\item [Accuracy]
	The accuracy of an error message is measured based upon the following characteristics:
	\begin{enumerate}
		\item 	How does the provided error description exactly outlines the discovered error
				Providing the user with a good description of the encountered syntax issue will support a fast error resolution.
				When the issue is vaguely described without pinpointing the exact issue, the error resolution will be time consuming.
		\item 	Pinpointing the correct error
				An error can invoke multiple subsequent issues. 
				These issues are however irrelevant for the user and the Ampersand parser should provide the exact origin of the issues.
		\item 	Quality of Hint
			The parser provides a hint together with the error message to support the user with the error resolution.
	\end {enumerate}
    \item[Conciseness]
	Providing a good error description is one thing, but this message can be hidden between several other error messages, resulting from the initial error.
	It is unlikely that the user will easily find the exact originating issue in his source file when he is overwhelmed with a multitude of error messages.
\end {description}

Based on the objective Ampersand parser error characteristics, we defined the following definition to distinguish between good, bad and average error messages:

\begin{description}
	\item [Bad error message,]a message is considered to be bad if one of the criteria blow is fulfilled:
		\begin{description}
			\item [Position]
			The position has a deviation of more then 1 line or 10 column positions from where the actual error is made.
			\item [Accuracy]
				\begin{enumerate}
					\item 	The provided error description is useless for the user to determine the actual error.
					\item 	The provided error description is not appointing the main, originating error without any correlation towards this main error.
				\end {enumerate}
			\item[Conciseness]
			More then distinct 3 errors are mentioned by the Ampersand parser.
		\end {description}
	\item [Average error message,] a message is considered to be of average quality if one of the criteria blow is fulfilled:
		\begin{description}
			\item [Position]
			The position has a deviation between 5 and  10 columns positions from where the actual error is made.
			\item [Accuracy]
				\begin{enumerate}
					\item 	The provided error description is not an exact description of the error but provides however useful information to discover the actual issue.
					\item 	The provided error description is not appointing the main, originating error but the link to the actual error can be discovered based on the provided information
					\item 	The provided hint is incorrect.
				\end {enumerate}
			\item[Conciseness]
			2 or 3 errors are mentioned by the Ampersand parser.
		\end {description}
		
	\item [Good error message,] we can state that any error message being not bad nor average is good.
\end {description}

\subsubsection{Error message gathering process}

To support the as-is analysis, an exhaustive list of all possible error messages is created.
This as-is analysis will be used as a reference base to verify the implementation of the new error mechanism of Parsec.
The errors are invoked by simulating all possible syntax errors that will invoke an error within the Ampersand Parser.
Each syntax statement is therefore manipulated, introducing one specific error at the time, and the resulting issue is then recorded together with the actual erroneous statement.
The exact same statements are afterwards pushed through the new parser, making it possible to make a quantitative`before and after' analysis.
Special attention is given to avoid redundant errors that could influence the quantitative analysis. 
An example of such an redundant error is the use of a capital letter in defining a specific reference. 
Although these references are used in several syntax statements, there is only one procedure in the parser to check all references starting with a capital letter.
An improvement in the error message of this check may only be taken into account one time.

\subsubsection{Error message results}
After the thorough analysis of the old and new Ampersand parser, the results are summarized in the table below.
The actual list, containing the syntax statements with the corresponding error messages is available as an appendix to this document.

% Please add the following required packages to your document preamble:
% \usepackage[table,xcdraw]{xcolor}
% If you use beamer only pass "xcolor=table" option, i.e. \documentclass[xcolor=table]{beamer}
\begin{table}[h]
	\begin{tabular}{lllll}
               & \multicolumn{2}{l}{Old parser} & \multicolumn{2}{l}{New parser} \\
               & \#          & \%               & \#          & \%               \\
		Errors good    & 19          & 22,35\%          & 70          & 82,35\%          \\
		Errors average & 18          & 21,18\%          & 14          & 16,47\%          \\
		Errors bad     & 48          & 56,47\%          & 1           & 1,18\%           \\
		\rowcolor[HTML]{9B9B9B} 
		{\bf Total}    & {\bf 85}    & {\bf 100,00\%}   & {\bf 85}    & {\bf 100,00\%}  
	\end{tabular}
\end{table}

The table clearly visualizes that there indeed was an issue with the error messages generated by the old Ampersand parser.
By implementing the Parsec parser, an improvement of 360\% is made towards the creation of good error messages in which 82\% of the generated error messages are good, compared to 22\% in the old Parser.

One bad error message remains in the system, after analysis, the resolution of this remaining bad error would require to much effort, hence increased complexity, compared to the value gain and therefore, this message is kept as is.

General remarks and highlights:
 \begin{enumerate}
	\item 	If the error qualification was defined in a less strict way, the number would be slightly different as some bad errors would be judged as being average. 
		In the comparison between the old and new parser, the exact same qualification method is used while the syntax issue was, as well, identical.
		Due to this approach, we can conclude that the achieved results are justified.
	\item 	Error positioning was already correct in the old Ampersand parser, this is maintained in the new parser
	\item 	The principle of Parsec to end parsing after the first error is found has a massive improvement on the bad error messages due the conciseness issues.
		Nearly 50\% of the average and bad messages changed to good error messages just due to this principle.
	\item 	Some error messages were extremely elaborated due to which the standard Windows command prompt was unable to fully show the error message.
\end {enumerate}

\subsubsection{Conclusion}
Based on the defined error qualification method, we can clearly conclude that the implementation of the new Ampersand Parser has a very positive effect on the error messages quality.
The percentage of good error is now on an acceptable level and the remaining error all, except for 1, of average quality, still giving valuable information used for error resolution.

To avoid the risk of additional complexity in the parser code, decreasing the code maintainability, we judged that this result is  the best compromise between error message quality maintainability.
