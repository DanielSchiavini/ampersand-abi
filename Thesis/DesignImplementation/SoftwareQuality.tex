% !TEX root = ../Thesis.tex

\subsection{Software quality factors}
\label{design:software-quality}
In our project plan \citepr{plan} multiple non-functional requirements are included in the project scope.
Improving the code maintainability is one of the most important non-functional requirements.

To assure that these non-functional requirements are correctly addressed, we defined some measures to adhere to during the full project life cycle.

\subsubsection{Documentation}
All important design decisions we made together with the code we delivered need to be documented.
This documentation is needed for the Ampersand team to have a clear insight in the way the new parser is structured and how it is integrated in Ampersand.
The availability of this documentation is crucial for the maintainability of the new parser.
The following documentation is delivered as a result of this ABI project:
\dict{ABI}{Afstudeerproject Bachelor Informatica (graduation project for the computer science bachelor)}%
%
\begin{description}
  \item[System design]
    A general system overview of the new system, describing the goal and purpose of each module, how it is designed to achieve its goals and how it is integrated in the system architecture.
   The system design is integrated in this thesis document.
  \item[Code annotations]
    Haddock is the de-facto standard for generating Haskell documentation. 
    This documentation generator generates HTML based on the comments in the Haskell source code.
    It is important to remark that Haddock normally only generates documentation of the functions that are exported by each module.
    It is however important that all functions are well documented, including the internal ones, and therefore, the internal functions are documented using regular, non-Haddock, code annotations.
    The code annotations, in the source code, together with the Haddock documentation are delivered as an appendix to this document.
  \item[EBNF comments]
    The EBNF structure is the most important documentation of how the Ampersand syntax is composed and how the parser functions are defined.
    As described in \autoref{analysis:grammar}, the actual EBNF is retrieved through reverse engineering.
    Each parser function corresponds to a specific EBNF syntax rule and this rule is consistently annotated in the code just above the parse function.
  A specific markup (\code{------}) is used to tag the EBNF rules.
  This allows us to automatically extract the EBNF rules from source code and export them to other formats.
\end{description}

\subsubsection{Readability}
  In the as-is analysis of the current parser, we noticed that the code has been through several feature additions over the past years.
  These repetitive small changes reflected in parts of the code which has become too elaborated or sometimes even obsolete.

  Each code statement in the parser and lexer is analyzed by the project team and, where possible, refactored to be as concise as possible.
  The delivered code is now as short as possible without compromising the readability of the code.

  In addition, the code review tool HLint is used.
  This tool provides a full overview of several code optimization suggestions that can further optimize the readability of the source code. 
  These suggestions cover topics such as redundant brackets, parameter reductions and shortcut notations.
  All HLint warnings regarding the input subsystem are fully addressed before the code is delivered to the customer.
  The HLint report for the delivered code is available in the \hyperref[app:docs]{project documentation (appendix)}.

\subsubsection{Performance}
  Performance is a requirement often made in software engineering projects that is difficult to measure before the software is actually used in a production environment.
  For the new Ampersand parser, we proactively identified the topics that could have an impact on the resulting parsing performance.
  In the design of the new parser, a performance aspect that makes an important difference is the parser backtracking (with the \code{try} function).
  Several refactorings in the grammar are carried through to avoid the use of the \code{try} function. 
  A full list of remaining backtracking productions, together with our suggestions how to solve them, is provided in section \autoref{design:backtracking}.
