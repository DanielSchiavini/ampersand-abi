% !TEX root = ../Thesis.tex

%algemeen overzicht domeinen en technieken plusminus 2 pagina's
\subsection{Overview}
\label{domain:overview}
Before the actual parser work can start, it is imperative that the project team establishes a solid foundation of the existing system and the environment in which the system is used.
This phase of the project (domains \& techniques) is meant to assure that this knowledge is sufficiently gathered through individual research.
A set of relevant topics needed to support the team during the project execution is listed and allocated to the project team members.
The allocation was based on the aspect that Maarten took the more functional topics where Daniel addressed the more technical topics.
This way, both the domains and the techniques are well covered.

The research of Maarten was focused on the goals of the Ampersand approach as a whole, within the domain of formal specification techniques.
It focuses on the Ampersand vision, the methodology, the way Ampersand is used in practice and the future road-map of the Ampersand approach.
This is depicted in \autoref{domain:approach}.

Daniel investigated the domain of user friendly error messages and how to create them.
The second part of Daniel's research investigated the technical considerations to take into account while selecting a new parser.
Finally, based on the acquired insight, a new parser was selected to be proposed to our project customer.
His research is explained in \autoref{domain:parsing}.
