% !TEX root = ../Thesis.tex

% individueel verslag onderzoek deeldomein en bijhorende technieken, plusminus 5 pagina's
% details van domein en technieken in relatie met het onderzoeksproject
% academische verantwoording gemaakte keuzen

\subsection{The Ampersand approach}
\label{domain:approach}
%TODO: Expand more on this research, this text is not understandable by itself
The research on the Ampersand Approach is done by Maarten Baertsoen.
The objective was to understand the vision of Ampersand in relation towards other formal methods within the same domain.

%TODO: Maarten: What are the differentiators?
Ampersand is not the first approach to use formal specifications to support software engineering projects, but it has some important differentiators to cope with known drawbacks.
The project teams need to know these differentiators to assure that they are respected through the project life cycle.

The research starts with a problem statement with regards to the elaboration of software requirements.
A historical overview together with the current status of formal specification methods, trying to solve this problem statement, is outlined.
Besides the significant improvements of these formal methods, several striking drawbacks relate to the use of formal methods are identified.

Against this list of limitations and drawbacks, the vision behind the Ampersand Approach and the methodology including the supporting tools, is explained together with how this is realized.

\dict{SWOT}{Strengths, Weaknesses, Opportunities and Threats}
The actual usage of the current Ampersand in practice is investigated to summarize the benefits and further improvement areas of the approach.
The research is closed with a personal SWOT analysis of the Ampersand Approach.

The complete research report is available as a separate document \citepr{ampersand-approach}.
