% !TEX root = ../Thesis.tex

% individueel verslag onderzoek deeldomein en bijhorende technieken, plusminus 5 pagina's
% details van domein en technieken in relatie met het onderzoeksproject
% academische verantwoording gemaakte keuzen

\subsection{Parsing libraries \& friendly errors}
\label{domain:parsing}
The research on parsing libraries and user-friendly error messages was done by Daniel Schiavini.
The objective was to gather enough technical information to support the design and implementation of the new Ampersand parser.
It focuses on knowledge acquisition in two interrelated fronts: a search for the parsing library best suited for this project and defining what constitutes good error messages.
In this section we attempt to give a short summary of the choices made.
Details are described more carefully in a separate document \citepr{parsing}.

%The first thing we noticed is that there are two main manners of building a parser: it is possible to define the grammar and let the parser be generated by an specific tool (e.g. YACC \citeac{}); or it is possible to build the parser in Haskell directly, with the advent of a parsing library.
%A parser generator is able to execute statical analysis on the input grammar.
%Besides, generators are often able to recognize more complicated grammars by running Left-to-right, and picking the Rightmost derivation, being called therefore an LR parser.

\dict{EBNF}{Backus-Naur Form}%
Generally, there are two options for constructing a parser:
The first option is to construct the parser in the language of choice, i.e. Haskell for this project.
Another possibility is to use a domain-specific-language (DSL) to describe the grammar (usually annotated BNF), and let separate software generate the actual parsing code.

In the context of the new Ampersand parser, some advantages of using a parser generator, instead of handwriting the code, are the possible static analysis and optimizations that a generator can do, the performance that can be achieved and the guarantee that the BNF notation is always up-to-date.

On the other hand, the advantages of building the parser in Haskell instead of using a generator, are the flexibility of the programming language, the simplicity of the tool chain, the fact that the rest of the project is also in Haskell and most importantly the fact that both the running code and the generated errors are easier to understand.

Since it seems to be harder to generate good error messages when the user and the programmers understand less of the parser workings, our choice was to build the parser in Haskell and to use a parsing library.

The second research question is which parsing library was the most suitable.
Two libraries were investigated: The uu-parsinglib and Parsec.
Considering the library design, documentation, features, generated errors and error customization, Parsec was judged to be the best suited library for the new parser.
This advice to use Parsec was accepted by the customer.

A list of important consideration points on developing good feedback has been collected through the literature.
Finally, other factors besides error messages are also very important for giving useful feedback.
One of the most important factors is that the documentation should be always kept available, up-to-date, clear and concise.

The complete research report is available as a separate document \citepr{parsing}.
