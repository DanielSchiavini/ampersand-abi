% !TEX root = ../Thesis.tex

\subsection{Goals}
Business rules promise to improve the quality and efficiency of the requirements elicitation process in information system projects.
Once the business rules are consistently and coherently defined, they form a knowledge repository used for the provisioning of information within organizations.
However, there used to be a large difference between the rules as seen by the Business Rules Manifesto and way these were supported by software systems.
With this issue in mind, the Ampersand approach was developed, giving focus to rules formulated in agreements instead of actions.

Ampersand is an approach for the use of business rules to define the business processes.
Users describe the business rules in a formal language (ADL) and Ampersand compiles those rules into functional specification, documentation and working software prototypes.

Over the years, the compiler has been extensively used, extended and improved.
However, the parser component has never been carefully analyzed for improvements.
In particular, the quality of the error messages (hence the user friendliness) is an attention point.

Our assignment, given by Prof.dr. Stef Joosten, is thus to redesign and implement a new parser for Ampersand, providing user friendly error feedback.
In order to improve the error messages, Daniel Schiavini has researched the qualities of error messages and the options for parsing within Haskell.
The main conclusion was to rewrite the parser using another parsing library, namely Parsec.
On the other hand, Maarten Baertsoen researched business rules and the Ampersand approach more thoroughly.

To guarantee the quality of the software delivered, we implemented a new library for testing the parser automatically.
We added the possibility to `pretty print' the parsing tree back into ADL code.
Several efforts have been made to improve the code quality in the aspects of readability, extensibility, maintainability, documentation and performance.

During development, we often merged the development code with the other developments.
While this cost some effort, it allowed us to take the new parser into production as soon as it was ready.

Besides improvements in the parser itself, we have studied the ADL grammar itself.
The grammar documentation was not up-to-date, so it was our task to reverse engineer the code to determine the recognized language.
The updated grammar is now available as code annotations and documentation.
In order to make the grammar more clear, performant and unambiguous, refactorings were applied without changing the language accepted by the parser.

Finally, a thorough analysis on the error message quality showed that the old parser gave good error messages in 22\% of the time, while the new parser improved this percentage up to 82\%.
Also, bad quality messages went down from 56\% to 1\%.

In this presentation, we will show how the new parser achieved it goals and how this will allow for Ampersand to continue growing.
This will save time and effort for students, researchers and commercial users.

By understanding what kind of messages the Ampersand users expect, we have developed a new parser for Ampersand based on the Parsec library.
This new parser improves both the error messages and the software properties related to readability, extensibility and maintainability.
Besides the parser itself, a suite of automated tests has been developed to assure the quality of the parsing results.

In this thesis we show how the parser was implemented and how this improved parser will allow Ampersand to continue growing.