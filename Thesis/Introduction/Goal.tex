% !TEX root = ../Thesis.tex

\subsection{Goal of this document}
The main objective of this phase is to gather information that will support the execution of the project.
This document contains the results of the research on domain and techniques that will support the project group.
It focuses on knowledge acquisition in two interrelated fronts:
\begin{description}
	\item[Haskell parsing libraries]
	In order to build the new Ampersand Parser (or refactor the current one), a research is done to choose the library best suited for the development.
	The appropriate library is chosen based on its design, documentation, features and generated errors.
	
	\item[User-friendly error messages]
	The most important feature of the parser that will be built, is that it should generate user-friendly error messages.
	To understand what kinds of messages can be (and should be) generated, a research will be done on what good errors are and how to generate them.
	This part of the research is done by consulting literature.
\end{description}
%
The results of both subjects culminate in a single section with research conclusions.
