\documentclass[a4paper,12pt,titlepage]{scrartcl}
\usepackage{scrpage2}
\usepackage[utf8x]{inputenc}
\usepackage[T1]{fontenc}
\usepackage[top=2.5cm, bottom=2.5cm, left=2cm, right=2cm, landscape]{geometry}
\usepackage[affil-it]{authblk}
\usepackage{url}
\usepackage[hidelinks]{hyperref}
\usepackage{graphicx}
\usepackage[table,xcdraw]{xcolor}
\usepackage{longtable}
\usepackage{multicol}

% Here we set up the header, meta-information and front matter
\date{December 16, 2014}      %// Today's date will appear when this is commented out.
\newcommand{\version}{1.0}

% title page
\author{Daniel S. C. Schiavini en Maarten Baertoen}
\affil{Open Universiteit Nederland\\
    Faculteit Management, Science and Technology \\
	T61327 - Afstudeerproject bachelor informatica}
\title{Error comparison report}
\subtitle{Useful feedback in the Ampersand parser\\
	~\\
	Phase 3d: Project Documentation}
\publishers{Version \version}

% header
\pagestyle{scrheadings}
\setheadsepline{0.2pt}
\clearscrheadings
\automark[section]{chapter}
\ihead{Daniel S.C. Schiavini en Maarten Baertsoen}
\ohead{Parsing libraries \& error messages}
\cfoot{\pagemark}

% code listings
\usepackage{listings}
\usepackage{color}

\definecolor{mygreen}{rgb}{0,0.6,0}
\definecolor{mygray}{rgb}{0.5,0.5,0.5}
\definecolor{mymauve}{rgb}{0.58,0,0.82}
\lstset{%
    basicstyle=\small\ttfamily,
    breakatwhitespace=true,          % sets if automatic breaks should only happen at whitespace
    breaklines=true,                 % sets automatic line breaking
    commentstyle=\color{mygreen},    % comment style
    keepspaces=true,                 % keeps spaces in text, useful for keeping indentation of code (possibly needs columns=flexible)
    keywordstyle=\color{blue},       % keyword style
    numbersep=5pt,                   % how far the line-numbers are from the code
    numberstyle=\tiny\color{mygray}, % the style that is used for the line-numbers
    stepnumber=1,                    % the step between two line-numbers. If it's 1, each line will be numbered
    stringstyle=\color{mymauve},     % string literal style
}

\lstnewenvironment{haskell}{\lstset{language=Haskell,numbers=left, otherkeywords={}, deletekeywords={}}}{}
\lstnewenvironment{adl}{\lstset{language=Haskell,numbers=left, otherkeywords={INCLUDE,CONTEXT,ENDCONTEXT,EXTENDS,THEMES,META,PATTERN,ENDPATTERN,PROCESS,ENDPROCESS,INTERFACE,CLASS,FOR,BOX,ROWS,TABS,COLS,INITIAL,SQLPLUG,PHPPLUG,TYPE,POPULATION,CONTAINS,UNI,INJ,SUR,TOT,SYM,ASY,TRN,RFX,IRF,AUT,PROP,ALWAYS,RULE,MESSAGE,VIOLATION,SRC,TGT,TEST,RELATION,MEANING,CONCEPT,IDENT,VIEW,ENDVIEW,DEFAULT,TXT,PRIMHTML,TEMPLATE,KEY,IMPORT,SPEC,ISA,IS,I,V,CLASSIFY,PRAGMA,PURPOSE,IN,REF,ENGLISH,DUTCH,REST,HTML,LATEX,MARKDOWN,ONE,BYPLUG,ROLE,EDITS,MAINTAINS}, deletekeywords={String}}}{}
\lstnewenvironment{ebnf}{\lstset{language=Haskell, numbers=none, otherkeywords={::=,=>}, deletekeywords={String}}}{}

\newcommand{\code}[1]{\texttt{\small #1}}

% Now the document starts
\begin{document}
\maketitle
\newpage
  \begin{multicols}{3}
  \small
  \tableofcontents
  \end{multicols}
\newpage
  \section{Introduction}
  This document is part of the graduation project `Useful feedback in the Ampersand parser' of the computer science bachelor at the Open University of the Netherlands.
  It presents a comparison between the error messages presented by the previous and the new Ampersand parser.

\section{Error list}
\subsection{Error type 01}
  \begin{description}
  \item[Incorrect ADL]~\\
\begin{adl}
use of 'CORNCEPT' instead of CONCEPT\end{adl}
  \item[Previous error]~\\
\begin{haskell}
Error(s) found:

before upper case identifier CORNCEPT at line 7, column 1 of file "Hypotheek.adl
"
Expecting "ENDPATTERN" or (lower case identifier ?lc? or "CLASSIFY" or "CONCEPT"
 or "IDENT" or "KEY" or "POPULATION" or "PURPOSE" or "RELATION" or "ROLE" or "RU
LE" or "SPEC" or "VIEW" ...)*
Try inserting symbol "CONCEPT"\end{haskell}
  \item[Previous evaluation]~\\
    \begin{itemize}
    \item \textbf{Accurate:} Good
    \item \textbf{Intuitive:} Good
    \item \textbf{Succint:} Good
    \item \textbf{Evaluation: Good}
    \end{itemize}
  \item[New error]~\\
\begin{haskell}
[PE "ArchitectureAndDesign/Syntax/Hypotheek.adl" (line 7, column 1):
unexpected Upper case identifier CORNCEPT
expecting Keyword "RULE", Keyword "CLASSIFY", Keyword "RELATION", Keyword "CONCEPT", Keyword "SPEC", Keyword "IDENT", Keyword "VIEW", Keyword "KEY", Keyword "PURPOSE", Keyword "POPULATION" or Keyword "ENDPATTERN"]\end{haskell}
  \item[New evaluation]~\\
    \begin{itemize}
    \item \textbf{Accurate:} Good
    \item \textbf{Intuitive:} Good
    \item \textbf{Succint:} Good
    \item \textbf{Evaluation: Good
}
    \end{itemize}
  \end{description}

\hrulefill

\subsection{Error type 02}
  \begin{description}
  \item[Incorrect ADL]~\\
\begin{adl}
clientName :: Client -Z> Name =
    [ ("Client_1"      , "Martijn")
    ; ("Client_2"      , "Stef")
    ]
\end{adl}
  \item[Previous error]~\\
\begin{haskell}
Error(s) found:

before "-" at line 14, column 22 of file "testfle_mba.adl"
Expecting symbol [ or "*" or "->" or "<-"
Try inserting symbol symbol [

==============================

before upper case identifier Z at line 14, column 23 of file "testfle_mba.adl"
Expecting symbol ] or "*" or "0" or "1"
Try inserting symbol symbol ]

==============================

before ">" at line 14, column 24 of file "testfle_mba.adl"
Expecting symbol [ or lower case identifier ?lc? or "." or "=" or "BYPLUG" or "CLASSIFY" or "CONCEPT" or "ENDCONTEXT" or "IDENT" or "INCLUDE" or "INTERFACE" or
"KEY" or "META" or "PATTERN" or "PHPPLUG" or "POPULATION" or "PRAGMA" or "PROCESS" or "PURPOSE" or "RELATION" or "RULE" or "SPEC" or "SQLPLUG" or "THEMES" or "V
IEW" or ("MEANING" ...)*
Try deleting symbol ">" at line 14, column 24 of file "testfile_mba.adl"

==============================

before upper case identifier Name at line 14, column 26 of file "testfle_mba.adl
" Expecting lower case identifier ?lc? or "." or "=" or "CLASSIFY" or "CONCEPT" or  "ENDCONTEXT" or "IDENT" or "INCLUDE" or "INTERFACE" or "KEY" or "META" or "PATTERN" or "PHPPLUG" or "POPULATION" or "PROCESS" or "PURPOSE" or "RELATION" or "RULE" or "SPEC" or "SQLPLUG" or "THEMES" or "VIEW"
Try deleting symbol upper case identifier Name at line 14, column 26 of file "testfile_mba.adl"\end{haskell}
  \item[Previous evaluation]~\\
    \begin{itemize}
    \item \textbf{Accurate:} Good
    \item \textbf{Intuitive:} Acceptable
    \item \textbf{Succint:} Bad
    \item \textbf{Evaluation: Bad}
    \end{itemize}
  \item[New error]~\\
\begin{haskell}
PE "ArchitectureAndDesign/Syntax/testfile_mba.adl" (line 14, column 22):
unexpected Operator '-'
expecting Operator '*', Operator '->', Operator '<-' or Symbol '['\end{haskell}
  \item[New evaluation]~\\
    \begin{itemize}
    \item \textbf{Accurate:} Good
    \item \textbf{Intuitive:} Good
    \item \textbf{Succint:} Good
    \item \textbf{Evaluation: Good
}
    \end{itemize}
  \end{description}

\hrulefill

\subsection{Error type 03}
  \begin{description}
  \item[Incorrect ADL]~\\
\begin{adl}
clientName :: Client ** Name =
    [ ("Client_1"      , "Martijn")
    ; ("Client_2"      , "Stef")
    ]\end{adl}
  \item[Previous error]~\\
\begin{haskell}
Error(s) found:

before "*" at line 14, column 23 of file "testfile_mba.adl"
Expecting upper case identifier ?uc? or string ""
Try deleting symbol "*" at line 14, column 23 of file "testfile_mba.adl"\end{haskell}
  \item[Previous evaluation]~\\
    \begin{itemize}
    \item \textbf{Accurate:} Good
    \item \textbf{Intuitive:} Acceptable
    \item \textbf{Succint:} Good
    \item \textbf{Evaluation: Acceptable}
    \end{itemize}
  \item[New error]~\\
\begin{haskell}
PE "ArchitectureAndDesign/Syntax/testfile_mba.adl" (line 14, column 23):
unexpected Operator '*'\end{haskell}
  \item[New evaluation]~\\
    \begin{itemize}
    \item \textbf{Accurate:} Good
    \item \textbf{Intuitive:} Good
    \item \textbf{Succint:} Good
    \item \textbf{Evaluation: Good
}
    \end{itemize}
  \end{description}

\hrulefill

\subsection{Error type 04}
  \begin{description}
  \item[Incorrect ADL]~\\
\begin{adl}
clientName :: Client [] Name =
    [ ("Client_1"      , "Martijn")
    ; ("Client_2"      , "Stef")
    ]\end{adl}
  \item[Previous error]~\\
\begin{haskell}
Error(s) found:

before symbol ] at line 14, column 23 of file "testfile_mba.adl"
Expecting "*" or "-" or "0" or "1"
Try inserting symbol "-"

\end{haskell}
  \item[Previous evaluation]~\\
    \begin{itemize}
    \item \textbf{Accurate:} Good
    \item \textbf{Intuitive:} Acceptable
    \item \textbf{Succint:} Good
    \item \textbf{Evaluation: Acceptable}
    \end{itemize}
  \item[New error]~\\
\begin{haskell}
PE "ArchitectureAndDesign/Syntax/testfile_mba.adl" (line 14, column 23):
unexpected Symbol ']'
expecting Integer 0, Hexadecimal 0x, Octal 0o, Integer 1, Hexadecimal 0x1, Octal 0o1, Operator '*' or Operator '-'\end{haskell}
  \item[New evaluation]~\\
    \begin{itemize}
    \item \textbf{Accurate:} Good
    \item \textbf{Intuitive:} Good
    \item \textbf{Succint:} Good
    \item \textbf{Evaluation: Good
}
    \end{itemize}
  \end{description}

\hrulefill

\subsection{Error type 05}
  \begin{description}
  \item[Incorrect ADL]~\\
\begin{adl}
clientName :: Client [0..2-*] Name =
    [ ("Client_1"      , "Martijn")
    ; ("Client_2"      , "Stef")
    ]\end{adl}
  \item[Previous error]~\\
\begin{haskell}
Error(s) found:

before decimal Integer 2 at line 14, column 26 of file "testfile_mba.adl"
Expecting "*" or "1"
Try deleting symbol decimal Integer 2 at line 14, column 26 of file "testfile_mb
a.adl"

==============================

before "-" at line 14, column 27 of file "testfile_mba.adl"
Expecting "*" or "1"
Try inserting symbol "1"\end{haskell}
  \item[Previous evaluation]~\\
    \begin{itemize}
    \item \textbf{Accurate:} Good
    \item \textbf{Intuitive:} Good
    \item \textbf{Succint:} Acceptable
    \item \textbf{Evaluation: Acceptable}
    \end{itemize}
  \item[New error]~\\
\begin{haskell}
PE "ArchitectureAndDesign/Syntax/testfile_mba.adl" (line 14, column 26):
unexpected Integer 2
expecting Integer 1, Hexadecimal 0x1, Octal 0o1 or Operator '*'\end{haskell}
  \item[New evaluation]~\\
    \begin{itemize}
    \item \textbf{Accurate:} Good
    \item \textbf{Intuitive:} Good
    \item \textbf{Succint:} Good
    \item \textbf{Evaluation: Good
}
    \end{itemize}
  \end{description}

\hrulefill

\subsection{Error type 06}
  \begin{description}
  \item[Incorrect ADL]~\\
\begin{adl}
clientName :: Client [0..1] Name =
    [ ("Client_1"      , "Martijn")
    ; ("Client_2"      , "Stef")
    ]\end{adl}
  \item[Previous error]~\\
\begin{haskell}
Error(s) found:

before symbol ] at line 14, column 27 of file "testfile_mba.adl"
Expecting "-"
Try inserting symbol "-"
\end{haskell}
  \item[Previous evaluation]~\\
    \begin{itemize}
    \item \textbf{Accurate:} Good
    \item \textbf{Intuitive:} Acceptable
    \item \textbf{Succint:} Good
    \item \textbf{Evaluation: Acceptable}
    \end{itemize}
  \item[New error]~\\
\begin{haskell}
PE "ArchitectureAndDesign/Syntax/testfile_mba.adl" (line 14, column 27):
unexpected Symbol ']'
expecting Operator '-'\end{haskell}
  \item[New evaluation]~\\
    \begin{itemize}
    \item \textbf{Accurate:} Good
    \item \textbf{Intuitive:} Acceptable
    \item \textbf{Succint:} Good
    \item \textbf{Evaluation: Acceptable
}
    \end{itemize}
  \end{description}

\hrulefill

\subsection{Error type 07}
  \begin{description}
  \item[Incorrect ADL]~\\
\begin{adl}
clientName :: Client [0-1] Name =
    [ ("Client_1"      , "Martijn")
    ; ("Client_2"      , "Stef")
    ]\end{adl}
  \item[Previous error]~\\
\begin{haskell}
Error(s) found:

before "-" at line 14, column 24 of file "testfile_mba.adl"
Expecting ".."
Try inserting symbol ".."

==============================

before "-" at line 14, column 24 of file "testfile_mba.adl"
Expecting "*" or "1"
Try inserting symbol "1"\end{haskell}
  \item[Previous evaluation]~\\
    \begin{itemize}
    \item \textbf{Accurate:} Good
    \item \textbf{Intuitive:} Good
    \item \textbf{Succint:} Acceptable
    \item \textbf{Evaluation: Acceptable}
    \end{itemize}
  \item[New error]~\\
\begin{haskell}
PE "ArchitectureAndDesign/Syntax/testfile_mba.adl" (line 14, column 24):
unexpected Operator '-'
expecting Operator '..'\end{haskell}
  \item[New evaluation]~\\
    \begin{itemize}
    \item \textbf{Accurate:} Good
    \item \textbf{Intuitive:} Good
    \item \textbf{Succint:} Good
    \item \textbf{Evaluation: Good
}
    \end{itemize}
  \end{description}

\hrulefill

\subsection{Error type 08}
  \begin{description}
  \item[Incorrect ADL]~\\
\begin{adl}
clientName :: Client [0-a] Name =
    [ ("Client_1"      , "Martijn")
    ; ("Client_2"      , "Stef")
    ]\end{adl}
  \item[Previous error]~\\
\begin{haskell}
Error(s) found:

before "-" at line 14, column 24 of file "testfile_mba.adl"
Expecting ".."
Try inserting symbol ".."

==============================

before "-" at line 14, column 24 of file "testfile_mba.adl"
Expecting "*" or "1"
Try inserting symbol "1"

==============================

before lower case identifier a at line 14, column 25 of file "testfile_mba.adl"
Expecting symbol ] or "*" or "0" or "1"
Try deleting symbol lower case identifier a at line 14, column 25 of file "testf
ile_mba.adl"\end{haskell}
  \item[Previous evaluation]~\\
    \begin{itemize}
    \item \textbf{Accurate:} Good
    \item \textbf{Intuitive:} Good
    \item \textbf{Succint:} Acceptable
    \item \textbf{Evaluation: Acceptable}
    \end{itemize}
  \item[New error]~\\
\begin{haskell}
PE "ArchitectureAndDesign/Syntax/testfile_mba.adl" (line 14, column 24):
unexpected Operator '-'
expecting Operator '..'
\end{haskell}
  \item[New evaluation]~\\
    \begin{itemize}
    \item \textbf{Accurate:} Good
    \item \textbf{Intuitive:} Good
    \item \textbf{Succint:} Good
    \item \textbf{Evaluation: Good
}
    \end{itemize}
  \end{description}

\hrulefill

\subsection{Error type 09}
  \begin{description}
  \item[Incorrect ADL]~\\
\begin{adl}
clientName :: Client [*-1..0] Name =
    [ ("Client_1"      , "Martijn")
    ; ("Client_2"      , "Stef")
    ]\end{adl}
  \item[Previous error]~\\
\begin{haskell}
Error(s) found:

before "0" at line 14, column 28 of file "testfile_mba.adl"
Expecting "*" or "1"
Try deleting symbol "0" at line 14, column 28 of file "testfile_mba.adl"

==============================

before symbol ] at line 14, column 29 of file "testfile_mba.adl"
Expecting "*" or "1"
Try inserting symbol "1"\end{haskell}
  \item[Previous evaluation]~\\
    \begin{itemize}
    \item \textbf{Accurate:} Good
    \item \textbf{Intuitive:} Good
    \item \textbf{Succint:} Acceptable
    \item \textbf{Evaluation: Acceptable}
    \end{itemize}
  \item[New error]~\\
\begin{haskell}
PE "ArchitectureAndDesign/Syntax/testfile_mba.adl" (line 14, column 28):
unexpected Integer 0
expecting Integer 1, Hexadecimal 0x1, Octal 0o1 or Operator '*'\end{haskell}
  \item[New evaluation]~\\
    \begin{itemize}
    \item \textbf{Accurate:} Good
    \item \textbf{Intuitive:} Good
    \item \textbf{Succint:} Good
    \item \textbf{Evaluation: Good
}
    \end{itemize}
  \end{description}

\hrulefill

\subsection{Error type 10}
  \begin{description}
  \item[Incorrect ADL]~\\
\begin{adl}
clientName :: 0Client [*-1..1] Name =
    [ ("Client_1"      , "Martijn")
    ; ("Client_2"      , "Stef")
    ]\end{adl}
  \item[Previous error]~\\
\begin{haskell}
Error(s) found:

before "0" at line 14, column 15 of file "testfile_mba.adl"
Expecting upper case identifier ?uc? or string ""
Try deleting symbol "0" at line 14, column 15 of file "testfile_mba.adl"\end{haskell}
  \item[Previous evaluation]~\\
    \begin{itemize}
    \item \textbf{Accurate:} Good
    \item \textbf{Intuitive:} Good
    \item \textbf{Succint:} Good
    \item \textbf{Evaluation: Good}
    \end{itemize}
  \item[New error]~\\
\begin{haskell}
PE "ArchitectureAndDesign/Syntax/testfile_mba.adl" (line 14, column 15):
unexpected Integer 0\end{haskell}
  \item[New evaluation]~\\
    \begin{itemize}
    \item \textbf{Accurate:} Good
    \item \textbf{Intuitive:} Good
    \item \textbf{Succint:} Good
    \item \textbf{Evaluation: Good
}
    \end{itemize}
  \end{description}

\hrulefill

\subsection{Error type 11}
  \begin{description}
  \item[Incorrect ADL]~\\
\begin{adl}
clientName : Client [*-1..1] Name =
    [ ("Client_1"      , "Martijn")
    ; ("Client_2"      , "Stef")
    ]\end{adl}
  \item[Previous error]~\\
\begin{haskell}
Error(s) found:

before ":" at line 14, column 12 of file "testfile_mba.adl"
Expecting "::"
Try deleting symbol ":" at line 14, column 12 of file "testfile_mba.adl"

==============================

before upper case identifier Client at line 14, column 14 of file "testfile_mba.
adl"
Expecting "::"
Try inserting symbol "::"\end{haskell}
  \item[Previous evaluation]~\\
    \begin{itemize}
    \item \textbf{Accurate:} Good
    \item \textbf{Intuitive:} Good
    \item \textbf{Succint:} Acceptable
    \item \textbf{Evaluation: Acceptable}
    \end{itemize}
  \item[New error]~\\
\begin{haskell}
PE "ArchitectureAndDesign/Syntax/testfile_mba.adl" (line 14, column 12):
unexpected Operator ':'
expecting Operator '::'\end{haskell}
  \item[New evaluation]~\\
    \begin{itemize}
    \item \textbf{Accurate:} Good
    \item \textbf{Intuitive:} Good
    \item \textbf{Succint:} Good
    \item \textbf{Evaluation: Good
}
    \end{itemize}
  \end{description}

\hrulefill

\subsection{Error type 12}
  \begin{description}
  \item[Incorrect ADL]~\\
\begin{adl}
RELATION RelTest [RelationCR]\end{adl}
  \item[Previous error]~\\
\begin{haskell}
Error(s) found:

before upper case identifier RelTest at line 29, column 10 of file "testfile_mba
.adl"
Expecting lower case identifier ?lc?
Try deleting symbol upper case identifier RelTest at line 29, column 10 of file
"testfile_mba.adl"

==============================

before symbol [ at line 29, column 18 of file "testfile_mba.adl"
Expecting lower case identifier ?lc?
Try inserting symbol lower case identifier ?lc?\end{haskell}
  \item[Previous evaluation]~\\
    \begin{itemize}
    \item \textbf{Accurate:} Good
    \item \textbf{Intuitive:} Acceptable
    \item \textbf{Succint:} Acceptable
    \item \textbf{Evaluation: Acceptable}
    \end{itemize}
  \item[New error]~\\
\begin{haskell}
PE "ArchitectureAndDesign/Syntax/testfile_mba.adl" (line 29, column 10):
unexpected Upper case identifier RelTest\end{haskell}
  \item[New evaluation]~\\
    \begin{itemize}
    \item \textbf{Accurate:} Good
    \item \textbf{Intuitive:} Good
    \item \textbf{Succint:} Good
    \item \textbf{Evaluation: Good
}
    \end{itemize}
  \end{description}

\hrulefill

\subsection{Error type 13}
  \begin{description}
  \item[Incorrect ADL]~\\
\begin{adl}
RELATION relTest [relationCR]\end{adl}
  \item[Previous error]~\\
\begin{haskell}
Error(s) found:

before lower case identifier relationCR at line 29, column 19 of file "testfile_
mba.adl"
Expecting upper case identifier ?uc? or "ONE" or string ""
Try deleting symbol lower case identifier relationCR at line 29, column 19 of fi
le "testfile_mba.adl"

==============================

before symbol ] at line 29, column 29 of file "testfile_mba.adl"
Expecting upper case identifier ?uc? or "ONE" or string ""
Try inserting symbol "ONE"
\end{haskell}
  \item[Previous evaluation]~\\
    \begin{itemize}
    \item \textbf{Accurate:} Good
    \item \textbf{Intuitive:} Good
    \item \textbf{Succint:} Acceptable
    \item \textbf{Evaluation: Acceptable}
    \end{itemize}
  \item[New error]~\\
\begin{haskell}
PE "ArchitectureAndDesign/Syntax/testfile_mba.adl" (line 29, column 19):
unexpected Lower case identifier relationCR
expecting Keyword "ONE"\end{haskell}
  \item[New evaluation]~\\
    \begin{itemize}
    \item \textbf{Accurate:} Good
    \item \textbf{Intuitive:} Acceptable
    \item \textbf{Succint:} Good
    \item \textbf{Evaluation: Acceptable
}
    \end{itemize}
  \end{description}

\hrulefill

\subsection{Error type 14}
  \begin{description}
  \item[Incorrect ADL]~\\
\begin{adl}
RELATION relTest [ONE-ONNE]\end{adl}
  \item[Previous error]~\\
\begin{haskell}
Error(s) found:

before "-" at line 19, column 22 of file "testfile_mba.adl"
Expecting symbol ] or "*"
Try deleting symbol "-" at line 19, column 22 of file "testfile_mba.adl"

==============================

before upper case identifier ONNE at line 19, column 23 of file "testfile_mba.ad
l"
Expecting symbol ]
Try deleting symbol upper case identifier ONNE at line 19, column 23 of file "te
stfile_mba.adl"\end{haskell}
  \item[Previous evaluation]~\\
    \begin{itemize}
    \item \textbf{Accurate:} Good
    \item \textbf{Intuitive:} Good
    \item \textbf{Succint:} Acceptable
    \item \textbf{Evaluation: Acceptable}
    \end{itemize}
  \item[New error]~\\
\begin{haskell}
PE "ArchitectureAndDesign/Syntax/testfile_mba.adl" (line 19, column 22):
unexpected Operator '-'
expecting Operator '*' or Symbol ']'\end{haskell}
  \item[New evaluation]~\\
    \begin{itemize}
    \item \textbf{Accurate:} Good
    \item \textbf{Intuitive:} Good
    \item \textbf{Succint:} Good
    \item \textbf{Evaluation: Good
}
    \end{itemize}
  \end{description}

\hrulefill

\subsection{Error type 15}
  \begin{description}
  \item[Incorrect ADL]~\\
\begin{adl}
clientName :: client [*-1..1] Name =
    [ ("Client_1"      , "Martijn")
    ; ("Client_2"      , "Stef")
    ]\end{adl}
  \item[Previous error]~\\
\begin{haskell}
Error(s) found:

before lower case identifier client at line 27, column 15 of file "testfile_mba.
adl"
Expecting upper case identifier ?uc? or string ""
Try deleting symbol lower case identifier client at line 27, column 15 of file "
testfile_mba.adl"

==============================

before symbol [ at line 27, column 22 of file "testfile_mba.adl"
Expecting upper case identifier ?uc? or string ""
Try inserting symbol upper case identifier ?uc?\end{haskell}
  \item[Previous evaluation]~\\
    \begin{itemize}
    \item \textbf{Accurate:} Good
    \item \textbf{Intuitive:} Good
    \item \textbf{Succint:} Acceptable
    \item \textbf{Evaluation: Acceptable}
    \end{itemize}
  \item[New error]~\\
\begin{haskell}
PE "ArchitectureAndDesign/Syntax/testfile_mba.adl" (line 14, column 15):
unexpected Lower case identifier client\end{haskell}
  \item[New evaluation]~\\
    \begin{itemize}
    \item \textbf{Accurate:} Good
    \item \textbf{Intuitive:} Good
    \item \textbf{Succint:} Good
    \item \textbf{Evaluation: Good
}
    \end{itemize}
  \end{description}

\hrulefill

\subsection{Error type 16}
  \begin{description}
  \item[Incorrect ADL]~\\
\begin{adl}
"clientName :: client [*-1..1] Name =
    [ (""Client_1""      , ""Martijn"")
    ; (""Client_2""      , ""Stef"")
    ]"\end{adl}
  \item[Previous error]~\\
\begin{haskell}
Error(s) found:

before error in scanner: Unterminated string literal at line 27, column 1 of fil
e "testfile_mba.adl"
Expecting lower case identifier ?lc? or "CLASSIFY" or "CONCEPT" or "ENDCONTEXT"
or "IDENT" or "INCLUDE" or "INTERFACE" or "KEY" or "META" or "PATTERN" or "PHPPL
UG" or "POPULATION" or "PROCESS" or "PURPOSE" or "RELATION" or "RULE" or "SPEC"
or "SQLPLUG" or "THEMES" or "VIEW"
Try deleting symbol error in scanner: Unterminated string literal at line 27, co
lumn 1 of file "testfile_mba.adl"

==============================

before symbol [ at line 28, column 5 of file "testfile_mba.adl"
Expecting lower case identifier ?lc? or "CLASSIFY" or "CONCEPT" or "ENDCONTEXT"
or "IDENT" or "INCLUDE" or "INTERFACE" or "KEY" or "META" or "PATTERN" or "PHPPL
UG" or "POPULATION" or "PROCESS" or "PURPOSE" or "RELATION" or "RULE" or "SPEC"
or "SQLPLUG" or "THEMES" or "VIEW"
Try deleting symbol symbol [ at line 28, column 5 of file "testfile_mba.adl"

==============================

before symbol ( at line 28, column 7 of file "testfile_mba.adl"
Expecting lower case identifier ?lc? or "CLASSIFY" or "CONCEPT" or "ENDCONTEXT"
or "IDENT" or "INCLUDE" or "INTERFACE" or "KEY" or "META" or "PATTERN" or "PHPPL
UG" or "POPULATION" or "PROCESS" or "PURPOSE" or "RELATION" or "RULE" or "SPEC"
or "SQLPLUG" or "THEMES" or "VIEW"
Try deleting symbol symbol ( at line 28, column 7 of file "testfile_mba.adl"

==============================

before string "" at line 28, column 8 of file "testfile_mba.adl"
Expecting lower case identifier ?lc? or "CLASSIFY" or "CONCEPT" or "ENDCONTEXT"
or "IDENT" or "INCLUDE" or "INTERFACE" or "KEY" or "META" or "PATTERN" or "PHPPL
UG" or "POPULATION" or "PROCESS" or "PURPOSE" or "RELATION" or "RULE" or "SPEC"
or "SQLPLUG" or "THEMES" or "VIEW"
Try deleting symbol string "" at line 28, column 8 of file "testfile_mba.adl"

==============================

before upper case identifier Client_1 at line 28, column 10 of file "testfile_mb
a.adl"
Expecting lower case identifier ?lc? or "CLASSIFY" or "CONCEPT" or "ENDCONTEXT"
or "IDENT" or "INCLUDE" or "INTERFACE" or "KEY" or "META" or "PATTERN" or "PHPPL
UG" or "POPULATION" or "PROCESS" or "PURPOSE" or "RELATION" or "RULE" or "SPEC"
or "SQLPLUG" or "THEMES" or "VIEW"
Try deleting symbol upper case identifier Client_1 at line 28, column 10 of file
 "testfile_mba.adl"

==============================

before string "" at line 28, column 18 of file "testfile_mba.adl"
Expecting lower case identifier ?lc? or "CLASSIFY" or "CONCEPT" or "ENDCONTEXT"
or "IDENT" or "INCLUDE" or "INTERFACE" or "KEY" or "META" or "PATTERN" or "PHPPL
UG" or "POPULATION" or "PROCESS" or "PURPOSE" or "RELATION" or "RULE" or "SPEC"
or "SQLPLUG" or "THEMES" or "VIEW"
Try inserting symbol "THEMES"

==============================

before upper case identifier Martijn at line 28, column 30 of file "testfile_mba
.adl"
Expecting symbol , or lower case identifier ?lc? or "CLASSIFY" or "CONCEPT" or "
ENDCONTEXT" or "IDENT" or "INCLUDE" or "INTERFACE" or "KEY" or "META" or "PATTER
N" or "PHPPLUG" or "POPULATION" or "PROCESS" or "PURPOSE" or "RELATION" or "RULE
" or "SPEC" or "SQLPLUG" or "THEMES" or "VIEW"
Try inserting symbol "PATTERN"

==============================

before string "" at line 28, column 37 of file "testfile_mba.adl"
Expecting "ENDPATTERN" or (lower case identifier ?lc? or "CLASSIFY" or "CONCEPT"
 or "IDENT" or "KEY" or "POPULATION" or "PURPOSE" or "RELATION" or "ROLE" or "RU
LE" or "SPEC" or "VIEW" ...)*
Try inserting symbol "ENDPATTERN"

==============================

before string "" at line 28, column 37 of file "testfile_mba.adl"
Expecting lower case identifier ?lc? or "CLASSIFY" or "CONCEPT" or "ENDCONTEXT"
or "IDENT" or "INCLUDE" or "INTERFACE" or "KEY" or "META" or "PATTERN" or "PHPPL
UG" or "POPULATION" or "PROCESS" or "PURPOSE" or "RELATION" or "RULE" or "SPEC"
or "SQLPLUG" or "THEMES" or "VIEW"
Try inserting symbol "RULE"

==============================

before symbol ) at line 28, column 39 of file "testfile_mba.adl"
Expecting ":"
Try deleting symbol symbol ) at line 28, column 39 of file "testfile_mba.adl"

==============================

before ";" at line 29, column 5 of file "testfile_mba.adl"
Expecting ":"
Try inserting symbol ":"

==============================

before ";" at line 29, column 5 of file "testfile_mba.adl"
Expecting symbol ( or lower case identifier ?LC? or "I" or "V" or atom '' or ("-
" ...)*
Try inserting symbol lower case identifier ?LC?

==============================

before string "" at line 29, column 8 of file "testfile_mba.adl"
Expecting symbol ( or lower case identifier ?LC? or "I" or "V" or atom '' or ("-
" ...)*
Try deleting symbol string "" at line 29, column 8 of file "testfile_mba.adl"

==============================

before upper case identifier Client_2 at line 29, column 10 of file "testfile_mb
a.adl"
Expecting symbol ( or lower case identifier ?LC? or "I" or "V" or atom '' or ("-
" ...)*
Try deleting symbol upper case identifier Client_2 at line 29, column 10 of file
 "testfile_mba.adl"

==============================

before string "" at line 29, column 18 of file "testfile_mba.adl"
Expecting symbol ( or lower case identifier ?LC? or "I" or "V" or atom '' or ("-
" ...)*
Try inserting symbol lower case identifier ?LC?

==============================

before string "" at line 29, column 18 of file "testfile_mba.adl"
Expecting symbol ) or symbol [ or "!" or "#" or "-" or "/" or "/\\" or ";" or "<
>" or "\\" or "\\/" or ("*" or "+" or "~" ...)*
Try inserting symbol symbol )

==============================

before string "" at line 29, column 18 of file "testfile_mba.adl"
Expecting lower case identifier ?lc? or "-" or "/" or "/\\" or ";" or "<>" or "=
" or "CLASSIFY" or "CONCEPT" or "ENDCONTEXT" or "IDENT" or "INCLUDE" or "INTERFA
CE" or "KEY" or "META" or "PATTERN" or "PHPPLUG" or "POPULATION" or "PROCESS" or
 "PURPOSE" or "RELATION" or "RULE" or "SPEC" or "SQLPLUG" or "THEMES" or "VIEW"
or "VIOLATION" or "\\" or "\\/" or "|-" or ("*" or "+" or "~" ...)* or ("MEANING
" ...)* or ("MESSAGE" ...)*
Try inserting symbol "THEMES"

==============================

before upper case identifier Stef at line 29, column 30 of file "testfile_mba.ad
l"
Expecting symbol , or lower case identifier ?lc? or "CLASSIFY" or "CONCEPT" or "
ENDCONTEXT" or "IDENT" or "INCLUDE" or "INTERFACE" or "KEY" or "META" or "PATTER
N" or "PHPPLUG" or "POPULATION" or "PROCESS" or "PURPOSE" or "RELATION" or "RULE
" or "SPEC" or "SQLPLUG" or "THEMES" or "VIEW"
Try inserting symbol "PATTERN"

==============================

before string "" at line 29, column 34 of file "testfile_mba.adl"
Expecting "ENDPATTERN" or (lower case identifier ?lc? or "CLASSIFY" or "CONCEPT"
 or "IDENT" or "KEY" or "POPULATION" or "PURPOSE" or "RELATION" or "ROLE" or "RU
LE" or "SPEC" or "VIEW" ...)*
Try inserting symbol "ENDPATTERN"

==============================

before string "" at line 29, column 34 of file "testfile_mba.adl"
Expecting lower case identifier ?lc? or "CLASSIFY" or "CONCEPT" or "ENDCONTEXT"
or "IDENT" or "INCLUDE" or "INTERFACE" or "KEY" or "META" or "PATTERN" or "PHPPL
UG" or "POPULATION" or "PROCESS" or "PURPOSE" or "RELATION" or "RULE" or "SPEC"
or "SQLPLUG" or "THEMES" or "VIEW"
Try inserting symbol "RULE"

==============================

before symbol ) at line 29, column 36 of file "testfile_mba.adl"
Expecting ":"
Try inserting symbol ":"

==============================

before symbol ) at line 29, column 36 of file "testfile_mba.adl"
Expecting symbol ( or lower case identifier ?LC? or "I" or "V" or atom '' or ("-
" ...)*
Try deleting symbol symbol ) at line 29, column 36 of file "testfile_mba.adl"

==============================

before symbol ] at line 30, column 5 of file "testfile_mba.adl"
Expecting symbol ( or lower case identifier ?LC? or "I" or "V" or atom '' or ("-
" ...)*
Try deleting symbol symbol ] at line 30, column 5 of file "testfile_mba.adl"

==============================

before error in scanner: Unterminated string literal at line 30, column 6 of fil
e "testfile_mba.adl"
Expecting symbol ( or lower case identifier ?LC? or "I" or "V" or atom '' or ("-
" ...)*
Try deleting symbol error in scanner: Unterminated string literal at line 30, co
lumn 6 of file "testfile_mba.adl"

==============================

before "::" at line 34, column 15 of file "testfile_mba.adl"
Expecting symbol [ or lower case identifier ?lc? or "!" or "#" or "-" or "/" or
"/\\" or ";" or "<>" or "=" or "CLASSIFY" or "CONCEPT" or "ENDCONTEXT" or "IDENT
" or "INCLUDE" or "INTERFACE" or "KEY" or "META" or "PATTERN" or "PHPPLUG" or "P
OPULATION" or "PROCESS" or "PURPOSE" or "RELATION" or "RULE" or "SPEC" or "SQLPL
UG" or "THEMES" or "VIEW" or "VIOLATION" or "\\" or "\\/" or "|-" or ("*" or "+"
 or "~" ...)* or ("MEANING" ...)* or ("MESSAGE" ...)*
Try inserting symbol lower case identifier ?lc?
\end{haskell}
  \item[Previous evaluation]~\\
    \begin{itemize}
    \item \textbf{Accurate:} Good
    \item \textbf{Intuitive:} Acceptable
    \item \textbf{Succint:} Bad
    \item \textbf{Evaluation: Bad}
    \end{itemize}
  \item[New error]~\\
\begin{haskell}
PE Lexer error LexerError line 27:1, file ArchitectureAndDesign/Syntax/testfile_mba.adl (NonTerminatedChar (Just "clientName :: client [*-1..1] Name =\r"))\end{haskell}
  \item[New evaluation]~\\
    \begin{itemize}
    \item \textbf{Accurate:} Good
    \item \textbf{Intuitive:} Good
    \item \textbf{Succint:} Good
    \item \textbf{Evaluation: Good
}
    \end{itemize}
  \end{description}

\hrulefill

\subsection{Error type 17}
  \begin{description}
  \item[Incorrect ADL]~\\
\begin{adl}
RELATION relTest [ONE*ONE]]\end{adl}
  \item[Previous error]~\\
\begin{haskell}
Error(s) found:

before symbol ] at line 19, column 27 of file "testfile_mba.adl"
Expecting symbol [ or lower case identifier ?lc? or "." or "=" or "BYPLUG" or "C
LASSIFY" or "CONCEPT" or "ENDCONTEXT" or "IDENT" or "INCLUDE" or "INTERFACE" or
"KEY" or "META" or "PATTERN" or "PHPPLUG" or "POPULATION" or "PRAGMA" or "PROCES
S" or "PURPOSE" or "RELATION" or "RULE" or "SPEC" or "SQLPLUG" or "THEMES" or "V
IEW" or ("MEANING" ...)*
Try deleting symbol symbol ] at line 19, column 27 of file "testfile_mba.adl"\end{haskell}
  \item[Previous evaluation]~\\
    \begin{itemize}
    \item \textbf{Accurate:} Good
    \item \textbf{Intuitive:} Good
    \item \textbf{Succint:} Good
    \item \textbf{Evaluation: Good}
    \end{itemize}
  \item[New error]~\\
\begin{haskell}
PE "ArchitectureAndDesign/Syntax/testfile_mba.adl" (line 19, column 27):
unexpected Symbol ']'
expecting Keyword "META", Keyword "PATTERN", Keyword "PROCESS", Keyword "RULE", Keyword "CLASSIFY", Keyword "RELATION", Keyword "CONCEPT", Keyword "SPEC", Keyword "IDENT", Keyword "VIEW", Keyword "KEY", Keyword "INTERFACE", Keyword "SQLPLUG", Keyword "PHPPLUG", Keyword "PURPOSE", Keyword "POPULATION", Keyword "THEMES", Keyword "INCLUDE" or Keyword "ENDCONTEXT"
\end{haskell}
  \item[New evaluation]~\\
    \begin{itemize}
    \item \textbf{Accurate:} Good
    \item \textbf{Intuitive:} Good
    \item \textbf{Succint:} Good
    \item \textbf{Evaluation: Good
}
    \end{itemize}
  \end{description}

\hrulefill

\subsection{Error type 18}
  \begin{description}
  \item[Incorrect ADL]~\\
\begin{adl}
RELATION relTest [[ONE*ONE]\end{adl}
  \item[Previous error]~\\
\begin{haskell}
Error(s) found:

before symbol [ at line 19, column 19 of file "testfile_mba.adl"
Expecting upper case identifier ?uc? or "ONE" or string ""
Try deleting symbol symbol [ at line 19, column 19 of file "testfile_mba.adl"
\end{haskell}
  \item[Previous evaluation]~\\
    \begin{itemize}
    \item \textbf{Accurate:} Good
    \item \textbf{Intuitive:} Good
    \item \textbf{Succint:} Good
    \item \textbf{Evaluation: Good}
    \end{itemize}
  \item[New error]~\\
\begin{haskell}
PE "ArchitectureAndDesign/Syntax/testfile_mba.adl" (line 19, column 19):
unexpected Symbol '['
expecting Keyword "ONE"\end{haskell}
  \item[New evaluation]~\\
    \begin{itemize}
    \item \textbf{Accurate:} Good
    \item \textbf{Intuitive:} Good
    \item \textbf{Succint:} Good
    \item \textbf{Evaluation: Good
}
    \end{itemize}
  \end{description}

\hrulefill

\subsection{Error type 19}
  \begin{description}
  \item[Incorrect ADL]~\\
\begin{adl}
CONTEXT DeliverySimple IN ENGLISH REST2\end{adl}
  \item[Previous error]~\\
\begin{haskell}
Error(s) found:

before upper case identifier REST2 at line 1, column 35 of file "testfile_mba.ad
l"
Expecting "ENDCONTEXT" or "HTML" or "LATEX" or "MARKDOWN" or "REST" or (lower ca
se identifier ?lc? or "CLASSIFY" or "CONCEPT" or "IDENT" or "INCLUDE" or "INTERF
ACE" or "KEY" or "META" or "PATTERN" or "PHPPLUG" or "POPULATION" or "PROCESS" o
r "PURPOSE" or "RELATION" or "RULE" or "SPEC" or "SQLPLUG" or "THEMES" or "VIEW"
 ...)*
Try deleting symbol upper case identifier REST2 at line 1, column 35 of file "te
stfile_mba.adl"\end{haskell}
  \item[Previous evaluation]~\\
    \begin{itemize}
    \item \textbf{Accurate:} Good
    \item \textbf{Intuitive:} Good
    \item \textbf{Succint:} Good
    \item \textbf{Evaluation: Good}
    \end{itemize}
  \item[New error]~\\
\begin{haskell}
PE "ArchitectureAndDesign/Syntax/testfile_mba.adl" (line 1, column 35):
unexpected Upper case identifier REST2
expecting Keyword "REST", Keyword "HTML", Keyword "LATEX", Keyword "MARKDOWN", Keyword "META", Keyword "PATTERN", Keyword "PROCESS", Keyword "RULE", Keyword "CLASSIFY", Keyword "RELATION", Keyword "CONCEPT", Keyword "SPEC", Keyword "IDENT", Keyword "VIEW", Keyword "KEY", Keyword "INTERFACE", Keyword "SQLPLUG", Keyword "PHPPLUG", Keyword "PURPOSE", Keyword "POPULATION", Keyword "THEMES", Keyword "INCLUDE" or Keyword "ENDCONTEXT"
\end{haskell}
  \item[New evaluation]~\\
    \begin{itemize}
    \item \textbf{Accurate:} Good
    \item \textbf{Intuitive:} Acceptable
    \item \textbf{Succint:} Good
    \item \textbf{Evaluation: Good
}
    \end{itemize}
  \end{description}

\hrulefill

\subsection{Error type 20}
  \begin{description}
  \item[Incorrect ADL]~\\
\begin{adl}
CONTEXT DeliverySimple REST2 IN ENGLISH \end{adl}
  \item[Previous error]~\\
\begin{haskell}
Error(s) found:

before upper case identifier REST2 at line 1, column 24 of file "testfile_mba.ad
l"
Expecting "ENDCONTEXT" or "HTML" or "IN" or "LATEX" or "MARKDOWN" or "REST" or (
lower case identifier ?lc? or "CLASSIFY" or "CONCEPT" or "IDENT" or "INCLUDE" or
 "INTERFACE" or "KEY" or "META" or "PATTERN" or "PHPPLUG" or "POPULATION" or "PR
OCESS" or "PURPOSE" or "RELATION" or "RULE" or "SPEC" or "SQLPLUG" or "THEMES" o
r "VIEW" ...)*
Try deleting symbol upper case identifier REST2 at line 1, column 24 of file "te
stfile_mba.adl"
\end{haskell}
  \item[Previous evaluation]~\\
    \begin{itemize}
    \item \textbf{Accurate:} Good
    \item \textbf{Intuitive:} Acceptable
    \item \textbf{Succint:} Good
    \item \textbf{Evaluation: Acceptable}
    \end{itemize}
  \item[New error]~\\
\begin{haskell}
PE "ArchitectureAndDesign/Syntax/testfile_mba.adl" (line 1, column 24):
unexpected Upper case identifier REST2
expecting Keyword "IN"
\end{haskell}
  \item[New evaluation]~\\
    \begin{itemize}
    \item \textbf{Accurate:} Good
    \item \textbf{Intuitive:} Good
    \item \textbf{Succint:} Good
    \item \textbf{Evaluation: Good
}
    \end{itemize}
  \end{description}

\hrulefill

\subsection{Error type 21}
  \begin{description}
  \item[Incorrect ADL]~\\
\begin{adl}
CONTEXT DeliverySimple IN ENGLISHD\end{adl}
  \item[Previous error]~\\
\begin{haskell}
Error(s) found:

before upper case identifier ENGLISHD at line 1, column 27 of file "testfile_mba
.adl"
Expecting "DUTCH" or "ENGLISH"
Try deleting symbol upper case identifier ENGLISHD at line 1, column 27 of file
"testfile_mba.adl"

==============================

before lower case identifier clientName at line 14, column 1 of file "testfile_m
ba.adl"
Expecting "DUTCH" or "ENGLISH"
Try inserting symbol "DUTCH"\end{haskell}
  \item[Previous evaluation]~\\
    \begin{itemize}
    \item \textbf{Accurate:} Good
    \item \textbf{Intuitive:} Good
    \item \textbf{Succint:} Acceptable
    \item \textbf{Evaluation: Acceptable}
    \end{itemize}
  \item[New error]~\\
\begin{haskell}
PE "ArchitectureAndDesign/Syntax/testfile_mba.adl" (line 1, column 27):
unexpected Upper case identifier ENGLISHD
expecting Keyword "DUTCH" or Keyword "ENGLISH"\end{haskell}
  \item[New evaluation]~\\
    \begin{itemize}
    \item \textbf{Accurate:} Good
    \item \textbf{Intuitive:} Good
    \item \textbf{Succint:} Good
    \item \textbf{Evaluation: Good
}
    \end{itemize}
  \end{description}

\hrulefill

\subsection{Error type 22}
  \begin{description}
  \item[Incorrect ADL]~\\
\begin{adl}
CONTEXT DeliverySimple in ENGLISH\end{adl}
  \item[Previous error]~\\
\begin{haskell}
Error(s) found:

before "ENGLISH" at line 1, column 27 of file "testfile_mba.adl"
Expecting "::"
Try deleting symbol "ENGLISH" at line 1, column 27 of file "testfile_mba.adl"

==============================

before lower case identifier clientName at line 14, column 1 of file "testfile_m
ba.adl"
Expecting "::"
Try deleting symbol lower case identifier clientName at line 14, column 1 of fil
e "testfile_mba.adl"\end{haskell}
  \item[Previous evaluation]~\\
    \begin{itemize}
    \item \textbf{Accurate:} Good
    \item \textbf{Intuitive:} Acceptable
    \item \textbf{Succint:} Acceptable
    \item \textbf{Evaluation: Acceptable}
    \end{itemize}
  \item[New error]~\\
\begin{haskell}
PE "ArchitectureAndDesign/Syntax/testfile_mba.adl" (line 1, column 24):
unexpected Lower case identifier in
expecting Keyword "IN"\end{haskell}
  \item[New evaluation]~\\
    \begin{itemize}
    \item \textbf{Accurate:} Good
    \item \textbf{Intuitive:} Good
    \item \textbf{Succint:} Good
    \item \textbf{Evaluation: Good
}
    \end{itemize}
  \end{description}

\hrulefill

\subsection{Error type 23}
  \begin{description}
  \item[Incorrect ADL]~\\
\begin{adl}
CONTEXT deliverySimple IN ENGLISH\end{adl}
  \item[Previous error]~\\
\begin{haskell}
Error(s) found:

before lower case identifier deliverySimple at line 1, column 9 of file "testfil
e_mba.adl"
Expecting upper case identifier ?uc? or string ""
Try deleting symbol lower case identifier deliverySimple at line 1, column 9 of
file "testfile_mba.adl"

==============================

before "IN" at line 1, column 24 of file "testfile_mba.adl"
Expecting upper case identifier ?uc? or string ""
Try inserting symbol upper case identifier ?uc?\end{haskell}
  \item[Previous evaluation]~\\
    \begin{itemize}
    \item \textbf{Accurate:} Good
    \item \textbf{Intuitive:} Good
    \item \textbf{Succint:} Acceptable
    \item \textbf{Evaluation: Acceptable}
    \end{itemize}
  \item[New error]~\\
\begin{haskell}
PE "ArchitectureAndDesign/Syntax/testfile_mba.adl" (line 1, column 9):
unexpected Lower case identifier deliverySimple\end{haskell}
  \item[New evaluation]~\\
    \begin{itemize}
    \item \textbf{Accurate:} Good
    \item \textbf{Intuitive:} Good
    \item \textbf{Succint:} Good
    \item \textbf{Evaluation: Good
}
    \end{itemize}
  \end{description}

\hrulefill

\subsection{Error type 24}
  \begin{description}
  \item[Incorrect ADL]~\\
\begin{adl}
CONTEXT DeliverySimple INCLUDE 0FILEPATH IN ENGLISH\end{adl}
  \item[Previous error]~\\
\begin{haskell}
Error(s) found:

before "0" at line 1, column 32 of file "testfile_mba.adl"
Expecting string ""
Try deleting symbol "0" at line 1, column 32 of file "testfile_mba.adl"

==============================

before upper case identifier FILEPATH at line 1, column 33 of file "testfile_mba
.adl"
Expecting string ""
Try inserting symbol string ""

==============================

before upper case identifier FILEPATH at line 1, column 33 of file "testfile_mba
.adl"
Expecting lower case identifier ?lc? or "CLASSIFY" or "CONCEPT" or "ENDCONTEXT"
or "IDENT" or "INCLUDE" or "INTERFACE" or "KEY" or "META" or "PATTERN" or "PHPPL
UG" or "POPULATION" or "PROCESS" or "PURPOSE" or "RELATION" or "RULE" or "SPEC"
or "SQLPLUG" or "THEMES" or "VIEW"
Try inserting symbol "PURPOSE"

==============================

before upper case identifier FILEPATH at line 1, column 33 of file "testfile_mba
.adl"
Expecting "CONCEPT" or "CONTEXT" or "IDENT" or "INTERFACE" or "PATTERN" or "PROC
ESS" or "RELATION" or "RULE" or "VIEW"
Try inserting symbol "CONCEPT"

==============================

before lower case identifier clientName at line 14, column 1 of file "testfile_m
ba.adl"
Expecting "HTML" or "LATEX" or "MARKDOWN" or "REF" or "REST" or explanation {+-}

Try inserting symbol explanation {+-}

\end{haskell}
  \item[Previous evaluation]~\\
    \begin{itemize}
    \item \textbf{Accurate:} Good
    \item \textbf{Intuitive:} Bad
    \item \textbf{Succint:} Bad
    \item \textbf{Evaluation: Bad}
    \end{itemize}
  \item[New error]~\\
\begin{haskell}
PE "ArchitectureAndDesign/Syntax/testfile_mba.adl" (line 1, column 24):
unexpected Keyword "INCLUDE"
expecting Keyword "IN"\end{haskell}
  \item[New evaluation]~\\
    \begin{itemize}
    \item \textbf{Accurate:} Good
    \item \textbf{Intuitive:} Good
    \item \textbf{Succint:} Good
    \item \textbf{Evaluation: Good
}
    \end{itemize}
  \end{description}

\hrulefill

\subsection{Error type 25}
  \begin{description}
  \item[Incorrect ADL]~\\
\begin{adl}
CONTEXT DeliverySimple IN ENGLISH INCLUDE 0FILEPATH\end{adl}
  \item[Previous error]~\\
\begin{haskell}
Error(s) found:

before "0" at line 1, column 43 of file "testfile_mba.adl"
Expecting string ""
Try deleting symbol "0" at line 1, column 43 of file "testfile_mba.adl"

==============================

before upper case identifier FILEPATH at line 1, column 44 of file "testfile_mba
.adl"
Expecting string ""
Try deleting symbol upper case identifier FILEPATH at line 1, column 44 of file
"testfile_mba.adl"

==============================

before lower case identifier clientName at line 14, column 1 of file "testfile_m
ba.adl"
Expecting string ""
Try inserting symbol string ""
\end{haskell}
  \item[Previous evaluation]~\\
    \begin{itemize}
    \item \textbf{Accurate:} Good
    \item \textbf{Intuitive:} Acceptable
    \item \textbf{Succint:} Bad
    \item \textbf{Evaluation: Bad}
    \end{itemize}
  \item[New error]~\\
\begin{haskell}
PE "ArchitectureAndDesign/Syntax/testfile_mba.adl" (line 1, column 43):
unexpected Integer 0\end{haskell}
  \item[New evaluation]~\\
    \begin{itemize}
    \item \textbf{Accurate:} Good
    \item \textbf{Intuitive:} Good
    \item \textbf{Succint:} Good
    \item \textbf{Evaluation: Good
}
    \end{itemize}
  \end{description}

\hrulefill

\subsection{Error type 26}
  \begin{description}
  \item[Incorrect ADL]~\\
\begin{adl}
CONTEXT

ENDCONTEXT3\end{adl}
  \item[Previous error]~\\
\begin{haskell}
Error(s) found:

before upper case identifier ENDCONTEXT3 at line 237, column 1 of file "testfile
_mba.adl"
Expecting lower case identifier ?lc? or "CLASSIFY" or "CONCEPT" or "ENDCONTEXT"
or "IDENT" or "INCLUDE" or "INTERFACE" or "KEY" or "META" or "PATTERN" or "PHPPL
UG" or "POPULATION" or "PROCESS" or "PURPOSE" or "RELATION" or "RULE" or "SPEC"
or "SQLPLUG" or "THEMES" or "VIEW"
Try deleting symbol upper case identifier ENDCONTEXT3 at line 237, column 1 of f
ile "testfile_mba.adl"

==============================

unexpected end of input
Expecting "ENDCONTEXT"
Try inserting symbol "ENDCONTEXT"
\end{haskell}
  \item[Previous evaluation]~\\
    \begin{itemize}
    \item \textbf{Accurate:} Good
    \item \textbf{Intuitive:} Good
    \item \textbf{Succint:} Acceptable
    \item \textbf{Evaluation: Acceptable}
    \end{itemize}
  \item[New error]~\\
\begin{haskell}
PE "ArchitectureAndDesign/Syntax/testfile_mba.adl" (line 237, column 1):
unexpected Upper case identifier ENDCONTEXT3
expecting Keyword "META", Keyword "PATTERN", Keyword "PROCESS", Keyword "RULE", Keyword "CLASSIFY", Keyword "RELATION", Keyword "CONCEPT", Keyword "SPEC", Keyword "IDENT", Keyword "VIEW", Keyword "KEY", Keyword "INTERFACE", Keyword "SQLPLUG", Keyword "PHPPLUG", Keyword "PURPOSE", Keyword "POPULATION", Keyword "THEMES", Keyword "INCLUDE" or Keyword "ENDCONTEXT"
\end{haskell}
  \item[New evaluation]~\\
    \begin{itemize}
    \item \textbf{Accurate:} Good
    \item \textbf{Intuitive:} Good
    \item \textbf{Succint:} Good
    \item \textbf{Evaluation: Good
}
    \end{itemize}
  \end{description}

\hrulefill

\subsection{Error type 27}
  \begin{description}
  \item[Incorrect ADL]~\\
\begin{adl}
META "authors" "Jan" "Paul"\end{adl}
  \item[Previous error]~\\
\begin{haskell}
Error(s) found:

before string "Paul" at line 11, column 22 of file "testfile_mba.adl"
Expecting lower case identifier ?lc? or "CLASSIFY" or "CONCEPT" or "ENDCONTEXT"
or "IDENT" or "INCLUDE" or "INTERFACE" or "KEY" or "META" or "PATTERN" or "PHPPL
UG" or "POPULATION" or "PROCESS" or "PURPOSE" or "RELATION" or "RULE" or "SPEC"
or "SQLPLUG" or "THEMES" or "VIEW"
Try deleting symbol string "Paul" at line 11, column 22 of file "testfile_mba.ad
l"

\end{haskell}
  \item[Previous evaluation]~\\
    \begin{itemize}
    \item \textbf{Accurate:} Good
    \item \textbf{Intuitive:} Good
    \item \textbf{Succint:} Good
    \item \textbf{Evaluation: Good}
    \end{itemize}
  \item[New error]~\\
\begin{haskell}
PE "ArchitectureAndDesign/Syntax/testfile_mba.adl" (line 11, column 22):
unexpected String "Paul"
expecting Keyword "META", Keyword "PATTERN", Keyword "PROCESS", Keyword "RULE", Keyword "CLASSIFY", Keyword "RELATION", Keyword "CONCEPT", Keyword "SPEC", Keyword "IDENT", Keyword "VIEW", Keyword "KEY", Keyword "INTERFACE", Keyword "SQLPLUG", Keyword "PHPPLUG", Keyword "PURPOSE", Keyword "POPULATION", Keyword "THEMES", Keyword "INCLUDE" or Keyword "ENDCONTEXT"
\end{haskell}
  \item[New evaluation]~\\
    \begin{itemize}
    \item \textbf{Accurate:} Good
    \item \textbf{Intuitive:} Good
    \item \textbf{Succint:} Good
    \item \textbf{Evaluation: Good
}
    \end{itemize}
  \end{description}

\hrulefill

\subsection{Error type 28}
  \begin{description}
  \item[Incorrect ADL]~\\
\begin{adl}
META "authors" \end{adl}
  \item[Previous error]~\\
\begin{haskell}
Error(s) found:

before lower case identifier clientName at line 17, column 1 of file "testfile_m
ba.adl"
Expecting string ""
Try inserting symbol string ""
\end{haskell}
  \item[Previous evaluation]~\\
    \begin{itemize}
    \item \textbf{Accurate:} Good
    \item \textbf{Intuitive:} Acceptable
    \item \textbf{Succint:} Good
    \item \textbf{Evaluation: Acceptable}
    \end{itemize}
  \item[New error]~\\
\begin{haskell}
PE "ArchitectureAndDesign/Syntax/testfile_mba.adl" (line 17, column 1):
unexpected Lower case identifier clientName\end{haskell}
  \item[New evaluation]~\\
    \begin{itemize}
    \item \textbf{Accurate:} Good
    \item \textbf{Intuitive:} Bad
    \item \textbf{Succint:} Good
    \item \textbf{Evaluation: Bad
}
    \end{itemize}
  \end{description}

\hrulefill

\subsection{Error type 29}
  \begin{description}
  \item[Incorrect ADL]~\\
\begin{adl}
METZA "authors"  "Jan"\end{adl}
  \item[Previous error]~\\
\begin{haskell}
Error(s) found:

before upper case identifier METZA at line 11, column 1 of file "testfile_mba.ad
l"
Expecting "ENDCONTEXT" or "HTML" or "LATEX" or "MARKDOWN" or "REST" or (lower ca
se identifier ?lc? or "CLASSIFY" or "CONCEPT" or "IDENT" or "INCLUDE" or "INTERF
ACE" or "KEY" or "META" or "PATTERN" or "PHPPLUG" or "POPULATION" or "PROCESS" o
r "PURPOSE" or "RELATION" or "RULE" or "SPEC" or "SQLPLUG" or "THEMES" or "VIEW"
 ...)*
Try inserting symbol "CONCEPT"\end{haskell}
  \item[Previous evaluation]~\\
    \begin{itemize}
    \item \textbf{Accurate:} Good
    \item \textbf{Intuitive:} Good
    \item \textbf{Succint:} Good
    \item \textbf{Evaluation: Good}
    \end{itemize}
  \item[New error]~\\
\begin{haskell}
PE "ArchitectureAndDesign/Syntax/testfile_mba.adl" (line 11, column 1):
unexpected Upper case identifier METZA
expecting Keyword "REST", Keyword "HTML", Keyword "LATEX", Keyword "MARKDOWN", Keyword "META", Keyword "PATTERN", Keyword "PROCESS", Keyword "RULE", Keyword "CLASSIFY", Keyword "RELATION", Keyword "CONCEPT", Keyword "SPEC", Keyword "IDENT", Keyword "VIEW", Keyword "KEY", Keyword "INTERFACE", Keyword "SQLPLUG", Keyword "PHPPLUG", Keyword "PURPOSE", Keyword "POPULATION", Keyword "THEMES", Keyword "INCLUDE" or Keyword "ENDCONTEXT"
\end{haskell}
  \item[New evaluation]~\\
    \begin{itemize}
    \item \textbf{Accurate:} Good
    \item \textbf{Intuitive:} Good
    \item \textbf{Succint:} Good
    \item \textbf{Evaluation: Good
}
    \end{itemize}
  \end{description}

\hrulefill

\subsection{Error type 30}
  \begin{description}
  \item[Incorrect ADL]~\\
\begin{adl}
METZA "authors"  "Jan""\end{adl}
  \item[Previous error]~\\
\begin{haskell}
Error(s) found:

before error in scanner: Unterminated string literal at line 11, column 22 of fi
le "testfile_mba.adl"
Expecting lower case identifier ?lc? or "CLASSIFY" or "CONCEPT" or "ENDCONTEXT"
or "IDENT" or "INCLUDE" or "INTERFACE" or "KEY" or "META" or "PATTERN" or "PHPPL
UG" or "POPULATION" or "PROCESS" or "PURPOSE" or "RELATION" or "RULE" or "SPEC"
or "SQLPLUG" or "THEMES" or "VIEW"
Try deleting symbol error in scanner: Unterminated string literal at line 11, co
lumn 22 of file "testfile_mba.adl"before error in scanner: Unterminated string literal at line 11, column 19 of fi
le "testfile_mba.adl"
Expecting lower case identifier ?lc? or "CLASSIFY" or "CONCEPT" or "ENDCONTEXT"
or "IDENT" or "INCLUDE" or "INTERFACE" or "KEY" or "META" or "PATTERN" or "PHPPL
UG" or "POPULATION" or "PROCESS" or "PURPOSE" or "RELATION" or "RULE" or "SPEC"
or "SQLPLUG" or "THEMES" or "VIEW"
Try deleting symbol error in scanner: Unterminated string literal at line 11, co
lumn 19 of file "testfile_mba.adl"
\end{haskell}
  \item[Previous evaluation]~\\
    \begin{itemize}
    \item \textbf{Accurate:} Good
    \item \textbf{Intuitive:} Acceptable
    \item \textbf{Succint:} Good
    \item \textbf{Evaluation: Acceptable}
    \end{itemize}
  \item[New error]~\\
\begin{haskell}
PE Lexer error LexerError line 11:22, file ArchitectureAndDesign/Syntax/testfile_mba.adl (NonTerminatedChar (Just "\r"))\end{haskell}
  \item[New evaluation]~\\
    \begin{itemize}
    \item \textbf{Accurate:} Good
    \item \textbf{Intuitive:} Acceptable
    \item \textbf{Succint:} Good
    \item \textbf{Evaluation: Acceptable
}
    \end{itemize}
  \end{description}

\hrulefill

\subsection{Error type 31}
  \begin{description}
  \item[Incorrect ADL]~\\
\begin{adl}
PATTEZRN Arbeidsduur

--[Arbeidsrelaties]--
werkgever :: Arbeidsrelatie -> Persoon
PRAGMA "In" "is" "de werkgever"
MEANING "Elke arbeidsrelatie benoemt expliciet welke (rechts)persoon de rol van werkgever vervult."
PURPOSE RELATION werkgever
{+Om de werkgever te kunnen bepalen gaan we ervan uit dat elke arbeidsrelatie precies een werkgever heeft.-}

ENDPATTERN\end{adl}
  \item[Previous error]~\\
\begin{haskell}
Error(s) found:

before upper case identifier PATTEZRN at line 10, column 1 of file "testfile_mba
.adl"
Expecting "ENDCONTEXT" or "HTML" or "LATEX" or "MARKDOWN" or "REST" or (lower ca
se identifier ?lc? or "CLASSIFY" or "CONCEPT" or "IDENT" or "INCLUDE" or "INTERF
ACE" or "KEY" or "META" or "PATTERN" or "PHPPLUG" or "POPULATION" or "PROCESS" o
r "PURPOSE" or "RELATION" or "RULE" or "SPEC" or "SQLPLUG" or "THEMES" or "VIEW"
 ...)*
Try deleting symbol upper case identifier PATTEZRN at line 10, column 1 of file
"testfile_mba.adl"

==============================

before upper case identifier Arbeidsduur at line 10, column 10 of file "testfile
_mba.adl"
Expecting "ENDCONTEXT" or (lower case identifier ?lc? or "CLASSIFY" or "CONCEPT"
 or "IDENT" or "INCLUDE" or "INTERFACE" or "KEY" or "META" or "PATTERN" or "PHPP
LUG" or "POPULATION" or "PROCESS" or "PURPOSE" or "RELATION" or "RULE" or "SPEC"
 or "SQLPLUG" or "THEMES" or "VIEW" ...)*
Try deleting symbol upper case identifier Arbeidsduur at line 10, column 10 of f
ile "testfile_mba.adl"

==============================

before "ENDPATTERN" at line 19, column 1 of file "testfile_mba.adl"
Expecting lower case identifier ?lc? or "CLASSIFY" or "CONCEPT" or "ENDCONTEXT"
or "IDENT" or "INCLUDE" or "INTERFACE" or "KEY" or "META" or "PATTERN" or "PHPPL
UG" or "POPULATION" or "PROCESS" or "PURPOSE" or "RELATION" or "RULE" or "SPEC"
or "SQLPLUG" or "THEMES" or "VIEW"
Try deleting symbol "ENDPATTERN" at line 19, column 1 of file "testfile_mba.adl"

\end{haskell}
  \item[Previous evaluation]~\\
    \begin{itemize}
    \item \textbf{Accurate:} Good
    \item \textbf{Intuitive:} Good
    \item \textbf{Succint:} Acceptable
    \item \textbf{Evaluation: Acceptable}
    \end{itemize}
  \item[New error]~\\
\begin{haskell}
PE "ArchitectureAndDesign/Syntax/testfile_mba.adl" (line 10, column 1):
unexpected Upper case identifier PATTEZRN
expecting Keyword "REST", Keyword "HTML", Keyword "LATEX", Keyword "MARKDOWN", Keyword "META", Keyword "PATTERN", Keyword "PROCESS", Keyword "RULE", Keyword "CLASSIFY", Keyword "RELATION", Keyword "CONCEPT", Keyword "SPEC", Keyword "IDENT", Keyword "VIEW", Keyword "KEY", Keyword "INTERFACE", Keyword "SQLPLUG", Keyword "PHPPLUG", Keyword "PURPOSE", Keyword "POPULATION", Keyword "THEMES", Keyword "INCLUDE" or Keyword "ENDCONTEXT"
\end{haskell}
  \item[New evaluation]~\\
    \begin{itemize}
    \item \textbf{Accurate:} Good
    \item \textbf{Intuitive:} Good
    \item \textbf{Succint:} Good
    \item \textbf{Evaluation: Good
}
    \end{itemize}
  \end{description}

\hrulefill

\subsection{Error type 32}
  \begin{description}
  \item[Incorrect ADL]~\\
\begin{adl}
BYPLUG [UNIS] BYPLUG\end{adl}
  \item[Previous error]~\\
\begin{haskell}
Error(s) found:

before upper case identifier UNIS at line 14, column 9 of file "testfile_mba.adl
"
Expecting symbol ] or ("ASY" or "INJ" or "IRF" or "PROP" or "RFX" or "SUR" or "S
YM" or "TOT" or "TRN" or "UNI" ...)*
Try deleting symbol upper case identifier UNIS at line 14, column 9 of file "tes
tfile_mba.adl"
\end{haskell}
  \item[Previous evaluation]~\\
    \begin{itemize}
    \item \textbf{Accurate:} Good
    \item \textbf{Intuitive:} Good
    \item \textbf{Succint:} Good
    \item \textbf{Evaluation: Good}
    \end{itemize}
  \item[New error]~\\
\begin{haskell}
PE "ArchitectureAndDesign/Syntax/testfile_mba.adl" (line 14, column 9):
unexpected Upper case identifier UNIS
expecting Keyword "UNI", Keyword "INJ", Keyword "SUR", Keyword "TOT", Keyword "SYM", Keyword "ASY", Keyword "TRN", Keyword "RFX", Keyword "IRF", Keyword "AUT", Keyword "PROP" or Symbol ']'
\end{haskell}
  \item[New evaluation]~\\
    \begin{itemize}
    \item \textbf{Accurate:} Good
    \item \textbf{Intuitive:} Good
    \item \textbf{Succint:} Good
    \item \textbf{Evaluation: Good
}
    \end{itemize}
  \end{description}

\hrulefill

\subsection{Error type 33}
  \begin{description}
  \item[Incorrect ADL]~\\
\begin{adl}
BYPLUG [UNI, TRN, ] BYPLUG\end{adl}
  \item[Previous error]~\\
\begin{haskell}
Error(s) found:

before symbol ] at line 14, column 19 of file "testfile_mba.adl"
Expecting "ASY" or "INJ" or "IRF" or "PROP" or "RFX" or "SUR" or "SYM" or "TOT"
or "TRN" or "UNI"
Try inserting symbol "UNI"
\end{haskell}
  \item[Previous evaluation]~\\
    \begin{itemize}
    \item \textbf{Accurate:} Good
    \item \textbf{Intuitive:} Acceptable
    \item \textbf{Succint:} Good
    \item \textbf{Evaluation: Acceptable}
    \end{itemize}
  \item[New error]~\\
\begin{haskell}
PE "ArchitectureAndDesign/Syntax/testfile_mba.adl" (line 14, column 19):
unexpected Symbol ']'
expecting Keyword "UNI", Keyword "INJ", Keyword "SUR", Keyword "TOT", Keyword "SYM", Keyword "ASY", Keyword "TRN", Keyword "RFX", Keyword "IRF", Keyword "AUT" or Keyword "PROP"
\end{haskell}
  \item[New evaluation]~\\
    \begin{itemize}
    \item \textbf{Accurate:} Good
    \item \textbf{Intuitive:} Good
    \item \textbf{Succint:} Good
    \item \textbf{Evaluation: Good
}
    \end{itemize}
  \end{description}

\hrulefill

\subsection{Error type 34}
  \begin{description}
  \item[Incorrect ADL]~\\
\begin{adl}
BYPLUG [UNI, TRN] BYPLUGS\end{adl}
  \item[Previous error]~\\
\begin{haskell}
Error(s) found:

before upper case identifier BYPLUGS at line 14, column 19 of file "testfile_mba
.adl"
Expecting lower case identifier ?lc? or "." or "=" or "BYPLUG" or "CLASSIFY" or
"CONCEPT" or "ENDPATTERN" or "IDENT" or "KEY" or "POPULATION" or "PRAGMA" or "PU
RPOSE" or "RELATION" or "ROLE" or "RULE" or "SPEC" or "VIEW" or ("MEANING" ...)*

Try deleting symbol upper case identifier BYPLUGS at line 14, column 19 of file
"testfile_mba.adl"\end{haskell}
  \item[Previous evaluation]~\\
    \begin{itemize}
    \item \textbf{Accurate:} Good
    \item \textbf{Intuitive:} Good
    \item \textbf{Succint:} Good
    \item \textbf{Evaluation: Good}
    \end{itemize}
  \item[New error]~\\
\begin{haskell}
PE "ArchitectureAndDesign/Syntax/testfile_mba.adl" (line 14, column 19):
unexpected Upper case identifier BYPLUGS
expecting Keyword "RULE", Keyword "CLASSIFY", Keyword "RELATION", Keyword "CONCEPT", Keyword "SPEC", Keyword "IDENT", Keyword "VIEW", Keyword "KEY", Keyword "PURPOSE", Keyword "POPULATION" or Keyword "ENDPATTERN"\end{haskell}
  \item[New evaluation]~\\
    \begin{itemize}
    \item \textbf{Accurate:} Good
    \item \textbf{Intuitive:} Good
    \item \textbf{Succint:} Good
    \item \textbf{Evaluation: Good
}
    \end{itemize}
  \end{description}

\hrulefill

\subsection{Error type 35}
  \begin{description}
  \item[Incorrect ADL]~\\
\begin{adl}
PRAGMA "In" d\end{adl}
  \item[Previous error]~\\
\begin{haskell}
Error(s) found:

before "MEANING" at line 16, column 1 of file "testfile_mba.adl"
Expecting "::"
Try deleting symbol "MEANING" at line 16, column 1 of file "testfile_mba.adl"

==============================

before string "Elke arbeidsrelatie benoemt expliciet welke (rechts)persoon de ro
l van werkgever vervult." at line 16, column 9 of file "testfile_mba.adl"
Expecting "::"
Try inserting symbol "::"

==============================

before "PURPOSE" at line 17, column 1 of file "testfile_mba.adl"
Expecting symbol [ or "*" or "->" or "<-"
Try inserting symbol "*"

==============================

before "PURPOSE" at line 17, column 1 of file "testfile_mba.adl"
Expecting upper case identifier ?uc? or string ""
Try inserting symbol upper case identifier ?uc?\end{haskell}
  \item[Previous evaluation]~\\
    \begin{itemize}
    \item \textbf{Accurate:} Good
    \item \textbf{Intuitive:} Good
    \item \textbf{Succint:} Bad
    \item \textbf{Evaluation: Bad}
    \end{itemize}
  \item[New error]~\\
\begin{haskell}
PE "ArchitectureAndDesign/Syntax/testfile_mba.adl" (line 16, column 1):
unexpected Keyword "MEANING"
expecting Operator '::'
\end{haskell}
  \item[New evaluation]~\\
    \begin{itemize}
    \item \textbf{Accurate:} Good
    \item \textbf{Intuitive:} Good
    \item \textbf{Succint:} Good
    \item \textbf{Evaluation: Good
}
    \end{itemize}
  \end{description}

\hrulefill

\subsection{Error type 36}
  \begin{description}
  \item[Incorrect ADL]~\\
\begin{adl}
PRAGMQA "In"\end{adl}
  \item[Previous error]~\\
\begin{haskell}
Error(s) found:

before upper case identifier PRAGMQA at line 15, column 1 of file "testfile_mba.
adl"
Expecting lower case identifier ?lc? or "." or "=" or "CLASSIFY" or "CONCEPT" or
 "ENDPATTERN" or "IDENT" or "KEY" or "POPULATION" or "PRAGMA" or "PURPOSE" or "R
ELATION" or "ROLE" or "RULE" or "SPEC" or "VIEW" or ("MEANING" ...)*
Try inserting symbol "CONCEPT"

==============================

before "MEANING" at line 16, column 1 of file "testfile_mba.adl"
Expecting "::"
Try deleting symbol "MEANING" at line 16, column 1 of file "testfile_mba.adl"

==============================

before string "Elke arbeidsrelatie benoemt expliciet welke (rechts)persoon de ro
l van werkgever vervult." at line 16, column 9 of file "testfile_mba.adl"
Expecting "::"
Try inserting symbol "::"

==============================

before "PURPOSE" at line 17, column 1 of file "testfile_mba.adl"
Expecting symbol [ or "*" or "->" or "<-"
Try inserting symbol "*"

==============================

before "PURPOSE" at line 17, column 1 of file "testfile_mba.adl"
Expecting upper case identifier ?uc? or string ""
Try inserting symbol upper case identifier ?uc?\end{haskell}
  \item[Previous evaluation]~\\
    \begin{itemize}
    \item \textbf{Accurate:} Good
    \item \textbf{Intuitive:} Good
    \item \textbf{Succint:} Bad
    \item \textbf{Evaluation: Bad}
    \end{itemize}
  \item[New error]~\\
\begin{haskell}
PE "ArchitectureAndDesign/Syntax/testfile_mba.adl" (line 15, column 1):
unexpected Upper case identifier PRAGMQA
expecting Keyword "RULE", Keyword "CLASSIFY", Keyword "RELATION", Keyword "CONCEPT", Keyword "SPEC", Keyword "IDENT", Keyword "VIEW", Keyword "KEY", Keyword "PURPOSE", Keyword "POPULATION" or Keyword "ENDPATTERN"\end{haskell}
  \item[New evaluation]~\\
    \begin{itemize}
    \item \textbf{Accurate:} Good
    \item \textbf{Intuitive:} Good
    \item \textbf{Succint:} Good
    \item \textbf{Evaluation: Good
}
    \end{itemize}
  \end{description}

\hrulefill

\subsection{Error type 37}
  \begin{description}
  \item[Incorrect ADL]~\\
\begin{adl}
PURPOSE RELATION werkgever
--{+Om de werkgever te kunnen bepalen gaan we ervan uit dat elke arbeidsrelatie precies een werkgever heeft.-}\end{adl}
  \item[Previous error]~\\
\begin{haskell}
Error(s) found:

before "ENDPATTERN" at line 22, column 1 of file "testfile_mba.adl"
Expecting symbol [ or "HTML" or "IN" or "LATEX" or "MARKDOWN" or "REF" or "REST"
 or explanation {+-}
Try inserting symbol explanation {+-}
\end{haskell}
  \item[Previous evaluation]~\\
    \begin{itemize}
    \item \textbf{Accurate:} Good
    \item \textbf{Intuitive:} Acceptable
    \item \textbf{Succint:} Good
    \item \textbf{Evaluation: Acceptable}
    \end{itemize}
  \item[New error]~\\
\begin{haskell}
PE "ArchitectureAndDesign/Syntax/testfile_mba.adl" (line 22, column 1):
unexpected Keyword "ENDPATTERN"
expecting Symbol '[', Keyword "IN", Keyword "REST", Keyword "HTML", Keyword "LATEX", Keyword "MARKDOWN" or Keyword "REF"\end{haskell}
  \item[New evaluation]~\\
    \begin{itemize}
    \item \textbf{Accurate:} Good
    \item \textbf{Intuitive:} Acceptable
    \item \textbf{Succint:} Good
    \item \textbf{Evaluation: Acceptable
}
    \end{itemize}
  \end{description}

\hrulefill

\subsection{Error type 38}
  \begin{description}
  \item[Incorrect ADL]~\\
\begin{adl}
MEANING IN ENGLISH "Elke arbeidsrelatie benoemt expliciet welke (rechts)persoon de rol van werkgever vervult."
= [abc]\end{adl}
  \item[Previous error]~\\
\begin{haskell}
Error(s) found:

before symbol ] at line 17, column 7 of file "testfile_mba.adl"
Expecting "*"
Try inserting symbol "*"

==============================

before symbol ] at line 17, column 7 of file "testfile_mba.adl"
Expecting lower case identifier ?lc? or upper case identifier ?uc? or atom '' or
 decimal Integer 1
Try inserting symbol atom ''
\end{haskell}
  \item[Previous evaluation]~\\
    \begin{itemize}
    \item \textbf{Accurate:} Good
    \item \textbf{Intuitive:} Acceptable
    \item \textbf{Succint:} Good
    \item \textbf{Evaluation: Acceptable}
    \end{itemize}
  \item[New error]~\\
\begin{haskell}
PE "ArchitectureAndDesign/Syntax/testfile_mba.adl" (line 17, column 4):
unexpected Lower case identifier abc
expecting Symbol '(' or Symbol ']'\end{haskell}
  \item[New evaluation]~\\
    \begin{itemize}
    \item \textbf{Accurate:} Good
    \item \textbf{Intuitive:} Acceptable
    \item \textbf{Succint:} Good
    \item \textbf{Evaluation: Acceptable
}
    \end{itemize}
  \end{description}

\hrulefill

\subsection{Error type 39}
  \begin{description}
  \item[Incorrect ADL]~\\
\begin{adl}
MEANING IN ENGLISH "Elke arbeidsrelatie benoemt expliciet welke (rechts)persoon de rol van werkgever vervult."
= [(0,0)]\end{adl}
  \item[Previous error]~\\
\begin{haskell}
Error(s) found:

before "0" at line 17, column 5 of file "testfile_mba.adl"
Expecting string ""
Try deleting symbol "0" at line 17, column 5 of file "testfile_mba.adl"

==============================

before symbol , at line 17, column 6 of file "testfile_mba.adl"
Expecting string ""
Try inserting symbol string ""

==============================

before "0" at line 17, column 7 of file "testfile_mba.adl"
Expecting string ""
Try deleting symbol "0" at line 17, column 7 of file "testfile_mba.adl"

==============================

before symbol ) at line 17, column 8 of file "testfile_mba.adl"
Expecting string ""
Try inserting symbol string ""\end{haskell}
  \item[Previous evaluation]~\\
    \begin{itemize}
    \item \textbf{Accurate:} Good
    \item \textbf{Intuitive:} Good
    \item \textbf{Succint:} Bad
    \item \textbf{Evaluation: Bad}
    \end{itemize}
  \item[New error]~\\
\begin{haskell}
PE "ArchitectureAndDesign/Syntax/testfile_mba.adl" (line 17, column 5):
unexpected Integer 0\end{haskell}
  \item[New evaluation]~\\
    \begin{itemize}
    \item \textbf{Accurate:} Good
    \item \textbf{Intuitive:} Good
    \item \textbf{Succint:} Good
    \item \textbf{Evaluation: Good
}
    \end{itemize}
  \end{description}

\hrulefill

\subsection{Error type 40}
  \begin{description}
  \item[Incorrect ADL]~\\
\begin{adl}
MEANING IN ENGLISH "Elke arbeidsrelatie benoemt expliciet welke (rechts)persoon de rol van werkgever vervult."
= [(1,"0")]\end{adl}
  \item[Previous error]~\\
\begin{haskell}
Error(s) found:

before symbol ] at line 17, column 7 of file "testfile_mba.adl"
Expecting "*"
Try inserting symbol "*"

==============================

before symbol ] at line 17, column 7 of file "testfile_mba.adl"
Expecting lower case identifier ?lc? or upper case identifier ?uc? or atom '' or
 decimal Integer 1
Try inserting symbol atom ''


D:\ampersand-fork\ArchitectureAndDesign\Syntax>ampersand testfile_mba.adl
Error(s) found:

before "1" at line 17, column 5 of file "testfile_mba.adl"
Expecting string ""
Try deleting symbol "1" at line 17, column 5 of file "testfile_mba.adl"

==============================

before symbol , at line 17, column 6 of file "testfile_mba.adl"
Expecting string ""
Try inserting symbol string ""
\end{haskell}
  \item[Previous evaluation]~\\
    \begin{itemize}
    \item \textbf{Accurate:} Good
    \item \textbf{Intuitive:} Acceptable
    \item \textbf{Succint:} Good
    \item \textbf{Evaluation: Acceptable}
    \end{itemize}
  \item[New error]~\\
\begin{haskell}
PE "ArchitectureAndDesign/Syntax/testfile_mba.adl" (line 17, column 15):
unexpected Integer 1\end{haskell}
  \item[New evaluation]~\\
    \begin{itemize}
    \item \textbf{Accurate:} Good
    \item \textbf{Intuitive:} Acceptable
    \item \textbf{Succint:} Good
    \item \textbf{Evaluation: Acceptable
}
    \end{itemize}
  \end{description}

\hrulefill

\subsection{Error type 41}
  \begin{description}
  \item[Incorrect ADL]~\\
\begin{adl}
MEANING IN ENGLISH "Elke arbeidsrelatie benoemt expliciet welke (rechts)persoon de rol van werkgever vervult."
= [("a","0")\end{adl}
  \item[Previous error]~\\
\begin{haskell}
Error(s) found:

before "PURPOSE" at line 18, column 1 of file "testfile_mba.adl"
Expecting symbol ] or (";" ...)*
Try inserting symbol symbol ]

\end{haskell}
  \item[Previous evaluation]~\\
    \begin{itemize}
    \item \textbf{Accurate:} Good
    \item \textbf{Intuitive:} Good
    \item \textbf{Succint:} Good
    \item \textbf{Evaluation: Good}
    \end{itemize}
  \item[New error]~\\
\begin{haskell}
PE "ArchitectureAndDesign/Syntax/testfile_mba.adl" (line 18, column 1):
unexpected Keyword "PURPOSE"
expecting Operator ';' or Symbol ']'\end{haskell}
  \item[New evaluation]~\\
    \begin{itemize}
    \item \textbf{Accurate:} Good
    \item \textbf{Intuitive:} Good
    \item \textbf{Succint:} Good
    \item \textbf{Evaluation: Good
}
    \end{itemize}
  \end{description}

\hrulefill

\subsection{Error type 42}
  \begin{description}
  \item[Incorrect ADL]~\\
\begin{adl}
MEANING IN ENGLISH "Elke arbeidsrelatie benoemt expliciet welke (rechts)persoon de rol van werkgever vervult."
== [("a","0")]        \end{adl}
  \item[Previous error]~\\
\begin{haskell}
Error(s) found:

before "=" at line 17, column 2 of file "testfile_mba.adl"
Expecting symbol [
Try deleting symbol "=" at line 17, column 2 of file "testfile_mba.adl"
\end{haskell}
  \item[Previous evaluation]~\\
    \begin{itemize}
    \item \textbf{Accurate:} Good
    \item \textbf{Intuitive:} Good
    \item \textbf{Succint:} Good
    \item \textbf{Evaluation: Good}
    \end{itemize}
  \item[New error]~\\
\begin{haskell}
PE "ArchitectureAndDesign/Syntax/testfile_mba.adl" (line 17, column 2):
unexpected Operator '='
expecting Symbol '['\end{haskell}
  \item[New evaluation]~\\
    \begin{itemize}
    \item \textbf{Accurate:} Good
    \item \textbf{Intuitive:} Good
    \item \textbf{Succint:} Good
    \item \textbf{Evaluation: Good
}
    \end{itemize}
  \end{description}

\hrulefill

\subsection{Error type 43}
  \begin{description}
  \item[Incorrect ADL]~\\
\begin{adl}
MEANING IN ENGLISH "Elke arbeidsrelatie benoemt expliciet welke (rechts)persoon de rol van werkgever vervult."
== [("a","0");]     \end{adl}
  \item[Previous error]~\\
\begin{haskell}
Expecting symbol [
Try deleting symbol "=" at line 17, column 2 of file "testfile_mba.adl"

==============================

before symbol ] at line 17, column 15 of file "testfile_mba.adl"
Expecting symbol (
Try deleting symbol symbol ] at line 17, column 15 of file "testfile_mba.adl"

==============================

before "PURPOSE" at line 18, column 1 of file "testfile_mba.adl"
Expecting symbol (
Try inserting symbol symbol (

==============================

before "PURPOSE" at line 18, column 1 of file "testfile_mba.adl"
Expecting string ""
Try inserting symbol string ""

==============================

before "PURPOSE" at line 18, column 1 of file "testfile_mba.adl"
Expecting symbol ,
Try inserting symbol symbol ,

==============================

before "PURPOSE" at line 18, column 1 of file "testfile_mba.adl"
Expecting string ""
Try inserting symbol string ""

==============================

before "PURPOSE" at line 18, column 1 of file "testfile_mba.adl"
Expecting symbol )
Try inserting symbol symbol )

==============================

before "PURPOSE" at line 18, column 1 of file "testfile_mba.adl"
Expecting symbol ] or ";"
Try inserting symbol symbol ]

\end{haskell}
  \item[Previous evaluation]~\\
    \begin{itemize}
    \item \textbf{Accurate:} Good
    \item \textbf{Intuitive:} Good
    \item \textbf{Succint:} Bad
    \item \textbf{Evaluation: Bad}
    \end{itemize}
  \item[New error]~\\
\begin{haskell}
PE "ArchitectureAndDesign/Syntax/testfile_mba.adl" (line 17, column 2):
unexpected Operator '='
expecting Symbol '['\end{haskell}
  \item[New evaluation]~\\
    \begin{itemize}
    \item \textbf{Accurate:} Good
    \item \textbf{Intuitive:} Good
    \item \textbf{Succint:} Good
    \item \textbf{Evaluation: Good
}
    \end{itemize}
  \end{description}

\hrulefill

\subsection{Error type 44}
  \begin{description}
  \item[Incorrect ADL]~\\
\begin{adl}
MEANING IN ENGLISH "Elke arbeidsrelatie benoemt expliciet welke (rechts)persoon de rol van werkgever vervult."
= [("a","0");(1)] \end{adl}
  \item[Previous error]~\\
\begin{haskell}
Error(s) found:

before "1" at line 17, column 15 of file "testfile_mba.adl"
Expecting string ""
Try inserting symbol string ""

==============================

before "1" at line 17, column 15 of file "testfile_mba.adl"
Expecting symbol ,
Try deleting symbol "1" at line 17, column 15 of file "testfile_mba.adl"

==============================

before symbol ) at line 17, column 16 of file "testfile_mba.adl"
Expecting symbol ,
Try inserting symbol symbol ,

==============================

before symbol ) at line 17, column 16 of file "testfile_mba.adl"
Expecting string ""
Try inserting symbol string ""
\end{haskell}
  \item[Previous evaluation]~\\
    \begin{itemize}
    \item \textbf{Accurate:} Good
    \item \textbf{Intuitive:} Acceptable
    \item \textbf{Succint:} Good
    \item \textbf{Evaluation: Acceptable}
    \end{itemize}
  \item[New error]~\\
\begin{haskell}
PE "ArchitectureAndDesign/Syntax/testfile_mba.adl" (line 17, column 15):\end{haskell}
  \item[New evaluation]~\\
    \begin{itemize}
    \item \textbf{Accurate:} Good
    \item \textbf{Intuitive:} Acceptable
    \item \textbf{Succint:} Good
    \item \textbf{Evaluation: Acceptable
}
    \end{itemize}
  \end{description}

\hrulefill

\subsection{Error type 45}
  \begin{description}
  \item[Incorrect ADL]~\\
\begin{adl}
werkgever :: Arbeidsrelatie -> Persoon 
BYPLUG [UNI, TRN] BYPLUG
PRAGMA "In"
MEANING IN ENGLISH "Elke arbeidsrelatie benoemt expliciet welke (rechts)persoon de rol van werkgever vervult."
= [("a","0");("","")]
DEFINE SRC "test"\end{adl}
  \item[Previous error]~\\
\begin{haskell}
Error(s) found:

before upper case identifier DEFINE at line 18, column 1 of file "testfile_mba.a
dl"
Expecting lower case identifier ?lc? or "." or "CLASSIFY" or "CONCEPT" or "ENDPA
TTERN" or "IDENT" or "KEY" or "POPULATION" or "PURPOSE" or "RELATION" or "ROLE"
or "RULE" or "SPEC" or "VIEW"
Try inserting symbol "ENDPATTERN"

==============================

before upper case identifier DEFINE at line 18, column 1 of file "testfile_mba.a
dl"
Expecting lower case identifier ?lc? or "CLASSIFY" or "CONCEPT" or "ENDCONTEXT"
or "IDENT" or "INCLUDE" or "INTERFACE" or "KEY" or "META" or "PATTERN" or "PHPPL
UG" or "POPULATION" or "PROCESS" or "PURPOSE" or "RELATION" or "RULE" or "SPEC"
or "SQLPLUG" or "THEMES" or "VIEW"
Try inserting symbol "RULE"

==============================

before "SRC" at line 18, column 8 of file "testfile_mba.adl"
Expecting ":"
Try inserting symbol ":"

==============================

before "SRC" at line 18, column 8 of file "testfile_mba.adl"
Expecting symbol ( or lower case identifier ?LC? or "I" or "V" or atom '' or ("-
" ...)*
Try inserting symbol lower case identifier ?LC?

==============================

before "SRC" at line 18, column 8 of file "testfile_mba.adl"
Expecting symbol [ or lower case identifier ?lc? or "!" or "#" or "-" or "/" or
"/\\" or ";" or "<>" or "=" or "CLASSIFY" or "CONCEPT" or "ENDCONTEXT" or "IDENT
" or "INCLUDE" or "INTERFACE" or "KEY" or "META" or "PATTERN" or "PHPPLUG" or "P
OPULATION" or "PROCESS" or "PURPOSE" or "RELATION" or "RULE" or "SPEC" or "SQLPL
UG" or "THEMES" or "VIEW" or "VIOLATION" or "\\" or "\\/" or "|-" or ("*" or "+"
 or "~" ...)* or ("MEANING" ...)* or ("MESSAGE" ...)*
Try deleting symbol "SRC" at line 18, column 8 of file "testfile_mba.adl"

==============================

before string "test" at line 18, column 12 of file "testfile_mba.adl"
Expecting lower case identifier ?lc? or "CLASSIFY" or "CONCEPT" or "ENDCONTEXT"
or "IDENT" or "INCLUDE" or "INTERFACE" or "KEY" or "META" or "PATTERN" or "PHPPL
UG" or "POPULATION" or "PROCESS" or "PURPOSE" or "RELATION" or "RULE" or "SPEC"
or "SQLPLUG" or "THEMES" or "VIEW"
Try deleting symbol string "test" at line 18, column 12 of file "testfile_mba.ad
l"

==============================

before "ENDPATTERN" at line 25, column 1 of file "testfile_mba.adl"
Expecting lower case identifier ?lc? or "CLASSIFY" or "CONCEPT" or "ENDCONTEXT"
or "IDENT" or "INCLUDE" or "INTERFACE" or "KEY" or "META" or "PATTERN" or "PHPPL
UG" or "POPULATION" or "PROCESS" or "PURPOSE" or "RELATION" or "RULE" or "SPEC"
or "SQLPLUG" or "THEMES" or "VIEW"
Try deleting symbol "ENDPATTERN" at line 25, column 1 of file "testfile_mba.adl"
\end{haskell}
  \item[Previous evaluation]~\\
    \begin{itemize}
    \item \textbf{Accurate:} Good
    \item \textbf{Intuitive:} Good
    \item \textbf{Succint:} Bad
    \item \textbf{Evaluation: Bad}
    \end{itemize}
  \item[New error]~\\
\begin{haskell}
unexpected Integer 1\end{haskell}
  \item[New evaluation]~\\
    \begin{itemize}
    \item \textbf{Accurate:} Good
    \item \textbf{Intuitive:} Good
    \item \textbf{Succint:} Good
    \item \textbf{Evaluation: Good
}
    \end{itemize}
  \end{description}

\hrulefill

\subsection{Error type 46}
  \begin{description}
  \item[Incorrect ADL]~\\
\begin{adl}
RULE orderInAssortment :: orderOf |- orderedAt; sells 
MEANING "Products ordered at a vendor must be sold by that vendor"
MESSAGE "A product was ordered at a vendor that does not sell it"
VIOLATION (SRC orderedAt;vendorName, TXT " does not sell ", TGT productName)\end{adl}
  \item[Previous error]~\\
\begin{haskell}
Error(s) found:

before "::" at line 223, column 24 of file "testfile_mba.adl"
Expecting symbol [ or lower case identifier ?lc? or "!" or "#" or "-" or "/" or
"/\\" or ":" or ";" or "<>" or "=" or "CLASSIFY" or "CONCEPT" or "ENDPROCESS" or
 "IDENT" or "KEY" or "POPULATION" or "PURPOSE" or "RELATION" or "ROLE" or "RULE"
 or "SPEC" or "VIEW" or "VIOLATION" or "\\" or "\\/" or "|-" or ("*" or "+" or "
~" ...)* or ("MEANING" ...)* or ("MESSAGE" ...)*
Try deleting symbol "::" at line 223, column 24 of file "testfile_mba.adl"

==============================

before lower case identifier orderOf at line 223, column 27 of file "testfile_mb
a.adl"
Expecting ":"
Try inserting symbol ":"\end{haskell}
  \item[Previous evaluation]~\\
    \begin{itemize}
    \item \textbf{Accurate:} Good
    \item \textbf{Intuitive:} Good
    \item \textbf{Succint:} Acceptable
    \item \textbf{Evaluation: Acceptable}
    \end{itemize}
  \item[New error]~\\
\begin{haskell}
PE "ArchitectureAndDesign/Syntax/testfile_mba.adl" (line 223, column 24):
unexpected Operator '::'
expecting Keyword "RULE", Keyword "CLASSIFY", Keyword "RELATION", Keyword "ROLE", Keyword "CONCEPT", Keyword "SPEC", Keyword "IDENT", Keyword "VIEW", Keyword "KEY", Keyword "PURPOSE", Keyword "POPULATION" or Keyword "ENDPROCESS"\end{haskell}
  \item[New evaluation]~\\
    \begin{itemize}
    \item \textbf{Accurate:} Good
    \item \textbf{Intuitive:} Good
    \item \textbf{Succint:} Good
    \item \textbf{Evaluation: Good
}
    \end{itemize}
  \end{description}

\hrulefill

\subsection{Error type 47}
  \begin{description}
  \item[Incorrect ADL]~\\
\begin{adl}
RULE 1onlyAcceptOwn : orderAccepted |- orderedAt\end{adl}
  \item[Previous error]~\\
\begin{haskell}
Error(s) found:

before "1" at line 228, column 6 of file "testfile_mba.adl"
Expecting symbol ( or lower case identifier ?LC? or upper case identifier ?uc? o
r "I" or "V" or string "" or atom '' or ("-" ...)*
Try deleting symbol "1" at line 228, column 6 of file "testfile_mba.adl"
\end{haskell}
  \item[Previous evaluation]~\\
    \begin{itemize}
    \item \textbf{Accurate:} Good
    \item \textbf{Intuitive:} Acceptable
    \item \textbf{Succint:} Good
    \item \textbf{Evaluation: Acceptable}
    \end{itemize}
  \item[New error]~\\
\begin{haskell}
PE "ArchitectureAndDesign/Syntax/testfile_mba.adl" (line 228, column 6):
unexpected Integer 1
expecting Operator '-', Keyword "I", Keyword "V" or Symbol '('\end{haskell}
  \item[New evaluation]~\\
    \begin{itemize}
    \item \textbf{Accurate:} Good
    \item \textbf{Intuitive:} Acceptable
    \item \textbf{Succint:} Good
    \item \textbf{Evaluation: Acceptable
}
    \end{itemize}
  \end{description}

\hrulefill

\subsection{Error type 48}
  \begin{description}
  \item[Incorrect ADL]~\\
\begin{adl}
VIEW 1client: Client(clientName, TXT ", ", clientAddress, TXT " in ", clientCity)\end{adl}
  \item[Previous error]~\\
\begin{haskell}
Error(s) found:

before "1" at line 48, column 6 of file "testfile_mba.adl"
Expecting lower case identifier ?LC? or upper case identifier ?UC? or string "?S
TR?"
Try deleting symbol "1" at line 48, column 6 of file "testfile_mba.adl"
\end{haskell}
  \item[Previous evaluation]~\\
    \begin{itemize}
    \item \textbf{Accurate:} Good
    \item \textbf{Intuitive:} Good
    \item \textbf{Succint:} Good
    \item \textbf{Evaluation: Good}
    \end{itemize}
  \item[New error]~\\
\begin{haskell}
PE "ArchitectureAndDesign/Syntax/testfile_mba.adl" (line 48, column 6):
unexpected Integer 1\end{haskell}
  \item[New evaluation]~\\
    \begin{itemize}
    \item \textbf{Accurate:} Good
    \item \textbf{Intuitive:} Good
    \item \textbf{Succint:} Good
    \item \textbf{Evaluation: Good
}
    \end{itemize}
  \end{description}

\hrulefill

\subsection{Error type 49}
  \begin{description}
  \item[Incorrect ADL]~\\
\begin{adl}
VIEW Client: Client{Test}(clientName, TXT ", ", clientAddress, TXT " in ", clientCity)\end{adl}
  \item[Previous error]~\\
\begin{haskell}
Error(s) found:

before symbol { at line 48, column 20 of file "testfile_mba.adl"
Expecting symbol (
Try inserting symbol symbol (

==============================

before symbol { at line 48, column 20 of file "testfile_mba.adl"
Expecting symbol ( or lower case identifier ?LC? or upper case identifier ?UC? o
r "I" or "PRIMHTML" or "TXT" or "V" or string "?STR?" or atom '' or ("-" ...)*
Try inserting symbol lower case identifier ?LC?

==============================

before symbol ( at line 48, column 26 of file "testfile_mba.adl"
Expecting ":"
Try deleting symbol symbol ( at line 48, column 26 of file "testfile_mba.adl"

==============================

before lower case identifier clientName at line 48, column 27 of file "testfile_
mba.adl"
Expecting ":"
Try inserting symbol ":"\end{haskell}
  \item[Previous evaluation]~\\
    \begin{itemize}
    \item \textbf{Accurate:} Good
    \item \textbf{Intuitive:} Acceptable
    \item \textbf{Succint:} Bad
    \item \textbf{Evaluation: Bad}
    \end{itemize}
  \item[New error]~\\
\begin{haskell}
PE "ArchitectureAndDesign/Syntax/testfile_mba.adl" (line 48, column 25):
unexpected Symbol '}'
expecting Operator ':'\end{haskell}
  \item[New evaluation]~\\
    \begin{itemize}
    \item \textbf{Accurate:} Good
    \item \textbf{Intuitive:} Good
    \item \textbf{Succint:} Good
    \item \textbf{Evaluation: Good
}
    \end{itemize}
  \end{description}

\hrulefill

\subsection{Error type 50}
  \begin{description}
  \item[Incorrect ADL]~\\
\begin{adl}
VIEW Client: Client, Client(clientName, TXT ", ", clientAddress, TXT " in ", clientCity)\end{adl}
  \item[Previous error]~\\
\begin{haskell}
Error(s) found:

before symbol , at line 48, column 20 of file "testfile_mba.adl"
Expecting symbol (
Try deleting symbol symbol , at line 48, column 20 of file "testfile_mba.adl"

==============================

before upper case identifier Client at line 48, column 22 of file "testfile_mba.
adl"
Expecting symbol (
Try deleting symbol upper case identifier Client at line 48, column 22 of file "
testfile_mba.adl"

\end{haskell}
  \item[Previous evaluation]~\\
    \begin{itemize}
    \item \textbf{Accurate:} Good
    \item \textbf{Intuitive:} Acceptable
    \item \textbf{Succint:} Acceptable
    \item \textbf{Evaluation: Acceptable}
    \end{itemize}
  \item[New error]~\\
\begin{haskell}
PE "ArchitectureAndDesign/Syntax/testfile_mba.adl" (line 48, column 20):
unexpected Symbol ','
expecting Keyword "DEFAULT", Symbol '{' or Symbol '('\end{haskell}
  \item[New evaluation]~\\
    \begin{itemize}
    \item \textbf{Accurate:} Good
    \item \textbf{Intuitive:} Good
    \item \textbf{Succint:} Good
    \item \textbf{Evaluation: Good
}
    \end{itemize}
  \end{description}

\hrulefill

\subsection{Error type 51}
  \begin{description}
  \item[Incorrect ADL]~\\
\begin{adl}
VIEW Client: Client(clientName, TXT ", "; clientAddress, TXT " in ", clientCity)\end{adl}
  \item[Previous error]~\\
\begin{haskell}
Error(s) found:

before ";" at line 48, column 41 of file "testfile_mba.adl"
Expecting symbol ) or symbol ,
Try inserting symbol symbol ,

==============================

before ";" at line 48, column 41 of file "testfile_mba.adl"
Expecting symbol ( or lower case identifier ?LC? or upper case identifier ?UC? o
r "I" or "PRIMHTML" or "TXT" or "V" or string "?STR?" or atom '' or ("-" ...)*
Try inserting symbol lower case identifier ?LC?
\end{haskell}
  \item[Previous evaluation]~\\
    \begin{itemize}
    \item \textbf{Accurate:} Good
    \item \textbf{Intuitive:} Good
    \item \textbf{Succint:} Acceptable
    \item \textbf{Evaluation: Acceptable}
    \end{itemize}
  \item[New error]~\\
\begin{haskell}
PE "ArchitectureAndDesign/Syntax/testfile_mba.adl" (line 48, column 41):
unexpected Operator ';'
expecting Symbol ',' or Symbol ')'\end{haskell}
  \item[New evaluation]~\\
    \begin{itemize}
    \item \textbf{Accurate:} Good
    \item \textbf{Intuitive:} Good
    \item \textbf{Succint:} Good
    \item \textbf{Evaluation: Good
}
    \end{itemize}
  \end{description}

\hrulefill

\subsection{Error type 52}
  \begin{description}
  \item[Incorrect ADL]~\\
\begin{adl}
VIEW Client: Client(clientName, TXdT ", ", clientAddress, TXT " in ", clientCity)\end{adl}
  \item[Previous error]~\\
\begin{haskell}
Error(s) found:

before string ", " at line 48, column 38 of file "testfile_mba.adl"
Expecting symbol { or ":"
Try inserting symbol symbol {

==============================

before "TXT" at line 48, column 59 of file "testfile_mba.adl"
Expecting (lower case identifier ?lc? or upper case identifier ?uc? or string ""
 ...)+
Try deleting symbol "TXT" at line 48, column 59 of file "testfile_mba.adl"

==============================

before symbol ) at line 48, column 81 of file "testfile_mba.adl"
Expecting symbol , or symbol } or (lower case identifier ?lc? or upper case iden
tifier ?uc? or string "" ...)*
Try deleting symbol symbol ) at line 48, column 81 of file "testfile_mba.adl"

==============================

before lower case identifier vendorName at line 54, column 1 of file "testfile_m
ba.adl"
Expecting symbol }
Try inserting symbol symbol }

==============================

before lower case identifier vendorName at line 54, column 1 of file "testfile_m
ba.adl"
Expecting ":"
Try inserting symbol ":"

==============================

before "::" at line 54, column 12 of file "testfile_mba.adl"
Expecting symbol ) or symbol , or symbol [ or "!" or "#" or "-" or "/" or "/\\"
or ";" or "<>" or "\\" or "\\/" or ("*" or "+" or "~" ...)*
Try inserting symbol symbol )

==============================

before "::" at line 54, column 12 of file "testfile_mba.adl"
Expecting lower case identifier ?lc? or "CLASSIFY" or "CONCEPT" or "ENDCONTEXT"
or "IDENT" or "INCLUDE" or "INTERFACE" or "KEY" or "META" or "PATTERN" or "PHPPL
UG" or "POPULATION" or "PROCESS" or "PURPOSE" or "RELATION" or "RULE" or "SPEC"
or "SQLPLUG" or "THEMES" or "VIEW"
Try inserting symbol lower case identifier ?lc?\end{haskell}
  \item[Previous evaluation]~\\
    \begin{itemize}
    \item \textbf{Accurate:} Good
    \item \textbf{Intuitive:} Good
    \item \textbf{Succint:} Bad
    \item \textbf{Evaluation: Bad}
    \end{itemize}
  \item[New error]~\\
\begin{haskell}
PE "ArchitectureAndDesign/Syntax/testfile_mba.adl" (line 48, column 38):
unexpected String ", "
expecting Symbol '{' or Operator ':'\end{haskell}
  \item[New evaluation]~\\
    \begin{itemize}
    \item \textbf{Accurate:} Good
    \item \textbf{Intuitive:} Acceptable
    \item \textbf{Succint:} Good
    \item \textbf{Evaluation: Acceptable
}
    \end{itemize}
  \end{description}

\hrulefill

\subsection{Error type 53}
  \begin{description}
  \item[Incorrect ADL]~\\
\begin{adl}
VIEWs Client: Client(Client:clientName, PRIMHTML ", ", clientAddress, TXT " in ", clientCity)\end{adl}
  \item[Previous error]~\\
\begin{haskell}
Error(s) found:

before upper case identifier VIEWs at line 48, column 1 of file "testfile_mba.ad
l"
Expecting lower case identifier ?lc? or "." or "CLASSIFY" or "CONCEPT" or "ENDCO
NTEXT" or "IDENT" or "INCLUDE" or "INTERFACE" or "KEY" or "META" or "PATTERN" or
 "PHPPLUG" or "POPULATION" or "PROCESS" or "PURPOSE" or "RELATION" or "RULE" or
"SPEC" or "SQLPLUG" or "THEMES" or "VIEW"
Try deleting symbol upper case identifier VIEWs at line 48, column 1 of file "te
stfile_mba.adl"

==============================

before upper case identifier Client at line 48, column 7 of file "testfile_mba.a
dl"
Expecting lower case identifier ?lc? or "CLASSIFY" or "CONCEPT" or "ENDCONTEXT"
or "IDENT" or "INCLUDE" or "INTERFACE" or "KEY" or "META" or "PATTERN" or "PHPPL
UG" or "POPULATION" or "PROCESS" or "PURPOSE" or "RELATION" or "RULE" or "SPEC"
or "SQLPLUG" or "THEMES" or "VIEW"
Try inserting symbol "IDENT"

==============================

before "PRIMHTML" at line 48, column 41 of file "testfile_mba.adl"
Expecting symbol ( or lower case identifier ?LC? or upper case identifier ?UC? o
r "I" or "V" or string "?STR?" or atom '' or ("-" ...)*
Try deleting symbol "PRIMHTML" at line 48, column 41 of file "testfile_mba.adl"

==============================

before symbol , at line 48, column 54 of file "testfile_mba.adl"
Expecting symbol { or ":"
Try deleting symbol symbol , at line 48, column 54 of file "testfile_mba.adl"

==============================

before lower case identifier clientAddress at line 48, column 56 of file "testfi
le_mba.adl"
Expecting ":"
Try inserting symbol ":"

==============================

before "TXT" at line 48, column 71 of file "testfile_mba.adl"
Expecting symbol ( or lower case identifier ?LC? or upper case identifier ?UC? o
r "I" or "V" or string "?STR?" or atom '' or ("-" ...)*
Try deleting symbol "TXT" at line 48, column 71 of file "testfile_mba.adl"

==============================

before symbol , at line 48, column 81 of file "testfile_mba.adl"
Expecting symbol { or ":"
Try deleting symbol symbol , at line 48, column 81 of file "testfile_mba.adl"

==============================

before lower case identifier clientCity at line 48, column 83 of file "testfile_
mba.adl"
Expecting ":"
Try inserting symbol ":"
\end{haskell}
  \item[Previous evaluation]~\\
    \begin{itemize}
    \item \textbf{Accurate:} Good
    \item \textbf{Intuitive:} Good
    \item \textbf{Succint:} Bad
    \item \textbf{Evaluation: Bad}
    \end{itemize}
  \item[New error]~\\
\begin{haskell}
PE "ArchitectureAndDesign/Syntax/testfile_mba.adl" (line 48, column 1):
unexpected Upper case identifier VIEWs
expecting Keyword "META", Keyword "PATTERN", Keyword "PROCESS", Keyword "RULE", Keyword "CLASSIFY", Keyword "RELATION", Keyword "CONCEPT", Keyword "SPEC", Keyword "IDENT", Keyword "VIEW", Keyword "KEY", Keyword "INTERFACE", Keyword "SQLPLUG", Keyword "PHPPLUG", Keyword "PURPOSE", Keyword "POPULATION", Keyword "THEMES", Keyword "INCLUDE" or Keyword "ENDCONTEXT"
\end{haskell}
  \item[New evaluation]~\\
    \begin{itemize}
    \item \textbf{Accurate:} Good
    \item \textbf{Intuitive:} Good
    \item \textbf{Succint:} Good
    \item \textbf{Evaluation: Good
}
    \end{itemize}
  \end{description}

\hrulefill

\subsection{Error type 54}
  \begin{description}
  \item[Incorrect ADL]~\\
\begin{adl}
INTERFACE Overview {client} (lient) : I[ONE]\end{adl}
  \item[Previous error]~\\
\begin{haskell}
Error(s) found:

before symbol ( at line 109, column 29 of file "testfile_mba.adl"
Expecting ":" or "FOR"
Try inserting symbol ":"

==============================

before ":" at line 109, column 37 of file "testfile_mba.adl"
Expecting "!" or "#" or "-" or "/" or "/\\" or ";" or "<>" or "BOX" or "INTERFAC
E" or "\\" or "\\/" or ("*" or "+" or "~" ...)*
Try inserting symbol "BOX"

==============================

before ":" at line 109, column 37 of file "testfile_mba.adl"
Expecting symbol [
Try inserting symbol symbol [

==============================

before ":" at line 109, column 37 of file "testfile_mba.adl"
Expecting lower case identifier ?LC? or upper case identifier ?UC? or string "?S
TR?"
Try inserting symbol lower case identifier ?LC?

==============================

before "INTERFACE" at line 120, column 1 of file "testfile_mba.adl"
Expecting symbol ] or (symbol , ...)*
Try inserting symbol symbol ]
\end{haskell}
  \item[Previous evaluation]~\\
    \begin{itemize}
    \item \textbf{Accurate:} Good
    \item \textbf{Intuitive:} Good
    \item \textbf{Succint:} Bad
    \item \textbf{Evaluation: Bad}
    \end{itemize}
  \item[New error]~\\
\begin{haskell}
PE "ArchitectureAndDesign/Syntax/testfile_mba.adl" (line 109, column 29):
unexpected Symbol '('
expecting Keyword "FOR" or Operator ':'\end{haskell}
  \item[New evaluation]~\\
    \begin{itemize}
    \item \textbf{Accurate:} Good
    \item \textbf{Intuitive:} Acceptable
    \item \textbf{Succint:} Good
    \item \textbf{Evaluation: Good
}
    \end{itemize}
  \end{description}

\hrulefill

\subsection{Error type 55}
  \begin{description}
  \item[Incorrect ADL]~\\
\begin{adl}
INTERFACE Client (clientName, clientAddress, clientCity, orderReceived) FORT Client, Customer, "test" : I[Client]\end{adl}
  \item[Previous error]~\\
\begin{haskell}
Error(s) found:

before upper case identifier FORT at line 120, column 73 of file "testfile_mba.a
dl"
Expecting symbol { or ":" or "FOR"
Try inserting symbol symbol {

==============================

before ":" at line 120, column 103 of file "testfile_mba.adl"
Expecting symbol , or symbol } or (lower case identifier ?lc? or upper case iden
tifier ?uc? or string "" ...)*
Try inserting symbol symbol }
\end{haskell}
  \item[Previous evaluation]~\\
    \begin{itemize}
    \item \textbf{Accurate:} Good
    \item \textbf{Intuitive:} Good
    \item \textbf{Succint:} Acceptable
    \item \textbf{Evaluation: Acceptable}
    \end{itemize}
  \item[New error]~\\
\begin{haskell}
PE "ArchitectureAndDesign/Syntax/testfile_mba.adl" (line 120, column 73):
unexpected Upper case identifier FORT
expecting Symbol '{', Keyword "FOR" or Operator ':'\end{haskell}
  \item[New evaluation]~\\
    \begin{itemize}
    \item \textbf{Accurate:} Good
    \item \textbf{Intuitive:} Good
    \item \textbf{Succint:} Good
    \item \textbf{Evaluation: Good
}
    \end{itemize}
  \end{description}

\hrulefill

\subsection{Error type 56}
  \begin{description}
  \item[Incorrect ADL]~\\
\begin{adl}
INTERFACE Client "test" (clientName, clientAddress, clientCity, orderReceived) FOR Client, Customer, "test" : I[Client]\end{adl}
  \item[Previous error]~\\
\begin{haskell}
Error(s) found:

before string "test" at line 120, column 18 of file "testfile_mba.adl"
Expecting symbol ( or symbol { or ":" or "FOR"
Try deleting symbol string "test" at line 120, column 18 of file "testfile_mba.a
dl"
\end{haskell}
  \item[Previous evaluation]~\\
    \begin{itemize}
    \item \textbf{Accurate:} Good
    \item \textbf{Intuitive:} Good
    \item \textbf{Succint:} Good
    \item \textbf{Evaluation: Good}
    \end{itemize}
  \item[New error]~\\
\begin{haskell}
PE "ArchitectureAndDesign/Syntax/testfile_mba.adl" (line 120, column 18):
unexpected String "test"
expecting Keyword "CLASS", Symbol '(', Symbol '{', Keyword "FOR" or Operator ':'\end{haskell}
  \item[New evaluation]~\\
    \begin{itemize}
    \item \textbf{Accurate:} Good
    \item \textbf{Intuitive:} Good
    \item \textbf{Succint:} Good
    \item \textbf{Evaluation: Good
}
    \end{itemize}
  \end{description}

\hrulefill

\subsection{Error type 57}
  \begin{description}
  \item[Incorrect ADL]~\\
\begin{adl}
INTERFACE Client (clientName, clientAddress, clientCity, orderReceived) FOR Client, Customer, "test" :: I[Client]\end{adl}
  \item[Previous error]~\\
\begin{haskell}
Error(s) found:

before "::" at line 120, column 102 of file "testfile_mba.adl"
Expecting symbol , or ":"
Try deleting symbol "::" at line 120, column 102 of file "testfile_mba.adl"

==============================

before "I" at line 120, column 105 of file "testfile_mba.adl"
Expecting ":"
Try inserting symbol ":"

\end{haskell}
  \item[Previous evaluation]~\\
    \begin{itemize}
    \item \textbf{Accurate:} Good
    \item \textbf{Intuitive:} Good
    \item \textbf{Succint:} Acceptable
    \item \textbf{Evaluation: Acceptable}
    \end{itemize}
  \item[New error]~\\
\begin{haskell}
PE "ArchitectureAndDesign/Syntax/testfile_mba.adl" (line 120, column 102):
unexpected Operator '::'
expecting Symbol ',' or Operator ':'
\end{haskell}
  \item[New evaluation]~\\
    \begin{itemize}
    \item \textbf{Accurate:} Good
    \item \textbf{Intuitive:} Good
    \item \textbf{Succint:} Good
    \item \textbf{Evaluation: Good
}
    \end{itemize}
  \end{description}

\hrulefill

\subsection{Error type 58}
  \begin{description}
  \item[Incorrect ADL]~\\
\begin{adl}
BOX <Test> [ "Name"   : clientName
    , "Street" : clientAddress\end{adl}
  \item[Previous error]~\\
\begin{haskell}
Error(s) found:

before operator < at line 121, column 5 of file "testfile_mba.adl"
Expecting symbol [
Try deleting symbol operator < at line 121, column 5 of file "testfile_mba.adl"

==============================

before upper case identifier Test at line 121, column 6 of file "testfile_mba.ad
l"
Expecting symbol [
Try deleting symbol upper case identifier Test at line 121, column 6 of file "te
stfile_mba.adl"

==============================

before ">" at line 121, column 10 of file "testfile_mba.adl"
Expecting symbol [
Try deleting symbol ">" at line 121, column 10 of file "testfile_mba.adl"
\end{haskell}
  \item[Previous evaluation]~\\
    \begin{itemize}
    \item \textbf{Accurate:} Good
    \item \textbf{Intuitive:} Good
    \item \textbf{Succint:} Acceptable
    \item \textbf{Evaluation: Acceptable}
    \end{itemize}
  \item[New error]~\\
\begin{haskell}
PE "ArchitectureAndDesign/Syntax/testfile_mba.adl" (line 121, column 5):
unexpected Operator '<'
expecting Symbol '<' or Symbol '['
\end{haskell}
  \item[New evaluation]~\\
    \begin{itemize}
    \item \textbf{Accurate:} Good
    \item \textbf{Intuitive:} Good
    \item \textbf{Succint:} Good
    \item \textbf{Evaluation: Good
}
    \end{itemize}
  \end{description}

\hrulefill

\subsection{Error type 59}
  \begin{description}
  \item[Incorrect ADL]~\\
\begin{adl}
ROWS [ "Name"   : clientName
    , "Street" : clientAddress
    , "City"   : clientCity\end{adl}
  \item[Previous error]~\\
\begin{haskell}
Error(s) found:

before upper case identifier ROWS at line 121, column 1 of file "testfile_mba.ad
l"
Expecting "!" or "#" or "-" or "/" or "/\\" or ";" or "<>" or "BOX" or "INTERFAC
E" or "\\" or "\\/" or ("*" or "+" or "~" ...)*
Try deleting symbol upper case identifier ROWS at line 121, column 1 of file "te
stfile_mba.adl"

==============================

before symbol [ at line 121, column 6 of file "testfile_mba.adl"
Expecting "BOX" or "INTERFACE"
Try inserting symbol "BOX"

\end{haskell}
  \item[Previous evaluation]~\\
    \begin{itemize}
    \item \textbf{Accurate:} Good
    \item \textbf{Intuitive:} Good
    \item \textbf{Succint:} Acceptable
    \item \textbf{Evaluation: Acceptable}
    \end{itemize}
  \item[New error]~\\
\begin{haskell}
Geen issue!\end{haskell}
  \item[New evaluation]~\\
    \begin{itemize}
    \item \textbf{Accurate:} Good
    \item \textbf{Intuitive:} Good
    \item \textbf{Succint:} Good
    \item \textbf{Evaluation: Good
}
    \end{itemize}
  \end{description}

\hrulefill

\subsection{Error type 60}
  \begin{description}
  \item[Incorrect ADL]~\\
\begin{adl}
BOX [[ "All clients"  : V[ONE*Client]
   , "All vendors"  : V[ONE*Vendor] 
   , "All products" : V[ONE*Product]
   , "All orders"   : V[ONE*Order]
     BOX [ product : orderOf;productName
         , client  : orderedBy;clientName
         , vendor  :orderedAt;vendorName
         ] 
   ]
   INTERFACE test]\end{adl}
  \item[Previous error]~\\
\begin{haskell}
Error(s) found:

before symbol [ at line 110, column 6 of file "testfile_mba.adl"
Expecting lower case identifier ?LC? or upper case identifier ?UC? or string "?S
TR?"
Try deleting symbol symbol [ at line 110, column 6 of file "testfile_mba.adl"

==============================

before symbol ] at line 119, column 18 of file "testfile_mba.adl"
Expecting symbol ( or symbol { or ":" or "FOR"
Try deleting symbol symbol ] at line 119, column 18 of file "testfile_mba.adl"

==============================

before "INTERFACE" at line 121, column 1 of file "testfile_mba.adl"
Expecting ":"
Try inserting symbol ":"

==============================

before "INTERFACE" at line 121, column 1 of file "testfile_mba.adl"
Expecting symbol ( or lower case identifier ?LC? or "I" or "V" or atom '' or ("-
" ...)*
Try inserting symbol lower case identifier ?LC?

==============================

before symbol ( at line 121, column 18 of file "testfile_mba.adl"
Expecting lower case identifier ?lc? or "CLASSIFY" or "CONCEPT" or "ENDCONTEXT"
or "IDENT" or "INCLUDE" or "INTERFACE" or "KEY" or "META" or "PATTERN" or "PHPPL
UG" or "POPULATION" or "PROCESS" or "PURPOSE" or "RELATION" or "RULE" or "SPEC"
or "SQLPLUG" or "THEMES" or "VIEW"
Try inserting symbol "RULE"

==============================

before symbol , at line 121, column 29 of file "testfile_mba.adl"
Expecting symbol ) or symbol [ or "!" or "#" or "-" or "/" or "/\\" or ";" or "<
>" or "\\" or "\\/" or ("*" or "+" or "~" ...)*
Try deleting symbol symbol , at line 121, column 29 of file "testfile_mba.adl"

==============================

before lower case identifier clientAddress at line 121, column 31 of file "testf
ile_mba.adl"
Expecting symbol )
Try deleting symbol lower case identifier clientAddress at line 121, column 31 o
f file "testfile_mba.adl"

==============================

before symbol , at line 121, column 44 of file "testfile_mba.adl"
Expecting symbol )
Try deleting symbol symbol , at line 121, column 44 of file "testfile_mba.adl"

==============================

before lower case identifier clientCity at line 121, column 46 of file "testfile
_mba.adl"
Expecting symbol )
Try deleting symbol lower case identifier clientCity at line 121, column 46 of f
ile "testfile_mba.adl"

==============================

before symbol , at line 121, column 56 of file "testfile_mba.adl"
Expecting symbol )
Try deleting symbol symbol , at line 121, column 56 of file "testfile_mba.adl"

==============================

before lower case identifier orderReceived at line 121, column 58 of file "testf
ile_mba.adl"
Expecting symbol )
Try deleting symbol lower case identifier orderReceived at line 121, column 58 o
f file "testfile_mba.adl"

==============================

before "FOR" at line 121, column 73 of file "testfile_mba.adl"
Expecting lower case identifier ?lc? or "!" or "#" or "-" or "/" or "/\\" or ";"
 or "<>" or "=" or "CLASSIFY" or "CONCEPT" or "ENDCONTEXT" or "IDENT" or "INCLUD
E" or "INTERFACE" or "KEY" or "META" or "PATTERN" or "PHPPLUG" or "POPULATION" o
r "PROCESS" or "PURPOSE" or "RELATION" or "RULE" or "SPEC" or "SQLPLUG" or "THEM
ES" or "VIEW" or "VIOLATION" or "\\" or "\\/" or "|-" or ("*" or "+" or "~" ...)
* or ("MEANING" ...)* or ("MESSAGE" ...)*
Try inserting symbol "INTERFACE"

==============================

before "FOR" at line 121, column 73 of file "testfile_mba.adl"
Expecting lower case identifier ?LC? or upper case identifier ?UC? or string "?S
TR?"
Try inserting symbol lower case identifier ?LC?
\end{haskell}
  \item[Previous evaluation]~\\
    \begin{itemize}
    \item \textbf{Accurate:} Good
    \item \textbf{Intuitive:} Good
    \item \textbf{Succint:} Bad
    \item \textbf{Evaluation: Bad}
    \end{itemize}
  \item[New error]~\\
\begin{haskell}
PE "ArchitectureAndDesign/Syntax/testfile_mba.adl" (line 110, column 6):
unexpected Symbol '['\end{haskell}
  \item[New evaluation]~\\
    \begin{itemize}
    \item \textbf{Accurate:} Good
    \item \textbf{Intuitive:} Good
    \item \textbf{Succint:} Good
    \item \textbf{Evaluation: Good
}
    \end{itemize}
  \end{description}

\hrulefill

\subsection{Error type 61}
  \begin{description}
  \item[Incorrect ADL]~\\
\begin{adl}
MEANING IN DUTCH RfEST {+ test -}\end{adl}
  \item[Previous error]~\\
\begin{haskell}
Error(s) found:

before upper case identifier RfEST at line 225, column 18 of file "testfile_mba.
adl"
Expecting "HTML" or "LATEX" or "MARKDOWN" or "REST" or string "" or explanation
{+-}
Try deleting symbol upper case identifier RfEST at line 225, column 18 of file "
testfile_mba.adl"\end{haskell}
  \item[Previous evaluation]~\\
    \begin{itemize}
    \item \textbf{Accurate:} Good
    \item \textbf{Intuitive:} Good
    \item \textbf{Succint:} Good
    \item \textbf{Evaluation: Good}
    \end{itemize}
  \item[New error]~\\
\begin{haskell}
PE "ArchitectureAndDesign/Syntax/testfile_mba.adl" (line 225, column 18):
unexpected Upper case identifier RfEST
expecting Keyword "REST", Keyword "HTML", Keyword "LATEX" or Keyword "MARKDOWN"
\end{haskell}
  \item[New evaluation]~\\
    \begin{itemize}
    \item \textbf{Accurate:} Good
    \item \textbf{Intuitive:} Good
    \item \textbf{Succint:} Good
    \item \textbf{Evaluation: Good
}
    \end{itemize}
  \end{description}

\hrulefill

\subsection{Error type 62}
  \begin{description}
  \item[Incorrect ADL]~\\
\begin{adl}
MEANING REST IN DUTCH  {+ test -}\end{adl}
  \item[Previous error]~\\
\begin{haskell}
Error(s) found:

before "IN" at line 225, column 14 of file "testfile_mba.adl"
Expecting string "" or explanation {+-}
Try inserting symbol string ""

==============================

before "IN" at line 225, column 14 of file "testfile_mba.adl"
Expecting lower case identifier ?lc? or "CLASSIFY" or "CONCEPT" or "ENDPROCESS"
or "IDENT" or "KEY" or "MEANING" or "POPULATION" or "PURPOSE" or "RELATION" or "
ROLE" or "RULE" or "SPEC" or "VIEW" or "VIOLATION" or ("MESSAGE" ...)*
Try inserting symbol "MESSAGE"\end{haskell}
  \item[Previous evaluation]~\\
    \begin{itemize}
    \item \textbf{Accurate:} Good
    \item \textbf{Intuitive:} Acceptable
    \item \textbf{Succint:} Acceptable
    \item \textbf{Evaluation: Acceptable}
    \end{itemize}
  \item[New error]~\\
\begin{haskell}
PE "ArchitectureAndDesign/Syntax/testfile_mba.adl" (line 225, column 14):
unexpected Keyword "IN"\end{haskell}
  \item[New evaluation]~\\
    \begin{itemize}
    \item \textbf{Accurate:} Good
    \item \textbf{Intuitive:} Good
    \item \textbf{Succint:} Good
    \item \textbf{Evaluation: Good
}
    \end{itemize}
  \end{description}

\hrulefill

\subsection{Error type 63}
  \begin{description}
  \item[Incorrect ADL]~\\
\begin{adl}
RULE OrderInAssortment : orderOf |- orderedAt; sells 
MESSAGE "A product was ordered at a vendor that does not sell it"
MEANING IN DUTCH  {+ test -}
VIOLATION (SRC orderedAt;vendorName, TXT " does not sell ", TGT productName)
\end{adl}
  \item[Previous error]~\\
\begin{haskell}
Error(s) found:

before "MEANING" at line 226, column 1 of file "testfile_mba.adl"
Expecting lower case identifier ?lc? or "CLASSIFY" or "CONCEPT" or "ENDPROCESS"
or "IDENT" or "KEY" or "MESSAGE" or "POPULATION" or "PURPOSE" or "RELATION" or "
ROLE" or "RULE" or "SPEC" or "VIEW" or "VIOLATION"
Try inserting symbol "RULE"

==============================

before "MEANING" at line 226, column 1 of file "testfile_mba.adl"
Expecting symbol ( or lower case identifier ?lc? or upper case identifier ?uc? o
r "I" or "V" or string "" or atom '' or ("-" ...)*
Try inserting symbol lower case identifier ?LC?
\end{haskell}
  \item[Previous evaluation]~\\
    \begin{itemize}
    \item \textbf{Accurate:} Good
    \item \textbf{Intuitive:} Good
    \item \textbf{Succint:} Acceptable
    \item \textbf{Evaluation: Acceptable}
    \end{itemize}
  \item[New error]~\\
\begin{haskell}
PE "ArchitectureAndDesign/Syntax/testfile_mba.adl" (line 226, column 1):
unexpected Keyword "MEANING"
expecting Keyword "RULE", Keyword "CLASSIFY", Keyword "RELATION", Keyword "ROLE", Keyword "CONCEPT", Keyword "SPEC", Keyword "IDENT", Keyword "VIEW", Keyword "KEY", Keyword "PURPOSE", Keyword "POPULATION" or Keyword "ENDPROCESS"
\end{haskell}
  \item[New evaluation]~\\
    \begin{itemize}
    \item \textbf{Accurate:} Good
    \item \textbf{Intuitive:} Good
    \item \textbf{Succint:} Good
    \item \textbf{Evaluation: Good
}
    \end{itemize}
  \end{description}

\hrulefill

\subsection{Error type 64}
  \begin{description}
  \item[Incorrect ADL]~\\
\begin{adl}
VIOLATION (SRC orderedAt;vendorName+1, TXT " does not sell ", TGT productName)\end{adl}
  \item[Previous error]~\\
\begin{haskell}
Error(s) found:

before "1" at line 227, column 37 of file "testfile_mba.adl"
Expecting symbol ) or "*" or "+" or "-" or "/" or "/\\" or ";" or "<>" or "\\" o
r "\\/" or "~" or (symbol , ...)*
Try deleting symbol "1" at line 227, column 37 of file "testfile_mba.adl"

\end{haskell}
  \item[Previous evaluation]~\\
    \begin{itemize}
    \item \textbf{Accurate:} Good
    \item \textbf{Intuitive:} Good
    \item \textbf{Succint:} Good
    \item \textbf{Evaluation: Good}
    \end{itemize}
  \item[New error]~\\
\begin{haskell}
PE "ArchitectureAndDesign/Syntax/testfile_mba.adl" (line 227, column 37):
unexpected Integer 1
expecting Operator '~', Operator '*', Operator '+', Operator ';', Operator '/', Operator '\', Operator '<>', Operator '-', Operator '/\', Operator '\/', Symbol ',' or Symbol ')'\end{haskell}
  \item[New evaluation]~\\
    \begin{itemize}
    \item \textbf{Accurate:} Good
    \item \textbf{Intuitive:} Good
    \item \textbf{Succint:} Good
    \item \textbf{Evaluation: Good
}
    \end{itemize}
  \end{description}

\hrulefill

\subsection{Error type 65}
  \begin{description}
  \item[Incorrect ADL]~\\
\begin{adl}
VIOLATION (SRC orderedAt;vendorName; TXT " does not sell ", TGT productName)\end{adl}
  \item[Previous error]~\\
\begin{haskell}
Error(s) found:

before "TXT" at line 227, column 38 of file "testfile_mba.adl"
Expecting symbol ( or lower case identifier ?LC? or "I" or "V" or atom '' or ("-
" ...)*
Try inserting symbol lower case identifier ?LC?

==============================

before "TXT" at line 227, column 38 of file "testfile_mba.adl"
Expecting symbol ) or symbol [ or "-" or "/" or "/\\" or ";" or "<>" or "\\" or
"\\/" or (symbol , ...)* or ("*" or "+" or "~" ...)*
Try inserting symbol symbol ,
\end{haskell}
  \item[Previous evaluation]~\\
    \begin{itemize}
    \item \textbf{Accurate:} Good
    \item \textbf{Intuitive:} Good
    \item \textbf{Succint:} Acceptable
    \item \textbf{Evaluation: Acceptable}
    \end{itemize}
  \item[New error]~\\
\begin{haskell}
PE "ArchitectureAndDesign/Syntax/testfile_mba.adl" (line 227, column 38):
unexpected Keyword "TXT"
expecting Operator '-', Keyword "I", Keyword "V" or Symbol '('\end{haskell}
  \item[New evaluation]~\\
    \begin{itemize}
    \item \textbf{Accurate:} Good
    \item \textbf{Intuitive:} Acceptable
    \item \textbf{Succint:} Good
    \item \textbf{Evaluation: Acceptable
}
    \end{itemize}
  \end{description}

\hrulefill

\subsection{Error type 66}
  \begin{description}
  \item[Incorrect ADL]~\\
\begin{adl}
productPrice :: Product -> Price BYPLUG [UNI, PORP] BYPLUG \end{adl}
  \item[Previous error]~\\
\begin{haskell}
Error(s) found:

before upper case identifier PORP at line 81, column 47 of file "testfile_mba.ad
l"
Expecting "ASY" or "INJ" or "IRF" or "PROP" or "RFX" or "SUR" or "SYM" or "TOT"
or "TRN" or "UNI"
Try deleting symbol upper case identifier PORP at line 81, column 47 of file "te
stfile_mba.adl"

==============================

before symbol ] at line 81, column 51 of file "testfile_mba.adl"
Expecting "ASY" or "INJ" or "IRF" or "PROP" or "RFX" or "SUR" or "SYM" or "TOT"
or "TRN" or "UNI"
Try inserting symbol "UNI"

\end{haskell}
  \item[Previous evaluation]~\\
    \begin{itemize}
    \item \textbf{Accurate:} Good
    \item \textbf{Intuitive:} Good
    \item \textbf{Succint:} Acceptable
    \item \textbf{Evaluation: Acceptable}
    \end{itemize}
  \item[New error]~\\
\begin{haskell}
PE "ArchitectureAndDesign/Syntax/testfile_mba.adl" (line 81, column 47):
unexpected Upper case identifier PORP
expecting Keyword "UNI", Keyword "INJ", Keyword "SUR", Keyword "TOT", Keyword "SYM", Keyword "ASY", Keyword "TRN", Keyword "RFX", Keyword "IRF", Keyword "AUT" or Keyword "PROP"\end{haskell}
  \item[New evaluation]~\\
    \begin{itemize}
    \item \textbf{Accurate:} Good
    \item \textbf{Intuitive:} Good
    \item \textbf{Succint:} Good
    \item \textbf{Evaluation: Good
}
    \end{itemize}
  \end{description}

\hrulefill

\subsection{Error type 67}
  \begin{description}
  \item[Incorrect ADL]~\\
\begin{adl}
productPrice :: Product -> Price BYPLUG [UNI, PROP] BYPLUG PRAGMA\end{adl}
  \item[Previous error]~\\
\begin{haskell}
Error(s) found:

before "=" at line 82, column 3 of file "testfile_mba.adl"
Expecting (string "" ...)+
Try inserting symbol string ""\end{haskell}
  \item[Previous evaluation]~\\
    \begin{itemize}
    \item \textbf{Accurate:} Good
    \item \textbf{Intuitive:} Good
    \item \textbf{Succint:} Good
    \item \textbf{Evaluation: Good}
    \end{itemize}
  \item[New error]~\\
\begin{haskell}
PE "ArchitectureAndDesign/Syntax/testfile_mba.adl" (line 82, column 3):
unexpected Operator '='\end{haskell}
  \item[New evaluation]~\\
    \begin{itemize}
    \item \textbf{Accurate:} Good
    \item \textbf{Intuitive:} Acceptable
    \item \textbf{Succint:} Good
    \item \textbf{Evaluation: Acceptable
}
    \end{itemize}
  \end{description}

\hrulefill

\subsection{Error type 68}
  \begin{description}
  \item[Incorrect ADL]~\\
\begin{adl}
productPrice :: Product -> Price BYPLUG [UNI, PROP] BYPLUG PRAGMA "TEST"
MEANING IN DUTCH  {+ test -}
  = [ ("Product_1"; "10,00 euro")
    ; ("Product_2", "0,75 euro")
    ; ("Product_3", "6,95 euro")
    ; ("Product_4", "8,50 euro")
    ; ("Product_5", "4,50 euro")
    ]\end{adl}
  \item[Previous error]~\\
\begin{haskell}
Expecting symbol ,
Try deleting symbol ";" at line 83, column 19 of file "testfile_mba.adl"

==============================

before string "10,00 euro" at line 83, column 21 of file "testfile_mba.adl"
Expecting symbol ,
Try inserting symbol symbol ,
\end{haskell}
  \item[Previous evaluation]~\\
    \begin{itemize}
    \item \textbf{Accurate:} Good
    \item \textbf{Intuitive:} Good
    \item \textbf{Succint:} Acceptable
    \item \textbf{Evaluation: Acceptable}
    \end{itemize}
  \item[New error]~\\
\begin{haskell}
PE "ArchitectureAndDesign/Syntax/testfile_mba.adl" (line 83, column 19):
unexpected Operator ';'
expecting Symbol ','\end{haskell}
  \item[New evaluation]~\\
    \begin{itemize}
    \item \textbf{Accurate:} Good
    \item \textbf{Intuitive:} Good
    \item \textbf{Succint:} Good
    \item \textbf{Evaluation: Good
}
    \end{itemize}
  \end{description}

\hrulefill

\subsection{Error type 69}
  \begin{description}
  \item[Incorrect ADL]~\\
\begin{adl}
productPrice :: Product -> Price BYPLUG [UNI, PROP] BYPLUG PRAGMA "TEST"
MEANING IN DUTCH  {+ test -}
  = [ ("Product_1", "10,00 euro")
    , ("Product_2", "0,75 euro")
    ; ("Product_3", "6,95 euro")
    ; ("Product_4", "8,50 euro")
    ; ("Product_5", "4,50 euro")
    ]\end{adl}
  \item[Previous error]~\\
\begin{haskell}
Error(s) found:

before symbol , at line 84, column 5 of file "testfile_mba.adl"
Expecting symbol ] or (";" ...)*
Try deleting symbol symbol , at line 84, column 5 of file "testfile_mba.adl"

==============================

before symbol ( at line 84, column 7 of file "testfile_mba.adl"
Expecting symbol ] or (";" ...)*
Try inserting symbol ";"
\end{haskell}
  \item[Previous evaluation]~\\
    \begin{itemize}
    \item \textbf{Accurate:} Good
    \item \textbf{Intuitive:} Good
    \item \textbf{Succint:} Acceptable
    \item \textbf{Evaluation: Acceptable}
    \end{itemize}
  \item[New error]~\\
\begin{haskell}
PE "ArchitectureAndDesign/Syntax/testfile_mba.adl" (line 84, column 5):
unexpected Symbol ','
expecting Operator ';' or Symbol ']'
\end{haskell}
  \item[New evaluation]~\\
    \begin{itemize}
    \item \textbf{Accurate:} Good
    \item \textbf{Intuitive:} Good
    \item \textbf{Succint:} Good
    \item \textbf{Evaluation: Good
}
    \end{itemize}
  \end{description}

\hrulefill

\subsection{Error type 70}
  \begin{description}
  \item[Incorrect ADL]~\\
\begin{adl}
productName :: Product -> Name
  = [ "Product_1" * "Inner tube"
    ; "Product_2" * "Outer tube"
    ]\end{adl}
  \item[Previous error]~\\
\begin{haskell}
Error(s) found:

before string "Product_1" at line 74, column 7 of file "testfile_mba.adl"
Expecting symbol ] or (symbol ( ...)* or (lower case identifier ?lc? or upper ca
se identifier ?uc? or atom '' or decimal Integer 1 ...)*
Try inserting symbol symbol ]

==============================

before string "Product_1" at line 74, column 7 of file "testfile_mba.adl"
Expecting lower case identifier ?lc? or "." or "CLASSIFY" or "CONCEPT" or "ENDCO
NTEXT" or "IDENT" or "INCLUDE" or "INTERFACE" or "KEY" or "META" or "PATTERN" or
 "PHPPLUG" or "POPULATION" or "PROCESS" or "PURPOSE" or "RELATION" or "RULE" or
"SPEC" or "SQLPLUG" or "THEMES" or "VIEW"
Try inserting symbol lower case identifier ?lc?

==============================

before string "Product_1" at line 74, column 7 of file "testfile_mba.adl"
Expecting "::"
Try inserting symbol "::"

==============================

before ";" at line 75, column 5 of file "testfile_mba.adl"
Expecting symbol [ or lower case identifier ?lc? or "." or "=" or "BYPLUG" or "C
LASSIFY" or "CONCEPT" or "ENDCONTEXT" or "IDENT" or "INCLUDE" or "INTERFACE" or
"KEY" or "META" or "PATTERN" or "PHPPLUG" or "POPULATION" or "PRAGMA" or "PROCES
S" or "PURPOSE" or "RELATION" or "RULE" or "SPEC" or "SQLPLUG" or "THEMES" or "V
IEW" or ("MEANING" ...)*
Try inserting symbol "RULE"

==============================

before ";" at line 75, column 5 of file "testfile_mba.adl"
Expecting symbol ( or lower case identifier ?lc? or upper case identifier ?uc? o
r "I" or "V" or string "" or atom '' or ("-" ...)*
Try inserting symbol lower case identifier ?LC?

==============================

before string "Product_2" at line 75, column 7 of file "testfile_mba.adl"
Expecting symbol ( or lower case identifier ?LC? or "I" or "V" or atom '' or ("-
" ...)*
Try inserting symbol lower case identifier ?LC?

==============================

before string "Product_2" at line 75, column 7 of file "testfile_mba.adl"
Expecting symbol [ or lower case identifier ?lc? or "-" or "/" or "/\\" or ";" o
r "<>" or "=" or "CLASSIFY" or "CONCEPT" or "ENDCONTEXT" or "IDENT" or "INCLUDE"
 or "INTERFACE" or "KEY" or "META" or "PATTERN" or "PHPPLUG" or "POPULATION" or
"PROCESS" or "PURPOSE" or "RELATION" or "RULE" or "SPEC" or "SQLPLUG" or "THEMES
" or "VIEW" or "VIOLATION" or "\\" or "\\/" or "|-" or ("*" or "+" or "~" ...)*
or ("MEANING" ...)* or ("MESSAGE" ...)*
Try inserting symbol symbol [
\end{haskell}
  \item[Previous evaluation]~\\
    \begin{itemize}
    \item \textbf{Accurate:} Good
    \item \textbf{Intuitive:} Bad
    \item \textbf{Succint:} Bad
    \item \textbf{Evaluation: Bad}
    \end{itemize}
  \item[New error]~\\
\begin{haskell}
PE "ArchitectureAndDesign/Syntax/testfile_mba.adl" (line 75, column 5):
unexpected Operator ';'
expecting Symbol ',' or Symbol ']'\end{haskell}
  \item[New evaluation]~\\
    \begin{itemize}
    \item \textbf{Accurate:} Good
    \item \textbf{Intuitive:} Good
    \item \textbf{Succint:} Good
    \item \textbf{Evaluation: Good
}
    \end{itemize}
  \end{description}

\hrulefill

\subsection{Error type 71}
  \begin{description}
  \item[Incorrect ADL]~\\
\begin{adl}
productName :: Product -> Name
  = ( "Product_1" * "Inner tube"
    , "Product_2" * "Outer tube"
    )\end{adl}
  \item[Previous error]~\\
\begin{haskell}
Error(s) found:

before symbol ( at line 74, column 5 of file "testfile_mba.adl"
Expecting symbol [
Try inserting symbol symbol [

==============================

before "*" at line 74, column 19 of file "testfile_mba.adl"
Expecting symbol ,
Try deleting symbol "*" at line 74, column 19 of file "testfile_mba.adl"

==============================

before string "Inner tube" at line 74, column 21 of file "testfile_mba.adl"
Expecting symbol ,
Try deleting symbol string "Inner tube" at line 74, column 21 of file "testfile_
mba.adl"

==============================

before "*" at line 75, column 19 of file "testfile_mba.adl"
Expecting symbol )
Try deleting symbol "*" at line 75, column 19 of file "testfile_mba.adl"

==============================

before string "Outer tube" at line 75, column 21 of file "testfile_mba.adl"
Expecting symbol )
Try deleting symbol string "Outer tube" at line 75, column 21 of file "testfile_
mba.adl"

==============================

before lower case identifier productPrice at line 78, column 1 of file "testfile
_mba.adl"
Expecting symbol ] or (";" ...)*
Try inserting symbol symbol ]
\end{haskell}
  \item[Previous evaluation]~\\
    \begin{itemize}
    \item \textbf{Accurate:} Good
    \item \textbf{Intuitive:} Good
    \item \textbf{Succint:} Bad
    \item \textbf{Evaluation: Bad}
    \end{itemize}
  \item[New error]~\\
\begin{haskell}
PE "ArchitectureAndDesign/Syntax/testfile_mba.adl" (line 74, column 5):
unexpected Symbol '('
expecting Symbol '[\end{haskell}
  \item[New evaluation]~\\
    \begin{itemize}
    \item \textbf{Accurate:} Good
    \item \textbf{Intuitive:} Good
    \item \textbf{Succint:} Good
    \item \textbf{Evaluation: Good
}
    \end{itemize}
  \end{description}

\hrulefill

\subsection{Error type 72}
  \begin{description}
  \item[Incorrect ADL]~\\
\begin{adl}
CONCEPT "PrestatieIndicator" BYPLUG "Een prestatie-indicator is een variable om prestaties van ondernemingen te analyseren." TYPE TESe\end{adl}
  \item[Previous error]~\\
\begin{haskell}
Error(s) found:

before upper case identifier TESe at line 23, column 131 of file "testfile_mba.a
dl"
Expecting string ""
Try deleting symbol upper case identifier TESe at line 23, column 131 of file "t
estfile_mba.adl"

==============================

before "ENDPATTERN" at line 25, column 1 of file "testfile_mba.adl"
Expecting string ""
Try inserting symbol string ""
\end{haskell}
  \item[Previous evaluation]~\\
    \begin{itemize}
    \item \textbf{Accurate:} Good
    \item \textbf{Intuitive:} Good
    \item \textbf{Succint:} Acceptable
    \item \textbf{Evaluation: Acceptable}
    \end{itemize}
  \item[New error]~\\
\begin{haskell}
PE "ArchitectureAndDesign/Syntax/testfile_mba.adl" (line 23, column 131):
unexpected Upper case identifier TESe\end{haskell}
  \item[New evaluation]~\\
    \begin{itemize}
    \item \textbf{Accurate:} Good
    \item \textbf{Intuitive:} Good
    \item \textbf{Succint:} Good
    \item \textbf{Evaluation: Good
}
    \end{itemize}
  \end{description}

\hrulefill

\subsection{Error type 73}
  \begin{description}
  \item[Incorrect ADL]~\\
\begin{adl}
SPEC PrestatieIndicator ISA "Functie"""\end{adl}
  \item[Previous error]~\\
\begin{haskell}
Error(s) found:

before string "" at line 26, column 38 of file "testfile_mba.adl"
Expecting lower case identifier ?lc? or "CLASSIFY" or "CONCEPT" or "ENDPATTERN"
or "IDENT" or "KEY" or "POPULATION" or "PURPOSE" or "RELATION" or "ROLE" or "RUL
E" or "SPEC" or "VIEW"
Try deleting symbol string "" at line 26, column 38 of file "testfile_mba.adl"\end{haskell}
  \item[Previous evaluation]~\\
    \begin{itemize}
    \item \textbf{Accurate:} Good
    \item \textbf{Intuitive:} Good
    \item \textbf{Succint:} Good
    \item \textbf{Evaluation: Good}
    \end{itemize}
  \item[New error]~\\
\begin{haskell}
PE "ArchitectureAndDesign/Syntax/testfile_mba.adl" (line 26, column 38):
unexpected String ""
expecting Keyword "RULE", Keyword "CLASSIFY", Keyword "RELATION", Keyword "CONCEPT", Keyword "SPEC", Keyword "IDENT", Keyword "VIEW", Keyword "KEY", Keyword "PURPOSE", Keyword "POPULATION" or Keyword "ENDPATTERN"\end{haskell}
  \item[New evaluation]~\\
    \begin{itemize}
    \item \textbf{Accurate:} Good
    \item \textbf{Intuitive:} Good
    \item \textbf{Succint:} Good
    \item \textbf{Evaluation: Good
}
    \end{itemize}
  \end{description}

\hrulefill

\subsection{Error type 74}
  \begin{description}
  \item[Incorrect ADL]~\\
\begin{adl}
CLASIFY PrestatieIndicator ISA "Functie"\end{adl}
  \item[Previous error]~\\
\begin{haskell}
PE "ArchitectureAndDesign/Syntax/testfile_mba.adl" (line 26, column 38):
unexpected String ""
expecting Keyword "RULE", Keyword "CLASSIFY", Keyword "RELATION", Keyword "CONCEPT", Keyword "SPEC", Keyword "IDENT", Keyword "VIEW", Keyword "KEY", Keyword "PURPOSE", Keyword "POPULATION" or Keyword "ENDPATTERN"\end{haskell}
  \item[Previous evaluation]~\\
    \begin{itemize}
    \item \textbf{Accurate:} Good
    \item \textbf{Intuitive:} Good
    \item \textbf{Succint:} Good
    \item \textbf{Evaluation: Good}
    \end{itemize}
  \item[New error]~\\
\begin{haskell}
PE "ArchitectureAndDesign/Syntax/testfile_mba.adl" (line 26, column 1):
unexpected Upper case identifier CLASIFY
expecting Keyword "RULE", Keyword "CLASSIFY", Keyword "RELATION", Keyword "CONCEPT", Keyword "SPEC", Keyword "IDENT", Keyword "VIEW", Keyword "KEY", Keyword "PURPOSE", Keyword "POPULATION" or Keyword "ENDPATTERN"
\end{haskell}
  \item[New evaluation]~\\
    \begin{itemize}
    \item \textbf{Accurate:} Good
    \item \textbf{Intuitive:} Good
    \item \textbf{Succint:} Good
    \item \textbf{Evaluation: Good
}
    \end{itemize}
  \end{description}

\hrulefill

\subsection{Error type 75}
  \begin{description}
  \item[Incorrect ADL]~\\
\begin{adl}
{+Om de werkgever te kunnen bepalen gaan we ervan uit dat elke arbeidsrelatie precies een werkgever heeft.-}\end{adl}
  \item[Previous error]~\\
\begin{haskell}
Error(s) found:

before "=" at line 86, column 3 of file "testfile_mba.adl"
Expecting lower case identifier ?lc? or "CLASSIFY" or "CONCEPT" or "ENDPATTERN"
or "IDENT" or "KEY" or "POPULATION" or "PURPOSE" or "RELATION" or "ROLE" or "RUL
E" or "SPEC" or "VIEW"
Try inserting symbol "ENDPATTERN"

==============================

before "=" at line 86, column 3 of file "testfile_mba.adl"
Expecting lower case identifier ?lc? or "CLASSIFY" or "CONCEPT" or "ENDCONTEXT"
or "IDENT" or "INCLUDE" or "INTERFACE" or "KEY" or "META" or "PATTERN" or "PHPPL
UG" or "POPULATION" or "PROCESS" or "PURPOSE" or "RELATION" or "RULE" or "SPEC"
or "SQLPLUG" or "THEMES" or "VIEW"
Try inserting symbol "RULE"

==============================

before "=" at line 86, column 3 of file "testfile_mba.adl"
Expecting symbol ( or lower case identifier ?lc? or upper case identifier ?uc? o
r "I" or "V" or string "" or atom '' or ("-" ...)*
Try inserting symbol lower case identifier ?LC?

==============================

before symbol [ at line 86, column 5 of file "testfile_mba.adl"
Expecting symbol ( or lower case identifier ?LC? or "I" or "V" or atom '' or ("-
" ...)*
Try inserting symbol lower case identifier ?LC?

==============================

before symbol ( at line 86, column 7 of file "testfile_mba.adl"
Expecting upper case identifier ?uc? or "ONE" or string ""
Try deleting symbol symbol ( at line 86, column 7 of file "testfile_mba.adl"

==============================

before symbol , at line 86, column 19 of file "testfile_mba.adl"
Expecting symbol ] or "*"
Try deleting symbol symbol , at line 86, column 19 of file "testfile_mba.adl"

==============================

before string "10,00 euro" at line 86, column 21 of file "testfile_mba.adl"
Expecting symbol ]
Try deleting symbol string "10,00 euro" at line 86, column 21 of file "testfile_
mba.adl"

==============================

before symbol ) at line 86, column 33 of file "testfile_mba.adl"
Expecting symbol ]
Try deleting symbol symbol ) at line 86, column 33 of file "testfile_mba.adl"

==============================

before ";" at line 87, column 5 of file "testfile_mba.adl"
Expecting symbol ]
Try inserting symbol symbol ]

==============================

before string "Product_2" at line 87, column 8 of file "testfile_mba.adl"
Expecting symbol ( or lower case identifier ?LC? or "I" or "V" or atom '' or ("-
" ...)*
Try inserting symbol lower case identifier ?LC?

==============================

before string "Product_2" at line 87, column 8 of file "testfile_mba.adl"
Expecting symbol ) or symbol [ or "!" or "#" or "-" or "/" or "/\\" or ";" or "<
>" or "\\" or "\\/" or ("*" or "+" or "~" ...)*
Try inserting symbol symbol )

==============================

before string "Product_2" at line 87, column 8 of file "testfile_mba.adl"
Expecting lower case identifier ?lc? or "-" or "/" or "/\\" or ";" or "<>" or "C
LASSIFY" or "CONCEPT" or "ENDCONTEXT" or "IDENT" or "INCLUDE" or "INTERFACE" or
"KEY" or "META" or "PATTERN" or "PHPPLUG" or "POPULATION" or "PROCESS" or "PURPO
SE" or "RELATION" or "RULE" or "SPEC" or "SQLPLUG" or "THEMES" or "VIEW" or "VIO
LATION" or "\\" or "\\/" or ("*" or "+" or "~" ...)* or ("MEANING" ...)* or ("ME
SSAGE" ...)*
Try inserting symbol "THEMES"

==============================

before symbol ) at line 87, column 32 of file "testfile_mba.adl"
Expecting symbol , or lower case identifier ?lc? or "CLASSIFY" or "CONCEPT" or "
ENDCONTEXT" or "IDENT" or "INCLUDE" or "INTERFACE" or "KEY" or "META" or "PATTER
N" or "PHPPLUG" or "POPULATION" or "PROCESS" or "PURPOSE" or "RELATION" or "RULE
" or "SPEC" or "SQLPLUG" or "THEMES" or "VIEW"
Try deleting symbol symbol ) at line 87, column 32 of file "testfile_mba.adl"

==============================

before ";" at line 88, column 5 of file "testfile_mba.adl"
Expecting lower case identifier ?lc? or "CLASSIFY" or "CONCEPT" or "ENDCONTEXT"
or "IDENT" or "INCLUDE" or "INTERFACE" or "KEY" or "META" or "PATTERN" or "PHPPL
UG" or "POPULATION" or "PROCESS" or "PURPOSE" or "RELATION" or "RULE" or "SPEC"
or "SQLPLUG" or "THEMES" or "VIEW"
Try inserting symbol "RULE"

==============================

before ";" at line 88, column 5 of file "testfile_mba.adl"
Expecting symbol ( or lower case identifier ?lc? or upper case identifier ?uc? o
r "I" or "V" or string "" or atom '' or ("-" ...)*
Try inserting symbol lower case identifier ?LC?

==============================

before string "Product_3" at line 88, column 8 of file "testfile_mba.adl"
Expecting symbol ( or lower case identifier ?LC? or "I" or "V" or atom '' or ("-
" ...)*
Try inserting symbol lower case identifier ?LC?

==============================

before string "Product_3" at line 88, column 8 of file "testfile_mba.adl"
Expecting symbol ) or symbol [ or "!" or "#" or "-" or "/" or "/\\" or ";" or "<
>" or "\\" or "\\/" or ("*" or "+" or "~" ...)*
Try inserting symbol symbol )

==============================

before string "Product_3" at line 88, column 8 of file "testfile_mba.adl"
Expecting lower case identifier ?lc? or "-" or "/" or "/\\" or ";" or "<>" or "=
" or "CLASSIFY" or "CONCEPT" or "ENDCONTEXT" or "IDENT" or "INCLUDE" or "INTERFA
CE" or "KEY" or "META" or "PATTERN" or "PHPPLUG" or "POPULATION" or "PROCESS" or
 "PURPOSE" or "RELATION" or "RULE" or "SPEC" or "SQLPLUG" or "THEMES" or "VIEW"
or "VIOLATION" or "\\" or "\\/" or "|-" or ("*" or "+" or "~" ...)* or ("MEANING
" ...)* or ("MESSAGE" ...)*
Try inserting symbol "THEMES"

==============================

before symbol ) at line 88, column 32 of file "testfile_mba.adl"
Expecting symbol , or lower case identifier ?lc? or "CLASSIFY" or "CONCEPT" or "
ENDCONTEXT" or "IDENT" or "INCLUDE" or "INTERFACE" or "KEY" or "META" or "PATTER
N" or "PHPPLUG" or "POPULATION" or "PROCESS" or "PURPOSE" or "RELATION" or "RULE
" or "SPEC" or "SQLPLUG" or "THEMES" or "VIEW"
Try deleting symbol symbol ) at line 88, column 32 of file "testfile_mba.adl"

==============================

before ";" at line 89, column 5 of file "testfile_mba.adl"
Expecting lower case identifier ?lc? or "CLASSIFY" or "CONCEPT" or "ENDCONTEXT"
or "IDENT" or "INCLUDE" or "INTERFACE" or "KEY" or "META" or "PATTERN" or "PHPPL
UG" or "POPULATION" or "PROCESS" or "PURPOSE" or "RELATION" or "RULE" or "SPEC"
or "SQLPLUG" or "THEMES" or "VIEW"
Try inserting symbol "RULE"

==============================

before ";" at line 89, column 5 of file "testfile_mba.adl"
Expecting symbol ( or lower case identifier ?lc? or upper case identifier ?uc? o
r "I" or "V" or string "" or atom '' or ("-" ...)*
Try inserting symbol lower case identifier ?LC?

==============================

before string "Product_4" at line 89, column 8 of file "testfile_mba.adl"
Expecting symbol ( or lower case identifier ?LC? or "I" or "V" or atom '' or ("-
" ...)*
Try inserting symbol lower case identifier ?LC?

==============================

before string "Product_4" at line 89, column 8 of file "testfile_mba.adl"
Expecting symbol ) or symbol [ or "!" or "#" or "-" or "/" or "/\\" or ";" or "<
>" or "\\" or "\\/" or ("*" or "+" or "~" ...)*
Try inserting symbol symbol )

==============================

before string "Product_4" at line 89, column 8 of file "testfile_mba.adl"
Expecting lower case identifier ?lc? or "-" or "/" or "/\\" or ";" or "<>" or "=
" or "CLASSIFY" or "CONCEPT" or "ENDCONTEXT" or "IDENT" or "INCLUDE" or "INTERFA
CE" or "KEY" or "META" or "PATTERN" or "PHPPLUG" or "POPULATION" or "PROCESS" or
 "PURPOSE" or "RELATION" or "RULE" or "SPEC" or "SQLPLUG" or "THEMES" or "VIEW"
or "VIOLATION" or "\\" or "\\/" or "|-" or ("*" or "+" or "~" ...)* or ("MEANING
" ...)* or ("MESSAGE" ...)*
Try inserting symbol "THEMES"

==============================

before symbol ) at line 89, column 32 of file "testfile_mba.adl"
Expecting symbol , or lower case identifier ?lc? or "CLASSIFY" or "CONCEPT" or "
ENDCONTEXT" or "IDENT" or "INCLUDE" or "INTERFACE" or "KEY" or "META" or "PATTER
N" or "PHPPLUG" or "POPULATION" or "PROCESS" or "PURPOSE" or "RELATION" or "RULE
" or "SPEC" or "SQLPLUG" or "THEMES" or "VIEW"
Try deleting symbol symbol ) at line 89, column 32 of file "testfile_mba.adl"

==============================

before ";" at line 90, column 5 of file "testfile_mba.adl"
Expecting lower case identifier ?lc? or "CLASSIFY" or "CONCEPT" or "ENDCONTEXT"
or "IDENT" or "INCLUDE" or "INTERFACE" or "KEY" or "META" or "PATTERN" or "PHPPL
UG" or "POPULATION" or "PROCESS" or "PURPOSE" or "RELATION" or "RULE" or "SPEC"
or "SQLPLUG" or "THEMES" or "VIEW"
Try inserting symbol "RULE"

==============================

before ";" at line 90, column 5 of file "testfile_mba.adl"
Expecting symbol ( or lower case identifier ?lc? or upper case identifier ?uc? o
r "I" or "V" or string "" or atom '' or ("-" ...)*
Try inserting symbol lower case identifier ?LC?

==============================

before string "Product_5" at line 90, column 8 of file "testfile_mba.adl"
Expecting symbol ( or lower case identifier ?LC? or "I" or "V" or atom '' or ("-
" ...)*
Try inserting symbol lower case identifier ?LC?

==============================

before string "Product_5" at line 90, column 8 of file "testfile_mba.adl"
Expecting symbol ) or symbol [ or "!" or "#" or "-" or "/" or "/\\" or ";" or "<
>" or "\\" or "\\/" or ("*" or "+" or "~" ...)*
Try inserting symbol symbol )

==============================

before string "Product_5" at line 90, column 8 of file "testfile_mba.adl"
Expecting lower case identifier ?lc? or "-" or "/" or "/\\" or ";" or "<>" or "=
" or "CLASSIFY" or "CONCEPT" or "ENDCONTEXT" or "IDENT" or "INCLUDE" or "INTERFA
CE" or "KEY" or "META" or "PATTERN" or "PHPPLUG" or "POPULATION" or "PROCESS" or
 "PURPOSE" or "RELATION" or "RULE" or "SPEC" or "SQLPLUG" or "THEMES" or "VIEW"
or "VIOLATION" or "\\" or "\\/" or "|-" or ("*" or "+" or "~" ...)* or ("MEANING
" ...)* or ("MESSAGE" ...)*
Try inserting symbol "THEMES"

==============================

before symbol ) at line 90, column 32 of file "testfile_mba.adl"
Expecting symbol , or lower case identifier ?lc? or "CLASSIFY" or "CONCEPT" or "
ENDCONTEXT" or "IDENT" or "INCLUDE" or "INTERFACE" or "KEY" or "META" or "PATTER
N" or "PHPPLUG" or "POPULATION" or "PROCESS" or "PURPOSE" or "RELATION" or "RULE
" or "SPEC" or "SQLPLUG" or "THEMES" or "VIEW"
Try deleting symbol symbol ) at line 90, column 32 of file "testfile_mba.adl"

==============================

before symbol ] at line 91, column 5 of file "testfile_mba.adl"
Expecting lower case identifier ?lc? or "CLASSIFY" or "CONCEPT" or "ENDCONTEXT"
or "IDENT" or "INCLUDE" or "INTERFACE" or "KEY" or "META" or "PATTERN" or "PHPPL
UG" or "POPULATION" or "PROCESS" or "PURPOSE" or "RELATION" or "RULE" or "SPEC"
or "SQLPLUG" or "THEMES" or "VIEW"
Try deleting symbol symbol ] at line 91, column 5 of file "testfile_mba.adl"
\end{haskell}
  \item[Previous evaluation]~\\
    \begin{itemize}
    \item \textbf{Accurate:} Good
    \item \textbf{Intuitive:} Good
    \item \textbf{Succint:} Bad
    \item \textbf{Evaluation: Bad}
    \end{itemize}
  \item[New error]~\\
\begin{haskell}
PE "ArchitectureAndDesign/Syntax/testfile_mba.adl" (line 86, column 3):
unexpected Operator '='
expecting Keyword "RULE", Keyword "CLASSIFY", Keyword "RELATION", Keyword "CONCEPT", Keyword "SPEC", Keyword "IDENT", Keyword "VIEW", Keyword "KEY", Keyword "PURPOSE", Keyword "POPULATION" or Keyword "ENDPATTERN"\end{haskell}
  \item[New evaluation]~\\
    \begin{itemize}
    \item \textbf{Accurate:} Good
    \item \textbf{Intuitive:} Good
    \item \textbf{Succint:} Good
    \item \textbf{Evaluation: Good
}
    \end{itemize}
  \end{description}

\hrulefill

\subsection{Error type 76}
  \begin{description}
  \item[Incorrect ADL]~\\
\begin{adl}
CLASSIFY PrestatieIndicator ISA ("Functie" /\ "Functie")\end{adl}
  \item[Previous error]~\\
\begin{haskell}
Error(s) found:

before symbol ( at line 27, column 33 of file "testfile_mba.adl"
Expecting upper case identifier ?uc? or string ""
Try deleting symbol symbol ( at line 27, column 33 of file "testfile_mba.adl"

==============================

before "/\\" at line 27, column 43 of file "testfile_mba.adl"
Expecting lower case identifier ?lc? or "CLASSIFY" or "CONCEPT" or "ENDPATTERN"
or "IDENT" or "KEY" or "POPULATION" or "PURPOSE" or "RELATION" or "ROLE" or "RUL
E" or "SPEC" or "VIEW"
Try inserting symbol "ENDPATTERN"

==============================

before "/\\" at line 27, column 43 of file "testfile_mba.adl"
Expecting lower case identifier ?lc? or "CLASSIFY" or "CONCEPT" or "ENDCONTEXT"
or "IDENT" or "INCLUDE" or "INTERFACE" or "KEY" or "META" or "PATTERN" or "PHPPL
UG" or "POPULATION" or "PROCESS" or "PURPOSE" or "RELATION" or "RULE" or "SPEC"
or "SQLPLUG" or "THEMES" or "VIEW"
Try inserting symbol "RULE"

==============================

before "/\\" at line 27, column 43 of file "testfile_mba.adl"
Expecting symbol ( or lower case identifier ?lc? or upper case identifier ?uc? o
r "I" or "V" or string "" or atom '' or ("-" ...)*
Try inserting symbol lower case identifier ?LC?

==============================

before string "Functie" at line 27, column 46 of file "testfile_mba.adl"
Expecting symbol ( or lower case identifier ?LC? or "I" or "V" or atom '' or ("-
" ...)*
Try deleting symbol string "Functie" at line 27, column 46 of file "testfile_mba
.adl"

==============================

before symbol ) at line 27, column 55 of file "testfile_mba.adl"
Expecting symbol ( or lower case identifier ?LC? or "I" or "V" or atom '' or ("-
" ...)*
Try inserting symbol symbol (

==============================

before symbol ) at line 27, column 55 of file "testfile_mba.adl"
Expecting symbol ( or lower case identifier ?LC? or "I" or "V" or atom '' or ("-
" ...)*
Try inserting symbol lower case identifier ?LC?

==============================

before "ENDPATTERN" at line 29, column 1 of file "testfile_mba.adl"
Expecting lower case identifier ?lc? or "!" or "#" or "-" or "/" or "/\\" or ";"
 or "<>" or "=" or "CLASSIFY" or "CONCEPT" or "ENDCONTEXT" or "IDENT" or "INCLUD
E" or "INTERFACE" or "KEY" or "META" or "PATTERN" or "PHPPLUG" or "POPULATION" o
r "PROCESS" or "PURPOSE" or "RELATION" or "RULE" or "SPEC" or "SQLPLUG" or "THEM
ES" or "VIEW" or "VIOLATION" or "\\" or "|-" or ("*" or "+" or "~" ...)* or ("ME
ANING" ...)* or ("MESSAGE" ...)*
Try deleting symbol "ENDPATTERN" at line 29, column 1 of file "testfile_mba.adl"
\end{haskell}
  \item[Previous evaluation]~\\
    \begin{itemize}
    \item \textbf{Accurate:} Good
    \item \textbf{Intuitive:} Good
    \item \textbf{Succint:} Bad
    \item \textbf{Evaluation: Bad}
    \end{itemize}
  \item[New error]~\\
\begin{haskell}
PE "ArchitectureAndDesign/Syntax/testfile_mba.adl" (line 27, column 33):
unexpected Symbol '('\end{haskell}
  \item[New evaluation]~\\
    \begin{itemize}
    \item \textbf{Accurate:} Good
    \item \textbf{Intuitive:} Good
    \item \textbf{Succint:} Good
    \item \textbf{Evaluation: Good
}
    \end{itemize}
  \end{description}

\hrulefill

\subsection{Error type 77}
  \begin{description}
  \item[Incorrect ADL]~\\
\begin{adl}
CLASSIFY PrestatieIndicator ISA "Functie" /\ "Functie"\end{adl}
  \item[Previous error]~\\
\begin{haskell}
Error(s) found:

before "/\\" at line 27, column 43 of file "testfile_mba.adl"
Expecting lower case identifier ?lc? or "CLASSIFY" or "CONCEPT" or "ENDPATTERN"
or "IDENT" or "KEY" or "POPULATION" or "PURPOSE" or "RELATION" or "ROLE" or "RUL
E" or "SPEC" or "VIEW"
Try deleting symbol "/\\" at line 27, column 43 of file "testfile_mba.adl"

==============================

before string "Functie" at line 27, column 46 of file "testfile_mba.adl"
Expecting "ENDPATTERN"
Try deleting symbol string "Functie" at line 27, column 46 of file "testfile_mba
.adl"

\end{haskell}
  \item[Previous evaluation]~\\
    \begin{itemize}
    \item \textbf{Accurate:} Good
    \item \textbf{Intuitive:} Good
    \item \textbf{Succint:} Acceptable
    \item \textbf{Evaluation: Acceptable}
    \end{itemize}
  \item[New error]~\\
\begin{haskell}
PE "ArchitectureAndDesign/Syntax/testfile_mba.adl" (line 27, column 43):
unexpected Operator '/\'
expecting Keyword "RULE", Keyword "CLASSIFY", Keyword "RELATION", Keyword "CONCEPT", Keyword "SPEC", Keyword "IDENT", Keyword "VIEW", Keyword "KEY", Keyword "PURPOSE", Keyword "POPULATION" or Keyword "ENDPATTERN"
\end{haskell}
  \item[New evaluation]~\\
    \begin{itemize}
    \item \textbf{Accurate:} Good
    \item \textbf{Intuitive:} Good
    \item \textbf{Succint:} Good
    \item \textbf{Evaluation: Good
}
    \end{itemize}
  \end{description}

\hrulefill

\subsection{Error type 78}
  \begin{description}
  \item[Incorrect ADL]~\\
\begin{adl}
PATTERN
CONTEXT DeliverySimple IN ENGLISH

ENDCONTEXT

ENDPATTERN\end{adl}
  \item[Previous error]~\\
\begin{haskell}
Error(s) found:

before "CONTEXT" at line 29, column 1 of file "testfile_mba.adl"
Expecting lower case identifier ?lc? or "CLASSIFY" or "CONCEPT" or "ENDPATTERN"
or "IDENT" or "KEY" or "POPULATION" or "PURPOSE" or "RELATION" or "ROLE" or "RUL
E" or "SPEC" or "VIEW"
Try inserting symbol "PURPOSE"

==============================

before "ENDCONTEXT" at line 31, column 1 of file "testfile_mba.adl"
Expecting "HTML" or "LATEX" or "MARKDOWN" or "REF" or "REST" or explanation {+-}

Try deleting symbol "ENDCONTEXT" at line 31, column 1 of file "testfile_mba.adl"


==============================

before "ENDPATTERN" at line 33, column 1 of file "testfile_mba.adl"
Expecting explanation {+-}
Try inserting symbol explanation {+-}\end{haskell}
  \item[Previous evaluation]~\\
    \begin{itemize}
    \item \textbf{Accurate:} Good
    \item \textbf{Intuitive:} Good
    \item \textbf{Succint:} Acceptable
    \item \textbf{Evaluation: Acceptable}
    \end{itemize}
  \item[New error]~\\
\begin{haskell}
PE "ArchitectureAndDesign/Syntax/testfile_mba.adl" (line 29, column 1):
unexpected Keyword "CONTEXT"
expecting Keyword "RULE", Keyword "CLASSIFY", Keyword "RELATION", Keyword "CONCEPT", Keyword "SPEC", Keyword "IDENT", Keyword "VIEW", Keyword "KEY", Keyword "PURPOSE", Keyword "POPULATION" or Keyword "ENDPATTERN"
\end{haskell}
  \item[New evaluation]~\\
    \begin{itemize}
    \item \textbf{Accurate:} Good
    \item \textbf{Intuitive:} Good
    \item \textbf{Succint:} Good
    \item \textbf{Evaluation: Good
}
    \end{itemize}
  \end{description}

\hrulefill

\subsection{Error type 79}
  \begin{description}
  \item[Incorrect ADL]~\\
\begin{adl}
RULE allAccepted: I |- orderAccepted; orderAccepted~ <> bestelling!\end{adl}
  \item[Previous error]~\\
\begin{haskell}
Error(s) found:

before "MEANING" at line 249, column 1 of file "testfile_mba.adl"
Expecting symbol ( or lower case identifier ?LC? or "I" or "V" or atom '' or ("-
" ...)*
Try inserting symbol lower case identifier ?LC?\end{haskell}
  \item[Previous evaluation]~\\
    \begin{itemize}
    \item \textbf{Accurate:} Good
    \item \textbf{Intuitive:} Good
    \item \textbf{Succint:} Acceptable
    \item \textbf{Evaluation: Acceptable}
    \end{itemize}
  \item[New error]~\\
\begin{haskell}
PE "ArchitectureAndDesign/Syntax/testfile_mba.adl" (line 249, column 1):
unexpected Keyword "MEANING"
expecting Operator '-', Keyword "I", Keyword "V" or Symbol '('
\end{haskell}
  \item[New evaluation]~\\
    \begin{itemize}
    \item \textbf{Accurate:} Good
    \item \textbf{Intuitive:} Good
    \item \textbf{Succint:} Good
    \item \textbf{Evaluation: Good
}
    \end{itemize}
  \end{description}

\hrulefill

\subsection{Error type 80}
  \begin{description}
  \item[Incorrect ADL]~\\
\begin{adl}
RULE allAccepted: I |- orderAccepted; orderAccepted~ <> bestelling!levering/pakje --\end{adl}
  \item[Previous error]~\\
\begin{haskell}
Error(s) found:

before "/" at line 248, column 76 of file "testfile_mba.adl"
Expecting symbol [ or lower case identifier ?lc? or "!" or "-" or "/\\" or "CLAS
SIFY" or "CONCEPT" or "ENDPROCESS" or "IDENT" or "KEY" or "POPULATION" or "PURPO
SE" or "RELATION" or "ROLE" or "RULE" or "SPEC" or "VIEW" or "VIOLATION" or "\\/
" or ("*" or "+" or "~" ...)* or ("MEANING" ...)* or ("MESSAGE" ...)*
Try inserting symbol "RULE"

==============================

before "/" at line 248, column 76 of file "testfile_mba.adl"
Expecting symbol ( or lower case identifier ?lc? or upper case identifier ?uc? o
r "I" or "V" or string "" or atom '' or ("-" ...)*
Try inserting symbol lower case identifier ?LC?\end{haskell}
  \item[Previous evaluation]~\\
    \begin{itemize}
    \item \textbf{Accurate:} Good
    \item \textbf{Intuitive:} Good
    \item \textbf{Succint:} Acceptable
    \item \textbf{Evaluation: Acceptable}
    \end{itemize}
  \item[New error]~\\
\begin{haskell}
PE "ArchitectureAndDesign/Syntax/testfile_mba.adl" (line 248, column 76):
unexpected Operator '/'
expecting Keyword "RULE", Keyword "CLASSIFY", Keyword "RELATION", Keyword "ROLE", Keyword "CONCEPT", Keyword "SPEC", Keyword "IDENT", Keyword "VIEW", Keyword "KEY", Keyword "PURPOSE", Keyword "POPULATION" or Keyword "ENDPROCESS"
\end{haskell}
  \item[New evaluation]~\\
    \begin{itemize}
    \item \textbf{Accurate:} Good
    \item \textbf{Intuitive:} Good
    \item \textbf{Succint:} Good
    \item \textbf{Evaluation: Good
}
    \end{itemize}
  \end{description}

\hrulefill

\subsection{Error type 81}
  \begin{description}
  \item[Incorrect ADL]~\\
\begin{adl}
RULE allAccepted: (I |- orderAccepted; orderAccepted~ <> bestelling) \end{adl}
  \item[Previous error]~\\
\begin{haskell}
Error(s) found:

before "|-" at line 248, column 22 of file "testfile_mba.adl"
Expecting symbol ) or symbol [ or "!" or "#" or "-" or "/" or "/\\" or ";" or "<
>" or "\\" or "\\/" or ("*" or "+" or "~" ...)*
Try inserting symbol symbol )

==============================

before symbol ) at line 248, column 68 of file "testfile_mba.adl"
Expecting symbol [ or lower case identifier ?lc? or "!" or "#" or "-" or "/\\" o
r ";" or "CLASSIFY" or "CONCEPT" or "ENDPROCESS" or "IDENT" or "KEY" or "POPULAT
ION" or "PURPOSE" or "RELATION" or "ROLE" or "RULE" or "SPEC" or "VIEW" or "VIOL
ATION" or "\\/" or ("*" or "+" or "~" ...)* or ("MEANING" ...)* or ("MESSAGE" ..
.)*
Try deleting symbol symbol ) at line 248, column 68 of file "testfile_mba.adl"
\end{haskell}
  \item[Previous evaluation]~\\
    \begin{itemize}
    \item \textbf{Accurate:} Good
    \item \textbf{Intuitive:} Good
    \item \textbf{Succint:} Acceptable
    \item \textbf{Evaluation: Acceptable}
    \end{itemize}
  \item[New error]~\\
\begin{haskell}
PE "ArchitectureAndDesign/Syntax/testfile_mba.adl" (line 248, column 22):
unexpected Operator '|-'
expecting Symbol '[', Operator '~', Operator '*', Operator '+', Operator ';', Operator '!', Operator '#', Operator '/', Operator '\', Operator '<>', Operator '-', Operator '/\', Operator '\/' or Symbol ')'\end{haskell}
  \item[New evaluation]~\\
    \begin{itemize}
    \item \textbf{Accurate:} Good
    \item \textbf{Intuitive:} Good
    \item \textbf{Succint:} Good
    \item \textbf{Evaluation: Good
}
    \end{itemize}
  \end{description}

\hrulefill

\subsection{Error type 82}
  \begin{description}
  \item[Incorrect ADL]~\\
\begin{adl}
RULE allAccepted: I |- orderAccepted; orderAccepted- -- == TOT\end{adl}
  \item[Previous error]~\\
\begin{haskell}
Error(s) found:

before "MEANING" at line 249, column 1 of file "testfile_mba.adl"
Expecting symbol ( or lower case identifier ?LC? or "I" or "V" or atom '' or ("-
" ...)*
Try inserting symbol lower case identifier ?LC?
\end{haskell}
  \item[Previous evaluation]~\\
    \begin{itemize}
    \item \textbf{Accurate:} Good
    \item \textbf{Intuitive:} Acceptable
    \item \textbf{Succint:} Good
    \item \textbf{Evaluation: Acceptable}
    \end{itemize}
  \item[New error]~\\
\begin{haskell}
PE "ArchitectureAndDesign/Syntax/testfile_mba.adl" (line 249, column 1):
unexpected Keyword "MEANING"
expecting Operator '-', Keyword "I", Keyword "V" or Symbol '('\end{haskell}
  \item[New evaluation]~\\
    \begin{itemize}
    \item \textbf{Accurate:} Good
    \item \textbf{Intuitive:} Acceptable
    \item \textbf{Succint:} Good
    \item \textbf{Evaluation: Acceptable
}
    \end{itemize}
  \end{description}

\hrulefill

\subsection{Error type 83}
  \begin{description}
  \item[Incorrect ADL]~\\
\begin{adl}
RULE allAccepted: I ONE <> orderAccepted; orderAccepted~ -- == TOT\end{adl}
  \item[Previous error]~\\
\begin{haskell}
Error(s) found:

before "ONE" at line 248, column 21 of file "testfile_mba.adl"
Expecting symbol [ or lower case identifier ?lc? or "!" or "#" or "-" or "/" or
"/\\" or ";" or "<>" or "=" or "CLASSIFY" or "CONCEPT" or "ENDPROCESS" or "IDENT
" or "KEY" or "POPULATION" or "PURPOSE" or "RELATION" or "ROLE" or "RULE" or "SP
EC" or "VIEW" or "VIOLATION" or "\\" or "\\/" or "|-" or ("*" or "+" or "~" ...)
* or ("MEANING" ...)* or ("MESSAGE" ...)*
Try deleting symbol "ONE" at line 248, column 21 of file "testfile_mba.adl"

\end{haskell}
  \item[Previous evaluation]~\\
    \begin{itemize}
    \item \textbf{Accurate:} Good
    \item \textbf{Intuitive:} Good
    \item \textbf{Succint:} Good
    \item \textbf{Evaluation: Good}
    \end{itemize}
  \item[New error]~\\
\begin{haskell}
PE "ArchitectureAndDesign/Syntax/testfile_mba.adl" (line 248, column 21):
unexpected Keyword "ONE"
expecting Keyword "RULE", Keyword "CLASSIFY", Keyword "RELATION", Keyword "ROLE", Keyword "CONCEPT", Keyword "SPEC", Keyword "IDENT", Keyword "VIEW", Keyword "KEY", Keyword "PURPOSE", Keyword "POPULATION" or Keyword "ENDPROCESS"
\end{haskell}
  \item[New evaluation]~\\
    \begin{itemize}
    \item \textbf{Accurate:} Good
    \item \textbf{Intuitive:} Good
    \item \textbf{Succint:} Good
    \item \textbf{Evaluation: Good
}
    \end{itemize}
  \end{description}

\hrulefill

\subsection{Error type 84}
  \begin{description}
  \item[Incorrect ADL]~\\
\begin{adl}
RULE allAccepted: ATOM <> orderAccepted; orderAccepted~ -- == TOT\end{adl}
  \item[Previous error]~\\
\begin{haskell}
Error(s) found:

before upper case identifier ATOM at line 248, column 19 of file "testfile_mba.a
dl"
Expecting symbol ( or lower case identifier ?LC? or "I" or "V" or atom '' or ("-
" ...)*
Try deleting symbol upper case identifier ATOM at line 248, column 19 of file "t
estfile_mba.adl"

==============================

before "<>" at line 248, column 24 of file "testfile_mba.adl"
Expecting symbol ( or lower case identifier ?LC? or "I" or "V" or atom '' or ("-
" ...)*
Try inserting symbol lower case identifier ?LC?
\end{haskell}
  \item[Previous evaluation]~\\
    \begin{itemize}
    \item \textbf{Accurate:} Good
    \item \textbf{Intuitive:} Good
    \item \textbf{Succint:} Acceptable
    \item \textbf{Evaluation: Acceptable}
    \end{itemize}
  \item[New error]~\\
\begin{haskell}
PE "ArchitectureAndDesign/Syntax/testfile_mba.adl" (line 248, column 19):
unexpected Upper case identifier ATOM
expecting Operator '-', Keyword "I", Keyword "V" or Symbol '('
\end{haskell}
  \item[New evaluation]~\\
    \begin{itemize}
    \item \textbf{Accurate:} Good
    \item \textbf{Intuitive:} Good
    \item \textbf{Succint:} Good
    \item \textbf{Evaluation: Good
}
    \end{itemize}
  \end{description}

\hrulefill

\subsection{Error type 85}
  \begin{description}
  \item[Incorrect ADL]~\\
\begin{adl}
ROLE Client,Order MAINTAIN allReceived,allsent\end{adl}
  \item[Previous error]~\\
\begin{haskell}
Error(s) found:

before upper case identifier MAINTAIN at line 261, column 19 of file "testfile_m
ba.adl"
Expecting symbol , or "EDITS" or "MAINTAINS"
Try deleting symbol upper case identifier MAINTAIN at line 261, column 19 of fil
e "testfile_mba.adl"

==============================

before lower case identifier allReceived at line 261, column 28 of file "testfil
e_mba.adl"
Expecting "EDITS"
Try inserting symbol "EDITS"\end{haskell}
  \item[Previous evaluation]~\\
    \begin{itemize}
    \item \textbf{Accurate:} Good
    \item \textbf{Intuitive:} Good
    \item \textbf{Succint:} Acceptable
    \item \textbf{Evaluation: Acceptable}
    \end{itemize}
  \item[New error]~\\
\begin{haskell}
PE "ArchitectureAndDesign/Syntax/testfile_mba.adl" (line 261, column 19):
unexpected Upper case identifier MAINTAIN
expecting Symbol ',', Keyword "MAINTAINS" or Keyword "EDITS"\end{haskell}
  \item[New evaluation]~\\
    \begin{itemize}
    \item \textbf{Accurate:} Good
    \item \textbf{Intuitive:} Good
    \item \textbf{Succint:} Good
    \item \textbf{Evaluation: Good
}
    \end{itemize}
  \end{description}

\end{document}