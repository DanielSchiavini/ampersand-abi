% !TEX root = ../Documentation.tex

\subsection{Next steps (R-D)}
\label{subsec:design-next-steps}

During the parser re-implementation and code re-factoring to enhance the code maintainability, we noticed potential improvement topics in Ampersand.
Were possible, with respect to our project requirements and our milestones, we integrated these topics in the new parser.
Some topics we could not handle due to time constraints or given the undesired impact on the surrounding modules.
It would be a pity if these improvement ideas were lost as these can help the Ampersand team to further enhance the tool.
This section summarizes our improvement suggestions, both generic as specific ones, for the Ampersand tool.

\begin{description}
  \item[Syntax improvements]
    In the syntax, we discovered some statements which are not pure LL(k) statements.
    The Parsec \texttt{try} function allows the backtracking in the parser, avoiding that input is consumed which is still needed in another parse statements if the parse function can not succeed successfully.
    Using \texttt{try} we can handle this situation but the backtracking has a negative impact of the parser performance whilst it adds complexity to the parser module. 
    Our suggestion is to further optimize the Ampersand syntax to establish a pure predictive syntax in which no backtracking is needed.
    This will not only improve the parser performance and maintainability, the syntax simplifications will .
    The syntax statements in which backtracking is needed, including the reasons why, are listed in section \autoref{subsec:backtracking}.

  \item[Warnings]
    An improvement point in the new lexer is that warnings are now supported. 
    Warnings are however not yet integrated in the Ampersand tool.
    There is no need to stop the compiling process for warnings, as the reasons behind them can make sense.
    Warnings can, however, support the user identifying the cause of unexpected results, although the compilation could be completed successfully.
    Our suggestion is to add the list of warnings to the design artifacts of Ampersand, available for the user to reflect on is needed.

  \item[Uniform parse tree structure]
  
   In the parse tree, not all data types are constructed in correspondence to the syntax of the Ampersand language.
   Several restructuring functions are used in the parser to reformulate the result of the parse functions to match the constructor type in the parse tree.
   These functions can be recognized in the parser as rebuild or reorder functions.
   An example of such a rebuild action is given below:

   The syntax notation of \texttt{ViewAtt} is defined as: \texttt{ViewAtt ::= LabelProps? Term}
   
   The parser function will use the order of the notation to extract the needed information:

   \begin{verbatim}
          pViewAtt :: AmpParser P_ObjectDef
          pViewAtt = rebuild <$> currPos <*> optLabelProps <*> pTerm
              where rebuild pos (nm, strs) ctx = P_Obj nm pos ctx mView msub strs
                    mView = Nothing
                    msub  = Nothing
    \end{verbatim}
 
   When we look at the data type as defined in the parse tree, the following order is defined:
   \begin{verbatim}
         data P_ObjDef a =
                  P_Obj { obj_nm :: String          
                             , obj_pos :: Origin  
                             , obj_ctx :: Term a
                             , obj_mView :: Maybe String 
                             , obj_msub :: Maybe (P_SubIfc a)  
                             , obj_strs :: [[String]]
                             } 
    \end{verbatim}

    As the order of the parameters in the parse tree and the parser are different, the intermediate function \texttt{rebuild} is used to align both types to each other.
   Our personal experience was that these rebuild functions make the parser code more difficult to understand and hence, to maintain.
   To further decrease the code complexity, we suggest to try to eliminate these rebuild functions by restructuring some data types in the parse tree.

  \item[Manual overrule of error message]
    Our analysis of the new error messages showed that the quality of these improved distinctively.
    Parsec provides to possibility to overrule the standard Parsec error message by an own formulated message.
    During our implementation we decided to stick to the standard if the standard error message was, at least, sufficient.
    If the Ampersand teams want to tweak some error message after all, this can be realized using the \texttt{<?>} Parsec function.
    Placing the \texttt{<?>} after the parser, followed by a string, changes the standard Parsec text after the keyword `expecting'.

\end{description}
