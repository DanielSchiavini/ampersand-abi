\documentclass[a4paper,12pt,abstracton,titlepage]{scrartcl}
\usepackage{scrpage2}
\usepackage[utf8]{inputenc}
\usepackage[T1]{fontenc}
\usepackage[top=2.5cm, bottom=2.5cm, left=2cm, right=2cm]{geometry}
\usepackage[affil-it]{authblk}
\usepackage{lipsum}
\usepackage{url}
\usepackage[hidelinks]{hyperref}
\usepackage{graphicx}
\usepackage[table,xcdraw]{xcolor}
\usepackage{longtable}
\usepackage{multicol}

%citations
\usepackage{multibib}
\newcites{ac}{Academic references}
\newcites{nac}{Informal references}

% code for generating glossary, from http://tex.stackexchange.com/a/5837/59718
\usepackage[acronym,toc]{glossaries}
\usepackage{glossary-mcols}
\newcommand{\dict}[2]{%
  \newglossaryentry{#1}{name=#1,description={#2}}%
  \glslink{#1}{}%
}
\makeglossaries

% Here we set up the header, meta-information and front matter
\date{December 16, 2014}      %// Today's date will appear when this is commented out.
\newcommand{\version}{1.0}

% title page
\author{Daniel S. C. Schiavini}
\affil{Open Universiteit Nederland, faculteit Informatica \\
	T61327 - Afstudeerproject bachelor informatica}
\title{Project Documentation}
\subtitle{Useful feedback in the Ampersand parser\\
	~\\
	Phase 3d}
\publishers{Version \version}

% header
\pagestyle{scrheadings}
\setheadsepline{0.2pt}
\clearscrheadings
\automark[section]{chapter}
\ihead{Daniel S.C. Schiavini en Maarten Baertsoen}
\ohead{Ampersand Parser: Project documentation}
\cfoot{\pagemark}

% URL's
\renewcommand*{\UrlFont}{\footnotesize\ttfamily}

% hyphenation
\hyphenation{
	gua-ran-tee
	pro-duct
	cor-res-pon-ding
	me-cha-nism
	know-ledge
	de-ve-lo-pers
	do-cu-men-ta-tion
	sa-tis-fac-tion
	Schi-a-vi-ni
	Ba-ert-so-en}

% Now the document starts
\begin{document}
\maketitle
\newpage

\tableofcontents
%\listoffigures
%\listoftables
\clearpage

% !TEX root = ../Documentation.tex
\section{Introduction}
\subsection{Identification}
This document contains the domain \& techniques analysis of the project `Useful feedback in the Ampersand parser'.
The document is the milestone product of the project phase 3a for Daniel S.C. Schiavini, as specified in the project planning \citenac{plan}.

This document is part of the graduation project of the computer science bachelor at the Open Universiteit Nederland.
The project `Useful feedback in the Ampersand parser' is executed in collaboration with Maarten Baertsoen, with support of the supervisor Dr. Bastiaan Heeren and examiner Prof.dr. Marko C.J.D. van Eekelen.
The assignment is given by Prof.dr. Stef Joosten, who researches how to further automate the design of business processes and information systems by the development of the Ampersand project.

Ampersand is an approach for the use of business rules to define the business processes.
Users describe the business rules in a formal language (ADL), and Ampersand compiles those rules into functional specification, documentation and working software
prototypes.
The main objective of this project is to improve the feedback and maintainability of the Ampersand parser.
See \citenac{plan} for more details on the project.

\subsection{Goals}
\lipsum[3]

\subsection{Document overview}
\lipsum[4]
% !TEX root = ../Documentation.tex

\section{Section}
\label{sec:section}
\lipsum[3]
\newpage
% !TEX root = ../Parsing.tex

\section{Conclusion}
\label{sec:conclusion}
In \autoref{sec:libraries}, the advice was given to use a combinator library for the new parser of Ampersand.
The main reason to avoid the parser generators is that it is hard to generate useful feedback.
Then, in \autoref{sec:errors}, it was made even more clear that besides generating good messages, those messages should also be customizable.

Therefore, the advice of this research is to use the combinator library that offers the highest level of customization in error messages, Parsec.
Although the uu-parsinglib seems to also be a very good choice, the experiences from the Helium compiler \citeac{helium-parser} should be also considered.
Besides, the Parsec library offers better support.

A list of important consideration points has also been collected through the literature and can be found in \autoref{sec:errors}, more specifically \ref{subsec:errors-ampersand}.

\part*{Appendices}
\addcontentsline{toc}{part}{Appendices}
\appendix
% !TEX root = ../Parsing.tex

\small
\printglossary[style=mcolindex,title=Glossary]
\label{sec:glossary}

\newpage
% !TEX root = ../Parsing.tex
\addcontentsline{toc}{section}{References}
\label{sec:bibliography}

\begin{thebibliography}{99}

\bibitem{plan}
	Planning for the project `Useful feedback in the Ampersand parser'\\
	Maarten Baertsoen and Daniel S. C. Schiavini\\
	Version 2.0 -- November 29, 2014\\
	\url{http://git.io/NeHuLg}

\bibitem{heeren-error}
	Top Quality Type Error Messages\\
	Bastiaan Heeren\\
	ISBN 90-393-4005-6, September 20, 2005\\
	\url{http://www.open.ou.nl/bhr/phdthesis}

\bibitem{monadic-parsing}
	Functional pearls -- Monadic Parsing in Haskell\\
	Graham Hutton (University of Nottingham) and Erik Meijer (University of Utrecht)\\
	\url{http://www.cs.nott.ac.uk/~gmh/monparsing.pdf}

\bibitem{convert-ebnf}
	 From EBNF to BNF \\
	 Christoph Zenger\\
	 June 4, 2000\\
	 \url{http://lampwww.epfl.ch/teaching/archive/compilation-ssc/2000/part4/parsing/node3.html}

\bibitem{bnf-ebnf}
	BNF and EBNF: What are they and how do they work?\\
	Lars Marius Garshol\\
	August 22, 2008\\
	\url{http://www.garshol.priv.no/download/text/bnf.html}

\bibitem{parser-examples}
	Haskell Parser Examples\\
	Geoff Hulette\\
	August 22, 2014\\
	\url{https://github.com/ghulette/haskell-parser-examples}

\bibitem{hugs-parser}
	Source code of the Hugs parser\\
	March 25, 2007\\
	\url{https://github.com/fuzxxl/Hugs/blob/master/src/parser.y}

\bibitem{ghc-parser}
	GHC: The Parser\\
	December 1, 2014\\
	\url{https://ghc.haskell.org/trac/ghc/wiki/Commentary/Compiler/Parser}
	%\url{https://ghc.haskell.org/trac/ghc/browser/ghc/compiler/parser/Parser.y}
	%https://www.haskell.org/pipermail/haskell-cafe/2013-August/109557.html

\bibitem{helium-parser}
	Helium, for Learning Haskell\\
	Bastiaan Heeren, Daan Leijen, Arjan van IJzendoorn\\
	Utrecht University\\
	\url{http://www.open.ou.nl/bhr/heeren-helium.pdf}
	
\bibitem{gcc-c-parser}
	GCC 4.1 Release Series Changes, New Features, and Fixes\\
	\url{https://gcc.gnu.org/gcc-3.4/changes.html}

\bibitem{gcc-cpp-parser}
	GCC 3.4 Release Series Changes, New Features, and Fixes\\
	\url{https://gcc.gnu.org/gcc-4.1/changes.html}
	
\end{thebibliography}

\end{document}