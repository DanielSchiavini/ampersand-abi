% !TEX root = ../Documentation.tex
\section{Introduction}
\subsection{Identification}
This document contains the documentation of the project `Useful feedback in the Ampersand parser'.
The document is the milestone product of the project phase 3d and contains the deliverables `analysis \& design' and `test report', as described in the project planning \citenac{plan}.

This document is part of the graduation project of the computer science bachelor at the Open Universiteit Nederland.
The project `Useful feedback in the Ampersand parser' is executed in collaboration with Maarten Baertsoen, with support of the supervisor Dr. Bastiaan Heeren and examiner Prof.dr. Marko C.J.D. van Eekelen.
The assignment is given by Prof.dr. Stef Joosten, who researches how to further automate the design of business processes and information systems by the development of the Ampersand project.

Ampersand is an approach for the use of business rules to define the business processes.
Users describe the business rules in a formal language (ADL), and Ampersand compiles those rules into functional specification, documentation and working software prototypes.
The main objective of this project is to improve the feedback and maintainability of the Ampersand parser.
See \citenac{plan} for more details on the project.

\subsection{Goals}
The goals of this document are to provide enough documentation to support the further development of the Ampersand parser after this project is concluded.
In order to provide sufficient knowledge, the following information is given:
\begin{itemize}
  \item Describe the analysis of the new Ampersand parser;
  \item Explain the design decisions taken during the analysis and development;
  \item Show how the solution fulfills the project requirements;
  \item Describe how the system was tested in a test report.
\end{itemize}

\subsection{Document overview}
This initial section gives an introduction to the document.
In \autoref{sec:analysis}, the analysis of the parser is described.
Afterward, in \autoref{sec:design}, the software design is depicted, and in \autoref{sec:tests} the test report is found.
A short conclusion is given in \autoref{sec:conclusion}

Finally, in the appendix, a glossary of terms, definitions and abbreviations is given, just as a list of references.
