\haddockmoduleheading{Database.Design.Ampersand.Basics.Auxiliaries}
\label{module:Database.Design.Ampersand.Basics.Auxiliaries}
\haddockbeginheader
{\haddockverb\begin{verbatim}
module Database.Design.Ampersand.Basics.Auxiliaries (
    eqClass,  eqCl,  getCycles,  transClosureMap,  combinations,  converse, 
    commaEng,  commaNL,  fst3,  snd3,  thd3,  Flippable(flp),  showTrace, 
    showTraceTag,  blockParenthesize,  addToLastLine,  indent
  ) where\end{verbatim}}
\haddockendheader

\begin{haddockdesc}
\item[\begin{tabular}{@{}l}
eqClass\ ::\ (a\ ->\ a\ ->\ Bool)\ ->\ {\char 91}a{\char 93}\ ->\ {\char 91}{\char 91}a{\char 93}{\char 93}
\end{tabular}]\haddockbegindoc
The \haddockid{eqClass} function takes an equality test function and a list and returns a list of lists such

 that each sublist in the result contains only equal elements, and all equal elements are in

 the same sublist.  For example,
\par
Example> eqClass \haddocktt{Mississippi} = {\char 91}\haddocktt{M},"iiii","ssss","pp"{\char 93}
\par

\end{haddockdesc}
\begin{haddockdesc}
\item[\begin{tabular}{@{}l}
eqCl\ ::\ Eq\ b\ =>\ (a\ ->\ b)\ ->\ {\char 91}a{\char 93}\ ->\ {\char 91}{\char 91}a{\char 93}{\char 93}
\end{tabular}]\haddockbegindoc
eqCl is used for gathering things that are equal wrt some criterion f.

   For instance, if you want to have persons with the same name:

    'eqCl name persons' produces a list,in which each element is a list of persons with the same name.

 Example> eqCl (==\haddocktt{s}) \haddocktt{Mississippi} = "ssss"
\par

\end{haddockdesc}
\begin{haddockdesc}
\item[\begin{tabular}{@{}l}
getCycles\ ::\ Eq\ a\ =>\ {\char 91}(a,\ {\char 91}a{\char 93}){\char 93}\ ->\ {\char 91}{\char 91}a{\char 93}{\char 93}
\end{tabular}]\haddockbegindoc
getCycles returns a list of cycles in the edges list (each edge is a pair of a from-vertex

   and a list of to-vertices)
\par

\end{haddockdesc}
\begin{haddockdesc}
\item[\begin{tabular}{@{}l}
transClosureMap\ ::\ (Eq\ a,\ Ord\ a)\ =>\ Map\ a\ {\char 91}a{\char 93}\ ->\ Map\ a\ {\char 91}a{\char 93}
\end{tabular}]\haddockbegindoc
Warshall's transitive closure algorithm
\par

\end{haddockdesc}
\begin{haddockdesc}
\item[\begin{tabular}{@{}l}
combinations\ ::\ {\char 91}{\char 91}a{\char 93}{\char 93}\ ->\ {\char 91}{\char 91}a{\char 93}{\char 93}
\end{tabular}]\haddockbegindoc
The \haddockid{combinations} function returns all possible combinations of lists of list.

 For example,
\par
\begin{quote}
{\haddockverb\begin{verbatim}
combinations [[1,2,3],[10,20],[4]] == [[1,10,4],[1,20,4],[2,10,4],[2,20,4],[3,10,4],[3,20,4]]
\end{verbatim}}
\end{quote}

\end{haddockdesc}
\begin{haddockdesc}
\item[
converse\ ::\ forall\ a\ b.\ (Ord\ a,\ Ord\ b)\ =>\ {\char 91}(a,\ {\char 91}b{\char 93}){\char 93}\ ->\ {\char 91}(b,\ {\char 91}a{\char 93}){\char 93}
]
\item[
commaEng\ ::\ String\ ->\ {\char 91}String{\char 93}\ ->\ String
]
\item[
commaNL\ ::\ String\ ->\ {\char 91}String{\char 93}\ ->\ String
]
\item[
fst3\ ::\ (a,\ b,\ c)\ ->\ a
]
\item[
snd3\ ::\ (a,\ b,\ c)\ ->\ b
]
\item[
thd3\ ::\ (a,\ b,\ c)\ ->\ c
]
\end{haddockdesc}
\begin{haddockdesc}
\item[\begin{tabular}{@{}l}
class\ Flippable\ a\ where
\end{tabular}]\haddockbegindoc
\haddockpremethods{}\textbf{Methods}
\begin{haddockdesc}
\item[\begin{tabular}{@{}l}
flp\ ::\ a\ ->\ a
\end{tabular}]
\end{haddockdesc}
\end{haddockdesc}
\begin{haddockdesc}
\item[\begin{tabular}{@{}l}
instance\ Flippable\ Prop\\instance\ Flippable\ SrcOrTgt
\end{tabular}]
\end{haddockdesc}
\begin{haddockdesc}
\item[
showTrace\ ::\ Show\ a\ =>\ a\ ->\ a
]
\item[
showTraceTag\ ::\ Show\ a\ =>\ String\ ->\ a\ ->\ a
]
\item[
blockParenthesize\ ::\ String\\\ \ \ \ \ \ \ \ \ \ \ \ \ \ \ \ \ \ \ \ \ ->\ String\ ->\ String\ ->\ {\char 91}{\char 91}String{\char 93}{\char 93}\ ->\ {\char 91}String{\char 93}
]
\item[
addToLastLine\ ::\ String\ ->\ {\char 91}String{\char 93}\ ->\ {\char 91}String{\char 93}
]
\item[
indent\ ::\ Int\ ->\ {\char 91}String{\char 93}\ ->\ {\char 91}String{\char 93}
]
\end{haddockdesc}