\haddockmoduleheading{Database.Design.Ampersand.Basics.Version}
\label{module:Database.Design.Ampersand.Basics.Version}
\haddockbeginheader
{\haddockverb\begin{verbatim}
module Database.Design.Ampersand.Basics.Version (
    ampersandVersionStr,  ampersandVersionWithoutBuildTimeStr,  fatalMsg
  ) where\end{verbatim}}
\haddockendheader

This module contains Version of Ampersand
\par

\begin{haddockdesc}
\item[\begin{tabular}{@{}l}
ampersandVersionStr\ ::\ String
\end{tabular}]\haddockbegindoc
String, containing the Ampersand version, including the build timestamp.
\par

\end{haddockdesc}
\begin{haddockdesc}
\item[\begin{tabular}{@{}l}
ampersandVersionWithoutBuildTimeStr\ ::\ String
\end{tabular}]\haddockbegindoc
String, containing the Ampersand version
\par

\end{haddockdesc}
\begin{haddockdesc}
\item[\begin{tabular}{@{}l}
fatalMsg\ ::\ String\ ->\ Int\ ->\ String\ ->\ a
\end{tabular}]\haddockbegindoc
a function to create error message in a structured way, containing the version of Ampersand.

   It throws an error, showing a (module)name and a number. This makes debugging pretty easy.
\par

\end{haddockdesc}