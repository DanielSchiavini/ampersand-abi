% !TEX root = ../Parsing.tex

\section{Knowledge acquisition}
\label{sec:knowledge-acquisition}

\section{Haskell parsing libraries}
	The current Ampersand parser is created with the Utrecht University parser combinator library (UU.Parsing).
	This library uses monads and combinators and has quite some good features.
	There are, however, other libraries that can provide other feature sets.
	
	The purpose of this research is to choose the library best suited to the development of the new Ampersand parser.
	Some of the libraries available are:
	\begin{itemize}
		\item Helium
		\item Parsec
		\item Polyparse
		\item Happy parser generator
	\end{itemize}
	There are also advantages and disadvantages on using monads and combinators, which must be researched beforehand.
	The result of this research could also be that the UU.Parsing library is the best choice.
	One big advantage for this library is that the current code base can be very helpful.

	The baseline approach for researching the libraries will be to check whether they have all required features and their corresponding effectiveness.
	Also, existing tools built with the parsing library will be checked, specially the maintainability of the code and quality of the error messages.

\section{User-friendly error messages}
	The most important feature of the parser that will be built, is that it should generate user-friendly error messages.
	To understand what kinds of messages can be (and should be) generated, a research will be done on what good errors are and how to generate them.
	
	This is also related to the choice of the parsing library, since the chosen library should support the generation of good errors without extensive effort.
	Coincidently, a good source of knowledge are the papers of the supervisor, Bastiaan Heeren, who has done his PhD thesis on the generation of top quality type error messages\cite{heeren-error}.

\section{Conclusion}
\lipsum[1]