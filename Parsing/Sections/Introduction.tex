% !TEX root = ../Parsing.tex
\section{Introduction}
\subsection{Identification}
This document contains the domain \& and techniques analysis of the project `Useful feedback in the Ampersand parser'.
The document is the milestone product of the project phase 3a for the project member Daniel S.C. Schiavini, as specified in the project planning \cite{plan}.

This document is part of the graduation project of the computer science bachelor at the Open Universiteit Nederland.
The project `Useful feedback in the Ampersand parser' is executed in collaboration with Maarten Baertsoen, with support of the supervisor Dr. Bastiaan Heeren and examiner Prof.dr. Marko C.J.D. van Eekelen.
The assignment is given by Dr. Stef Joosten, who researches how to further automate the design of business processes and information systems by the development of the Ampersand project.

Ampersand is an approach for the use of business rules to define the business processes.
Users describe the business rules in a formal language (ADL), and Ampersand compiles those rules into functional specification, documentation and working software
prototypes.
See \cite{plan} for more details on the project.

\subsection{Goals}
The main objective of this phase is to gather information that will support the execution of the project.
This document contains the results of the research on domain and techniques that will support the project group.
It focuses on knowledge acquisition in two interrelated fronts:
\begin{description}
	\item[Haskell parsing libraries]
	In order to build the new Ampersand Parser, a small research will be done to choose the library best suited for the development.
	The appropriate library is chosen based on its design, documentation, features and generated errors.
	
	\item[User-friendly error messages]
	The most important feature of the parser that will be built, is that it should generate user-friendly error messages.
	To understand what kinds of messages can be (and should be) generated, a research will be done on what good errors are and how to generate them.
	This part of the research is done by consulting literature.
\end{description}
%
The results of both subjects culminate in a single section with research conclusions.

\subsection{Document overview}
An introduction is given is this chapter.
Then, in \autoref{sec:libraries} the choices of user-friendly error messages are elaborated.
In \autoref{sec:errors} the qualities of user-friendly error messages are quickly described, and in \autoref{sec:conclusion} a conclusion is given.

Finally, in the appendix, a glossary of terms, definitions and abbreviations is given, just as a list of references.
