% !TEX root = ../Parsing.tex
\section{Introduction}
\subsection{Identification}
This document contains the domain \& and techniques analysis of the project `Useful feedback in the Ampersand parser'.
The document is the milestone product of the project phase 3a for the project member Daniel S.C. Schiavini, as specified in the project planning \cite{plan}.

This document is part of the graduation project of the computer science bachelor at the Open Universiteit Nederland.
The project `Useful feedback in the Ampersand parser' is executed in collaboration with Maarten Baertsoen, with support of the supervisor Dr. Bastiaan Heeren and examiner Prof.dr. Marko C.J.D. van Eekelen.
The assignment is given by Dr. Stef Joosten, who researches how to further automate the design of business processes and information systems by the development of the Ampersand project.

\subsection{Goal of this document}
The main objective of this phase is to gather information that will support the execution of the project.
This document is the description of the lessons learned that will be shared within the project group.
It focuses on knowledge acquisition in two fronts:
\begin{description}
	\item[Haskell parsing libraries]
	In order to build the new Ampersand Parser, a small research will be done to choose the library best suited.
	The appropriate library will be chosen based on its design, features and generated errors.
	
	\item[User-friendly error messages]
	The most important feature of the parser that will be built, is that it should generate user-friendly error messages.
	To understand what kinds of messages can be (and should be) generated, a research will be done on what good errors are and how to generate them.
\end{description}
%
Each subject is elaborated in a separate section, and a conclusion is given over both subjects.

TODO: motivation & background, research question, used methods\\

\subsection{Document overview}
An introduction is given is this chapter.
Then, in \autoref{sec:libraries} the choices of user-friendly error messages are elaborated.
In \autoref{sec:errors} the qualities of user-friendly error messages are quickly described, and in \autoref{sec:conclusion} a conclusion is given.

Finally, in the appendix, a glossary of terms, definitions and abbreviations is given, just as a list of references.
