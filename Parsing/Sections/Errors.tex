% !TEX root = ../Parsing.tex

\section{User-friendly error messages}
\label{sec:errors}

\subsection{Parsing errors}
When a parser is executed and is unable to recognize the input text, an error is raised.
The main problem when this happens, is that the parser cannot know for sure what the programmer meant to write.
Only the programmer can know exactly what the purpose of the invalid input was.

However, parsers should be able to recognize the most common errors, and support the user to correct them.
Spenke et al. \citeac{error-recovery} discuss the following assumptions regarding parsing errors:
\begin{enumerate}
	\item An incorrect program is very similar to a correct one;
	\item An error is very soon followed by a correct piece of program;
	\item There are symbols (e.g. keywords) that are omitted quite rarely, while some less important symbols are more frequently omitted (e.g. semicolons);
	\item There are some very reliable symbols which most likely do not occur by accident, but always in an specific context (e.g. then being part of an if statement);
	\item  If an error cannot be corrected by deletion and/or insertion of a few symbols, the reason is often a complete, misplaced syntactic unit, such as a whole expression where only a single constant is allowed;
	\item A frequent reason for syntax errors is typing errors. In addition, similar basic symbols, such as round and square brackets may easily be confused.
\end{enumerate}

\subsection{User-friendliness}
There is no formal definition of what user-friendly error messages are.
Indeed, each kind of user is different and may expect something else from the system.
For example, an expert programmer can understand much more of technical implementation details than an inexperienced student.

Therefore, it may be very important to be able to fine-tune the errors shown.
The developers of the compiler may want details of the inner workings of the system (e.g. the parse tree).
Students, on the other hand, may just want to see the most likely cause of their mistake.

\subsection{Implications for the Ampersand parser}
\label{subsec:errors-ampersand}
The following items have been identified as important in the error messages generated by parsers:
\begin{description}
	\item[Location] When an error is detected, it is crucial to point out where in the input text the error has been found.
		If the location is incorrect, the user will have to search through his/her entire program to find the syntax error \citeac{helium-parser,uu-doc,error-correcting,parsec}.
  \item[Production rules] Listing which product rules are applicable at that moment can help the user to choose the correct one \citeac{helium-parser}.
		However, students have been reported to get overwhelmed by the huge list of terminals currently generated by Ampersand in the case of errors \citeac{heeren-error}. 
	\item[Misspellings] Very often, errors happen because the users mistype keywords and/or symbols.
		Tokens that are invalid, but very close to valid input and often misspelled, should be recognized by the parser \citeac{helium-parser,error-recovery}.
	\item[Error recovery] When an error occurs, it is appropriate that the parser is able to recover from that error.
		This way, the parser can continue analyzing the input \citeac{error-correcting,error-recovery}.
		In some cases it may be wished to display all the different errors, and in some cases only the first one.
	\item[Error output] Most compilers output errors as plain text.
		However, other formats may be more appropriated for communicating with the user, e.g. syntax diagrams or HTML.
		Other formats are only possible if the development environment can correctly display the errors.
\end{description}
