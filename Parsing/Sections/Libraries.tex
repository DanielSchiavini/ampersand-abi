% !TEX root = ../Parsing.tex

\section{Haskell parsing libraries}
\label{sec:libraries}
\dict{UU.Parsing}{Haskell parsing library from the Utrecht University}%
The current Ampersand parser is created with the Utrecht University parser combinator library (UU.Parsing).
This library uses monads and combinators and has quite some good features.
There are, however, other libraries that can provide other feature sets.

The purpose of this research is to choose the library best suited to the development of the new Ampersand parser.
Some of the libraries available are:
\begin{itemize}
	\item Helium
	\item Parsec
	\item Polyparse
	\item Happy parser generator
\end{itemize}
There are also advantages and disadvantages on using monads and combinators, which must be researched beforehand.
The result of this research could also be that the UU.Parsing library is the best choice.
One big advantage for this library is that the current code base can be very helpful.

The baseline approach for researching the libraries will be to check whether they have all required features and their corresponding effectiveness.
Also, existing tools built with the parsing library will be checked, specially the maintainability of the code and quality of the error messages.
