\documentclass[a4paper,12pt,abstracton,titlepage]{scrartcl}
\usepackage{fancyhdr}
\usepackage[utf8]{inputenc}
\usepackage[T1]{fontenc}
\usepackage[top=2.5cm, bottom=2.5cm, left=2.5cm, right=2.5cm]{geometry}
\usepackage[affil-it]{authblk}
\usepackage{lipsum}
\usepackage[hidelinks]{hyperref}
\usepackage{graphicx}
\usepackage[table,xcdraw]{xcolor}
\usepackage{longtable}

% code for generating glossary, from http://tex.stackexchange.com/a/5837/59718
\usepackage[acronym,toc]{glossaries}
\newcommand{\dict}[2]{%
  \newglossaryentry{#1}{name=#1,description={#2}}%
  \glslink{#1}{}%
}
\makeglossaries

% Here we set up the header, meta-information and front matter
%\date{November 3, 2014}      %// Today's date will appear when this is commented out.
\newcommand{\version}{0.1}

\author{Daniel S. C. Schiavini}
\affil{Open Universiteit Nederland, faculteit Informatica \\
	T61327 - Afstudeerproject bachelor informatica}
\title{Haskell parsing libraries \& User-friendly error messages}
\subtitle{Fase 3a: Domain \& Techniques}
\publishers{Version \version}

\pagestyle{fancy}
\lhead{D.S.C. Schiavini}
\rhead{Parsing libraries \& error messages}
\cfoot{\thepage}
\setlength{\headheight}{15pt}

% requirements
\newcommand{\req}[3]{\noindent\textbf{\texttt{#1} #2}\\#3}

% hyphenation
\hyphenation{
	gua-ran-tee
	pro-duct
	cor-res-pon-ding
	me-cha-nism
	know-ledge
	de-ve-lo-pers
	do-cu-men-ta-tion
	Schi-a-vi-ni
	Ba-ert-so-en}

% Now the document starts
\begin{document}
\maketitle
\newpage

\tableofcontents
\listoffigures
\listoftables
\clearpage

% !TEX root = ../Documentation.tex
\section{Introduction}
\subsection{Identification}
This document contains the domain \& techniques analysis of the project `Useful feedback in the Ampersand parser'.
The document is the milestone product of the project phase 3a for Daniel S.C. Schiavini, as specified in the project planning \citenac{plan}.

This document is part of the graduation project of the computer science bachelor at the Open Universiteit Nederland.
The project `Useful feedback in the Ampersand parser' is executed in collaboration with Maarten Baertsoen, with support of the supervisor Dr. Bastiaan Heeren and examiner Prof.dr. Marko C.J.D. van Eekelen.
The assignment is given by Prof.dr. Stef Joosten, who researches how to further automate the design of business processes and information systems by the development of the Ampersand project.

Ampersand is an approach for the use of business rules to define the business processes.
Users describe the business rules in a formal language (ADL), and Ampersand compiles those rules into functional specification, documentation and working software
prototypes.
The main objective of this project is to improve the feedback and maintainability of the Ampersand parser.
See \citenac{plan} for more details on the project.

\subsection{Goals}
\lipsum[3]

\subsection{Document overview}
\lipsum[4]
\section{Knowledge acquisition (-)}
\label{sec:knowledge-acquisition}

\subsection{Research context}
Some ideas by Bastiaan:
\begin{itemize}
  \item more info on ampersand users
  \item user-friendly error messages (specific to compilers or not)
  \item parser-libraries
  \item Stef's research (3a)
  \item Haskell parsers (e.g. Helium), monadit or combinators
\end{itemize}

\subsection{Domain \& technology}
\subsubsection{Part Daniel}
\lipsum[1]

\subsubsection{Part Maarten}
\lipsum[1]

\subsection{Knowledge documentation}
\lipsum[1]

\newpage
\appendix
% !TEX root = ../Parsing.tex

\small
\printglossary[style=mcolindex,title=Glossary]
\label{sec:glossary}

\clearpage
% !TEX root = ../Parsing.tex
\addcontentsline{toc}{section}{References}
\label{sec:bibliography}

\begin{thebibliography}{99}

\bibitem{plan}
	Planning for the project `Useful feedback in the Ampersand parser'\\
	Maarten Baertsoen and Daniel S. C. Schiavini\\
	Version 2.0 -- November 29, 2014\\
	\url{http://git.io/NeHuLg}

\bibitem{heeren-error}
	Top Quality Type Error Messages\\
	Bastiaan Heeren\\
	ISBN 90-393-4005-6, September 20, 2005\\
	\url{http://www.open.ou.nl/bhr/phdthesis}

\bibitem{monadic-parsing}
	Functional pearls -- Monadic Parsing in Haskell\\
	Graham Hutton (University of Nottingham) and Erik Meijer (University of Utrecht)\\
	\url{http://www.cs.nott.ac.uk/~gmh/monparsing.pdf}

\bibitem{convert-ebnf}
	 From EBNF to BNF \\
	 Christoph Zenger\\
	 June 4, 2000\\
	 \url{http://lampwww.epfl.ch/teaching/archive/compilation-ssc/2000/part4/parsing/node3.html}

\bibitem{bnf-ebnf}
	BNF and EBNF: What are they and how do they work?\\
	Lars Marius Garshol\\
	August 22, 2008\\
	\url{http://www.garshol.priv.no/download/text/bnf.html}

\bibitem{parser-examples}
	Haskell Parser Examples\\
	Geoff Hulette\\
	August 22, 2014\\
	\url{https://github.com/ghulette/haskell-parser-examples}

\bibitem{hugs-parser}
	Source code of the Hugs parser\\
	March 25, 2007\\
	\url{https://github.com/fuzxxl/Hugs/blob/master/src/parser.y}

\bibitem{ghc-parser}
	GHC: The Parser\\
	December 1, 2014\\
	\url{https://ghc.haskell.org/trac/ghc/wiki/Commentary/Compiler/Parser}
	%\url{https://ghc.haskell.org/trac/ghc/browser/ghc/compiler/parser/Parser.y}
	%https://www.haskell.org/pipermail/haskell-cafe/2013-August/109557.html

\bibitem{helium-parser}
	Helium, for Learning Haskell\\
	Bastiaan Heeren, Daan Leijen, Arjan van IJzendoorn\\
	Utrecht University\\
	\url{http://www.open.ou.nl/bhr/heeren-helium.pdf}
	
\bibitem{gcc-c-parser}
	GCC 4.1 Release Series Changes, New Features, and Fixes\\
	\url{https://gcc.gnu.org/gcc-3.4/changes.html}

\bibitem{gcc-cpp-parser}
	GCC 3.4 Release Series Changes, New Features, and Fixes\\
	\url{https://gcc.gnu.org/gcc-4.1/changes.html}
	
\end{thebibliography}

\end{document}