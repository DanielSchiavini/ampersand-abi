% !TEX root = ../ResearchContext.tex

\section{Questions}
\label{sec:questions}
The following questions have been asked to a researched involved in the research, namely Lloyd Rutledge.
His answers are given after each of the questions.

~\\
\question{What are the largest user complaints when using Ampersand?}
\answer{~}

\question{Do users appreciate the error correction done by the previous parser?}
\answer{~}
  
\question{May better user feedback allow for a more efficient research on formal business rules?}
\answer{~}

\question{Has the maintainability of the previous parser been an issue in our research?}
\answer{~}
  
\question{How can a new Ampersand parser, and a better user feedback, support the objectives of the Ampersand project?}
\answer{~}

\question{Ampersand is nu reeds een antwoord op gekende issues van toepassing op methodologien mbt business requirements (praktische bruikbaarheid, volledigheid, gebruik van natuurlijke taal, opleiding, ...) Tevens staan bijkomende uitbreidingen op de agenda (agile, responsive UI op basis van een repository,...)\\
Wat zijn volgens jou (Stef) de komende uitdagingen die eraan komen en wat is de visie hoe Ampersand daarin past?\\
--> bijvoorbeeld: nu wordt er gebruik gemaakt van natuurlijke taal die volgens een vaste grammatica verwerkt wordt, de kennis en achterliggende IT processen om natuurkijke taal te interpreteren volgens midner strikte structuren wordt alsmaar sterker, kan het een evolutie zijn om al een interpretatie te doen op bbasis van gesproken tekst (ter vorming van de basis)}
\answer{~}

\question{Ampersand is reeds sterk en zal alsmaar sterker worden, wat is de visie en aanpak hoe Ampersand verder in de 'markt' gepositioneerd zal worden?}
\answer{~}

\question{Is er de intentie om Ampersand te commercialiseren? Of, minder financieel gefocust, in een soort vzw onder gebracht van waaruit een (semi) professionele support gegeven kan worden aan de methodologie en tools (cfr PMI, ITIL,…)}
\answer{~}

\question{Waar kunnen we binnen het ABI project rekening mee houden om de toekomstvisie van Ampersand te ondersteunen? (maintainability van de code is bv al 1 aspect)}
\answer{~}

\question{De drempel om bij te dragen aan de Ampersand tools is eerder hoog: Los van de methode is er ook de praktische utiwerking. Haskell is op zich al geen courante en eenvoudige taal, en de uitwerking is bij momenten behoorlijk complex. Gezien het open-source aspect en de ambitieuze roadmap: wat is de visie om contributors aan te trekken en effectief hun weg te laten vinden in de code, project...}
\answer{~}

\question{Er bestaan reeds een aantal parsers, nu gaan we Parsec inbouwen, maar de kans is zeker bestaande dat er binnen x aantal jaar opnieuw geavanceerdere parsers uitgewerkt zullen worden, wat is de visie hoe we deze evolutie momenteel al kunnen supporteren? Zal Ampersand in de toekomst, afhankelijk van de context, de mogelijkheid moeten bieden om verschillende parsers toe te passen (bv via een selectie in een menuitem)}
\answer{~}

\question{Mogelijke onderzoeksvraag: Formele talen beloven een vorm van zekerheid op het vlak van correctheid, maar hoe correct de software ook is, fouten met oorsprong in analyse en het achterliggende redeneren kunnen een perfect formeel model vormen, zonder het gewenste resultaat. De Focus van ampersand ligt op het faciliteren van het beredeneren tijdens de analyse fase. Kan een betere foutafhandeling in ampersand ertoe bijdragen dat dit aspect binnen formele talen beter afgedekt wordt en formele talen niet enkel een juist maar ook een gewenst resultaat bekomen.}
\answer{~}
