% !TEX root = ../ResearchContext.tex

\section{Expert consultation}
\label{sec:expert-consultation}
The following questions have been formulated during and after the literature research.
The questions have been asked to a researcher involved, namely Lloyd Rutledge.
Lloyd is currently an Assistant Professor at the Open Universiteit.
During the literature study, Lloyd already helped us finding interesting investigation topics and possible links to other research contexts during two alignment sessions.
The questions below are presented in the order they are listed below.
His answers are recorded literally after each of the questions, our own perception of the answers in relation to the research context are summarized in the conclusion, \autoref{sec:conclusion}.

~\\
\question{What do you use Ampersand for?}
\answer{
Students in our Masters course Rule-Based Design use Ampersand to implement business rules from case studies as a course assignment, which determines a large part of their course grade.
I also supervise several Masters Thesis students who perform their own research using Ampersand.
They often implement a case study in Ampersand to test hypothesis about how to establish business rules.
Our main current research theme is design patters for business rules.
We are starting a series of Masters Thesis research projects that propose rule design patterns and evaluate them in various case studies.
}

\question{What are the largest user complaints when using Ampersand?}
\answer{
Mostly complaints regard lack of documentation.
Inconsistency between new versions of the software and the older software that exists is also a concern.
}

\question{Do you know whether users appreciate the error correction done by the previous parser?}
\answer{
No, I do not.
}

\question{Would better user feedback allow for a more efficient research on formal business rules?}
\answer{
I see no direct relation between user feedback and research.
It may have an indirect effect by making all work with Ampersand easier.
}

\question{Has the maintainability of the previous parser been an issue in your research?}
\answer{
I have had no complaints from Masters Thesis researchers, just as the course students have not complained to me about the parser.
It could be because the students are often business and management students, and thus the course may often be seen as a management theory course.
And then the course is secondly seen as the application of formal logic to business rules.
But on that level it is more a mathematical exercise than a computer project.
That there is software to process the rules into charts may be seen by many students as an secondary tool for helping establish the logic formalisms and the more strictly written business rules in natural language.

The research that I directly execute is more about the Semantic Web, its logic and its tools.
}
  
\question{How can a new Ampersand parser and a better user feedback support the objectives of the Ampersand project?}
\answer{
By helping users more quickly write Ampersand scripts that function as expected and that deliver the desired insights.
}

\question{Is a better user feedback sufficient to optimize the learning path of students?
Or should a syntax simplification be considered for this target group?}
\answer{
There are too many aspects for optimizing learning paths to state that one tool would be sufficient.
Syntax simplification is another research area.
It has often been applied to business rules, both as non-computer and computer-processed syntaxes.
Our research's logical foundation is relation algebra, and so for us any simplified syntax would have to be equivalent to relation algebra.
Keywords may be easier in the beginning to work with than logical symbols.
But in the long run it is the form and structure of the logic itself that students need to learn to master.
}

\question{What are the next challenges for your research using the Ampersand tool?}
\answer{
My own direct research involves the relation between Ampersand, and thus relation algebra, to Semantic Web languages, and thus the tools that implement them, and also thus the logical formalism they're based on.
I will explore how Semantic Web tools can set up the equivalent of the business rules that Ampersand sets up.
Semantic Web tools such as Protégé have rich user interfaces such as GUI's for authoring data models and rules.
Parser concerns become then no longer relevant because the GUI interface restricts user interaction to making code that has to be valid.
If all Ampersand functionality could be programmed with such a tool, then many of the concerns motivated the Ampersand parser are addressed.
}

\question{Is it harder for you to define the business rules or to express and compile them in ADL?}
\answer{
Forming the logical structure is often a larger intellectual challenge then the grammar.
}

\question{Would you like to suggest any improvements to the ADL grammar?}
\answer{
A tool could replace the symbols with easily understood natural language keywords, but that only helps in the beginning.
}

\question{Is Ampersand the right tool for your research?}
\answer{
It is the right tool for my Masters students who research relation algebra.
My direct research involves the Semantic Web more, and in particular the relation between relation algebra and the Semantic Web.
Relation algebra is an expressive superset of the Semantic Web in terms of logic formalism.
Thus where I rely mostly on Ampersand is to explore the logic that the Semantic Web does directly support.
}
