% !TEX root = ../ResearchContext.tex

\section{Introduction}
\subsection{Identification}
This document contains the research context of the project `Useful feedback in the Ampersand parser'.
The document is the milestone product of the project phase 3b, as specified in the project planning \citenac{plan}.

This document is part of the graduation project of the computer science bachelor at the Open Universiteit Nederland.
The project `Useful feedback in the Ampersand parser' has the support of the supervisor Dr. Bastiaan Heeren and examiner Prof.dr. Marko C.J.D. van Eekelen.
The assignment is given by Prof.dr. Stef Joosten, who researches how to further automate the design of business processes and information systems by the development of the Ampersand project.

Ampersand is an approach for the use of business rules to define the business processes.
Users describe the business rules in a formal language (ADL), and Ampersand compiles those rules into functional specification, documentation and working software
prototypes.
The main objective of this project is to improve the feedback and maintainability of the Ampersand parser.
See \citenac{plan} for more details on the project.

\subsection{Goal of this document}
In this document we will investigate how the new Ampersand parser influences its research context.
In order to proceed with the investigation, we identify in which contexts the project is executed and for which context, what is its influence.

\subsection{Document overview}
An introduction is given is this section.
Then, in \autoref{sec:research} the short research on the subject is explained.
In \autoref{sec:questions} some questions are elaborated for a researcher.
In \autoref{sec:answers} the responses to these questions are given, followed by a short conclusion in \autoref{sec:conclusion}.

Finally, in the appendix, a glossary of terms, definitions and abbreviations is given, just as a list of references.
