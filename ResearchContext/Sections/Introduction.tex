% !TEX root = ../ResearchContext.tex

\section{Introduction}
\subsection{Identification}
This document contains the research context of the project `Useful feedback in the Ampersand parser'.
The document is the milestone product of the project phase 3b, as specified in the project planning \citenac{plan}.

This document is part of the graduation project of the computer science bachelor at the Open University of the Netherlands.
The project `Useful feedback in the Ampersand parser' is supervised by Dr. Bastiaan Heeren and examined by Prof.dr. Marko C.J.D. van Eekelen.
The assignment is given by Prof.dr. Stef Joosten, who researches how to further automate the design of business processes and information systems by the development of the Ampersand project.

Ampersand is an approach for the use of business rules to define an information system.
Users describe the business rules in a formal language (ADL), and Ampersand compiles those rules into functional specification, documentation and working software prototypes. 
By using natural language statements, business oriented users can better formulate and evaluate the stated requirements.
After the automatic conversion of these requirements, the stakeholders can better check and understand the actual needs of the new information system.
The main objective of this project is to improve the feedback and maintainability of the Ampersand parser.
See the project planning \citenac{plan} for more details on the project.

\subsection{Goal of this research}
In this project phase we will identify research contexts related to the Ampersand approach as a whole.
In order to proceed with the investigation, we identify the relevant contexts and for each context the corresponding influence of this project on the bigger picture is described.

The starting point of the investigation is done through literature research, allowing us to understand the actual research context.
For each research context, we identify tangent points with the Ampersand project.

During the literature study, key questions are formulated which are presented afterwards, together with the actual literature study, to a consult researcher.
For our research context investigation, we consulted the expert researcher Lloyd Rutledge, who was willing to help us widen our horizon.


\subsection{Document overview}
An introduction is given is this section.
The identification of potential investigation areas is done based on our previous knowledge gathered in phase 3b of the project (Domain \& Techniques) \citenac{parsing,ampersand-approach}, besides brainstorming with our project supervisor and our expert researcher.
In \autoref{sec:literature-study}, for each interesting research context we identified, the literature research on the subject is described.
In \autoref{sec:expert-consultation} the formulated questions towards our expert researcher are listed, including his responses.
This is followed by a short conclusion in \autoref{sec:conclusion} in which we reflect on the actual influence Ampersand has on his surrounding research contexts.

Finally, in the appendix, a glossary of terms, definitions and abbreviations is given, just as a list of references.
