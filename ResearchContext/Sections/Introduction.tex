% !TEX root = ../ResearchContext.tex

\section{Introduction}
\subsection{Identification}
This document contains the research context of the project `Useful feedback in the Ampersand parser'.
The document is the milestone product of the project phase 3b, as specified in the project planning \citenac{plan}.

This document is part of the graduation project of the computer science bachelor at the Open Universiteit Nederland.
The project `Useful feedback in the Ampersand parser' has the support of the supervisor Dr. Bastiaan Heeren and examiner Prof.dr. Marko C.J.D. van Eekelen.
The assignment is given by Prof.dr. Stef Joosten, who researches how to further automate the design of business processes and information systems by the development of the Ampersand project.

Ampersand is an approach for the use of business rules to define the supporting information system.
Users describe the business rules in a formal language (ADL), and Ampersand compiles those rules into functional specification, documentation and working software prototypes. 
By using natural language statements, business oriented user can better formulate and evaluate the stated requirements.
After the automatic conversion of these requirements, the system engineers can better check and understand the actual needs of the new information system.
The main objective of this project is to improve the feedback and maintainability of the Ampersand parser.
See the project planing \citenac{plan} for more details on the project.

\subsection{Goal of this research}
In this project phase we will identify research contexts related to the Ampersand approach as a whole.
In order to proceed with the investigation, we identify the interesting contexts and for each context, the corresponding influence of this project on the bigger picture.

The starting point of the investigation is done through literature research allowing us to understand the actual research context.
For each research context, we identify tangent points with the Ampersand project.

During the literature study, key questions are formed which are presented, together with the actual literature study, to a consult researcher.
For our research context investigation, we found researcher Lloyd Rutledge willing to help us widening our horizon.


\subsection{Document overview}
An introduction is given is this section.
Then, in \autoref{sec:research} for each interesting research context, the literature research on the subject is explained.
In \autoref{sec:questions} some questions are elaborated for a researcher and their responses are given.
This is followed by a short conclusion in \autoref{sec:conclusion}.

Finally, in the appendix, a glossary of terms, definitions and abbreviations is given, just as a list of references.
