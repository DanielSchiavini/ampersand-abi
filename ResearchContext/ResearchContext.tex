\documentclass[a4paper,12pt,abstracton,titlepage]{scrartcl}
\usepackage{scrpage2}
\usepackage[utf8]{inputenc}
\usepackage[T1]{fontenc}
\usepackage[top=2.5cm, bottom=2.5cm, left=2cm, right=2cm]{geometry}
\usepackage[affil-it]{authblk}
\usepackage{lipsum}
\usepackage{url}
\usepackage[hidelinks]{hyperref}
\usepackage{graphicx}
\usepackage[table,xcdraw]{xcolor}
\usepackage{longtable}
\usepackage{multicol}

%citations
\usepackage{multibib}
\newcites{ac}{Academic references}
\newcites{nac}{Informal references}

% code for generating glossary, from http://tex.stackexchange.com/a/5837/59718
\usepackage[acronym,toc]{glossaries}
\usepackage{glossary-mcols}
\newcommand{\dict}[2]{%
  \newglossaryentry{#1}{name=#1,description={#2}}%
  \glslink{#1}{}%
}
\makeglossaries

% Here we set up the header, meta-information and front matter
\date{December 16, 2014}      %// Today's date will appear when this is commented out.
\newcommand{\version}{0.1}

% title page
\author{Daniel S. C. Schiavini and Maarten Baertsoen}
\affil{Open Universiteit Nederland, faculteit Informatica \\
	T61327 - Afstudeerproject bachelor informatica}
\title{Research Context}
\subtitle{Useful feedback in the Ampersand parser\\
	~\\
	Phase 3b}
\publishers{Version \version}

% header
\pagestyle{scrheadings}
\setheadsepline{0.2pt}
\clearscrheadings
\automark[section]{chapter}
\ihead{Daniel S.C. Schiavini}
\ohead{Parsing libraries \& error messages}
\cfoot{\pagemark}

% URL's
\renewcommand*{\UrlFont}{\footnotesize\ttfamily}

% hyphenation
\hyphenation{
	gua-ran-tee
	pro-duct
	cor-res-pon-ding
	me-cha-nism
	know-ledge
	de-ve-lo-pers
	do-cu-men-ta-tion
	sa-tis-fac-tion
	Schi-a-vi-ni
	Ba-ert-so-en}

% Now the document starts
\begin{document}
\maketitle
\newpage

\tableofcontents
\listoffigures
\listoftables
\clearpage

% !TEX root = ../Documentation.tex
\section{Introduction}
\subsection{Identification}
This document contains the domain \& techniques analysis of the project `Useful feedback in the Ampersand parser'.
The document is the milestone product of the project phase 3a for Daniel S.C. Schiavini, as specified in the project planning \citenac{plan}.

This document is part of the graduation project of the computer science bachelor at the Open Universiteit Nederland.
The project `Useful feedback in the Ampersand parser' is executed in collaboration with Maarten Baertsoen, with support of the supervisor Dr. Bastiaan Heeren and examiner Prof.dr. Marko C.J.D. van Eekelen.
The assignment is given by Prof.dr. Stef Joosten, who researches how to further automate the design of business processes and information systems by the development of the Ampersand project.

Ampersand is an approach for the use of business rules to define the business processes.
Users describe the business rules in a formal language (ADL), and Ampersand compiles those rules into functional specification, documentation and working software
prototypes.
The main objective of this project is to improve the feedback and maintainability of the Ampersand parser.
See \citenac{plan} for more details on the project.

\subsection{Goals}
\lipsum[3]

\subsection{Document overview}
\lipsum[4]
% !TEX root = ../ResearchContext.tex

\section{Short research}
\label{sec:research}
By investigating the requirements defined in the project planning \citenac{plan}, we identify here two contexts in which the new Ampersand parser is implemented.
First, we identify the Ampersand context, thus the research for the formal definition of business rules and the generation of documentation and software based  on it.
Second, we identify the parser context, thus the research for better user feedback in parsers.
In the following subsections each of the two contexts are described, including our findings during the literature research.

\subsection{Ampersand}
The Ampersand project has the ambition to offer a holistic methodology, including supporting tools, to support organizations during a software development project.
A new way of business requirements gathering, using natural language, is introduced by Ampersand.
We define the Ampersand approach by the Ampersand Methodology including the corresponding tools to create design artifacts and working software prototypes.

\subsubsection{Rule based design}

One of  the objectives of Ampersand is to bridge the gap between business analysts and system designers using natural language.
This objective is part of a general research context describing information systems based on business rules.

Kestutis Kapocius and Tomas Danikauskas \citeac{kapoc353} outline the use of an Output Driven Requirements Specification Method, ODRES, in combination with a Business Rules, BR, approach. 
In this approach, the BR approach formulates business rules in natural language using the Business Rules Solution, BRS, RuleSpeak model, Ronald Ross  \citeac{RuleSpeak}.
%TODO: good reference to add of RuleSpeak.
The Ampersand Language, describing the rules and syntax, can be seen as a formalization of the RuleSpeak guidelines and best practices.

\subsubsection{Ampersand target group context}
The Ampersand approach currently has 2 target groups: business analysts, in a commercial context, and students, in an educational context.

In the commercial context, the Ampersand users are typically business analysts who don't necessarily have programming experience.
The goal of these users is to specify, often complex, business rules in the most efficient way possible while still avoiding ambiguity.
To achieve their goals, the strictness of formal languages are a `necessary evil'.
Therefore, syntax errors can be very frustrating for the users.

Experiences within the Open Universiteit have shown that business analysts need roughly 100 hours to learn how to formalize business rules using Ampersand \citeac{joosten2007deriving}.
The learning costs can be thus quite high for organizations and students, and any improvement that can be made in the accessibility of Ampersand is welcome.

In the educational context, the Ampersand Users are students focusing on the domain of Rule Based Design. 
Within the course assignments, the students must check the completeness and consistency of their rule sets using Ampersand. 
 
Lex Wedemeijer \citeac{CSERC2013_Wedemijer} states that the instructional software must be powerful, yet basic.
The advanced features, offered to professional users, is not needed for students, moreover, the usage of a rich modeling features has a negative impact on the students learning curve.
The conclusion is made by Lex Wedemeijer that Ampersand may not be the perfect tool to support these student due to the richness of the language.
A more straightforward language, referred to as the Essential Language, is proposed, with clear and unambiguous notations, which is more suited to support students in the domain of Rule Based Design.

Providing a clearer error feedback in Ampersand does not lower the hurdle of the syntax richness, it is however a step in the good direction to better support the students. 
An interesting next step is the possibility to re-use the Ampersand vision and tools to support the easier to learn syntax of The Essential Language. 
To options to realize the re-use of Ampersand, is to simplify Ampersand towards the Essential Language or to implement a translator between both syntaxes.
Given the similarities between both languages, the translator seems the best approach, additional research is however needed to confirm this statement.

Andrei Lapets \citeac{lapets2009improving} concludes that user accessibility has not been a priority in the design of most formal verification systems.
He concludes that the syntax is often unfamiliar and/or introduces an entire new environment the user has to learn how to work with.
Finally, he points out that formal systems often offer a bottom-up structure that forces users to implement the complete set of assumptions in their research domain; this is made worse by the fact that usually few external libraries are available.

There are several measures that can be taken in order to improve the accessibility.
For example, the syntax should be familiar, simple and concrete \citeac{lapets2009improving}.
However, this project does not have any influence in the syntax.
The syntax of the ADL language is a given, so our influence is constrained to better user errors and development maintainability.

\dict{IDE}{Integrated development environment}
Pim Bos \citeac{bos2013bedrijfsregels} made a comparison between the tools for semantic web and Ampersand.
He concludes that the tools (e.g. script editors or an IDE) for supporting the development of Ampersand projects are lacking.
The fact that the ADL-compiler is currently the only feedback tool for users, stresses how important the feedback from the compiler is important.

Another issue named by Bos is the fact that the Ampersand user base is small.
A large user base is important for organizations using the tool, since they need to find specialists to implement and maintain the system.
It is also important for students and researchers, who may have difficulty finding peers to support and review their work.
Hopefully a better user feedback may also facilitate a larger user base.

The issues related with the lack of libraries and the bottom-up approach are named by both Lapets \citeac{lapets2009improving} and Bos \citeac{bos2013bedrijfsregels}.
To address this problem the institute Informatiegilde has been created \citeac{foutvrije-specificaties}.
This institute aims to organize an open-source repository of rules that can be used by anyone.

\subsection{Parsing errors}
%TODO: how can we integrate some info regarding the parser research context in regards to natural language processors, chatbots...
Although the new Ampersand parser has no influence on the research for better user feedback in parsers, it is important to identify this context because it is of great importance for the project.
Our literature research was focused on the error messages for parsers implemented in Haskell, and then mainly the LL($k$) parsers built with the Parsec library, which was chosen during our domain and technique research \citenac{parsing}.

Parsec has extensive error messages; giving position, unexpected input, expected productions and general user messages.
The error messages are given in terms of high-level grammar productions and can be localized for different (natural) languages \citeac{parsec-fast}.
Special care must be taken when using backtracking (the $try$-function) in the parsec library, as it may negatively influence the error messages generated \citeac{parsec}\citenac{try-harmful}.

Swierstra and Duponcheel \citeac{swierstra1996deterministic} describe how automatic error correction can be implemented.
This automatic error correction was used in the previous Ampersand parser.
However, the shortcomings of error correction have been described in our techniques research \cite{parsing} so this possibility is not used in the new Ampersand parser.












% !TEX root = ../ResearchContext.tex

\section{Expert consultation}
\label{sec:expert-consultation}
The following questions have been formulated during and after the literature research.
The questions have been asked to a researcher involved in the domain of Rule Based Design and the Semantic Web, namely Lloyd Rutledge.
Lloyd is currently an Assistant Professor at the Open Universiteit.
During the literature study, Lloyd already helped us finding interesting investigation topics and possible links to other research contexts during two alignment sessions.
The questions below are presented in the order they are listed below.
His answers are recorded literally after each of the questions, our own perception of the answers in relation to the research context are summarized in the conclusion, \autoref{sec:conclusion}.

~\\
\question{What do you use Ampersand for?}
\answer{
Students in our Masters course Rule-Based Design use Ampersand to implement business rules from case studies as a course assignment, which determines a large part of their course grade.
I also supervise several Masters Thesis students who perform their own research using Ampersand.
They often implement a case study in Ampersand to test hypotheses about how to establish business rules.
Our main current research theme is design patters for business rules.
We are starting a series of Masters Thesis research projects that propose rule design patterns and evaluate them in various case studies.
}

\question{What are the largest user complaints when using Ampersand?}
\answer{
Mostly complaints regard lack of documentation.
Inconsistencies between new versions of the software and the older software that exists is also a concern.
}

\question{Do you know whether users appreciate the error correction done by the previous parser?}
\answer{
No, I do not.
}

\question{Would better user feedback allow for a more efficient research on formal business rules?}
\answer{
I see no direct relation between user feedback and research.
It may have an indirect effect by making all work with Ampersand easier.
}

\question{Has the maintainability of the previous parser been an issue in your research?}
\answer{
I have had no complaints from Masters Thesis researchers, just as the course students have not complained to me about the parser.
It could be because the students are often business and management students, and thus the course may often be seen as a management theory course.
And then the course is secondly seen as the application of formal logic to business rules.
But on that level it is more a mathematical exercise than a computer project.
That there is software to process the rules into charts may be seen by many students as a secondary tool for helping establish the logic formalisms and the more strictly written business rules in natural language.

The research that I directly execute is more about the Semantic Web, its logic and its tools.
}
  
\question{How can a new Ampersand parser and better user feedback support the objectives of the Ampersand project?}
\answer{
By helping users more quickly write Ampersand scripts that function as expected and that deliver the desired insights.
}

\question{Is a better user feedback sufficient to optimize the learning path of students?
Or should a syntax simplification be considered for this target group?}
\answer{
There are too many aspects for optimizing learning paths to state that one tool would be sufficient.
Syntax simplification is another research area.
It has often been applied to business rules, both as non-computer and computer-processed syntaxes.
Our research's logical foundation is relation algebra, and so for us any simplified syntax would have to be equivalent to relation algebra.
Keywords may be easier in the beginning to work with than logical symbols.
But in the long run it is the form and structure of the logic itself that students need to learn to master.
}

\question{What are the next challenges for your research using the Ampersand tool?}
\answer{
My own direct research involves the relation between Ampersand, and thus relation algebra, to Semantic Web languages, and thus the tools that implement them, and also thus the logical formalism they're based on.
I will explore how Semantic Web tools can set up the equivalent of the business rules that Ampersand sets up.
Semantic Web tools such as Protégé have rich user interfaces such as GUI's for authoring data models and rules.
Parser concerns become then no longer relevant because the GUI interface restricts user interaction to making code that has to be valid.
If all Ampersand functionality could be programmed with such a tool, then many of the concerns motivated the Ampersand parser are addressed.
}

\question{Is it harder for you to define the business rules or to express and compile them in ADL?}
\answer{
Forming the logical structure is often a larger intellectual challenge than the grammar.
}

\question{Would you like to suggest any improvements to the ADL grammar?}
\answer{
A tool could replace the symbols with easily understood natural language keywords, but that only helps in the beginning.
}

\question{Is Ampersand the right tool for your research?}
\answer{
It is the right tool for my Masters students who research relation algebra.
My direct research involves the Semantic Web more, and in particular the relation between relation algebra and the Semantic Web.
Relation algebra is an expressive superset of the Semantic Web in terms of logic formalism.
Thus where I rely mostly on Ampersand is to explore the logic that the Semantic Web does directly support.
}

% !TEX root = ../ResearchContext.tex

\section{Answers}
\label{sec:answers}
\lipsum[1]
\newpage
% !TEX root = ../Parsing.tex

\section{Conclusion}
\label{sec:conclusion}
In \autoref{sec:libraries}, the advice was given to use a combinator library for the new parser of Ampersand.
The main reason to avoid the parser generators is that it is hard to generate useful feedback.
Then, in \autoref{sec:errors}, it was made even more clear that besides generating good messages, those messages should also be customizable.

Therefore, the advice of this research is to use the combinator library that offers the highest level of customization in error messages, Parsec.
Although the uu-parsinglib seems to also be a very good choice, the experiences from the Helium compiler \citeac{helium-parser} should be also considered.
Besides, the Parsec library offers better support.

A list of important consideration points has also been collected through the literature and can be found in \autoref{sec:errors}, more specifically \ref{subsec:errors-ampersand}.

\part*{Appendices}
\addcontentsline{toc}{part}{Appendices}
\appendix
% !TEX root = ../Parsing.tex

\small
\printglossary[style=mcolindex,title=Glossary]
\label{sec:glossary}

\newpage
% !TEX root = ../Parsing.tex
\addcontentsline{toc}{section}{References}
\label{sec:bibliography}

\begin{thebibliography}{99}

\bibitem{plan}
	Planning for the project `Useful feedback in the Ampersand parser'\\
	Maarten Baertsoen and Daniel S. C. Schiavini\\
	Version 2.0 -- November 29, 2014\\
	\url{http://git.io/NeHuLg}

\bibitem{heeren-error}
	Top Quality Type Error Messages\\
	Bastiaan Heeren\\
	ISBN 90-393-4005-6, September 20, 2005\\
	\url{http://www.open.ou.nl/bhr/phdthesis}

\bibitem{monadic-parsing}
	Functional pearls -- Monadic Parsing in Haskell\\
	Graham Hutton (University of Nottingham) and Erik Meijer (University of Utrecht)\\
	\url{http://www.cs.nott.ac.uk/~gmh/monparsing.pdf}

\bibitem{convert-ebnf}
	 From EBNF to BNF \\
	 Christoph Zenger\\
	 June 4, 2000\\
	 \url{http://lampwww.epfl.ch/teaching/archive/compilation-ssc/2000/part4/parsing/node3.html}

\bibitem{bnf-ebnf}
	BNF and EBNF: What are they and how do they work?\\
	Lars Marius Garshol\\
	August 22, 2008\\
	\url{http://www.garshol.priv.no/download/text/bnf.html}

\bibitem{parser-examples}
	Haskell Parser Examples\\
	Geoff Hulette\\
	August 22, 2014\\
	\url{https://github.com/ghulette/haskell-parser-examples}

\bibitem{hugs-parser}
	Source code of the Hugs parser\\
	March 25, 2007\\
	\url{https://github.com/fuzxxl/Hugs/blob/master/src/parser.y}

\bibitem{ghc-parser}
	GHC: The Parser\\
	December 1, 2014\\
	\url{https://ghc.haskell.org/trac/ghc/wiki/Commentary/Compiler/Parser}
	%\url{https://ghc.haskell.org/trac/ghc/browser/ghc/compiler/parser/Parser.y}
	%https://www.haskell.org/pipermail/haskell-cafe/2013-August/109557.html

\bibitem{helium-parser}
	Helium, for Learning Haskell\\
	Bastiaan Heeren, Daan Leijen, Arjan van IJzendoorn\\
	Utrecht University\\
	\url{http://www.open.ou.nl/bhr/heeren-helium.pdf}
	
\bibitem{gcc-c-parser}
	GCC 4.1 Release Series Changes, New Features, and Fixes\\
	\url{https://gcc.gnu.org/gcc-3.4/changes.html}

\bibitem{gcc-cpp-parser}
	GCC 3.4 Release Series Changes, New Features, and Fixes\\
	\url{https://gcc.gnu.org/gcc-4.1/changes.html}
	
\end{thebibliography}

\end{document}